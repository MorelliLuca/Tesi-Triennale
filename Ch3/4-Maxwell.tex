\section{Elettromagnetismo in forma 4-vettoriale}
Nella Sezione \ref{sec:4-equazioniMotoEM} è stato ricavato il 4-tensore elettromagnetico $F^{\mu\nu}=\partial^\mu A^\nu-\partial^\nu A^\mu\ $\footnote{In questa formula si è adottata la notazione alternativa, che verrà utilizzata anche successivamente, per le derivate parziali, già introdotta nella Sezione \ref{sec:4-vettori}, secondo la quale: $\partial^\mu=\frac{\partial\ }{\partial x_\mu}$ e $\partial_\mu=\frac{\partial\ }{\partial x^\mu}$.}:
\begin{equation*}
        F^{\mu\nu}=
    \begin{pmatrix}
        0&-E_x/c&-E_y/c&-E_z/c\\
        E_x/c&0&-B_z&B_y\\
        E_y/c&B_z&0&-B_x\\
        E_z/c&-B_y&B_x&0\\
 \end{pmatrix},\qquad
 F_{\mu\nu}=
 \begin{pmatrix}
     0&E_x/c&E_y/c&E_z/c\\
     -E_x/c&0&-B_z&B_y\\
     -E_y/c&B_z&0&-B_x\\
     -E_z/c&-B_y&B_x&0\\
\end{pmatrix}.
\end{equation*}
Questo tensore fa le veci dei campi elettromagnetici nel formalismo 4-vettoriale dello spazio-tempo della relatività.
È quindi ora opportuno ricavare anche la forma delle equazioni di Maxwell \eqref{EquazioniMaxwell} 4-vettoriale così da completare il quadro dell'elettromagnetismo nel formalismo della relatività.
\subsection{4-corrente}\label{sec:4-corrente}
Nella Sezione \ref{sec:EquazioniMaxwell} si è già illustrato come sperimentalmente la carica elettrica sia una quantità invariante sotto cambio di sistema di riferimento. Questo, essendo un fatto sperimentale, continua a valere anche nella teoria relatività. Sempre nella Sezione \ref{sec:EquazioniMaxwell}, sono state introdotte due quantità sorgenti del campo elettro magnetico: la densità di carica volumetrica e la densità di corrente superficiale. Rispettivamente:
\begin{equation}
    \rho=\lim_{\Delta V\rightarrow 0}\frac{\Delta q}{\Delta V},\qquad\qquad \vec J=\rho\vec v(t,\vec x),\label{defRhoJ}
\end{equation}
dove $\Delta q$ è la carica contenuta nel volume $\Delta V$ e $\vec v$ è il campo di velocità dato dal moto delle singole cariche che compongono la distribuzione descritta da $\rho$.\\

Si supponga di avere due sistemi di riferimento: $K^0$ nel quale tutte le cariche sono ferme e un secondo $K$ nel quale queste si muovono a velocità $\vec{v}$. In questo modo dalla \eqref{defRhoJ} si possono definire due densità di carica, $\rho_0$ in $K^0$ e $\rho$ in $K$, tali che, presa una regione di spazio $\mathcal{V}_0$ e la sua trasformata secondo un boost di Lorentz $\mathcal{V}$, si abbia:
\begin{equation*}
    Q=\int_{\mathcal{V}_0}\rho_0(t^{(0)},\vec{ x^{(0)}})\ d^3x^{(0)}=\int_{\mathcal{V}}\rho(t,\vec x)\ d^3x.
\end{equation*}
Nella Sezione \ref{sec:ContrazioneDilatazione} si è mostrato che un parallelepipedo, in seguito ad un boost di Lorentz, subisce una contrazione del suo volume di un fattore $\gamma$. Questo stesso ragionamento può essere applicato all'elemento di volume infinitesimo in coordinate cartesiane che si trasforma: $d^3x^{(0)}=d^3x\gamma$. Così facendo si ottiene, dai due integrali sopra, considerando che il dominio di integrazione è arbitrario:
\begin{equation}
    \int_{\mathcal{V}}\gamma\rho_0(\Lambda^{-1}(t,\vec{ x}))\ d^3x-\int_{\mathcal{V}}\rho(t,\vec x)\ d^3x=0\quad\Rightarrow\quad\gamma\rho_0=\rho.
\end{equation}
Questa è la trasformazione della densità di carica da un sistema solidale alle cariche ad un altro sistema di rifeirmento inerziale. Si osservi che, per questa trasformazione, $\rho_0$ è uno scalare Lorentz invariante al pari del tempo proprio (infatti in ogni sistema di riferimento inerziale è possibile, misurando $\rho$ e la velocità del moto delle cariche, ottenere il valore di $\rho_0$ e che risulta il medesimo per tutti gli osservatori).\\
Dalla definizione di densità di corrente superficiale \eqref{defRhoJ} si ha quindi che nel sistema di riferimento $K$ vale $\vec{J}=\rho\vec v=\rho_0\gamma\vec{v}$.\\
Se ora si considera il 4-vettore velocità e lo si moltiplica per $\rho_0$ si ottiene un secondo 4-vettore (poiché quest'ultima è un invariante):
\begin{equation}
    J^\mu=u^\mu\rho_0=(\rho_0c,\vec v\rho_0)\gamma=(\rho c,\vec J),\label{4-corrente}
\end{equation} 
questo 4-vettore è chiamato \emph{4-corrente}.\\
Si osservi che se si calcola la 4-divergenza della 4-corrente si ottiene:
\begin{equation}
    \partial_\mu J^\mu=\frac{\partial \rho}{\partial t}+\vnabla\cdot\vec J,\label{continuitàCarica}
\end{equation}
che è parte dell' \emph{equazione di continuità della carica}. Inoltre questa quantità è uno scalare Lorentz invariante\footnote{In questo caso accade che, trasformando la 4-corrente e gli operatori di differenziazione, si ha: $\partial'_\mu J'^\mu=\partial_\nu (\Lambda^{-1})_\mu^\nu \Lambda_\lambda^\mu J^\lambda=\partial_\nu \delta_\lambda^\nu J^\lambda=\partial_\nu J^\nu$.}, per cui se si valuta la \eqref{continuitàCarica} nel sistema $K^0$ (dove risulta nulla essendo le cariche in tale sistema immobili) si ha che deve essere nulla in ogni riferimento inerziale, così che valga in ognuno di essi l'equazione di continuità: $\partial_\mu J^\mu=0$.

\subsection{Equazioni di Maxwell 4-vettoriali}\label{sec:4-Maxwell}
Si procederà ora a ricavare la forma 4-vettoriale delle equazioni di Maxwell. Per farlo si considerino le equazioni di Maxwell in cui sono state sostituiti il potenziale $\varphi$ e il potenziale vettore $\vec A$ (ottenuti nella Sezione \ref{sec:gauge}, Eq. \eqref{MaxwellPote1} e Eq. \eqref{MaxwellPote2}):
\begin{flalign*}
    &\vnabla^2\varphi+\vnabla\cdot \frac{\partial \vec A}{\partial t}=-\frac{\rho}{\epsilon_0},\\
    &\vnabla^2\vec A-\epsilon_0\mu_0\frac{\partial^2 \vec A}{\partial t^2}-\vnabla\bigg(\vnabla\cdot\vec A +\epsilon_0\mu_0\frac{\partial \varphi}{\partial t}\bigg)=-\mu_0\vec J.
\end{flalign*}
Le restanti equazioni di Maxwell, una volta sostituiti i potenziali, si riducono a semplici identità.\\
Indicando con lettere latine i soli indici ${1,2,3}$, gli operatori di differenziazione che appaiono nelle equazioni precedenti si scrivono:
\begin{equation*}
    \vnabla=\partial_i,\qquad \vnabla^2=\partial_i\partial_i,\qquad\frac{\partial}{\partial t}=c\partial_0.
\end{equation*}
In questo modo, utilizzando il potenziale vettore $A^\mu=(\frac{\varphi}{c},\vec A)$, le due equazioni di Maxwell \eqref{MaxwellPote1} e \eqref{MaxwellPote2} assumono la forma:
\begin{flalign*}
    &\partial_i\partial_i A^0+\partial_i\partial_0A^i=-\frac{\rho}{\epsilon_0}c,\\
    &\partial_i\partial_iA^k-\frac{\epsilon_0\mu_0}{c^2}\partial_0\partial_0A^k-\partial_k\bigg(\partial_iA^i +\frac{\epsilon_0\mu_0}{c^2}\partial_0A^0\bigg)=-\mu_0J^k,
\end{flalign*}
Ricordando che $\epsilon_0\mu_0=\frac{1}{c^2}$ e che nel passare da vettori covarianti a controvarianti la matrice metrica cambia solo il segno delle componenti spaziali di un 4-vettore si ottiene:
\begin{flalign*}
    &-\partial_i\partial^i A^0+\partial_i\partial^0A^i=\partial_i\partial^0A^i-\partial_i\partial^i A^0=-\mu_0\rho c,\\
    &-\partial_i\partial^iA^k-\partial_0\partial^0A^k+\partial^k(\partial_iA^i +\partial_0A^0)=\partial^k\partial_\mu A^\mu-\partial_\mu\partial^\mu A^k=-\mu_0J^k.
\end{flalign*}
A questo punto è necessario supporre che i potenziali abbiano sempre una regolarità tale da soddisfare le ipotesi del teorema di Schwarz, così che si possa scambiare l'ordine delle derivate parziali. Così facendo, sommando (e sottraendo) nella prima equazione opportuni termini si ha:
\begin{flalign*}
    &\partial_i\partial^0A^i-\partial_i\partial^i A^0+\partial_0\partial^0A^0-\partial_0\partial^0A^0=\partial_\mu\partial^0A^\mu-\partial_\mu\partial^\mu A^0=-\mu_0\rho c,\\
    &\partial^k\partial_\mu A^\mu-\partial_\mu\partial^\mu A^k=\partial_\mu\partial ^kA^\mu-\partial_\mu\partial^\mu A^k=-\mu_0J^k.
\end{flalign*}
Infine riconoscendo in queste equazioni che $\partial^\nu A^\mu-\partial^\mu A^\nu=-F^{\mu\nu}\ $\footnote{$F^{\mu\nu}$ dalla sua definizione è antisimmetrico per cui $F^{\mu\nu}=-F^{\nu\mu}$} e $\rho c=J^0$ si trova
\begin{flalign*}
    &\partial_\mu(\partial^0A^\mu-\partial^\mu A^0)=-\partial_\mu F^{\mu 0}=-\mu_0 J^0,\\
    &\partial_\mu(\partial ^kA^\mu-\partial^\mu A^k)=-\partial_\mu F^{\mu k}=-\mu_0J^k.
\end{flalign*}
Queste se riunite in un'unica equazione rappresentano la forma 4-vettoriale delle equazioni di Maxwell:
\begin{equation}
    \partial_\mu F^{\mu\nu}=\mu_0J^\nu.\label{4-Maxwell}
\end{equation} 
Così facendo risulta evidente la covarianza di queste equazioni, che è stata richiesta in primo luogo per ricavare la relatività speciale.