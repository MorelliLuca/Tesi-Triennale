Si consideri una carica $q$ piccola\footnote{Questa assunzione è necessaria in quanto si vuole considerare inizialmente che la carica non modifichi il campo elettromagnetico che ne influenza il moto.}, a questa è associata una lagrangiana $\mathcal{L} $ costituita da una parte di corpo libero e una parte d'interazione con i campi elettrici e magnetici esterni, si vuole ricavare la forma di questa seconda parte.\\Come si è detto nella sezione \ref{sec:EquazioniMaxwell} tale carica subisce la forza di Lorentz e siccome la relatività assume la formulazione dell'elettromagnetico di Maxwell come la formulazione corretta si può procedere ricavando la forma della lagrangiana d'interazione tra carica e campi da questa forza. Secondo la teoria di Maxwell è possibile descrivere i due campi tramite un potenziale scalare $\phi(t,\vec x)$ e un potenziale vettore $\vec{A}(t,\vec x)$ tali che:
\begin{equation}
    \vec{E}=-\vnabla\varphi-\frac{\partial \vec{A}}{\partial t} \qquad \vec B=\vnabla\wedge\vec A.\label{PotenzialiEM}
\end{equation}
È opportuno osservare che queste due quantità sono in realtà definite a meno di termini opportuni: il potenziale $\varphi$ è definito a meno di una costante rispetto alle coordinate spaziali che non varia il campo elettrico tramite l'operazione di gradiente e il potenziale vettore $\vec A$ è definito a meno di un gradiente di un campo scalare, siccome il rotore di un gradiente è sempre nullo. Questa arbitrarietà prende il nome di \emph{simmetria di Gauge}.\\
Sfruttando questi due potenziali la forza di Lorentz assume la forma:
\begin{flalign}
    \vec F&=q\bigg(\vec E(t,\vec x)+\vec v\wedge\vec B(t,\vec x)\bigg)=q\bigg(-\vnabla\varphi(t,\vec x)-\frac{\partial \vec{A}}{\partial t}(t,\vec x)+\vec v\wedge(\vnabla\wedge\vec A(t,\vec x))\bigg)\nonumber\\
    &=q\bigg(-\vnabla\varphi(t,\vec x)-\frac{\partial \vec{A}}{\partial t}(t,\vec x)+\vnabla(\vec v\cdot\vec A(t,\vec x))-(\vec v\cdot \vnabla)\vec A(t,\vec x)\bigg)\nonumber\\
    &=q\vnabla\bigg[-\varphi(t,\vec x)+\vec v\cdot \vec A(t,\vec x)\bigg]-q\frac{d}{dt}\vec A(t,\vec x)\nonumber\\
    &=-\bigg[\frac{d}{dt}\vnabla_{\vec v}-\vnabla\bigg](-q\varphi(t,\vec x)+q\vec v\cdot\vec A(t,\vec x)),\qquad \qquad\vnabla_{\vec{v}}=\bigg(\frac{\partial}{\partial v_x},\frac{\partial}{\partial v_y},\frac{\partial}{\partial v_z}\bigg).\label{FLorentzLagrangiana}
\end{flalign}
Si noti che nella \eqref{FLorentzLagrangiana} si è ottenuta una quantità scalare su cui agisce l'operatore di Eulero Lagrange in forma vettoriale (sezione \ref{sec:MinimaAzione}). Infatti in meccanica lagrangiana tale operatore agendo sulla lagrangiana d'interazione fornisce proprio la forza che media l'interazione, cambiata di segno.\\ La lagrangiana d'interazione elettromagnetica è quindi
\begin{equation}
    \mathcal{L}_{Int}=-q\varphi(t,\vec x)+q\vec v\cdot\vec A(t,\vec x),
\end{equation}
nota questa l'azione è data dall'integrale della somma di questa con la lagrangiana di corpo libero:
\begin{equation}
    \mathcal{S} [\vec x(t)]=\int_{t_1}^{t_2}\bigg(-mc^2\sqrt{1-\frac{|\vec v|^2}{c^2}}-q\varphi(t,\vec x)+q\vec v\cdot\vec A(t,\vec x)\bigg)\ dt.\label{lagrangianaInt}
\end{equation}
Come si è fatto per il caso della particella libera si può procedere calcolando il vettore impulso generalizzato e l'energia della particella carica interagente con campi elettrici e magnetici. Dalle definizioni \eqref{DefMomentoGeneralizzato} e \eqref{DefEnergiaMeccanica} utilizzate con $\mathcal{L} =\mathcal{L}_{Libera}+\mathcal{L}_{Int}$ si ottiene:
\begin{equation}
    \vec{P}=\frac{m\vec v}{\sqrt{1-\frac{|\vec v|^2}{c^2}}}+q\vec A=\vec{p}+q\vec A,\qquad E=\frac{mc^2}{\sqrt{1-\frac{|\vec v|^2}{c^2}}}+q\phi,\label{energiaImpulsoIntEM}
\end{equation}
dove si è indicato con $\vec P$ l'impulso generalizzato e con $\vec p$ l'impulso della particella libera, che viene in generale chiamato solamente impulso.\\Studiando sistemi di particelle interagenti con campi si indica l'energia che possiederebbe la particella se fosse libera con $E_{Lib}=mc^2\gamma$, questa è l'energia cinetica del sistema a meno di una costante che è l'energia a riposo della particella. \\

I potenziali utilizzati nella lagrangiana \eqref{lagrangianaInt} non sono però quantità misurabili per cui per studiare questi sistemi è opportuno ricavarli dalle loro definizioni \eqref{PotenzialiEM} conoscendo la forma del campo elettrico e magnetico. Nel caso più semplice possibile, ossia di campi costanti nel tempo ed uniformi\footnote{Con uniformi si intende costanti nelle tre dimensioni spaziali.}, i potenziali sono 
\begin{equation}
    \varphi=-\vec E\cdot\vec x, \qquad \qquad \vec A=\frac{\vec B\wedge \vec x}{2},\label{PotenzialiCostanti}
\end{equation}
infatti dalle definizioni si ha:
\begin{equation*}
    -\vnabla{\varphi}=(\vec E\cdot\vnabla)\vec x=\vec{E},\qquad \vnabla\wedge\vec A=\frac{1}{2}[ (\vnabla\cdot\vec x)\vec B-(\vec B\cdot \vnabla)\vec x]=\frac{1}{2}(3\vec B -\vec B)=\vec B.
\end{equation*}