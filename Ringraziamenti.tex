
\chapter*{Ringraziamenti}
\addcontentsline{toc}{chapter}{Ringraziamenti}
\markboth{Ringraziamenti}{Ringraziamenti}
Al termine di questi tre anni non posso non ringraziare tutte le persone che mi hanno accompagnato fino ad oggi. Primi tra tutti desidero ringraziare i miei genitori: fin da quando ero bambino hanno coltivato la mia curiosità da cui è nata la passione per la fisica. Grazie di cuore per l'amore che mi avete dato supportandomi sempre, specialmente quando ho incontrato le prime difficoltà della vita.\\Voglio ringraziare anche il professor Paolo Albano, che mi ha seguito nella preparazione di questa tesi, per tutto il tempo e le attenzioni che mi ha dedicato. Grazie al suo lavoro e alla sua disponibilità di confronto non solo ho avuto modo di apprendere gli argomenti trattati in questa sede ma ho anche potuto approfondire e consolidare molti aspetti fondamentali di quanto ho studiato in questi anni. Da lei ho anche imparato come approcciarsi ad un lavoro come questo e a come scriverlo affinché risulti chiaro e comprensibile.\\ Desidero ringraziare Ester, Luca Bergamini e Samuele che, fin da prima dell'inizio dell'università, sono sempre stati al mio fianco con la loro amicizia. La vostra vicinanza mi ha aiutato non solo durante la pandemia ma anche nel momento in cui ho dovuto lasciare casa per venire a vivere da solo a Bologna. Un particolare ringraziamento va a Sua Eccellenza Gianni Sacchi che in questi tre anni mi ha supportato specialmente nei momenti di crisi personale. Grazie a tutti voi di essere sempre stati per me un porto sicuro in cui fare ritorno.\\Desidero ringraziare tutti gli amici che ho incontrato a Bologna. Per primo Marco Piolanti che, tra gli alti e i bassi delle nostre vite universitarie, è sempre stato un vero amico su cui poter contare. Grazie per tutte le serate passate assieme e tutte le nostre infinite discussioni filosofiche. Non posso dimenticare Cristina con cui ho abitato in questi anni. Grazie per avermi sopportato come coinquilino ed esserti prestata come cavia per i miei esperimenti di cucina. Ringrazio anche Andrea, Damiano, Federico, Gabriele, Lorenzo, Luca, Matteo Bonazzi, Matteo Zandi, Marco Benazzi e Marco Mullazani con cui ho condiviso questo cammino di lezione in lezione, di esame in esame e di giornata in giornata. Tutti i pranzi, tutti i caffè al ginseng e tutti gli integrali complessi che abbiamo condiviso rimarranno per sempre un ricordo indelebile di questi anni. Infine, ma non per importanza, voglio ringraziare tutti i ragazzi del Treno dei Clochard. Ogni venerdì sera mi avete accolto come amico condividendo con me le vostre settimane. 
