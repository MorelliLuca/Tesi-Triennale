\section{Formalismo lagrangiano per la teoria dei campi}
In relatività l'interazione di una particella carica con un campo elettromagnetico è descritta da un'azione nella quale compaiono accoppiate la 4-velocità della particella e il 4-potenziale del campo. Questa è adatta a descrivere solamente particelle con carica piccola (tale da non modificare il campo elettromagnetico). Per descrivere il moto di cariche capaci di modificare il campo stesso è necessario che nell'azione relativistica compaia un termine che, tramite il principio di minima azione, determini anche la dinamica del campo. La teoria che riesce in questo intento è la \emph{teoria dei campi}.
\label{sec:lagCampi}
Seguendo la monografia di Barone \cite{Barone} si vuole ora descrivere questo formalismo.
\subsection{Il principio di minima azione per un campo}
Per prima cosa si supponga che per un campo $\varphi(x^\mu)$ sia possibile costruire una funzione $\mathfrak{L}(\varphi,\partial_\mu\varphi,x^\nu)$, chiamata \emph{densità di lagrangiana}, che ne descrive le proprietà.
Si definisce allora l'azione di un campo il seguente funzionale:
\begin{equation}
    \mathcal{S}[\varphi]=\int\int_{\Omega}\mathfrak{L}(\varphi,\partial_\mu\varphi,x^\nu)\ d^3x\ dt,\label{azioneCampo}
\end{equation}
dove l'integrale è calcolato su un dominio 4-dimensionale $\Omega$ \footnote{Questo volume può anche non essere finito.} nello spazio-tempo.\\
Il principio di minima azione afferma in questo caso che: supponendo che $\varphi$ e $\partial_\mu\varphi$ si annullino con sufficiente rapidità sulla frontiera $\partial\Omega$, il campo $\varphi(t,\vec x)$ realmente osservato nel sistema è tale da rendere stazionaria l'azione \eqref{azioneCampo}. Usando questo principio si possono ricavare le equazioni di Eulero-Lagrange per un campo.\\

 Si consideri un secondo campo (detto variazione di $\varphi$) $\delta\varphi(t,\vec x)$ che assume valori piccoli in $\Omega$ e si annulla sulla sua frontiera. Il campo $\varphi+\delta\varphi$ rispetta ancora le ipotesi del principio di minima azione. Si osservi che così facendo le coordinate e il 4-volume $\Omega$ non vengono variate. Affinché $\varphi$ sia il campo che rende stazionaria l'azione deve essere nulla la variazione $\delta\mathcal{S}$ generata da $\delta\varphi$:
\begin{equation}
    \delta\mathcal{S}=\mathcal{S}[\varphi+\delta\varphi]-\mathcal{S}[\varphi]\approx\int\int_{\Omega}\bigg[\frac{\partial\mathfrak{L}}{\partial \varphi}\delta\varphi+\frac{\partial\mathfrak{L}}{\partial \partial_\mu\varphi}\delta\partial_\mu\varphi\bigg]\ d^3x\ dt=0,\label{minimaAzioneCampiFull}
\end{equation}
dove l'integrando è lo sviluppo di Taylor al prim'ordine della densità di lagrangiana rispetto alle variabili che vengono variate.\\
In primo luogo si osservi che, siccome si sta considerando una sola variazione del campo e non delle coordinate, si ha
\begin{equation*}
    \delta\partial_\mu\varphi=\partial_\mu(\varphi+\delta\varphi)-\partial_\mu\varphi=\partial_\mu\varphi+\partial_\mu\delta\varphi-\partial_\mu\varphi=\partial_\mu\delta\varphi,
\end{equation*}
per cui la \eqref{minimaAzioneCampiFull} diventa:
\begin{equation}
    \delta\mathcal{S}\approx\int\int_{\Omega}\bigg[\frac{\partial\mathfrak{L}}{\partial \varphi}\delta\varphi+\frac{\partial\mathfrak{L}}{\partial \partial_\mu\varphi}\partial_\mu\delta\varphi\bigg]\ d^3x\ dt=0.\label{minimaAzioneCampiFull'}
\end{equation}
In più dimensioni vale la formula di integrazione per parti\footnote{Questa è data integrando: $\partial_\mu(f(x)V^\mu(x))=f(x)\partial_\mu V^\mu(x)+V^\mu(x)\partial_\mu f(x)$, $x\in \mathbb{R}^4$.} per un prodotto $V^\mu(x)\partial_\mu f(x) $ integrato su un 4-volume $\mathcal{V}$ ($x\in \mathbb{R}^4$):
\begin{equation*}
    \int_{\mathcal{V} }V^\mu(x)\partial_\mu f(x)\ d^4x=\int_{\mathcal{V} }\partial_\mu(f(x)V^\mu(x))\ d^4x-\int_{\mathcal{V} }f(x)\partial_\mu V^\mu(x)\ d^4x.
\end{equation*}
Utilizzando il Teorema di Gauss nel primo addendo della parte di destra di quest'ultima equazione si ha:
\begin{equation*}
\int_{\mathcal{V} }V^\mu(x)\partial_\mu f(x)\ d^4x=\int_{\partial\mathcal{V} }f(x)V^\mu(x)n_\mu\ d\sigma-    \int_{\mathcal{V} }f(x)\partial_\mu V^\mu(x)\ d^4x,
\end{equation*}
dove $d\sigma$ è l'elemento di 4-superfice di $\partial\mathcal{V}$ e $n_\mu$ è il 4-vettore normale ad esso.\\
Integrando per parti in questo modo il secondo addendo ad integrando della \eqref{minimaAzioneCampiFull'} si ha:
\begin{flalign*}
    \int\int_{\Omega}\bigg[\frac{\partial\mathfrak{L}}{\partial \varphi}\delta\varphi+\frac{\partial\mathfrak{L}}{\partial \partial_\mu\varphi}\partial_\mu&\delta\varphi\bigg] \ d^3x\ dt\\&=\int\int_{\Omega}\bigg[\frac{\partial\mathfrak{L}}{\partial \varphi}\delta\varphi-\partial_\mu\frac{\partial\mathfrak{L}}{\partial \partial_\mu\varphi}\delta\varphi\bigg]\ d^3x\ dt+\int_{\partial\Omega }\frac{\partial\mathfrak{L}}{\partial \partial_\mu\varphi}n_\mu\delta\varphi\ d\sigma=0,
\end{flalign*}
dove $d\sigma$ è l'elemento di 4-superfice di $\partial\Omega$ e $n_\mu$ è il 4-vettore normale ad esso.\\
Per le ipotesi ($\delta\varphi$ si annulla su $\partial\Omega$) l'ultimo addendo si annulla e si ottiene:
\begin{equation*}
    \delta\mathcal{S}\approx\int\int_{\Omega}\bigg[\frac{\partial\mathfrak{L}}{\partial \varphi}-\partial_\mu\frac{\partial\mathfrak{L}}{\partial \partial_\mu\varphi}\bigg]\delta\varphi\ d^3x\ dt.
\end{equation*}
Siccome questo integrale deve annullarsi per ogni variazione $\delta\varphi$ (che è arbitraria) è necessario che si annulli l'integrando. Si ottiene quindi:
\begin{equation}
    \partial_\mu\frac{\partial\mathfrak{L}}{\partial \partial_\mu\varphi}-\frac{\partial\mathfrak{L}}{\partial \varphi}=0\label{euleroLagrnageCampo}
\end{equation}
che è l'\emph{equazione di Eulero-Lagrange} per un campo scalare $\varphi$.\\Volendo ottenere le equazioni differenziali per un campo vettoriale $\varphi_\nu(x^\mu)$ è possibile considerare ogni componente come un campo scalare che è soluzione di una \eqref{euleroLagrnageCampo}. In questo modo si hanno tante equazioni di Eulero-Lagrange quante le componenti:
\begin{equation}
    \partial_\mu\frac{\partial\mathfrak{L}}{\partial \partial_\mu\varphi_\nu}-\frac{\partial\mathfrak{L}}{\partial \varphi_\nu}=0\label{euleroLagrnageCampi}.
\end{equation}
Infine si osservi che $\mathfrak{L}$ è definita a meno di una costante moltiplicativa e di una divergenza di un campo vettoriale. Infatti moltiplicando la densità di lagrangiana per una costante le equazioni differenziali che si ottengono dalla \eqref{minimaAzioneCampiFull'} non cambiano. Sommando invece a $\mathfrak{L}$ una divergenza di un campo scalare $\partial_\mu A^\mu(x^\nu)$ e usando il Teorema di Gauss si ottiene nell'azione un termine pari a:
\begin{equation*}
    \int\int_{\Omega}\partial_\mu A^\mu \ d^3x\ dt=\int_{\partial\Omega} A^\nu n_\nu\ d\sigma,
\end{equation*}
dove $d\sigma$ è l'elemento di 4-superfice di $\partial\Omega$ e $n_\mu$ è il 4-vettore normale ad esso.\\ Per ipotesi la variazione di questo termine deve annullarsi (siccome i campi da cui può dipendere $\mathfrak{L}$ devono annullarsi sulla frontiera di $\Omega$) non contribuendo alle equazioni che si ottengono dalle equazioni di Eulero-Lagrange \eqref{minimaAzioneCampiFull'}.\\

Per un sistema di più campi si definiscono i campi \emph{canonicamente coniugati}:
\begin{equation}
    \pi^\nu(x^\mu)=\frac{\partial \mathfrak{L}}{\partial\partial_{0}\varphi_\nu}(x^\mu)\label{campo coniugato}.
\end{equation}
Si definisce anche la \emph{densità di energia} tramite la trasformata di Legendre della densità di lagrangiana:
\begin{equation}
    \mathfrak{E} =\sum_{\nu}\pi^\nu\partial_0\varphi_\nu-\mathfrak{L} \label{densktà di energia}.
\end{equation}
