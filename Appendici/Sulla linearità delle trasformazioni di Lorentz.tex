\chapter{Lemmi e teoremi per le trasformazioni di Lorentz}
\label{chap:LinearitàLorentz}
 Nelle seguenti pagine sono riportati alcuni risultati utili per la derivazione delle trasformazioni di Lorentz tra cui una dimostrazione dell'affinità delle trasformazioni di Lorentz riportata da Fock in \cite{Fock}.\\

\section{Risultati generali}


\begin{lemma}
    Siano $l_1: \mathbb{R}^n\rightarrow\mathbb{R}$ una forma lineare e $q:\mathbb{R}^n\rightarrow\mathbb{R}$ una forma quadratica. Se vale $q|_{\ker f}=0$ allora $\exists l_2: \mathbb{R}^n\rightarrow\mathbb{R}$ forma lineare tale che:
    \begin{equation*}
        q=l_1l_2.
    \end{equation*}\label{lemm:A1}
\end{lemma}
\begin{proof}
    Per prima cosa si osservi che $\exists A\in M_{n\times n}(\mathbb{R} )$ tale che:
    \begin{equation}
        q(v)=\langle v,Av\rangle,\qquad \forall v\in\mathbb{R}^n,\label{qvAv}
    \end{equation}
    dove si è indicato con $\langle \cdot ,\cdot \rangle$ il prodotto scalare euclideo.\\Inoltre $\exists u\in\mathbb{R}^n$ tale che:
    \begin{equation}
        l_1(v)=\langle u,v\rangle,\qquad \forall v\in\mathbb{R}^n.\label{luv}\
    \end{equation}
    Si osservi che è possibile decomporre $v$ in una componente ortogonale ad $u$ ed una parallela:
    \begin{equation}
        v=v^\parallel +v^\bot =\frac{u \langle u,v\rangle}{\|u\|^2}+v-\frac{u\langle u,v\rangle}{\|u\|^2}.
    \label{Al1Decomposizione}
    \end{equation}  
    Esprimendo $q(v)$ secondo questa decomposizione si ottengono 3 addendi:
    \begin{flalign*}
            q(v)&=\langle  v^\parallel +v^\bot,A(v^\parallel +v^\bot)\rangle\\
            &=\ \langle v^\parallel ,Av^\parallel \rangle+2\langle v^\bot,Av^\parallel\rangle+\langle v^\bot,Av^\bot\rangle.
    \end{flalign*}
    Per la \eqref{qvAv} il termine $\langle v^\bot,Av^\bot\rangle$ è pari a $q(v^\bot)$ e per costruzione $l_1(v^\bot)=\langle u,v^\bot\rangle=0$. Ne segue che deve annullarsi $q(v^\bot)$ (per ipotesi $q|_{\ker f}=0$) per cui:
   \begin{equation*}
    q(v)=\langle v^\parallel ,Av^\parallel \rangle+2\langle v^\bot,Av^\parallel\rangle.
   \end{equation*}
   Sostituendo in quest'ultima relazione le espressioni di $v^\bot$ e $v^\parallel$, date dalla (\ref{Al1Decomposizione}), si ha:
   \begin{flalign*}
        q(v)&=\langle u ,Au \rangle\left(\frac{\langle u,v\rangle}{\|u\|^2} \right)^2 +2 \langle v^\bot,Au\rangle \frac{\langle u,v\rangle}{\|u\|^2}\\
        &=\ \langle u ,Au \rangle\left(\frac{\langle u,v\rangle}{\|u\|^2} \right)^2 +2 \langle v,Au\rangle \frac{\langle u,v\rangle}{\|u\|^2} -2\langle u ,Au \rangle\left(\frac{\langle u,v\rangle}{\|u\|^2} \right)^2\\
   \end{flalign*}
   Infine, se si raccoglie un termine $\langle u,v\rangle$ e si sommano i termini uguali, si ottiene un'espressione per $q(v)$ composta da un termine lineare rispetto a $v$ moltiplicato per il prodotto scalare $\langle u,v\rangle$:
   \begin{equation}
    q(v)=\langle u,v\rangle\left[2 \frac{\langle v,Au\rangle}{\|u\|^2}-\frac{\langle u ,Au \rangle\langle u,v\rangle}{\|u\|^4} \right]=l_1(v)l_2(v) \qquad \forall v\in\mathbb{R}^n
   \end{equation}
\end{proof}
Sia $v\in\mathbb{R}^4$, per comodità da ora in poi si scriverà $v=(v_0,v_1,v_2,v_3)=(v_0,\vec{v})$.
\begin{lemma}
     Sia $q:\mathbb{R}^4\rightarrow\mathbb{R}$ una forma quadratica tale che $\bar{v}_0^2-|\vec{\bar{v}}|^2=0$ implica $ q(\bar{v})=0$. Allora $\exists \lambda $ tale che $ q(v)=\lambda(v_0^2-|\vec{v}|^2)\ \forall v\in\mathbb{R}^4$.
    \label{lemm:A2}
\end{lemma}
\begin{proof}
    Si osservi che $q(v)$ può essere scomposto, seguendo le regole del prodotto righe per colonne $q(v)=v^tAv$ (dove $A\in M_{3\times3}(\mathbb{R})$ simmetrica), nella seguente somma:
    \begin{equation*}
        q(v)=a_{00}v_0^2+2v_0(a_{01}v_1+a_{02}v_2+a_{03}v_3)+\hat{q}(\vec{v})
    \end{equation*}
    dove $\hat{q}$ è ancora una forma quadratica in $\mathbb{R}^3$ che agisce su $\vec{v}$.\\Si consideri $\bar{v}=(-|\vec{v}|,\vec{v})$. Per costruzione $(\bar{v}_0)^2-|\vec{\bar{v}}|^2=|\vec{v}|^2-|\vec{v}|^2=0$, per cui per le ipotesi si ha $q(\bar{v})=0$. Essendo $q(\bar{v})=0$, posto $v=(v_0,\vec v)$, si può scrivere:
    \begin{flalign}
        \label{Al2Decomposizione}
            q(v)&=q(v)-q(\bar{v})\\\nonumber
            &=a_{00}(v_0^2-|\vec{v}|^2)+2(v_0-|\vec{v}|)(a_{01}v_1+a_{02}v_2+a_{03}v_3)+\hat{q}(\vec{v})-\hat{q}(\vec{\bar{v}})\\\nonumber
            &=a_{00}(v_0^2-|\vec{v}|^2)+2(v_0-|\vec{v}|)(a_{01}v_1+a_{02}v_2+a_{03}v_3)
    \end{flalign}
   inoltre calcolando la (\ref{Al2Decomposizione}) per $\bar{v}$ si ha:
   \begin{equation*}
    q(\bar{v})=-2|\vec{v}|(a_{01}v_1+a_{02}v_2+a_{03}v_3)=0.
   \end{equation*}
  Essendo $v$ arbitrario si ha che $a_{01}=a_{02}=a_{03}=0$ e quindi la tesi:
   \begin{equation}
    q(v)=a_{00}(v_0^2-|\vec{v}|^2)\quad \forall v\in\mathbb{R}^4
   \end{equation}
\end{proof}
\newpage
\section{Una dimostrazione della linearità delle trasformazioni di Lorentz}
Siano $x,x'\in\mathbb{R}^4$ due vettori dello spazio-tempo misurati in due sistemi di riferimento inerziali $K$ e $K'$. Si vogliono determinare le proprietà dell'applicazione $f:\mathbb{R}^4\rightarrow\mathbb{R}^4$ che trasforma i vettori misurati in $K$ negli stessi vettori misurati in $K'$ e viceversa considerando validi il principio di relatività e il principio di costanza della velocità della luce.\\

Per prima cosa si supporà che $f$ sia sufficentemente regolare, almeno $C^2$, ed è necessario che sia invertibile poichè deve poter trasformare sia da $K$ a $K'$, sia viceversa. Questa condizione è equivalente a richiedere che $\det J_f(x)\neq 0\ \forall x\in\mathbb{R}^4$ ($J_f(x)$ jacobiana di $f$).\\

Si considerino ora $\xi, \gamma \in \mathbb{R}^4$, rispettivamente detti punto iniziale e velocità della parametrizzazione. Questi consentono di parametrizzare un moto rettilineo uniforme nel seguente modo:
\begin{equation*}
    x=\xi+s\gamma \qquad s\in \mathbb{R}.
\end{equation*}
Infatti è possibile esplicitare la dipendenza di $x_0$ da $s$ per ottenere la forma di un moto rettilineo uniforme. Per questo motivo è necessario che $\gamma_0\neq 0$. 
\begin{equation*}
    s=\frac{x_0-\xi_0}{\gamma_0} \qquad \Rightarrow \qquad x_i=\xi_i+\frac{x_0-\xi_0}{\gamma_0}\gamma_i, \qquad i=1,2,3,
\end{equation*}
Secondo il principio di relatività se si calcola $f(x)=x'$ è necessario che $x'=\xi'+s'\gamma'$, così che un moto rettilineo uniforme resti tale in ogni sistema di riferimento inerziale. Derivando $f(\xi+s\gamma)$ rispetto al parametro $s'$ si ottiene:
\begin{equation*}
    \frac{dx'}{ds'}=\gamma'=J_f(\xi+s\gamma)\frac{ds}{ds'}\gamma.
\end{equation*}
Si consideri ora la i-esima componente di $\gamma'$, dove $i=0,1,2,3$\footnote{In questa appendice non si fa uso della convenzione per cui le lettere latine rappresentano solamente gli indici $1,2,3$.}, e la si divida per $\gamma_0$:
\begin{equation*}
    \frac{\gamma'_i}{\gamma'_0}=\frac{\sum_{k=0}^3\partial_kf_i(\xi+s\gamma)\gamma_k}{\sum_{k=0}^3\partial_kf_0(\xi+s\gamma)\gamma_k}
    \label{ARappGamma}
\end{equation*} 
dove $\partial_k=\frac{\partial}{\partial x_k}$. Da ora in poi si utilizzerà la convenzione degli indici ripetuti di Einstein per cui $\sum_k x_k y_k=x_k y_k$.\\Il rapporto $\frac{\gamma'_i}{\gamma'_0}$ è costante da cui segue immediatamente che $\frac{d}{ds}\frac{\gamma'_i}{\gamma'_0}=0$  e quindi:
\begin{flalign*}
        \frac{d}{ds}&\frac{\partial_kf_i(\xi+s\gamma)\gamma_k}{\partial_jf_0(\xi+s\gamma)\gamma_j}=\\&
    =\frac{\left(\partial^2_{hn}f_i(\xi+s\gamma)\gamma_h\gamma_n\right)\left(\partial_kf_0(\xi+s\gamma)\gamma_k\right)
    - \left(\partial_kf_i(\xi+s\gamma)\gamma_k\right)\left(\partial^2_{hn}f_0(\xi+s\gamma)\gamma_h\gamma_n\right)}{\left(\partial_jf_0(\xi+s\gamma)\gamma_j\right)^2}=0    
\end{flalign*}
che può annullarsi solo se si annulla il numeratore della frazione. Da questa osservazione e dal fatto che preso un $x\in \mathbb{R}^4 $ esistono $ \gamma \in\mathbb{R}^4, s \in \mathbb{R}$ tali che $x=\xi+s\gamma$ si ottiene:
\begin{equation}
    \left(\partial^2_{hn}f_i(x)\gamma_h\gamma_n\right)\left(\partial_kf_0(x)\gamma_k\right)
    = \left(\partial_kf_i(x)\gamma_k\right)\left(\partial^2_{hn}f_0(x)\gamma_h\gamma_n\right) \quad i=0,1,2,3  
    \label{ABigPSum}
\end{equation}
Siccome si è supposto inizialmente $\det J_f(x)\neq 0$, $ \forall\gamma\in \mathbb{R}^4$ non nullo, deve necessariamente esistere almeno un $j$ tale che $\partial_kf_j(x)\gamma_k\neq 0$. Inoltre, se si suppone di prendere $\bar{\gamma}\neq0$ tale che $\partial_kf_0(x)\bar{\gamma}_k=0$, per la (\ref{ABigPSum}) necessariamente anche $\partial^2_{hn}f_0(x)\bar{\gamma}_h\bar{\gamma}_n=0$.\\

Si applichi quindi il Lemma \ref{lemm:A1} che, in virtù di quanto appena osservato ($\partial_kf_0(x)\bar{\gamma}_k=0\Rightarrow \partial^2_{hn}f_0(x)\bar{\gamma}_h\bar{\gamma}_n=0$), consente di scrivere:
\begin{equation*}
    \partial^2_{hn}f_0(x)\gamma_h\gamma_n=\ \langle \psi(x),\gamma\rangle\ (\partial_kf_0(x)\gamma_k) \qquad   \forall x,\gamma\in \mathbb{R}^4 \quad i=0,1,2,3 \quad \psi(x)\in\mathbb{R}^4
\end{equation*}
che inserito nella (\ref{ABigPSum}) risulta in:
\begin{equation*}
    \left(\partial^2_{hn}f_i(x)\gamma_h\gamma_n\right)\left(\partial_kf_0(x)\gamma_k\right)
    = \left(\partial_kf_i(x)\gamma_k\right)\ \langle \psi(x),\gamma\rangle\ (\partial_kf_0(x)\bar{\gamma}_k).
\end{equation*}
Infine, siccome $\gamma'_0\neq 0$ (affinché sia $x'$ sia un moto rettilineo uniforme) a maggior ragione la sua espressione data dalla jacobiana di $f$ soddisfa $(\partial_kf_0(x)\gamma_k\frac{ds}{ds'})\neq 0$. È quindi possibile elidere da ambo i lati tale membro:
\begin{equation}
    \partial^2_{hn}f_i(x)\gamma_h\gamma_n=\left(\partial_kf_i(x)\gamma_k\right)\ \langle \psi(x),\gamma\rangle \qquad   \forall x,\gamma\in \mathbb{R}^4 \quad i=0,1,2,3.
    \label{APArtialScal}
\end{equation}
Si osservi che  $\partial^2_{hn}f_i(x)$ è una matrice simmetrica (per il teorema di Schwarz) per cui dall'espressione appena ottenuta si può scrivere:
\begin{equation}
    \partial^2_{hk}f_i(x)=\frac{1}{2}\left(\partial_kf_i(x)\psi_h(x)+\partial_hf_i(x)\psi_k(x)\right)
    \label{APArtialScalSimm}
\end{equation}

Si consideri ora $x$ e $x'$ parametrizzazioni del moto di un raggio di luce. Per il principio di costanza della velocità della luce se in $K$ $x$ è parametrizzato sul cono di luce, lo stesso deve avvenire in $K'$, il che è esprimibile matematicamente con le due condizioni:
\begin{equation*}
    \begin{gathered}
        \gamma_0^2-(\gamma_1^2+\gamma_2^2+\gamma_3^2)=0,\\
        (\gamma_0')^2-[(\gamma_1')^2+(\gamma_2')^2+(\gamma_3')^2]=0.
    \end{gathered}
\end{equation*}
Poiché si può esprimere ogni componente di $\gamma'$ tramite la jacobiana di $f$ e $\gamma$, come già si è fatto calcolando $\frac{\gamma'_i}{\gamma'_0}$:
\begin{equation*}
    \left(\partial_kf_0(x)\gamma_k\right)^2-\left[\sum_{i=1}^3\left(\partial_kf_i(x)\gamma_k\right)^2\right]=0
\end{equation*}
Il principio di costanza della velocità della luce consente di utilizzare il Lemma \ref{lemm:A2} (poiché l'annullarsi di una forma quadratica implica l'annullarsi dell'altra) da cui si ha che:
\begin{equation*}
    \left(\partial_kf_0(x)\gamma_k\right)^2-\left[\sum_{i=1}^3\left(\partial_kf_i(x)\gamma_k\right)^2\right]=
    \lambda(x)\left[\gamma_0^2-(\gamma_1^2+\gamma_2^2+\gamma_3^2)\right]
\end{equation*}
Sia ora $g_{ij}$ una matrice tale per cui $g_{00}=1$, $g_{ii}=-1$ per $i=1,2,3$ e $g_{ij}=0$ se $i\neq j$, così facendo:
\begin{equation*}
    \begin{gathered}
        \lambda(x)\left[\gamma_0^2-(\gamma_1^2+\gamma_2^2+\gamma_3^2)\right]= 
        \lambda(x)g_{kh}\gamma_k\gamma_h=g_{ij}(\partial_kf_i(x)\gamma_k)(\partial_hf_j(x)\gamma_h)\\
    \end{gathered}
\end{equation*}
\begin{equation}
   \Rightarrow\qquad \lambda(x)g_{hk}=g_{ij}(\partial_kf_i(x))(\partial_hf_j(x)).
    \label{APartialDelta}
\end{equation}
Si derivi ora rispetto a $x_s $ ad ambo i membri così da ottenere:
\begin{equation*}
    g_{hk} \partial_s\lambda =2g_{ij}\partial_{ks}^2f_i(x)\partial_hf_j(x).
\end{equation*}
Sostituendo la derivata seconda parziale tramite la (\ref{APArtialScalSimm}) questa espressione è semplificabile e si ottiene:
\begin{equation}
    g_{hk} \partial_s\lambda =g_{ij} \left[\partial_kf_i(x)\psi_s+\partial_sf_i(x)\psi_k\right]\partial_hf_j(x).
\end{equation}    
Infine applicando nuovamente la (\ref{APartialDelta}) si possono sostituire tutti i prodotti di derivate parziali:
\begin{equation*}
    g_{hk} \partial_s\lambda =g_{kh}\lambda\psi_s+g_{sh}\lambda \psi_k.
\end{equation*}
Si considerino queste due differenti casistiche:
\begin{itemize}
    \item siano $k=h\neq s$ allora
    \begin{equation*}
        g_{hk} \partial_s\lambda =g_{kh}\lambda\psi_s \Rightarrow \partial_s\lambda=\psi_s\lambda,
    \end{equation*}
    \item siano $k=h= s$ allora
    \begin{equation*}
        g_{hk} \partial_s\lambda =2g_{kh}\lambda\psi_s \Rightarrow \partial_s\lambda=2\psi_s\lambda.
    \end{equation*}
\end{itemize}
Queste due relazioni implicano però che $\lambda\psi_i=0$ (per $i=0,1,2,3$). Però $\lambda$  non può annullarsi altrimenti la trasformazione $f$ mapperebbe ogni moto in un moto a velocità luminare, il che violerebbe il principio di relatività.\\Segue quindi che $\psi_i=0$ (per $i=0,1,2,3$), per cui dalla (\ref{APArtialScalSimm}) risulta:
\begin{equation}
    \partial_{kh}^2f_i(x)=0 \qquad \qquad \forall i,k,h=0,1,2,3 \forall x \quad\in \mathbb{R}^4
\end{equation}
per cui $f(x)$ può essere esclusivamente una trasformazione affine.