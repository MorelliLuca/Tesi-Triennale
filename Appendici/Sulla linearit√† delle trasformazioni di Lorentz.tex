\chapter{Sulla linearità delle trasformazioni di Lorentz}
Le trasformazioni di Lorentz sono una famiglia di trasformazioni affini, ossia composte 
da una generica traslazione in quattro dimensioni e un'applicazione lineare. 
Questa loro proprietà è deducibile dai postulati di Einstein e solitamente, nella maggior 
parte delle trattazioni della teoria della Relatività, è data come proprietà ovvia. 
Nelle seguenti pagine è presentata la dimostrazione di tale proprietà.\\

\section{Prerequisiti matematici}
In primo luogo è necessario dimostrare due Lemmi che si rivelano utili anche per altri 
aspetti della teoria della Relatività.

\begin{lemma}
    Siano $v,u\in \mathbb{R}^n$, $<,>$ sia il prodotto scalare euclideo e siano definite $f_u: \mathbb{R}^n\rightarrow\mathbb{R}$, 
    tale che $f_u(v)=\ <u,v>$, e $q:\mathbb{R}^n\rightarrow\mathbb{R}$ forma quadratica tale che $q
    (v)=\ <v,Av>$, con $A\in M_{n\times n}(\mathbb{R})$ simmetrica. Se, fissato $\bar{v}$ tale che $f_u(\bar{v})=0$ implica che $q(\bar{v})=0$,
    allora $\exists \psi\in \mathbb{R}^n$ tale che $q(v)=f_u(v)<\psi,v> \forall v\in \mathbb{R}^n$.
    \label{lemm:A1}
\end{lemma}
\begin{proof}
    Si osservi che è possibile decomporre $v$ in una componente
    ortogonale ad $u$ ed una parallela:
    \begin{equation}
        v=v^\parallel +v^\bot =\frac{u\ <u,v>}{\|u\|^2}+v-\frac{u\ <u,v>}{\|u\|^2}+
    \label{Al1Decomposizione}
    \end{equation}  
    si esprima quindi $q(v)$ secondo questa decomposizione:
    \begin{equation*}
        \begin{gathered}
            q(v)=\ <v^\parallel +v^\bot,A(v^\parallel +v^\bot)>\\
            =\ <v^\parallel ,Av^\parallel >+2<v^\bot,Av^\parallel>+<v^\bot,Av^\bot>
        \end{gathered}
    \end{equation*}
    il termine $<v^\bot,Av^\bot>$ è per definizione $q(v^\bot)$ e poichè 
    per costruzione di $v^\bot$ si ha $f_u(v^\bot)=0$, per le ipotesi $q(v^\bot)$
    deve annullarsi per cui:
   \begin{equation*}
    q(v)=<v^\parallel ,Av^\parallel >+2<v^\bot,Av^\parallel>
   \end{equation*}
   sostituiendo le espressioni complete di $v^\bot$ e $v^\parallel$, dalla (\ref{Al1Decomposizione}), si ha:
   \begin{equation*}
    \begin{gathered}
        q(v)=<u ,Au >\left(\frac{<u,v>}{\|u\|^2} \right)^2 +2 <v^\bot,Au> \frac{<u,v>}{\|u\|^2}\\
        =\ <u ,Au >\left(\frac{<u,v>}{\|u\|^2} \right)^2 +2 <v,Au> \frac{<u,v>}{\|u\|^2} -2<u ,Au >\left(\frac{<u,v>}{\|u\|^2} \right)^2\\
    \end{gathered}
   \end{equation*}
   Infine se si raccoglie un termine $<u,v>$ e si sommano i termini uguali si ottiene un'espressione per $q(v)$ composta da un termine lineare 
   rispetto a $v$ moltiplicato per il prodotto scalare $<u,v>$:
   \begin{equation}
    q(v)=<u,v>\left[2 \frac{<v,Au>}{\|u\|^2}-\frac{<u ,Au ><u,v>}{\|u\|^4} \right] \qquad \forall v\in\mathbb{R}^4
   \end{equation}
\end{proof}

\begin{lemma}
    Sia $v\in\mathbb{R}^4$, per comodità si scriverà $v=(v_0,v_1,v_2,v_3)=(v_0,\vec{v})$, tale che $\vec{v}\neq 0$. Sia quindi $q:\mathbb{R}^4\rightarrow\mathbb{R}$ 
    una forma quadratica tale che se fissato $\bar{v}$ con $\bar{v}_0^2-|\vec{\bar{v}}|^2=0$ implica $ q(\bar{v})=0$ allora $\exists \lambda $ 
    tale che $ q(v)=\lambda(v_0^2-|\vec{v}|^2)\ \forall v\in\mathbb{R}^4$.
    \label{lemm:A2}
\end{lemma}
\begin{proof}
    Si osservi che $q(v)$ può essere scomposto, seguendo le regole del prodotto righe per colonne $q(v)=v^tAv$ (dove 
    $A\in M_{3x3}(\mathbb{R})$ simmetrica), nella seguente somma:
    \begin{equation*}
        q(v)=a_{00}v_0^2+2v_0(a_{01}v_1+a_{02}v_2+a_{03}v_3)+\hat{q}(\vec{v})
    \end{equation*}
    dove $\hat{q}$ è ancora una forma quadratica in $\mathbb{R}^3$ che agisce su $\vec{v}$.\\
    Si consideri ora un vettore $\bar{v}\in \mathbb{R}^4$ e tale che considerando $v$ allora $\bar{v}=(-|\vec{v}|,\vec{v})$; per costruzione
    $(\bar{v}_0)^2-|\vec{\bar{v}}|^2=|\vec{v}|^2-|\vec{v}|^2=0$, per cui per le ipotesi assunte si ha $q(\bar{v})=0$.\\
    Essendo $q(\bar{v})=0$ si può scrivere:
    \begin{equation}
        \begin{gathered}
            q(v)=q(v)-q(\bar{v})\\
            =a_{00}(v_0^2-|\vec{v}|^2)+2(v_0-|\vec{v}|)(a_{01}v_1+a_{02}v_2+a_{03}v_3)+\hat{q}(\vec{v})-\hat{q}(\vec{\bar{v}})\\
            =a_{00}(v_0^2-|\vec{v}|^2)+2(v_0-|\vec{v}|)(a_{01}v_1+a_{02}v_2+a_{03}v_3)
        \end{gathered}
       \label{Al2Decomposizione}
    \end{equation}
   inoltre calcolando la (\ref{Al2Decomposizione}) per $\bar{v}$ si ha:
   \begin{equation*}
    q(\bar{v})=-2|\vec{v}|(a_{01}v_1+a_{02}v_2+a_{03}v_3)=0
   \end{equation*}
   Per ipotesi $|\vec{v}|\neq0$ per cui $a_{01}=a_{02}=a_{03}=0$ che dimostra il lemma poichè implica:
   \begin{equation}
    q(v)=a_{00}(v_0^2-|\vec{v}|^2)\quad \forall v\in\mathbb{R}^4
   \end{equation}
\end{proof}

\section{La dimostrazione}
Siano $x,x'\in\mathbb{R}^4$ due vettori dello spazio-tempo misurati rispettivamente in due sistemi di riferimento inerziali $K$ e $K'$, 
si vogliono determinare le proprietà dell'applicazione $f:\mathbb{R}^4\rightarrow\mathbb{R}^4$ che trasforma i vettori misurati in $K$ 
in vettori misurati in $K'$ e viceversa considerando validi il principio di Relatività e il principio di costanza della velocità della luce.\\

Per prima cosa si supporà che $f$ sia sufficentemente regolare, almeno $C^2$, ed è necessario che sia invertibile poichè deve poter 
trasformare sia da $K$ a $K'$, sia viceversa: questa condizione è equivalente a richiedere che $\det J_f(x)\neq 0\ \forall x\in\mathbb{R}^4$.\\

Si considerino ora $\xi, \gamma \in \mathbb{R}^4$, ripettivamente detti punto iniziale e velocità della parametrizazione, 
che consentono di parametrizzare un moto rettilineo uniforme nel seguentue modo:
\begin{equation*}
    x=\xi+s\gamma \qquad s\in \mathbb{R}
\end{equation*}
infatti è possibile esplicitare la dipendenza di $x_0$ da $s$ per ottenere la forma di un moto rettilineo uniforme, per questo motivo è necessario che $\gamma_0\neq 0$. 
\begin{equation*}
    s=\frac{x_0-\xi_0}{\gamma_0} \qquad \Rightarrow \qquad x_i=\xi_i+\frac{x_0-\xi_0}{\gamma_0}\gamma_i \qquad i=1,2,3
\end{equation*}
Secondo il principio di relatività se si calcola $f(x)=x'$ è necessario che $x'=\xi'+s'\gamma'$ così che un moto rettilineo uniforme resti tale in ogni sistema 
di riferimento inerziale. Derivando $f(\xi+s\gamma)$ rispetto al parametro $s'$ si ottiene:
\begin{equation*}
    \frac{dx'}{ds'}=\gamma'=J_f(\xi+s\gamma)\frac{ds}{ds'}\gamma
\end{equation*}
si consideri ora la i-esima componente di $\gamma'$, dove $i=0,1,2,3$, e la si divida per $\gamma_0$:
\begin{equation*}
    \frac{\gamma'_i}{\gamma'_0}=\frac{\sum_{k=0}^3\partial_kf_i(\xi+s\gamma)\gamma_k}{\sum_{k=0}^3\partial_kf_0(\xi+s\gamma)\gamma_k}
    \label{ARappGamma}
\end{equation*} 
dove $\partial_k=\frac{\partial}{\partial x_k}$. Da ora in poi si utilizzerà la convenzione degli indici ripeturi di Einstein per cui $\sum_k x_k y_k=x_k y_k$.\\
Il rapporto $\frac{\gamma'_i}{\gamma'_0}$ è costante da cui segue immediatamente che $\frac{d}{ds'}\frac{\gamma'_i}{\gamma'_0}=0$  e quindi:
\begin{equation*}
    \begin{gathered}
        \frac{d}{ds'}\frac{\partial_kf_i(\xi+s\gamma)\gamma_k}{\partial_jf_0(\xi+s\gamma)\gamma_j}\\
    =\frac{\left(\partial^2_{hn}f_i(\xi+s\gamma)\gamma_h\gamma_n\right)\left(\partial_kf_0(\xi+s\gamma)\gamma_k\right)
    - \left(\partial_kf_i(\xi+s\gamma)\gamma_k\right)\left(\partial^2_{hn}f_0(\xi+s\gamma)\gamma_h\gamma_n\right)}{\left(\partial_jf_0(\xi+s\gamma)\gamma_j\right)^2}=0    
\end{gathered}
\end{equation*}
che può annullarsi solo se si annulla il numeratore della frazione. Da questa osservazione e dal fatto che preso un $x\in \mathbb{R}^4 $ esistono $ \gamma \in
\mathbb{R}^4, s \in \mathbb{R}$ tali che $x=\xi+s\gamma$ si ottiene:
\begin{equation}
    \left(\partial^2_{hn}f_i(x)\gamma_h\gamma_n\right)\left(\partial_kf_0(x)\gamma_k\right)
    = \left(\partial_kf_i(x)\gamma_k\right)\left(\partial^2_{hn}f_0(x)\gamma_h\gamma_n\right) \qquad \forall x,\gamma\in \mathbb{R}^4 \quad i=0,1,2,3  
    \label{ABigPSum}
\end{equation}
Siccome si è supposto inizialmente $\det J_f(x)\neq 0$ deve necessariamente esistere almeno un $j$ tale che $ \forall\gamma\in \mathbb{R}^4\quad\partial_kf_j(x)\gamma_k\neq 0$, 
inoltre se si suppone di prendere $\bar{\gamma}$ tale che $\partial_kf_0(x)\bar{\gamma}_k=0$ allora per la (\ref{ABigPSum}) necessariamente 
anche $\partial^2_{hn}f_0(x)\bar{\gamma}_h\bar{\gamma}_n=0$.\\

Si applichi quindi il Lemma (\ref{lemm:A1}) il quale, in virtù del fatto che $\partial_kf_0(x)\bar{\gamma}_k=0\Rightarrow 
\partial^2_{hn}f_0(x)\bar{\gamma}_h\bar{\gamma}_n=0$, consente di scrivere:
\begin{equation*}
    \partial^2_{hn}f_0(x)\gamma_h\gamma_n=\ <\psi(x),\gamma>\ (\partial_kf_0(x)\gamma_k) \qquad   \forall x,\gamma\in \mathbb{R}^4 \quad i=0,1,2,3  
\end{equation*}
che inserito nella (\ref{ABigPSum}) risulta in:
\begin{equation*}
    \left(\partial^2_{hn}f_i(x)\gamma_h\gamma_n\right)\left(\partial_kf_0(x)\gamma_k\right)
    = \left(\partial_kf_i(x)\gamma_k\right)\ <\psi(x),\gamma>\ (\partial_kf_0(x)\bar{\gamma}_k)
\end{equation*}
infine siccome $\gamma'_0\neq 0$ affinchè sia $x'$ sia un moto rettilneo unfiorme, a maggior ragione la su espressione data dall'immagine di $f$ soddisfa $(\partial_kf_0(x)\gamma_k\frac{ds}{ds'})\neq 0$ 
ed è quindi possibile elidere da ambo i lati tale membro:
\begin{equation}
    \partial^2_{hn}f_i(x)\gamma_h\gamma_n=\left(\partial_kf_i(x)\gamma_k\right)\ <\psi(x),\gamma> \qquad   \forall x,\gamma\in \mathbb{R}^4 \quad i=0,1,2,3
    \label{APArtialScal}
\end{equation}
Si osservi che  $\partial^2_{hn}f_i(x)$ è una matrice simmetrica per il teorema di Schwarz per cui dall'espressione appena ottenuta si può scrivere:
\begin{equation}
    \partial^2_{hk}f_i(x)=\frac{1}{2}\left(\partial_kf_i(x)\psi_h(x)+\partial_hf_i(x)\psi_k(x)\right)
    \label{APArtialScalSimm}
\end{equation}
\\
Si consideri ora $x$ e $x'$ parametrizazioni del moto di un raggio di luce, per il principio di costanza della velocità della luce 
se in $K$ $x$ è paramtrizzato sul cono di luce, lo stesso deve avvenire in $K'$, il che è esprimibile matematicamente con le due condizioni:
\begin{equation*}
    \begin{gathered}
        \gamma_0^2-(\gamma_1^2+\gamma_2^2+\gamma_3^2)=0\\
        (\gamma_0')^2-[(\gamma_1')^2+(\gamma_2')^2+(\gamma_3')^2]=0
    \end{gathered}
\end{equation*}
e poichè si può esprimere ogni componente di $\gamma'$ tramite la jacobiana di $f$ e $\gamma$, come già 
si è fatto calcolando $\frac{\gamma'_i}{\gamma'_0}$:
\begin{equation*}
    \left(\partial_kf_0(x)\gamma_k\right)^2-\left[\sum_{i=1}^3\left(\partial_kf_i(x)\gamma_k\right)^2\right]=0
\end{equation*}
Il principio di costanza della velocità della luce consente di utilizzare il lemma (\ref{lemm:A2}) poiché l'annullarsi di una
 forma quadratica implica l'annullarsi dell'altra, da cui si ha che:
\begin{equation*}
    \left(\partial_kf_0(x)\gamma_k\right)^2-\left[\sum_{i=1}^3\left(\partial_kf_i(x)\gamma_k\right)^2\right]=
    \lambda(x)\left[\gamma_0^2-(\gamma_1^2+\gamma_2^2+\gamma_3^2)\right]
\end{equation*}
Sia ora $e_i$ un vettore tale per cui $e_0=1$ altrimenti $e_i=-1$, così facendo:
\begin{equation*}
    \begin{gathered}
        \lambda(x)\left[\gamma_0^2-(\gamma_1^2+\gamma_2^2+\gamma_3^2)=0\right]= \lambda(x)e_i\gamma_i^2
        =e_i\left(\partial_kf_i(x)\gamma_k\right)^2\\
        \Rightarrow \qquad \lambda(x)e_k\gamma_k\gamma_h\delta_{kh}=e_i(\partial_kf_i(x)\gamma_k)^2=e_i(\partial_kf_i(x)\gamma_k)(\partial_hf_i(x)\gamma_h)\\
    \end{gathered}
\end{equation*}
\begin{equation}
   \Rightarrow\qquad \lambda(x)e_k\delta_{hk}=e_i(\partial_kf_i(x))(\partial_hf_i(x))
    \label{APartialDelta}
\end{equation}
Si derivi ora rispetto a $x_j $ ad ambo i membri così da ottenere:
\begin{equation*}
    e_k \delta_{hk}\partial_j\lambda =2e_i\partial_{kj}^2f_i(x)\partial_hf_i(x)
\end{equation*}
sostituendo la derivata seconda parziale tramite la (\ref{APArtialScalSimm}) questa espressione è semplificabile e si ottiene:
\begin{equation}
    e_k\delta_{hk}\partial_j\lambda = e_i\left[\partial_jf_i(x)\psi_k+\partial_kf_i(x)\psi_j\right]\partial_hf_i(x)
\end{equation}    
infine applicando nuovamente la (\ref{APartialDelta}) si possono sostituire tutti i prodotti di derivate parziali:
\begin{equation*}
    e_k \delta_{hk}\partial_j\lambda=e_j\lambda\delta_{jk}\psi_k+e_k\lambda\delta_{kh}\psi_j
\end{equation*}
Si considerino queste due differenti casitiche:
\begin{itemize}
    \item Siano $k=h\neq j$ allora:
    \begin{equation*}
        e_k\delta_{hk}\partial_j\lambda=\lambda e_k\psi_j \ \Rightarrow \partial_j\lambda=\psi_j\lambda
    \end{equation*}
    \item Siano $k=h= j$ allora:
    \begin{equation*}
        2\lambda e_k\psi_k=e_k\partial_k\lambda \ \Rightarrow 2\partial_k\lambda=\psi_k\lambda
    \end{equation*}
\end{itemize}
Queste due relazioni appena ottenute implicano però che $\psi_i=0$ per $i=0,1,2,3$ e quindi dalla 
(\ref{APArtialScalSimm}) si ha che:
\begin{equation}
    \partial_{kh}^2f_i(x)=0 \qquad \qquad \forall i,k,h=0,1,2,3 \forall x \quad\in \mathbb{R}^4
\end{equation}
per cui $f(x)$ può essere esclusivamente una trasfromazione affine.