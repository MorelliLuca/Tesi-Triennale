\subsection{Contrazione delle lunghezze e dilatazione dei tempi}
Come si è già visto il concetto di tempo assoluto non è conciliabile con i postulati di Einstein, questo fatto è ora deducibile dalle Trasformazioni di Lorentz (\ref{TrasformazioneLorentz}).\\
Si consideri in un sistema di riferimento inerziale $K$ il moto rettilineo ed uniforme a velocità $V$ di un corpo che emette un segnale luminoso ogni $\Delta t$. Considerando un secondo sistema di riferimento inerziale $K'$ solidale a tale corpo e tale da osservarlo nell'origine degli assi spaziali, allora le trasformazioni di Lorentz risulteranno:
\begin{equation*}
    \begin{cases}
        t=(t'+\frac{V}{c^2}x')\gamma\\
        x=(x'+Vt')\gamma\\
        y=y'\\
        z=z'
    \end{cases}
    \Longleftrightarrow \quad
    \begin{cases}
        t'=(t-\frac{V}{c^2}x)\gamma\\
        x'=(x-Vt)\gamma\\
        y'=y\\
        z'=z
    \end{cases}
    \qquad \gamma=\frac{1}{\sqrt{1-\frac{V^2}{c^2}}}.
\end{equation*}
Si possono quindi correlare i tempi di emissione misurati in $K$ con quelli in $K'$ considerando la trasformazione di Lorentz rispetto al corpo posto nell'origine del sistema $K'$ (r'=(ct',0,0,0)): 
\begin{equation}
    \begin{cases}
        t=t'\gamma\\
        x=Vt'\gamma\\
        y=0\\
        z=0
    \end{cases}
    \Rightarrow \Delta \tau=t'_1-t'_0=\frac{(t_1-t_0)}{\gamma}=\frac{\Delta t}{\gamma}=\Delta t \sqrt{1-\frac{V^2}{c^2}}.
    \label{dilatazioneTempi}
\end{equation}
L'intervallo di tempo $\Delta \tau$, misurato nel sistema $K'$ solidale al corpo, è detto tempo proprio di tale corpo e considerando che $\gamma$ assume valori maggiori di $1$ per ogni $V\neq 0$ risulta l'intervallo di tempo tra due eventi più breve misurabile tra tutti gli intervalli $\Delta t$ misurati in ogni sistema di riferimento inerziale.\\

Si consideri adesso il corpo come un parallelepipedo esteso di dimensioni $l_1\ l_2\ l_3$ rispettivamente lungo gli assi $x,\ y,\ z$ del sistema $K$ e $\lambda_1\ \lambda_2\ \lambda_3$ rispettivamente lungo gli assi $x',\ y',\ z'$ del sistema $K'$. Operativamente le dimensioni dell'oggetto sono misurate ponendo dei regoli diretti lungo gli assi di ogni sistema di riferimento e non in moto in essi, quindi ad un determinato istante si misurano contemporaneamente le posizioni delle estremità del corpo utilizzando i regoli. Si osservi che è necessario effettuare misure contemporanee della posizione di ogni coppia di estremità per assicurarsi di non misurare anche una componente dovuta allo spostamento del corpo avvenuto tra le misure non contemporanee. Si effettui la misura come è appena stata descritta nell'istante $t=0$, si ottene, facendo uso delle trasformazioni di Lorentz: 
\begin{equation}
    \begin{cases}
        0=(t'+\frac{V}{c^2}\lambda_1)\gamma\\
        l_1=(\lambda_1+Vt')\gamma\\
        l_2=\lambda_2\\
        l_3=\lambda_3
    \end{cases}
    \Rightarrow
    \begin{cases}
        t'=-\frac{V}{c^2}\lambda_1\\
        l_1=(1-\frac{V^2}{c^2})\lambda_1\gamma\\
        l_2=\lambda_2\\
        l_3=\lambda_3
    \end{cases}
    \Rightarrow
    \begin{cases}
        \lambda_1=l_1\gamma=\frac{l_1}{\sqrt{1-\frac{V^2}{c^2}}}\\
        \lambda_2=l_2\\
        \lambda_3=l_3
    \end{cases}.
    \label{contrazioneLunghezze}
\end{equation}   
 Si osserva quindi che anche le lunghezze non possono più essere considerate assolute, nella fattispecie la lunghezza di un corpo misurata nella direzione del suo moto risulta contratta rispetto alla medesima lunghezza misurata nel sistema $K'$ solidale al corpo, detta lunghezza propria. Inoltre, analogamente a come si è osservato con il tempo proprio, si ha che $\gamma > 1$ per ogni $V\neq0$ da cui si deduce che la lunghezza propria è la lunghezza maggiore tra tutte quelle misurabili in differenti sistemi di riferimento inerziali.\\
 
 Per concludere è possibile constatare che il fenomeno della contrazione delle lunghezze influenza anche il volume di un corpo, nel caso del parallelepipedo sopra considerato il volume $\mathcal{V}_0$ misurato in $K'$ diminuirà dello stesso fattore della lunghezza $\lambda_1$ se misurato in $K$:
 \begin{equation}
    V=l_1\ l_2\ l_3=\frac{\lambda_1\ \lambda_2\ \lambda_3}{\gamma}=\frac{V_0}{\gamma}.
    \label{contrazioneVolumi}
 \end{equation}
 Quest'ultima osservazione mette in luce un'ulteriore differenza tra la teoria che si sta descrivendo e la teoria classica, infatti poiché il volume di un copro può mutare in base al suo moto nella teoria della relatività non è possibile fare uso del così detto corpo rigido.
