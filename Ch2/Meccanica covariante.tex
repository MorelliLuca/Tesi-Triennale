\section{Meccanica lagrangiana covariante}Preso da \cite{Barone}\\
Le trasformazioni di Lorentz mettono in luce come la dimensione temporale si da considerare al pari di quelle spaziali. Inoltre, come si è già visto nella Sezione precedente, nell'ambito della relatività è più conveniente studiare quantità 4-vettoriali piuttosto che le grandezze caratteristiche della meccanica classica. Per questi motivi è opportuno analizzare anche un formulazione covariante$^{\cite{Barone}}$ della meccanica.\\

Si consideri una curva nello spazio-tempo $x^\mu$ parametrizzata da una variabile priva di significato fisico $\xi$. Il principio di minima azione descritto nella Sezione \ref{sec:MinimaAzione} richiede che fissati l'istante iniziale e finale le coordinate spaziali occupate dal sistema in tali istanti siano anch'esse determinate a priori. Si noti che associare ad un istante una posizione spaziale equivale geometricamente a fissare un punto nello spazio-tempo $\tilde{x}^\mu$. Da un punto di vista quadridimensionale è quindi necessario richiedere che per due valori $\xi_1$ e $\xi_2$ si abbia $x^\mu(\xi_1)=x^\mu_{iniziale}$ e $x^\mu(\xi_1)=x^\mu_{finale}$ con $x^\mu_{iniziale}$ $x^\mu_{finale}$ fissati.\\


