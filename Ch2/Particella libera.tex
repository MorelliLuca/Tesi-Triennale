\section{La lagrangiana relativistica}\
Si procederà in questa sezione ricavando la lagrangiana relativistica e quanto segue dalla forma di questa. Siccome una generica lagrangiana è data dalla somma di una parte di corpo libero $\mathcal{L}_{lib}$ con una seconda di interazione $\mathcal{L}_{int}$ è opportuno, in primis, dedurre la forma di $\mathcal{L}_{lib}$ studiando il caso della particella libera. Si seguiranno i ragionamenti di Landau \cite{Landau}.
\subsection{La particella libera}
Si consideri il moto di una particella nello spazio-tempo, il principio di minima azione determina tale moto indipendentemente dalla scelta del sistema di riferimento in cui esso è descritto.
Come si è già visto nella Sezione \ref{sec:DervTrasfLorentz} le trasformazioni di Lorentz conservano le lunghezze nello spazio-tempo, per questo motivo tale lunghezza per una curva del moto è un candidato ad essere l'azione relativistica. Una curva di questo tipo può essere parametrizzata dal tempo misurato nel sistema in cui si osserva il moto. Così facendo la curva sarà data dal 4-vettore posizione $x^\mu(t)=(ct,\vec{x}(t))$. La lunghezza della curva, e la supposta azione, è quindi:
\begin{equation}
    \label{azioneRel}
    \mathcal{S}[x^\mu(t)]=\alpha\int_{t_1}^{t_2} |\dot{x}^\mu(t)|\ dt= \alpha c\int_{t_1}^{t_2} \sqrt{1-\frac{|\vec{v}|^2}{c^2}}\ dt=\alpha c\int_{\tau_1}^{\tau_2}d\tau,
\end{equation}
dove si è indicato $\vec{v}=\dot{\vec{x}}$ che corrisponde al vettore tridimensionale velocità e con $\tau$ il tempo proprio\footnote{Si intende in questo caso il tempo misurato in un sistema di riferimento inerziale istantaneamente solidale al moto, nel quale la velocità della particella è nulla} della particella (Sezione \ref{sec:ContrazioneDilatazione}).\\

A questo punto si deve verificare se questa azione, nel limite classico (ossia quando il rapporto $\frac{|\vec{v}|^2}{c^2}$ tende a $0$), è riconducibile all'azione classica. Ricordando che la lagrangiana classica della particella libera è data da $\frac{m|\vec{v}|^2}{2} $:
\begin{equation*}
    \mathcal{L} =\alpha c \sqrt{1-\frac{|\vec{v}|^2}{c^2}}\longrightarrow \alpha c - \frac{\alpha c |\vec{v}|^2}{2c^2}\qquad \text{per}\ \frac{|\vec{v}|^2}{c^2}\rightarrow 0.
\end{equation*}
A meno di costanti moltiplicative e additive (che non mutano la forma delle soluzioni delle equazioni di Eulero-Lagrange), il limite classico ottenuto è quindi riconducibile alla lagrangiana classica se $\alpha=-mc$.\\
Questa evidenza quindi corrobora\footnote{In realtà solamente un confronto tra i risultati teorici che si dedurranno e quelli sperimentali può giustificare l'ipotesi iniziale.} l'ipotesi iniziale così che sia lecito affermare che la lagrangiana relativistica della particella libera sia:
\begin{equation}\label{FreeLagRel}
    \mathcal{L} = -mc^2\sqrt{1-\frac{|\vec{v}|^2}{c^2}}.
\end{equation}
\subsection{Energia e momenti}\label{sec:LagRelEnMo}
Identificata la lagrangiana relativistica per la particella libera è ora possibile sfruttare tutti gli strumenti illustarti nella Sezione \ref{sec:MinimaAzione} per studiare grandezze quali l'energia e i momenti di questo sistema.\\

La prima quantità che si può studiare è il momento associato alle coordinate cartesiane. Questo è il vettore  $(\frac{\partial \mathcal{L} }{\partial v_x},\ \frac{\partial \mathcal{L} }{\partial v_y},\ \frac{\partial \mathcal{L} }{\partial v_z})$, noto come impulso $\vec p$, e dalla (\ref{FreeLagRel}) risulta:
\begin{equation}
    \vec{p}=m\vec{v}\gamma=\frac{m\vec{v}}{\sqrt{1-\frac{|\vec{v}|^2}{c^2}}}.
    \label{impulsoRel}
\end{equation}
Anche relativisticamente, come in meccanica classica, l'impulso di una particella libera risulta conservato (infatti tutte le coordinate della particella libera sono cicliche). Questo implica che il moto di una particella libera è rettilineo ed uniforme.\\
Si osservi che, nel limite classico in cui $|\vec{v}|\ll c$, l'impulso diviene la classica espressione $\vec p=m\vec v$, mentre per velocità tendenti in modulo a $c$ l'impulso tende ad infinito.\\

Analogamente è possibile calcolare l'energia della particella libera ($E = \vec{p}\cdot\vec{v}-\mathcal{L} $) che risulta una quantità conservata (poiché non vi è dipendenza esplicita dal tempo nella lagrangiana). Dalla (\ref{FreeLagRel}) si ottiene:
\begin{equation}
    E = mc^2\gamma=\frac{mc^2}{\sqrt{1-\frac{|\vec{v}|^2}{c^2}}}.\label{energiaRel}
\end{equation}
In questo caso è opportuno osservare che, per particelle libere a riposo, l'energia non è nulla, a differenza di come accade in meccanica classica. Bensì, assume il valore $E_0=mc^2$, noto come energia a riposo della particella. Sottraendo all'energia totale del corpo il termine di energia a riposo si ottiene la quantità di energia esclusivamente dovuta al moto, ossia l'energia cinetica:
\begin{equation}
    T=mc^2\gamma-mc^2=mc^2(\gamma-1).
\end{equation}
Anche in questo caso, studiando il limite classico, si ritrova l'energia cinetica classica mentre nel limite di velocità prossime a $c$ si ottiene che l'energia del corpo tende ad infinito. Quest'ultima osservazione consente di affermare che $c$ è la velocità limite di ogni moto (infatti un qualsiasi corpo dotato di massa necessita di un'energia infinita per poter essere accelerato fino a tale velocità).\\

Talvolta è utile, per ottenere equazioni differenziali di più facile risoluzione di quelle date dalle equazioni di Eulero-Lagrange, esprimere il vettore velocità facendo uso dell'impulso e dell'energia. Dalla \eqref{impulsoRel} e dalla \eqref{energiaRel} si ha che:
\begin{equation}
    \vec v=\frac{\vec p}{m\gamma}=\frac{c^2\vec p}{mc^2\gamma}=\frac{c^2\vec p}{E}.\label{velPE}
\end{equation}    
Questa espressione è indipendente dal fattore $\gamma$ e, se si determinano $\vec p$ e $E$ in funzione del tempo, risulta integrabile (permette di ottenere le equazioni del moto).\\

Si consideri ora il 4-gradiente della funzione d'azione, siccome $S(q,t)$ è uno scalare Lorentz invariante, questo è un 4-vettore covariante (Sezione \ref{sec:4-vettori}). Come si è mostrato nella Sezione \ref{sec:MinimaAzione}, l'azione è legata ad energia e momenti dalle sue derivate parziali ($\vnabla S=\vec{p}$ e $\frac{\partial S }{\partial t}=-E$). Facendo uso di queste relazioni si ottiene la definizione del \emph{4-impulso}:
\begin{equation}
   \frac{\partial S}{\partial x^\mu}= \bigg(\frac{1}{c}\frac{\partial S }{\partial t},\frac{\partial S }{\partial x},\frac{\partial S }{\partial y},\frac{\partial S }{\partial z}\bigg)=\bigg(-\frac{E}{c},\vec{p}\bigg)\quad \Rightarrow \quad
   p^\mu=-\frac{\partial S }{\partial x_\mu}=\bigg(\frac{E}{c},\vec{p}\bigg).
\end{equation}
Si osservi che in relatività l'energia diviene una quantità intrinsecamente legata agli impulsi, tanto da essere la componente temporale del 4-impulso.\\

Studiando la variazione dell'azione \eqref{azioneRel}, per variazioni nello spazio-tempo, è possibile determinare alcune informazioni utili sul 4-impulso e sul moto delle particelle libere.
La curva del moto $\vec{q}(t)$, se espressa in coordinate cartesiane, è rappresentata nello spazio-tempo dal 4-vettore $x^\mu(t)=(ct,\vec{x}(t))$. Di questa si possono considerare variazioni $\delta x^\mu(t)$ che soddisfino le ipotesi del principio di minima azione ($\delta x^\mu(t_1)=\delta x^\mu(t_2)=0$), allora:
\begin{equation*}
    \mathcal{S}[x^\mu(t)]=-mc \int_{t_1}^{t_2} |\dot{x}^\mu(t)|\ dt=-mc \int_{t_1}^{t_2} \sqrt{\frac{d x^\mu}{dt}\frac{d x_\mu}{dt}}\ dt.
\end{equation*}
La variazione prima di questo integrale, sviluppando l'integranda secondo Taylor al prim'ordine (rispetto a $\frac{d x^\mu}{dt}$) e integrando per parti, risulta:
\begin{flalign}
    \delta \mathcal{S}&=-mc \int_{t_1}^{t_2} \frac{1}{2\sqrt{\dot x^\nu(t)\dot x_\nu(t)}}2\frac{dx^\mu}{dt}\frac{d}{dt}\delta x_\mu(t)\ dt\nonumber\\&=-\frac{mc}{\sqrt{\dot x^\nu(t)\dot x_\nu(t)}}\frac{dx^\mu}{dt}\delta x_\mu(t)\bigg|_{t_1}^{t_2}+mc \int_{t_1}^{t_2}\delta x_\mu(t) \frac{d}{dt}\bigg(\frac{1}{\sqrt{\dot x^\nu(t)\dot x_\nu(t)}}\frac{dx^\mu}{dt}\bigg)\ dt.\label{dSFreePart}
\end{flalign}
Si osservi che, siccome $\delta x^\mu(t_1)=\delta x^\mu(t_2)=0$, il primo addendo si annulla. Poiché, per il principio di minima azione, $\delta \mathcal{S}$ deve essere nulla, l'integranda del secondo addendo della \eqref{dSFreePart} deve annullarsi $\forall \delta x_\mu$. Osservando che $\frac{1}{\sqrt{\dot x^\nu(t)\dot x_\nu(t)}}=\frac{\gamma}{c}$ e $\frac{d}{d\tau}=\gamma\frac{d}{dt}$, dove $\tau$ indica il tempo proprio, si ottiene:
\begin{equation}
    \frac{d}{dt}\bigg(\frac{\gamma}{c} \frac{dx^\mu}{dt}\bigg)=\frac{d}{dt}\bigg(\frac{1}{c} \frac{dx^\mu}{d\tau}\bigg)=\frac{1}{c}\frac{du^\mu}{dt}=0\label{mru4-d}.
\end{equation}
Ovvero, la 4-velocità è costante e quindi la traiettoria di un particella libera nello spazio-tempo è una retta (poiché la 4-velocità è tangente a tale curva).\\

Si consideri invece il caso in cui l'integrale d'azione sia calcolato sulla traiettoria del moto ma il secondo estremo d'integrazione non sia fissato. In questo caso il primo addendo della \eqref{dSFreePart} non si annulla in $t_2$ mentre ad annullarsi (per la \eqref{mru4-d}) è l'integrale a secondo addendo. Ricordando $\frac{1}{|\dot{x}^\mu(t)|}=\frac{\gamma}{c}$ e $\frac{d}{d\tau}=\gamma\frac{d}{dt}$, si ha che il 4-impulso è quindi il prodotto tra la 4-velocità è la massa della particella libera:
\begin{equation}
    \delta S=-m\gamma\frac{dx^\mu}{dt}\delta x_\mu=-m\frac{dx^\mu}{d\tau}\delta x_\mu=-m u^\mu\delta x_\mu \quad \Rightarrow \quad p^\mu=m u^\mu.
\end{equation}

Poiché energia ed impulso costituiscono un 4-vettore è opportuno, analogamente a quanto accade per la 4-velocità, determinare la legge di trasformazione di queste due quantità. Considerando la trasformazione di Lorentz $p'^\mu=\Lambda_\nu^\mu p^\nu$ si ha:
\begin{equation}
    p'_x=\bigg(p_x-\frac{V}{c^2}E\bigg)\gamma,\qquad p'_y=p_y,\qquad p'_z=p_z,\qquad E'=(E-Vp_x)\gamma.
    \label{TrasfLorentzEI}
\end{equation}
Il formalismo quadrivettoriale garantisce che il modulo del 4-impulso sia Lorentz invariante. Per determinare tale quantità è quindi utile studiarla prima in un sistema di riferimento inerziale solidale al moto della particella (in questo sistema l'impulso e l'energia cinetica sono nulli). Detto $p'^\mu$ il 4-impulso in tale sistema, si ha
\begin{equation*}
    p'^\mu p'_\mu=\frac{E_0^2}{c^2}=m^2c^2.
\end{equation*}  
Siccome tale modulo è lo stesso in ogni sistema di riferimento inerziale si ottene la relazione tra l'energia e l'impulso di una particella:
\begin{equation}
    p^\mu p_\mu=\frac{E^2}{c^2}-|\vec p |^2=m^2c^2.
    \label{relazioneEnergiaImpulso}
\end{equation}
Da quest'ultima relazione è possibile ottenere \emph{l'equazione di Hamilton Jacobi} relativistica:
\begin{equation}
    p^\mu p_\mu=\frac{\partial S }{\partial x^\mu}\frac{\partial S }{\partial x_\mu}=\bigg(\frac{1}{c}\frac{\partial S }{\partial t}\bigg)^2-\bigg(\frac{\partial S }{\partial x}\bigg)^2-\bigg(\frac{\partial S }{\partial y}\bigg)^2-\bigg(\frac{\partial S }{\partial z}\bigg)^2=m^2c^2.
\end{equation}
