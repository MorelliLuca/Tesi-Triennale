\section{Sistemi di più particelle}
Seguendo Landau \cite{Landau}, si vuole adesso studiare un insieme di $N$ particelle non interagenti che formano un sistema isolato.
Questo è quindi composto da $N$ particelle libere (siccome non vi sono interazioni) e perciò si ha che le energie e i momenti delle singole particelle si conservano.\\

Per studiare questo tipo di sistemi è opportuno fare uso di coordinate e momenti ad hoc. Si consideri una sola particella delle $N$ descritta del 4-vettore $x^\mu$, se a tutto il sistema si fa compiere una rotazione infinitesima nello spazio-tempo questo 4-vettore muterà di una quantità infinitesima. La variazione di questo 4-vettore può essere espressa in termini di una trasformazione lineare con coefficienti infinitesimi:
\begin{equation*}
    \delta x^\mu=x_\nu\delta \Omega^{\mu\nu}.
\end{equation*}
Una rotazione quadridimensionale non varia la lunghezza del 4-vettore posizione così che, trascurando termini infinitesimi di ordine superiore al primo, si ha: 
\begin{equation*}
    x^\mu x_\mu=(x^\mu+\delta x^\mu)(x_\mu+\delta x_\mu)= x^\mu x_\mu+2x^\mu \delta x_\mu+ \delta x^\mu\delta x_\mu\ \Rightarrow \ x^\mu x^\nu \delta \Omega_{\mu \nu}=0.
\end{equation*}
Poiché $x^\mu$ non è stato scelto in maniera specifica, quanto appena detto vale per ogni vettore in $\mathbb{R}^4$. Si osservi che siccome $x^\mu x^\nu$ (arbitrario) forma un tensore simmetrico e la sua contrazione su $\delta \Omega_{\mu \nu}$ è nulla, allora quest'ultimo deve esse un tensore antisimmetrico\footnote{Presi due tensori $A$ e $B$, rispettivamente simmetrico e antisimmetrico, possono essere scritti come: $A^{\mu\nu}=\frac{1}{2}(A^{\mu\nu}+A^{\nu\mu})$ e $B^{\mu\nu}=\frac{1}{2}(B^{\mu\nu}-B^{\nu\mu})$. Allora $A^{\mu\nu}B^{\mu\nu}=\frac{1}{4}(A^{\mu\nu}B^{\mu\nu}-A^{\mu\nu}B^{\nu\mu}+A^{\nu\mu}-A^{\nu\mu}B^{\nu\mu})=0$.}:
\begin{equation}
    \delta\Omega_{\mu \nu}=-\delta\Omega_{\nu \mu}.
\end{equation}
Si è mostrato nella Sezione \ref{sec:MinimaAzione} che una variazione delle coordinate spaziotemporali rispetto alle quali è calcolata la funzione d'azione genera una variazione di questa. Considerando le precedenti osservazioni per ogni particella del sistema in questo caso si ottiene:
\begin{equation*}
    \delta S=-\sum_{\text{Particelle}}p^\mu \delta x_\mu=-\delta \Omega_{\mu \nu}\sum_{\text{Particelle}}p^\mu x^\nu.\footnote{In questa formula e da qui in avanti sono omessi gli indici che indicano la particella di appartenenza, ogni sommatoria sottintende che questa avviene su ogni particella del sistema.}
\end{equation*} 
Se si scompone il tensore $p^\mu x^\nu$ nella somma di una parte simmetrica e una antisimmetrica, siccome questo è contratto su un tensore antisimmetrico, si ottiene che la sola parte antisimmetrica dà un contributo non nullo: 
\begin{equation*}
    \delta S=-\frac{\delta \Omega_{\mu \nu}}{2}\sum_{\text{Particelle}}(p^\mu x^\nu-x^\nu p^\mu).
\end{equation*}
Poiché lo spazio-tempo è isotropo qualsiasi rotazione di tutto il sistema, che è isolato, non ne genera una mutazione. Per questo motivo se si considerano $\Omega_{\mu\nu}$ come alcune coordinate generalizzate queste non possono apparire in $\mathcal{L} $, il che le rende coordinate cicliche. Ne segue che il loro momento associato $\frac{\partial S}{\partial\Omega_{\mu \nu}}$ deve conservarsi e definendo 4-tensore \emph{momento angolare} la quantità
\begin{equation}
    M^{\mu\nu}=\sum_{\text{Particelle}}(p^\nu x^\mu-x^\mu p^\nu )
\end{equation}
questa deve essere conservata durante l'evoluzione del sistema. Si osservi che questo tensore è composto in alcune componenti dalla definizione classica di momento angolare $\vec{M}$, nella fattispecie:
\begin{equation*}
    M_x=M^{23},\ M_y=-M^{13},\ M_z=M^{12}.
\end{equation*}
Le componenti $M^{01},\ M^{02},\ M^{03}$ formano un secondo vettore pari a $\sum(ct\vec{p}-\frac{E\vec{x}}{c})$. Siccome deve conservarsi il 4-tensore momento angolare anche questi due vettori dovranno conservarsi. Nella fattispecie (siccome anche l'energia di ogni particella deve conservarsi) il secondo vettore diviso per l'energia totale del sistema deve essere ancora un vettore costante:
\begin{equation}
    \frac{\sum E \vec{x}}{\sum E}- t\frac{c^2\sum\vec{p}}{\sum E}=\text{costante}.\label{MotocmRelativ}
\end{equation}
Si definiscono rispettivamente i vettori
\begin{equation}
    \vec x_{CM}=\frac{\sum E \vec{x}}{\sum E} \qquad \vec v_{CM}=\frac{c^2\sum\vec{p}}{\sum E}\label{cmRelativ}
\end{equation}
centro di massa relativistico e velocità del centro di massa relativistica. Dalla (\ref{MotocmRelativ}) si ha che il vettore $\vec x_{CM}$ consente di descrivere il moto traslazionale dell'intero sistema come di un solo punto che si muove a velocità  costante $\vec v_{CM}$.\\
 Qualora le velocità dei punti del sistema fossero piccole rispetto a $c$ l'energia a riposo di ogni particella risulterebbe molto maggiore di quella cinetica. Così la formula delle coordinate del centro di massa relativistico tenderebbe a quella classica $\frac{\sum m\vec{x}}{\sum m}$.\\

Si consideri la trasformazione di Lorentz da un sistema arbitrario ad un sistema in moto a velocità $\vec v_{CM}$. Siccome $p_{tot}^\mu=\sum p^\mu$ \footnote{La somma di 4-vettori è ancora un 4-vettore.} e considerando che le trasformazioni di Lorentz sono lineari, si ha che questo si trasforma nella somma dei 4-impulsi del nuovo sistema di riferimento:
\begin{equation*}
    p^\mu_{Tot}\Lambda_\mu^\nu=\sum_{particelle} p^\mu\Lambda_\mu^\nu=\sum_{particelle} p'^\nu=p'^\nu_{Tot}.
\end{equation*}
Se si è orientato l'asse $x$ con la direzione del moto del centro di massa, dalle trasformazioni del 4-impulso \eqref{TrasfLorentzEI} si ha:
\begin{equation*}
    \vec{p'}_{Tot}=\bigg( \vec{p}_{Tot}-\frac{c^2 \vec{p}_{Tot}}{c^2 E_{Tot}}E_{Tot}\bigg)\gamma=0, \quad E_{Tot}'=\bigg(E_{Tot}-\frac{c^2(\vec{p}_{Tot})^2}{E_{Tot}}\bigg)\gamma=\sqrt{E^2_{Tot}-c^2(\vec{p}_{Tot})^2}. 
\end{equation*}
Nel sistema in moto a $\vec v_{CM}$ l'impulso totale è quindi nullo e l'energia del sistema è un invariante di Lorentz, infatti la sua espressione appena ricavata è un multiplo del modulo del 4-vettore impulso.