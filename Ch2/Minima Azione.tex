La meccanica classica, descritta nel capitolo \ref{sec:MC}, descrive il moto di sistemi di corpi tramite l'equazione di Newton (\ref{equazioneDiNewton}), si può dimostrare che questa però è del tutto equivalente al \emph{principio di minima azione}.\\ Questo principio assume che per ogni sistema meccanico sia possibile costruire una funzione $\mathcal{L}(q_1...q_n,\dot{q}_1...\dot{q}_n,t)$ detta \emph{lagrangiana}, dove $q_1...q_n$ sono le coordinate e $\dot{q}_1...\dot{q}_n$ sono le velocità date dagli $n$ gradi di libertà del sistema che si indicheranno solamente con $q$ e $\dot{q}$.\\ In questo modo il principio di minima azione afferma che: se si suppone che ad un istante $t_1$ iniziale e ad un istante finale $t_2$ il sistema si trovi a coordinate e velocità fissate l'evoluzione del sistema avverrà secondo un moto $(q(t),\dot{q}(t))$ tale che l'integrale 
\begin{equation}
    \label{Azione}
    S=\int_{t_1}^{t_2} \mathcal{L}(q,\dot{q},t)\ dt,
\end{equation}
detto \emph{azione}, assuma un valore estremale.\\

Si vogliono ora ottenere le equazioni che consentono, dato questo principio, di ottenere il moto $(q(t),\dot{q}(t))$: si consideri la funzione $q(t)=(q_1(t),...,q_n(t))$ che minimizza l'azione (\ref{Azione}) e una seconda funzione $\delta q(t)$, detta variazione di $q(t)$, tale che $\delta q (t_1)=\delta q(t_2)=0$ e che assuma valori piccoli nel suo dominio $[t_1,t_2]$, in questo modo si può costruire una seconda funzione $q(t)+\delta q(t)$ che soddisfa ancora le ipotesi per cui $\tilde{q}(t_1)=\tilde{q}(t_2)=0$.\\
La variazione $\delta S$, data dalla variazione $\delta q$, deve risultare nulla affinché $q(t)$ sia la funzione che rende estermale l'azione, così che il principio di minima azione sia equivalente a richiedere che
\begin{equation}
    \int_{t_1}^{t_2} \mathcal{L}(q+\delta q,\dot{q}+\delta \dot{q},t)\ dt-\int_{t_1}^{t_2} \mathcal{L}(q,\dot{q},t)\ dt=\int_{t_1}^{t_2}\sum_{i=1}^{n}\bigg[\frac{\partial\mathcal{L} }{\partial q_i}\delta q_i+\frac{\partial\mathcal{L} }{\partial \dot{q}_i}\delta \dot{q}_i \bigg]\ dt=0,
\end{equation}
nella quale si è espressa esplicitamente la variazione dell'azione rispetto alla variazione $\delta q$ e alla sua derivata prima $\delta \dot{q}$ tramite le derivate parziali della lagrangiana $\frac{\partial\mathcal{L} }{\partial q_i},\ \frac{\partial\mathcal{L} }{\partial \dot{q}_i}$.\\
Integrando per parti gli addendi contenenti $\delta \dot{q}_i=\frac{d}{dt}\delta q_i$ e ricordando che $\delta q (t)$ si annulla in $t_1$ e in $t_2$, si ottiene
\begin{equation*}
    \int_{t_1}^{t_2}\sum_{i=1}^{n}\bigg[\frac{\partial\mathcal{L} }{\partial q_i}\delta q_i-\frac{d}{dt}\frac{\partial\mathcal{L} }{\partial \dot{q}_i}\delta q_i \bigg]\ dt+\frac{\partial\mathcal{L} }{\partial q_i}\delta \dot{q}_i\bigg|_{t_1}^{t_2}=\int_{t_1}^{t_2}\sum_{i=1}^{n}\bigg[\frac{\partial\mathcal{L} }{\partial q_i}-\frac{d}{dt}\frac{\partial\mathcal{L} }{\partial \dot{q}_i} \bigg]\delta q_i\ dt=0.
\end{equation*}
Poiché questo integrale deve annullarsi per ogni $\delta q$ che rispetti le ipotesi fin qui fatte allora è necessario che si annulli identicamente l'integrando, nella fattispecie ogni suo addendo singolarmente siccome ognuno di questi è moltiplicato per un componente di $\delta q(t)$, così si ottengono le \emph{equazioni di Eulero Lagrange} che consentono di determinare la forma di $q(t)$:
\begin{equation}
    \label{eulero-lagrange}
    \frac{d}{dt}\frac{\partial\mathcal{L} }{\partial \dot{q}_i}-\frac{\partial\mathcal{L} }{\partial q_i}=0 \qquad \forall i\in{1,2,...,n}.
\end{equation}