\documentclass[12pt,a4paper]{report}

\usepackage[italian]{babel}

\usepackage{newlfont}

\usepackage{color}
\textwidth=450pt\oddsidemargin=0pt

\usepackage{graphicx}

\usepackage{titlesec}

\usepackage{pgfplots}
\pgfplotsset{compat=1.16}

\usepackage[none]{hyphenat}

\usepackage{amssymb}

\usepackage{amsmath}

\usepackage{amsthm}
\newtheorem{lemma}{Lemma}
\newtheorem{sublemma}{Lemma}[section]
\numberwithin{equation}{section}

\usepackage{multicol}

\usepackage[margin=1.2in]{geometry}

\usepackage{fancyhdr}
\pagestyle{fancy}
\fancyhead[RO,LE]{\textbf{Titolo Tesi}}
\fancyhead[LO,RE]{Luca Morelli}

\newcommand{\vnabla}{\vec{\nabla}}

\begin{document}

\begin{titlepage}
\begin{center}
	{{\Large{\textsc{Alma Mater Studiorum $\cdot$ Universit\`a di Bologna}}}} 
	\rule[0.1cm]{15.8cm}{0.1mm}
	\rule[0.5cm]{15.8cm}{0.6mm}
	\\\vspace{3mm}
	
	{\small{\bf Scuola di Scienze \\ 
			Dipartimento di Fisica e Astronomia\\
			Corso di Laurea in Fisica}}
	
\end{center}

\vspace{50mm}

\begin{center}\textcolor{black}{
		%
		% INSERIRE IL TITOLO DELLA TESI
		%
		{\LARGE{\bf Aspetti fisici e matematici della teoria\\\vspace{5mm} della relatività ristretta}}\\
}\end{center}

\vspace{50mm} \par \noindent

\begin{minipage}[t]{0.47\textwidth}
	%
	% INSERIRE IL NOME DEL RELATORE CON IL RELATIVO TITOLO DI DOTTORE O PROFESSORE
	%
	\large{\bf Relatore: \vspace{2mm}\\\textcolor{black}{
				Prof. Paolo Albano}}\\\\
\end{minipage}
%
\hfill
%
\begin{minipage}[t]{0.47\textwidth}\raggedleft \textcolor{black}{
		{\large{\bf Presentata da:
				\vspace{2mm}\\
				%
				% INSERIRE IL NOME DEL CANDIDATO
				%
				Luca Morelli  }}}
\end{minipage}

\vspace{40mm}

\begin{center}
	%
	% INSERIRE L'ANNO ACCADEMICO
	%
	Anno Accademico \textcolor{black}{ 2022/2023}
\end{center}


\end{titlepage}

\begin{center}
	\textbf{Abstract:}
\end{center}

Riassunto breve del documento da creare dove si sintetizzano tutti i punti trattati in maniera tale da rendere più immediata la selezione del documento da parte di un sventurato lettore.


\tableofcontents

\chapter*{Introduzione}   



\chapter{Fatti di Fisica Classica}
Nella seconda metà del diciannovesimo secolo la fisica era costituita fondamentalmente dalla meccanica, 
dalla termodinamica e dall'elettromagnetismo.\\ La meccanica, fondata da Newton e Galileo, descriveva il 
moto dei corpi e di fatto fungeva da modello per tutta la fisica.\\L'elettromagnetismo invece, in seguito 
a numerosi esperimenti, aveva trovato una completa spiegazione nelle Equazioni di Maxwell.\\
Poichè la nascita della Teoria della Relatività Speciale è strettamente connessa a queste due branche della fisica 
è necessario illustrare brevemente i loro fondamenti. 


\section{La meccanica e le trasformazioni di Galileo}
La meccanica classica, in tutte le sue possibili formulazioni, ha come fondamento una serie 
di osservazioni sperimentali che vengono utilizzate come principi da cui dedurre le leggi del moto.\\

Il primo fatto sperimentale che viene assunto è che lo spazio sia tridimensionale, isotropo, omogeneo e che rispetti la 
geometria euclidea mentre il tempo sia ad una sola dimensione. Inoltre si assume che le distanze spaziali e il tempo siano assoluti,
 ossia che ogni osservatore concordi sulla misura di queste.\\
Sulla base di queste assunzioni si può quindi scegliere un punto dello spazio-tempo come 
origine di un sistema di riferimento, ossia uno spazio vettoriale 
$\mathbb{R}^3\times\mathbb{R}$ che ha come vettore nullo il punto scelto. Da questo vengono scelte tre direzioni 
spaziali arbitrarie lungo cui sono orientati tre assi cartesiani, identificabili con la base di $\mathbb{R}^3$, e 
dal corrispondente punto temporale si inizia a misurare il tempo. Si osservi che tali scelte sono arbitrarie, 
come lo è la direzione degli assi corrispondenti ai vettori della base, questo poiché spazio e tempo sono isotropi 
ed omogenei.\\

Il secondo fatto sperimentale prende il nome di Principio di Relatività Galileiano e consiste 
nell'assunzione che esistano una serie di sistemi di riferimento detti inerziali, caratterizzati 
dalla proprietà di essere reciprocamente in moto rettilineo uniforme, in cui le leggi della natura 
in ogni istante assumono la stessa forma.\\

Infine si assume che, in un riferimento inerziale, le posizioni e le velocità dei punti di un sistema ad un tempo iniziale ne determinino 
in maniera univoca l'evoluzione secondo la legge:
\begin{equation}
	m\ddot{\vec{x}}=\vec{F}(\vec{x},\dot{\vec{x}},t)
	\label{equazioneDiNewton}
\end{equation}
dove $m$ è detta massa inerziale e $\vec{F}$ è una funzione caratteristica del sistema detta forza.\\

Si vuole quindi identificare quali applicazioni $\varphi:\mathbb{R}^3\times\mathbb{R}\rightarrow
\mathbb{R}^3\times\mathbb{R}$ consentono di cambiare sistema di riferimento inerziale, ossia quali 
trasformazioni non variano le leggi della natura. Queste applicazioni si chiamano Trasformazioni di 
Galileo e dalle evidenze sperimentali si conclude che queste sono costituite dalle 
composizioni di tre famiglie di applicazioni:
\begin{itemize}
	\item una generica traslazione spazio temporale dell'origine, dedotta dalla proprietà 
	di omogeneità dello spazio e del tempo:
	\begin{equation}
		\varphi_{\vec{r},s}(\vec{x},t)=(\vec{x}+\vec{r},t+s)
		\label{GalileoTraslazoine}
	\end{equation} 
\item una generica rotazione degli assi spaziali, dovuta alla proprietà di isotropia dello spazio:
\begin{equation}
	\varphi_{G}(\vec{x},t)=(G\vec{x},t) \qquad G\in M_{3\times3}(\mathbb{R}):G^{-1}=G^t
	\label{GalileoRotazione}
\end{equation} 
	\item una traslazione di moto rettilineo uniforme, ammissibile grazie alle proprietà dei sistemi 
	di riferimento inerziali:
\begin{equation}
	\varphi_{\vec{v}}(\vec{x},t)=(\vec{x}+\vec{v}t,t)
	\label{GalileoVelocità}
\end{equation} 
\end{itemize}
Quest'ultima tipologia di trasformazione è quella che viene comunemente studiata per caratterizzare le trasformazioni di sistemi inerziali. 
Si consideri quindi un sistema $K$, con coordinate $(\vec{x},t)$ e un sistema $K'$, con 
coordinate $(\vec{x'},t')$, in moto a velocità $\vec{V}$ rispetto a $K$, si scriverà allora la 
(\ref{GalileoVelocità}) come:
\begin{equation}
	\vec{x'}=\vec{x}-\vec{V}t \qquad t'=t
	\label{GalileoEasy}
\end{equation}
Si noti che questa trasformazione, essendo lineare, non muta la forma del vettore $\ddot{\vec{x}}$, infatti:
\begin{equation*}
	\frac{d^2}{dt'^2}(\vec{x'})=\frac{d^2}{dt^2}(\vec{x}-\vec{V}t)=\ddot{\vec{x}}+\frac{d}{dt}(\vec{V})=\ddot{\vec{x}}
\end{equation*} 
questo fatto, assieme al Principio di Relatività Galileiano applicato alla Legge di Newton (\ref{equazioneDiNewton}), 
impongono che le forze esercitate su di un punto e misurate in due sistemi inerziali differenti debbano essere le medesime. \\

Analogamente a quanto appena fatto si possono ricavare le trasformazioni per la velocità di un punto:
\begin{equation}
	\frac{d}{dt'}(\vec{x'})=\frac{d}{dt}(\vec{x}-\vec{V}t)=\dot{\vec{x}}+\vec{V}
\end{equation}
Si osserva che generale le Trasformazioni di Galileo compongono le velocità per somma algebrica.\\

Per determinare l'invarianza di un legge fisica rispetto alle trasformazioni (\ref{GalileoEasy}) può essere necessario 
studiare come si trasformino gli operatori di differenziazione. 
Se si vuole derivare una $f(\vec{x},t)$, dalla regola di Leibniz, si ha:
\begin{equation*}
	\begin{gathered}
		\frac{\partial}{\partial t'}=\frac{\partial t}{\partial t'}\frac{\partial}{\partial t}+
		\sum_{i=1}^{3}\frac{\partial x_i}{\partial t}\frac{\partial t}{\partial t'}
		\frac{\partial}{\partial x_i} \\
		\frac{\partial}{\partial x'_i}=\frac{\partial t}{\partial x'_i}\frac{\partial}{\partial t}+
		\sum_{i=1}^{3}\frac{\partial x_j}{\partial x'_i}\frac{\partial}{\partial x_j}
	\end{gathered}
\end{equation*}
osservando dalla (\ref{GalileoEasy}) che $\frac{\partial t}{\partial t'}=1$, che 
$\frac{\partial t}{\partial x'_i}=0$, che $\frac{\partial x_i}{\partial t}=V_i$ e che 
$\frac{\partial x_j}{\partial \xi_i}=\delta_{ij}$, dove $\delta_{ij}$ è una delta di Kronecker, 
si ottengono le trasformazioni degli operatori di differenziazione desiderate:
\begin{equation}
	\frac{\partial}{\partial t'}=\frac{\partial}{\partial t}+\vec{V}\cdot\vec{\nabla} \qquad \qquad
	\frac{\partial}{\partial x'_i}=\frac{\partial}{\partial x_i}
	\label{GalileoDifferenziale}
\end{equation}
Dalla (\ref{GalileoDifferenziale}) si deduce che tutti gli operatori che comprendono solamente derivate 
rispetto alle derivate spaziali sono lasciati inalterati dalle Trasformazioni di Galileo, così che per esempio $\vnabla'=\vnabla$.
\subsection{L'elettromagnetismo e le equazioni di Maxwell}
Per interpretare i fenomeni elettromagnetici, anche in questo caso, è necessario introdurre
una serie di osservazioni sperimentali: in primo luogo esiste una proprietà della materia 
detta carica elettrica, che non dipende dal sistema di riferimento in cui è misurata e che consente 
ai corpi di interagire con due campi vettoriali: 
il campo elettrico $\vec{E}$ e il campo magnetico $\vec{B}$.\\ Un corpo puntiforme di carica 
$q$ interagendo con questi subisce un forza $\vec{F}$ data da:
\begin{equation}
	\vec{F}=q(\vec{E}(\vec{x},t)+\vec{v}\wedge\vec{B}(\vec{x},t))
	\label{ForzaLorentz}
\end{equation}
dove $\vec{v}$ è il vettore velocità di tale corpo.\\
In secondo luogo gli esperimenti mostrarono che questi due campi rispettano una serie di equazioni 
dette equazioni di Maxwell:
\begin{equation}
	\begin{gathered}
		\vec{\nabla}\cdot\vec{E}=\frac{\rho}{\epsilon_0} \qquad \qquad \vec{\nabla}\cdot\vec{B}=0 \\
		\vec{\nabla}\wedge\vec{E}=-\frac{\partial\vec{B}}{\partial t} \qquad \qquad \vec{\nabla}\wedge
		\vec{B}=\mu_0\vec{J}+\epsilon_0\mu_0\frac{\partial\vec{E}}{\partial t}
		\label{EquazioniMaxwell}
	\end{gathered}
\end{equation}
dove $\rho$ è la densità di carica volumetrica, $\vec{J}$ è la densità di corrente superficiale e 
$\epsilon_0$, $\mu_0$ sono due costanti del vuoto.\\
Dalle equazioni di Maxwell (\ref{EquazioniMaxwell}) segue che le cariche sono sorgenti del campo 
elettrico mentre le correnti lo sono per 
il campo Magnetico. Per esempio è possibile mostrare che una carica puntiforme è sorgente di campo elettrico e 
tramite la forza di Lorentz (\ref{ForzaLorentz}) è possibile derivare la legge di Coulomb.\\
Si consideri una carica puntiforme $Q$ posta nell'origine di un sistema di riferimento e si prenda una sfera $\mathcal{S}$ di raggio $R$ 
contenete tale carica. Integrando la prima equazione di Maxwell e facendo uso del teorema di Gauss si ottiene:
\begin{equation*}
	\int_\mathcal{S}\vec{\nabla}\cdot\vec{E}\ d^3x=\int_\mathcal{S}\frac{\rho}{\epsilon_0}\ d^3x=\frac{Q}{\epsilon_0}=\oint_{\partial\mathcal{S}}\vec{E}\cdot\hat{n}\ d\Sigma
\end{equation*}
dove si è indicato con $\hat{n}$ il versore normale alla superficie $\mathcal{S}$ nel punto in cui si valuta l'integrando.\\
Si osservi che essendo lo spazio 
isotropo e la carica puntiforme allora qualsiasi rotazione degli assi coordinati mantiene immutato il sistema, ne segue che 
il Campo Elettrico generato dalla carica deve essere radiale e costante su superfici sferiche centrate nell'origine. Infatti un'ipotetica seconda carica 
fissa a distanza $R$ dall'origine del sistema di riferimento deve percepire sempre la medesima forza, indipendentemente dalla rotazione effettuata, 
che risulta connessa al campo elettrico per mezzo della (\ref{ForzaLorentz}), inoltre sempre dalle Equazioni di Maxwell, supponendo assenza di correnti 
si ha che il rotore di $\vec{E}$ risulta nullo. Grazie a queste deduzioni l'integrale di superficie si riduce in:
\begin{equation*}
	\oint_{\partial\mathcal{S}}\vec{E}\cdot\hat{n}\ d\Sigma=4\pi R^2|\vec{E}|=\frac{Q}{\epsilon_0}
\end{equation*}	
Tenendo conto che $\vec{E}$ è radiale, come si è dedotto, si ha la legge di Coulomb
\begin{equation}
	\vec{E}=\frac{1}{4\pi\epsilon_0}\frac{Q}{R^2}\hat{r} \quad \Rightarrow \quad \vec{F}=q\vec{E}=\frac{1}{4\pi\epsilon_0}\frac{qQ}{R^2}\hat{r}
\end{equation} 
dove si è indicato con $\hat{r}$ il versore radiale in coordinate sferiche.\\

Infine è importante per trattare la teoria della relatività osservare che è possibile ottenere, calcolando il rotore di ambo 
i membri delle ultime due equazioni di Maxwell
\begin{equation*}
	\vec{\nabla}\wedge\vec{\nabla}\wedge\vec{E}=-\frac{\partial}{\partial t}\vnabla\wedge\vec{B} 
	\qquad \vec{\nabla}\wedge\vec{\nabla}\wedge\vec{B}=\mu_0\epsilon_0\frac{\partial}{\partial t}
	\vnabla\wedge\vec{E}
\end{equation*}
e supponendo assenza di cariche, per cui $\rho=0$ e $\vec{J}=0$, due equazioni che descrivono 
onde di campo elettrico e magnetico nel vuoto:
\begin{equation}
	\vnabla^2\vec{E}=\mu_0\epsilon_0\frac{\partial^2\vec{E}}{\partial t^2} \qquad \vnabla^2\vec{B}=
	\mu_0\epsilon_0\frac{\partial^2\vec{B}}{\partial t^2}
\end{equation}
dove si è indicato l'operatore laplaciano con la notazione $\vnabla^2=(\frac{\partial^2 }{\partial x^2},\frac{\partial^2 }{\partial y^2},\frac{\partial^2 }{\partial z^2})$.\\
Queste onde si propagano con una velocità $\frac{1}{\sqrt{\mu_0\epsilon_0}}=299792458\  \frac{m}{s}$, 
che corrisponde con precisione ai valori sperimentalmente misurati della velocità della luce. Maxwell suppose allora che questa fosse quindi da intendere come un fenomeno elettromagnetico e successivi esperimenti, come quelli di Hertz, confermarono tale ipotesi.\\
Sulla base della teoria ondulatoria classica è però necessario identificare un mezzo nel quale queste onde possano 
propagarsi e rispetto al quale la loro velocità di propagazione debba essere intesa. Per questo motivo alla fine dell'ottocento venne ipotizzata l'esistenza di tale mezzo detto etere luminifero.

\chapter{L'articolo del 1905}
La Meccanica Newtoniana e l'Elettromagnetismo di Maxwell si rivelarono a gli occhi dei fisici dell'ottocento 
incompatibili fra loro, 
poichè le Equazioni di Maxwell non risultarono invarianti per le trasformazioni di Galileo. 
Proprio per questo motivo i fisici dell'epoca dovettero rivalutare i principi alla base delle leggi 
della natura fino a quando l'incompatibilità trovò una soluzione nel 1905 con la Relatività Ristretta di Einstein.
 Si ripercorreranno ora i passi che Einstein stesso indicò nel suo articolo del 1905 \cite{Einstein1905}.
\section{La non invarianza delle Equazioni di Maxwell}

 Si considerino due sistemi di riferimento $K$ e $K'$, inerziali e reciprocamente in moto a velocità $\vec{V}$, in ogni sistema 
 si misureranno rispettivamente $\vec{E},\ \vec{B}$ e $\vec{E'},\ \vec{B'}$.\\ Il Principio di Relatività Galileiana impone che 
 questi due campi, nei loro sistemi di riferimento, rispettino le Equazioni di Maxwell (\ref{EquazioniMaxwell}). Inoltre, siccome la forza (\ref{ForzaLorentz})
 deve essere la medesima in tutti i sistemi di riferimento inerziali, considerando una carica $q$ in moto con velocità $\vec{v}$ 
 in $K$ e $\vec{V}+\vec{v}$ in $K'$, deve valere:
 \begin{equation*}
	\begin{gathered}
		\vec{F'}=\vec{F}\quad\Rightarrow\quad \vec{E'}+\vec{v}\wedge\vec{B'}=\vec{E}+(\vec{V}+\vec{v})\wedge\vec{B}\\
         \Rightarrow\quad \vec{E'}+\vec{v}\wedge(\vec{B'}-\vec{B})=\vec{E}+\vec{V}\wedge\vec{B'}
	\end{gathered}
 \end{equation*}
Così facendo si possono ottenere le trasformazioni dei capi Elettrici e Magnetici tra sistemi inerziali. Queste però possono dipendere 
esclusivamente dalle proprietà dei due sistemi di riferimento 
considerati e nella fattispecie dalla loro velocità reciproca, per questo motivo il termine contenente $\vec{v}$, 
ossia la velocità della carica nel sistema $K$, deve annullarsi, per cui le trasformazioni risultano:
\begin{equation}
	\begin{cases}
		\vec{E'}(\vec{x'},t)=\vec{E}(\vec{x},t)+\vec{V}\wedge\vec{B}(\vec{x},t)\\
		\vec{B'}(\vec{x'},t)=\vec{B}(\vec{x},t)
	\end{cases}
	\label{TrasfGalileoEB}
\end{equation}
Bisogna ora studiare come si trasformino le grandezze generatici dei campi:
$\rho$ e $\vec{J}$. Se si considera una volume $\Delta V$, in cui è presente una carica 
$\Delta q$, allora la densità di carica è definita come:
\begin{equation*}
	\rho=\lim_{\Delta V\rightarrow 0}\frac{\Delta q}{\Delta V}
\end{equation*}
Siccome le lunghezze sono assunte essere assolute devono esserlo pure i volumi ed, 
essendo la carica non dipendente dal sistema di riferimento, si conclude che pure la 
densità di carica non lo è. Per quanto riguarda la densità di corrente superficiale, 
definita come $\vec{J}=\rho\vec{v}$,  
è sufficiente applicare le trasformazioni delle velocità tra due sistemi in moto reciproco 
a velocità $\vec{V}$ per ottenere:
\begin{equation}
	\vec{J'}=\vec{J}-\rho\vec{V}
\end{equation}
I risultati appena ottenuti consentono di determinare l'invarianza delle Equazioni di 
Maxwell per le Trasformazioni di Galileo.\\

Studiando la prima Equazione di Maxwell in $K'$ e considerandola valida in $K$, se si 
trasformano $E'$ in $E$ e analogamente per gli operatori di differenziazione secondo la (\ref{GalileoDifferenziale}), si ottiene una 
quantità che è nulla se questa equazione è valida in  $K'$:
\begin{flalign*}
	&\vnabla'\cdot\vec{E'}-\frac{\rho}{\epsilon_0}=\vnabla\cdot(\vec{E}+\vec{V}\wedge\vec{B})-\frac{\rho}{\epsilon_0}\\
	&=\left(\vnabla\cdot\vec{E}-\frac{\rho}{\epsilon_0}\right)-\vec{V}\cdot(\vnabla\wedge\vec{B})=-\vec{V}\cdot(\vnabla\wedge\vec{B})
\end{flalign*}
Il primo termine tra parentesi è identicamente nullo poiché valgono le Equazioni di Maxwell in $K$ 
mentre l'ultimo termine non è sempre nullo, nella fattispecie in presenza di campi elettrici variabili nel tempo, 
questo implica che quindi la prima Equazione di Maxwell non è invariante per le Trasformazioni di Galileo.\\

Se si studia la seconda con lo stesso procedimento si scopre che questa invece è invariante:
\begin{equation*}
	\vnabla'\cdot\vec{B'}=\vnabla\cdot\vec{B}=0
\end{equation*}

Analogamente per la terza equazione:
\begin{flalign*}
	&\vnabla'\wedge\vec{E'}+\frac{\partial \vec{B'}}{\partial t'}=
	\vnabla\wedge\vec{E'}+\frac{\partial \vec{B'}}{\partial t}+(\vec{V}\cdot\vnabla)\vec{B'}\\
	&=\vnabla\wedge(\vec{E}+\vec{V}\wedge\vec{B})+\frac{\partial \vec{B}}{\partial t}+(\vec{V}\cdot\vnabla)\vec{B}\\
	&=\left(\vnabla\wedge\vec{E}+\frac{\partial \vec{B}}{\partial t}\right)+\vnabla\wedge\vec{V}\wedge\vec{B}+(\vec{V}\cdot\vnabla)\vec{B}=0
\end{flalign*} 
infatti il termine tra parentesi è identicamente nullo poiché in $K$ vale la terza Equazione di Maxwell 
e gli addendi restanti si annullano se sviluppati tramite le regole di differenziazione ricordando che $\vec{V}$ è costante:
\begin{equation*}
	\vnabla\wedge\vec{V}\wedge\vec{B}+(\vec{V}\cdot\vnabla)\vec{B}=-(\vec{V}\cdot\vnabla)\vec{B}+(\vec{V}\cdot\vnabla)\vec{B}=0
\end{equation*}
L'ultima Equazione di Maxwell è invece non invariante infatti sempre con il medesimo procedimento si ottiene:
\begin{flalign*}
	&\vnabla '\wedge\vec{B'}-\mu_0\epsilon_0\frac{\partial \vec{E'}}{\partial t'}-\mu_0\vec{J'}=
	\vnabla\wedge\vec{B'}-\mu_0\epsilon_0\frac{\partial \vec{E'}}{\partial t}-\mu_0\epsilon_0(\vec{V}\cdot\vnabla)\vec{E'}-\mu_0\vec{J'} &&\\
	&=\vnabla\wedge\vec{B}-\mu_0\epsilon_0\frac{\partial \vec{E}}{\partial t}-\mu_0\epsilon_0\vec{V}\wedge\frac{\partial \vec{B}}{\partial t}-\mu_0\epsilon_0(\vec{V}\cdot\vnabla)(\vec{E}+\vec{V}\wedge\vec{B})-\mu_0\vec{J}+\mu_0\vec{V}\rho &&\\
	&=\left(\vnabla\wedge\vec{B}-\mu_0\epsilon_0\frac{\partial \vec{E}}{\partial t}-\mu_0\vec{J}\right)-\mu_0\epsilon_0\vec{V}\wedge\frac{\partial \vec{B}}{\partial t}-\mu_0\epsilon_0(\vec{V}\cdot\vnabla)(\vec{E}+\vec{V}\wedge\vec{B})+\mu_0\vec{V}\rho &&\\
	&=-\mu_0\epsilon_0\vec{V}\wedge(-\vnabla\wedge\vec{E})-\mu_0\epsilon_0(\vec{V}\cdot\vnabla)(\vec{E}+\vec{V}\wedge\vec{B})+\mu_0\vec{V}\rho &&\\
	&=-\mu_0\epsilon_0(\vec{V}\cdot\vnabla)(\vec{V}\wedge\vec{B})+\mu_0\vec{V}\rho
\end{flalign*}
che non è nullo in generale con le assunzioni fin qui fatte.\\

Si è quindi giunti alla conclusione che la teoria di Maxwell non è conciliabile con 
la meccanica di Newton e viceversa.
\subsection{I postulati di Einstein}\label{Sec:postulati}
Per giungere ad una formulazione coerente della dinamica dei corpi carichi Einstein, nel suo articolo del 1905$^{\cite{Einstein1905}}$, propose di modificare gli assunti alla base della meccanica classica per prediligere un modello coerente con l'elettromagnetismo di Maxwell.\\Infatti all'epoca erano noti alcuni risultati sperimentali che giocavano a sfavore della concezione classica della meccanica, primo fra tutti l'esperimento di Michelson e Morley che avrebbe dovuto consentire di misurare la velocità della Terra rispetto all'etere luminifero, detta vento d'etere. L'esperimento ebbe un esito inaspettato, infatti non fu possibile misurare alcun vento d'etere portando i fisici a tre possibili spiegazioni: o l'Etere si muove assieme alla Terra, o l'apparato sperimentale si contrae lungo la direzione del moto terrestre oppure non esiste alcun Etere e la luce si propaga alla medesima velocità in ogni direzione e per ogni osservatore.\\

Einstein pose quindi a fondamento della sua teoria due postulati:
\begin{itemize}
    \item Il principio di relatività: basato sull'assunzione che esistano  una serie di sistemi di riferimento detti inerziali, reciprocamente in moto rettilineo uniforme, in cui le leggi della fisica sono identicamente valide.
    \item Il principio di costanza della velocità della luce: il quale asserisce che la luce nello spazio vuoto si propaghi sempre con modulo della velocità determinato ed identico per ogni osservatore inerziale, che si indicherà con $c$. 
\end{itemize}
Il primo si rifà al principio di relatività galileiano mentre il secondo è una diretta conseguenza dell'esperimento di Michelson e Morley il cui risultato viene spiegato senza la necessità dell'introduzione dell'etere e di un sistema di riferimento privilegiato. Inoltre si continua ad intendere spazio e tempo come due enti omogenei e isotropi.\\

La primissima conseguenza dell'assunzione di questi due postulati è la non validità delle trasformazioni di Galileo. Infatti secondo queste un moto di velocità $\vec{v}$ in un sistema $K$, se osservato in un sistema $K'$, nel quale $K$ si muove a velocità $\vec{V}$, risulterà in un moto a velocità $\vec{v}+\vec{V}$. Il secondo postulato però richiede che se tale moto è di un fascio di luce questo debba risultare sia in $K$ che in $K'$ in un moto con modulo della velocità pari a $c$, in totale disaccordo con le trasformazioni di Galileo.\\

Inoltre nel suo articolo Einstein stesso propose, dopo aver enunciato i postulati, un'esperimento mentale che consente di mostrare come questi siano in diretto conflitto con le assunzioni classiche dell'assolutezza del tempo e delle lunghezze. Si considerino due orologi reciprocamente a riposo posizionati in due punti detti $A$ e $B$. Einstein osservò che ogni orologio è in grado di misurare intervalli temporali solamente per eventi che avvengono nello stesso punto in cui ognuno è posizionato, questo poiché diventa necessario tener conto della velocità della luce che quindi propagandosi genera dei ritardi nella percezione degli eventi lontani.\\ Il secondo postulato consente però di sincronizzare gli orologi così che sia possibile confrontare i tempi misurati in $A$ e in $B$. In primo luogo si ipotizzi di far partire all'istante $t_A=0$, misurato dal primo orologio, un fascio di luce che viaggia da $A$ e giunge in $B$ quando l'orologio posizionato in tale punto segna un tempo $t_B$. In $B$ il fascio è riflesso e fa ritorno in $A$ quando il relativo orologio segna un tempo $t_A$. Poiché per il principio di costanza della velocità della luce il fascio luminoso deve propagarsi in ogni direzione con la stessa velocità e la distanza tra i due orologi è fissata e costante allora il tempo impiegato dalla luce per andare da $A$ a $B$ e vice versa deve essere il medesimo. Si conclude quindi che i due orologi sono sincronizzati solamente se vale:
\begin{equation}
    2t_B=t_A
    \label{SinconizazioneOrologi}
\end{equation}
Chiaramente se l'orologio nel punto $A$ si è rivelato sincrono con quello nel punto $B$, tramite la procedura appena descritta, è chiaro che è altrettanto vero che quello posto in $B$ è sincrono con quello posto in $A$ e che se si considera un orologio posto in un terzo punto $C$ che risulta sincrono con quello posto in $B$ allora questo terzo orologio è sincrono con quello nel punto $A$.\\Così facendo è possibile assegnare un immaginario orologio ad ogni punto dello spazio in maniera tale che siano tutti sincroni tra loro e sia possibile determinare quando due eventi lontani fra loro avvengono nello stesso istante in un determinato sistema di riferimento inerziale.\\
Fatta propria questa osservazione è possibile procedere analizzando l'esperimento mentale: si consideri un regolo di lunghezza $l$ in un sistema di riferimento ad esso solidale detto $K$. Si considerino anche due orologi sincroni posti nelle due estremità del regolo dette $A$ e $B$. In un sistema $K'$ si osserva il regolo in moto a velocità $v$ lungo la direzione in cui la parte lunga del regolo poggia. Sia detto $A'$ il punto del sistema $K'$ in cui si osserva l'emissione del fascio di luce dalla prima estremità del regolo, $B'$ il punto, sempre in $K'$, in cui si osserva la riflessione del fascio nel secondo estremo del regolo ed infine $C'$ il punto in $K'$ in cui si osserva il fascio fare ritorno alla prima estremità. In ognuno dei tre punti è presente un orologio sincronizzato con gli altri del sistema $K'$ in maniera da osservare l'emissione del fascio ad un tempo $t'_{A'}=0$. Se $t'_{B'}$ è l'istante in cui si osserva la riflessione in $B'$, $t'_{C'}$ è l'istante in cui il fascio fa ritorno al primo estremo del regolo e $l'$ è la lunghezza del regolo misurata nel sistema $K'$ allora è possibile determinare in funzione di questi tempi le distanze tra i punti $A',\ B'$ e $C'$, infatti queste dipenderanno in parte dalla distanza percorsa del regolo e in parte dalla sua lunghezza:
\begin{flalign*}
    &\Delta x'_{A'B'}=x'_{B'}-x'_{A'}=v(t'_{B'}-t'_{A'})+l'=vt'_{B'}+l'\\
    &\Delta x'_{B'C'}=x'_{B'}-x'_{C'}=l'-v(t'_{C'}-t'_{B'})
\end{flalign*}
Queste due distanze sono quelle percorse dal fascio luminoso rispettivamente in tempi $t'_{B'}$ e $t'_{C'}-t'_{B'}$, per cui si ottiene:
\begin{flalign*}
    \Delta x'_{A'B'}=ct'_{B'}=vt'_{B'}+l' \quad &\Rightarrow\qquad t'_{B'}=\frac{l'}{c-v}\\
    \Delta x'_{B'C'}=c(t'_{C'}-t'_{B'})=l'-v(t'_{C'}-t'_{B'}) \qquad &\Rightarrow\quad t'_{C'}-t'_{B'}=\frac{l'}{c+v}
\end{flalign*}
Questa semplice osservazione consente di concludere che i postulati enunciati da Einstein non ammettono la possibilità di assumere tempi assoluti: infatti l'osservatore in $K'$ osserva che il fascio di luce impiega più tempo per giungere al secondo estremo di quanto ne trascorra tra la riflessione e il suo ritorno al primo estremo, mentre invece l'osservatore in $K$, solidale con il regolo, osserva tempi identici per questi due tragitti. Una volta sviluppata matematicamente questa nuova teoria della relatività si osserverà che in maniera analoga pure le lunghezze non possono più essere considerate assolute.


\appendix
\input{Appendici/Sulla linearità delle trasformazioni di Lorentz}

\bibliographystyle{plain}
\bibliography{ref}

\end{document}
