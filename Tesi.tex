\documentclass[12pt,a4paper]{book}

\usepackage[italian]{babel}
\usepackage{newlfont}
\usepackage{color}
\textwidth=450pt\oddsidemargin=0pt
\usepackage{graphicx}
\usepackage[justification=centering]{caption}
\usepackage{titlesec}
\usepackage{pgfplots}
\pgfplotsset{compat=1.16}
\usepackage[none]{hyphenat}
\usepackage{amssymb}
\usepackage{amsmath}
\usepackage{amsthm}
\newtheorem{lemma}{Lemma}
\newtheorem{sublemma}{Lemma}[section]
\newtheorem{thm}{Teorema}
\newtheorem{prop}{Proposizione}
\numberwithin{equation}{section}
\usepackage[margin=1in]{geometry}
\usepackage{multicol}
\usepackage{float}
\usepackage{subcaption}
\usepackage{blindtext}
\parindent=0pt
\usepackage{fancyhdr}
\pagestyle{fancy}
\fancyhead[LE]{\textbf{Aspetti fisici e matematici della relatività ristretta}}
\fancyhead[RE]{Luca Morelli}
\fancyhead[RO]{\textbf{\nouppercase{\leftmark}}}
\fancyhead[LO]{}
\fancyfoot[LE,RO]{\thepage}
\cfoot{}
\fancypagestyle{plain}{}
\usepackage[perpage]{footmisc}
\newcommand{\vnabla}{\vec{\nabla}}
\usepackage{hyperref}
\hypersetup{
pdfauthor={Luca Morelli},
pdftitle={Aspetti fisici e matematici della teoria della relatività ristretta},
colorlinks, linktocpage=true, pdfstartpage=1, pdfstartview=FitV,%
breaklinks=true, pdfpagemode=UseNone, pageanchor=true, pdfpagemode=UseOutlines,%
plainpages=false, bookmarksnumbered, bookmarksopen=true, bookmarksopenlevel=1,%
hypertexnames=true, pdfhighlight=/O,%
urlcolor=black, linkcolor=black, citecolor=black, pagecolor=black,%}
}
\usetikzlibrary{decorations.pathmorphing,patterns}
\def\changemargin#1#2{\list{}{\rightmargin#2\leftmargin#1}\item[]}
\let\endchangemargin=\endlist 
\begin{document}
\begin{sloppypar}
\frontmatter
\begin{titlepage}
\begin{center}
	{{\Large{\textsc{Alma Mater Studiorum $\cdot$ Universit\`a di Bologna}}}} 
	\rule[0.1cm]{15.8cm}{0.1mm}
	\rule[0.5cm]{15.8cm}{0.6mm}
	\\\vspace{3mm}
	
	{\small{\bf Scuola di Scienze \\ 
			Dipartimento di Fisica e Astronomia\\
			Corso di Laurea in Fisica}}
	
\end{center}

\vspace{50mm}

\begin{center}\textcolor{black}{
		%
		% INSERIRE IL TITOLO DELLA TESI
		%
		{\LARGE{\bf Aspetti fisici e matematici della teoria\\\vspace{5mm} della relatività ristretta}}\\
}\end{center}

\vspace{50mm} \par \noindent

\begin{minipage}[t]{0.47\textwidth}
	%
	% INSERIRE IL NOME DEL RELATORE CON IL RELATIVO TITOLO DI DOTTORE O PROFESSORE
	%
	\large{\bf Relatore: \vspace{2mm}\\\textcolor{black}{
				Prof. Paolo Albano}}\\\\
\end{minipage}
%
\hfill
%
\begin{minipage}[t]{0.47\textwidth}\raggedleft \textcolor{black}{
		{\large{\bf Presentata da:
				\vspace{2mm}\\
				%
				% INSERIRE IL NOME DEL CANDIDATO
				%
				Luca Morelli  }}}
\end{minipage}

\vspace{40mm}

\begin{center}
	%
	% INSERIRE L'ANNO ACCADEMICO
	%
	Anno Accademico \textcolor{black}{ 2022/2023}
\end{center}


\end{titlepage}\null\newpage
\begin{center}
	\textbf{Abstract:}
\end{center}

Riassunto breve del documento da creare dove si sintetizzano tutti i punti trattati in maniera tale da rendere più immediata la selezione del documento da parte di un sventurato lettore.

\section*{Notazioni utilizzate}
\addcontentsline{toc}{section}{Notazioni utilizzate}
\vspace*{\fill}
$\frac{d}{dt}$ è la derivata totale rispetto al tempo, indicata anche con un punto.\\

$\frac{\partial}{\partial x}$ è la derivata parziale rispetto a $x$.\\
\subsection*{Notazioni di grandezze tridimensionali}
Il simbolo $\vec{a}$ denota un vettore di $\mathbb{R}^3$ le cui cartesiane sono indicate con $a_x,\ a_y\ , a_z$.\\
$\vec{x}$ e $\vec{v}$ i sono vettori posizione e velocità.\\
$q_i$ e $\dot q_i$ sono le coordinate e le velocità generalizzate. \\
$\vec{p}$ e $E$ sono la quantità di moto (o impulso) e l'energia.\\
Il campo elettrico e magnetico sono indicati con $\vec{E}\ \text{e}\ \vec{B}$.\\
Si è deciso di fare uso delle unità del Sistema Internazionale così i campi elettrico e magnetico dipendono dalle costanti dielettriche e magnetiche del vuoto $\epsilon_0$ e $\mu_0$.\\
$\vec a\cdot\vec b$ indica il prodotto scalare euclideo.\\
$\vec a\wedge\vec b$ indica il prodotto vettoriale.\\
$\vnabla$ è l'operatore $(\frac{\partial}{\partial x},\frac{\partial}{\partial y},\frac{\partial}{\partial z})$.\\
\subsection*{Notazioni di grandezze quadridimensionali}
Con $A^\mu$ si indica un 4-vettore (quadrivettore) controvariante di $\mathbb{R}^4$ mentre con $A_\mu$ si intende un 4-vettore covariante.\\
Le componenti di $A^\mu$ sono date dal variare di $\mu$ (in generale di lettere greche) tra $0$ e $3$, si usa anche la notazione $A^\mu=(A^0,\vec{A})$.\\
Con lettere latine sono indicate le componenti spaziali di un 4-vettore, così che $A^i$ indichi solamente $A^1,\ A^2,\ A^3$.\\
Due indici ripetuti sottintendono una sommatoria $A^\mu B^\mu=\sum_{\mu=0} A^\mu B^\mu$.\\
$g_{\mu \nu}$ è la matrice metrica con segnatura $(+,-,-,-)$.\\
$|A^\mu|$ è la norma Minkowski data da: $A^\mu A_\mu=(A^0)^2-(A^1)^2-(A^2)^2-(A^3)^2=g_{\mu \nu}A^\mu A^\nu$.\\
 
\tableofcontents
\mainmatter

\chapter*{Introduzione}   



\chapter{Fondamenti di relatività ristretta}

\section{La matematica dei sistemi di riferimento inerziali}
\label{sec:MathSDRI}
I sistemi di riferimento costituiscono le fondamenta dello studio dei moti dei corpi sulle quali costruire leggi fisiche, è quindi opportuno descrivere matematicamente tali oggetti e le loro proprietà per poi addentrarsi nella teoria della relatività.
\subsection{I sistemi di riferimento inerziali}
Come ogni ambito della fisica la descrizione matematica dei sistemi di riferimento si basa su una serie di osservazioni sperimentali che vengono riformulate come assiomi.\\

Il primo fatto sperimentale che viene assunto è che lo spazio sia tridimensionale, isotropo e omogeneo (ossia che non esistano rispettivamente direzioni e punti privilegiati). Inoltre si assume che lo spazio sia modellizzabile con la 
geometria euclidea mentre il tempo sia ad una sola dimensione ed anch'esso isotropo ed omogeneo.\\
Sulla base di queste assunzioni si può scegliere un punto dello spazio-tempo come 
origine di uno spazio vettoriale $\mathbb{R}\times\mathbb{R}^3$. In altre parole si scelgono l'origine e tre direzioni spaziali arbitrarie (linearmente indipendenti) lungo le quali sono orientati tre assi cartesiani, identificabili con una base di $\mathbb{R}^3$.
Inoltre dall'istante corrispondente al punto spazio-temporale precedentemente scelto si inizia a misurare il tempo. Si osservi che tali scelte sono arbitrarie poiché spazio e tempo sono isotropi ed omogenei.\\

Per poter formulare delle leggi che descrivano la realtà fisica risulta necessario assumere l'esistenza di una serie di sistemi di riferimento, detti inerziali, in cui tali leggi siano valide indipendentemente dal sistema in cui sono descritte. Questi sistemi sono sperimentalmente caratterizzati dalla proprietà di essere reciprocamente in moto rettilineo uniforme, ossia il moto dell'origine di un sistema, rispetto ad un qualsiasi altro sistema di riferimento inerziale, deve essere nella forma: 
\begin{equation}
	\vec x(t)=\vec x_0+\vec vt \qquad \vec x_0,\vec v\in \mathbb{R}^3, \ t\in\mathbb{R},
\end{equation}
dove $\vec x$ rappresenta un punto nello spazio, $t$ un istante di tempo e $\vec v$ la velocità (costante) reciproca dei due sistemi di riferimento.\\
Una volta identificati questi particolari sistemi  si può enunciare il principio di relatività: \emph{in ogni sistema di riferimento inerziale tutte le leggi della fisica sono identiche}.\\
\textbf{Invarianti VS Covarianti}

\subsection{Trasformazioni di sistemi di riferimento inerziali}
Si vuole ora studiare quali siano le più generiche trasformazioni che consentono di ottenere la descrizione di un fenomeno in un sistema di riferimento inerziale $K'$ conoscendo la descrizione di tale fenomeno in un primo riferimento inerziale $K$. In questo modo la generica trasformazione da $K$ a $K'$ sarà un'applicazione $f:\mathbb{R}\times \mathbb{R}^3\rightarrow\mathbb{R}\times \mathbb{R}^3$ invertibile; quest'ultima proprietà è necessaria poiché, come deve essere possibile passare da $K$ a $K'$, deve essere possibile fare anche il contrario.\\
Per poter caratterizzare le proprietà di tale applicazione è ora necessario tradurre matematicamente la richiesta imposta dal principio di relatività. Si osservi che, per tale principio e per la definizione di sistema inerziale, è necessario che in seguito ad una trasformazione tutti i sistemi inerziali restino tali. In altre parole $f$ deve trasformare tutte le rette in rette (nello spazio-tempo), questa condizione consente di utilizzare un teorema di geometria che consente di identificare la famiglia di queste trasformazioni (si seguirà la dimostrazione data in \cite{LostThmOfGeometry}). 

\begin{thm}
Sia $f:\mathbb{R}^n\rightarrow\mathbb{R}^n\ (n>1)$ una funzione biettiva che trasforma tutte le rette in rette. Allora $f$ è una trasformazione affine. 
\label{thm:LinGenMain}
\end{thm}
Per procedere alla dimostrazione di questo teorema è utile iniziare dimostrandone un caso particolare, ossia il caso $n=2$. Per questo caso particolare è però necessario dimostrare in primo luogo un risultato preliminare:
\begin{prop}
	Sia $\varphi:\mathbb{R}\rightarrow\mathbb{R}$ un automorfismo di campo\footnote{$\varphi$ è un automorfismo di campo se presi $x,y\in \mathbb{R}$ valgano $\varphi(x+y)=\varphi(x)+\varphi(y)$ e $\varphi(xy)=\varphi(x)\varphi(y)$}, allora $\varphi=\text{id}$.
\end{prop}	
\begin{proof}
	Si osservi che $\varphi(1)=1$ (infatti se $x\neq0$ allora $\varphi(x)=\varphi(1\cdot x)=\varphi(1)\varphi(x)$). Inoltre $\varphi(0)=0$ (infatti $\forall x\in\mathbb{R}$ vale $\varphi(0\cdot x)=\varphi(0)\varphi(x)=\varphi(0)$, questa equazione implica $\varphi(0)(\varphi(x)-1)=0$ che risulta soddisfatta per ogni $x$ solo se $\varphi(0)$ è identicamente nullo). Queste proprietà garantiscono che $\varphi(-1)=-1$ (infatti $\varphi(1-1)=\varphi(1)+\varphi(-1)=\varphi(0)=0$).\\
	Si prenda ora $n\in\mathbb{N}$, $n$ è esprimibile come somma ripetuta $n$ volte dell'unità, quindi:
	\begin{equation*}
		\varphi(n)=\varphi(1+1+...+1)=n\varphi(1)=n.
	\end{equation*}
	Dalle proprietà precedentemente descritte è immediato che $\varphi(n)=n$ valga per ogni numero intero (poiché $\varphi(-n)=-1\varphi(n)=-n$).\\
	Si consideri adesso $q\in \mathbb{Q}$, allora $\varphi(q)=q$. Infatti, se $x,y\in\mathbb{R}$ tali che $xy=1$, per quanto già detto, $\varphi(xy)=\varphi(x)\varphi(y)=1$ e quindi $\varphi(\frac{1}{x})=\frac{1}{\varphi(x)}$. Poiché ogni numero razionale è esprimibile come rapporto di numeri interi si ha che pure questi sono conservati da $\varphi$.\\
	Siano $x,y\in \mathbb{R}$ con $x<y$, allora $\exists z\in\mathbb{R}$ tale che $z\neq0$ e $y-x=z^2$. Utilizzando le proprietà supposte per ipotesi si ha quindi che:
	\begin{equation*}
		\varphi(y)-\varphi(x)=\varphi(y-x)=\varphi(z^2)=\varphi(z)^2>0.
	\end{equation*}
	Questo implica che $\varphi$ conservi l'ordinamento di $\mathbb{R}$.\\
	Sia ora $x\in\mathbb{R}$, poiché $\varphi(x)\in\mathbb{R}$ e $\mathbb{Q}$ è denso in $\mathbb{R}$ allora se $x\neq\varphi(x)$ deve esistere un $q\in\mathbb{Q}$ compreso tra $\varphi(x)$ e $x$. Si supponga $\varphi(x)>x$, (in maniera del tutto analoga si può procedere supponendo $\varphi(x)<x$). Allora $\varphi(x)> q> x$. Poiché $\varphi$ conserva l'ordinamento si ha l'assurdo: $\varphi(q)>\varphi(x)$ e al contempo $\varphi(x)>q=\varphi(q)$. Da questo ragionamento si deduce che $\varphi=\text{id}$.
\end{proof}
\begin{thm}
	Sia $f:\mathbb{R}^2\rightarrow\mathbb{R}^2$ una funzione biettiva che trasforma le rette in rette. Allora $f$ è una trasformazione affine. 
	\label{thm:LinGen2}
\end{thm}
\begin{proof}
    Sia $A:\mathbb{R}^2\rightarrow\mathbb{R}^2$ una trasformazione affine invertibile tale che $(A\circ f)(0,0)=(0,0),\ (A\circ f)(1,0)=(1,0)$ e $(A\circ f)(0,1)=(0,1)$. Si osservi che anche $A\circ f$ trasforma rette in rette. Sia $g=A\circ f$.\\
	
	Si mostrerà inizialemente che $\forall x,y\in\mathbb{R} ^2$ vale la proprietà $g(x+y)=g(x)+g(y)$.\\Si considerino i punti $x,y$ e $(0,0)$ non giacenti sulla stessa retta nel piano $\mathbb{R}^2$. Si prendano due rette passanti rispettivamente per l'origine e $x$ e l'origine e $y$ e le rispettive rette parallele passanti per $y$ nel primo caso e $x$ nel secondo, come mostrato in Figura \ref{Fig:PianoLinGen}, queste ultime due rette si intersecheranno nel punto $x+y$, formando quindi un parallelogrammo.\\
	\begin{figure}[h!]
		\centering
		\begin{tikzpicture}[scale=0.8]
		\filldraw[black] (0,0) circle (1pt) node[anchor=south east]{$(0,0)$};
		\filldraw[black] (.5,2) circle (1pt) node[anchor=south east]{$y$};
		\filldraw[black] (2,.5) circle (1pt) node[anchor=north west]{$x$};
		\filldraw[black] (2.5,2.5) circle (1pt) node[anchor=north west]{$x+y$};

		\draw[black, thick] (-1,-0.245) -- (3,0.75);
		\draw[black, thick] (-.5,1.755) -- (3.5,2.75);
		\draw[black, thick] (-0.257,-1) -- (0.75,3);
		\draw[black, thick] (1.61,-1) -- (2.75,3.5);

		\filldraw[black] (0+10,0) circle (1pt) node[anchor=south east]{$(0,0)$};
		\filldraw[black] (.5+10,2) circle (1pt) node[anchor=south east]{$g(y)$};
		\filldraw[black] (2+10,.5) circle (1pt) node[anchor=north west]{$g(x)$};
		\filldraw[black] (2.5+10,2.5) circle (1pt) node[anchor=north west]{$g(x+y)$};

		\draw[black, thick] (-1+10,-0.245) -- (3+10,0.75);
		\draw[black, thick] (-.5+10,1.755) -- (3.5+10,2.75);
		\draw[black, thick] (-0.257+10,-1) -- (0.75+10,3);
		\draw[black, thick] (1.61+10,-1) -- (2.75+10,3.5);

		\draw[black, ultra thick,->] (5,1.5) -- (8,1.5);
		\node[fill=white] at (6.5,1.5) {$g$};
		
	\end{tikzpicture}
	\caption{Rappresentazione grafica di come agisce la funzione $g$ nel piano}
	\label{Fig:PianoLinGen}
	\end{figure}
	Per ipotesi queste quattro rette verranno mappate in altre quattro rette da $g$ e, poiché questa funzione è biettiva, le immagini di rette parallele saranno a loro volta parallele (altrimenti nel punto di incidenza si perderebbe l'iniettività di $g$), mentre i punti di incidenza potranno essere mappati solamente in altri punti di incidenza (poiché questi appartenendo ognuno a due rette dovranno, per biettività, appartenere ad entrambe le immagini delle due rette nel codominio). Questo sta a significare che le quattro rette così ottenute formeranno un nuovo parallelogrammo di vertici $g(x),\ g(y),\ g(x+y)$ e $(0,0)$, poiché per costruzione $g(0,0)=(0,0)$. La regola del parallelogramma garantisce quindi che $g(x)+g(y)=g(x+y)$.\\
	Se invece i punti $x,y$ e $(0,0)$ giacciono sulla stessa retta è sufficiente prendere un punto $z\in \mathbb{R}^2$ che non giaccia sulla medesima retta dei punti precedenti. Si può quindi utilizzare quanto dimostrato nel caso precedente per ottenere:
	\begin{equation*}
		g(x+y+z)=g(x+y)+g(z)= g(x)+g(y+z)=g(x)+g(y)+g(z)
	\end{equation*}
	da cui segue che $g(x+y)=g(x)+g(y)$ anche in questo caso.\\

    Poiché per ipotesi $g$ conserva rette in rette e per costruzione mappa l'origine nell'origine e i punti $(0,1)$ e $(1,0)$ in se stessi allora $g$ deve trasformare gli assi $x$ e $y$ in se stessi. Si possono quindi considerare due applicazioni $\alpha,\beta:\mathbb{R} \rightarrow\mathbb{R} $ tali che $g(x,y)=(\alpha(x),\beta(y))$ con  $x,y\in\mathbb{R}$. Si osservi che, per quanto dimostrato fino ad ora, $f(1,1)=f(1,0)+f(0,1)=(1,0)+(0,1)=(1,1)$. Analogamente a come si è detto per gli assi $x$ e $y$, anche la bisettrice del primo quadrante è quindi mappata in se stessa per cui, per ogni $x\in\mathbb{R}$, $\alpha(x)=\beta(x)$. Si proseguirà quindi studiando solo una di queste due funzioni (poiché quanto si dirà per una è valido anche per l'altra).\\

	Siano $a,b\in\mathbb{R}$ e si consideri una retta passante per l'origine.
	\begin{equation*}
		L=\{ (x,y)\in\mathbb{R}^2 \text{ tale che } y=ax\  \}.
	\end{equation*}
    Il punto $(1,a)\in L$ e poiché $g(1,a)=(1,\alpha(a))$ si ottiene che $L$ è mappata in una nuova retta passante per l'origine con coefficiente angolare $\alpha(a)$.
	\begin{equation*}
		g(L)=\{ (x,y)\in\mathbb{R}^2 \text{ tale che } y=\alpha(a)x\}.
	\end{equation*} 
	Pure $(b,ab)\in L$ e quindi $(\alpha(b),\alpha(ab))\in g(L)$, ricordando però che $g(L)$ è una retta passante per l'origine con coefficiente angolare $\alpha(a)$, si deduce che dovrà sussistere la seguente relazione: $\alpha(ab)=\alpha(a)\alpha(b)$.\\

	Si è quindi dimostrato che per $\alpha$ e $\beta$ valgono le seguenti relazioni per ogni $x,y\in\mathbb{R}$:
	\begin{flalign*}
		&&\alpha(x+y)=\alpha(x)+\alpha(y)\qquad \alpha(xy)=\alpha(x)\alpha(y)&&\\
		&&\beta(x+y)=\beta(x)+\beta(y)\qquad \beta(xy)=\beta(x)\beta(y).&&
	\end{flalign*}
	Queste relazioni implicano che $\alpha$ e $\beta$ siano due automorfismi di campo di $\mathbb{R}$ e, per la Proposizione 1, risultano entrambi l'identità, di conseguenza pure $g=A\circ f=\text{id}$. Infine applicando a $g$ la trasformazione inversa di $A$ (così da avere $A^{-1}\circ A\circ f=A^{-1}=f$) si è dimostrato che $f$ deve essere affine.
\end{proof}

Si procederà ora generalizzando questo teorema con la dimostrazione del Teorema \ref{thm:LinGenMain}.

\begin{proof}[Dimostrazione Teorema \ref{thm:LinGenMain} $(n>2)$]
	Si consideri la traslazione $T:\mathbb{R}^n \rightarrow\mathbb{R}^n$ tale che $(T\circ f)(0)=0$ e si definisca $g=T\circ f$.\\
	Siano $x,y\in\mathbb{R}^n$ e $\pi$ un piano tale che $x,y\in\pi$. Si prendano due rette $r_1,r_2\in\pi$ e incidenti in $0$, e un punto $P\in\pi$. Se si considerano altre due rette $r_3$ e $r_4$ passanti per $P$ e tali che $r_3$ sia parallela a $r_1$ e $r_4$ lo sia per $r_2$, così che $r_3$ intersecherà $r_2$ in un punto $P_2$ e analogamente $r_4$ intersecherà $r_1$ in un punto $P_1$, è possibile costruire un parallelogrammo con vertice $P$ e lati giacenti sulle rette considerate.
	\begin{figure}[h!]
		\centering
		\begin{tikzpicture}[scale=0.8]
		\filldraw[black] (0,0) circle (1pt) node[anchor=south east]{$0$};
		\filldraw[black] (.5,2) circle (1pt) node[anchor=south east]{$P_1$};
		\filldraw[black] (2,.5) circle (1pt) node[anchor=north west]{$P_2$};
		\filldraw[black] (2.5,2.5) circle (1pt) node[anchor=north west]{$P$};

		\draw[black, thick] (-1,-0.245) -- (3,0.75) node[anchor=south]{$r_2$};
		\draw[black, thick] (-.5,1.755) -- (3.5,2.75) node[anchor=south]{$r_4$};
		\draw[black, thick] (-0.257,-1) -- (0.75,3) node[anchor=west]{$r_1$};
		\draw[black, thick] (1.61,-1) -- (2.75,3.5) node[anchor=west]{$r_3$};

		\filldraw[black] (0+10,0) circle (1pt) node[anchor=south east]{$g(0)$};
		\filldraw[black] (.5+10,2) circle (1pt) node[anchor=south east]{$g(P_1)$};
		\filldraw[black] (2+10,.5) circle (1pt) node[anchor=north west]{$g(P_2)$};
		\filldraw[black] (2.5+10,2.5) circle (1pt) node[anchor=north west]{$g(P)$};

		\draw[black, thick] (-1+10,-0.245) -- (3+10,0.75) ;
		\draw[black, thick] (-.5+10,1.755) -- (3.5+10,2.75) ;
		\draw[black, thick] (-0.257+10,-1) -- (0.75+10,3) ; 
		\draw[black, thick] (1.61+10,-1) -- (2.75+10,3.5) ;

		\draw[black, ultra thick,->] (5,1.5) -- (8,1.5);
		\node[fill=white] at (6.5,1.5) {$g$};
		\node[fill=white] at (-1,1) {$\pi$};
		\node[fill=white] at (14.3,1) {$g(\pi)$};
		
	\end{tikzpicture}
	\caption{Rappresentazione grafica di come agisce la funzione $g$ nel piano}
	\label{Fig:PianoLinGen2}
\end{figure}
	 Poiché $g$ è biettiva e trasforma rette in rette allora, come si è già visto per dimostrare il Teorema \ref{thm:LinGen2}, questo parallelogrammo deve essere trasformato in un nuovo parallelogrammo con i lati giacenti sulle immagini delle precedenti rette e con i vertici dati dalle immagini dei vertici del precedente parallelogrammo. Detto $\pi'$ il piano contente le immagini di $r_1$ e $r_2$ allora, per costruzione del parallelogrammo, si avrà che $g(P)\in \pi'$ e poiché $P\in\pi$ è arbitrario si deduce che ogni punto di $\pi$ è mappato in $\pi'$ ossia $g$ trasforma piani in altri piani.\\

	Si consideri ora un'applicazione lineare invertibile $L$ tale che $(L\circ g)(\pi)=\pi$, in questo modo $L\circ g\big|_{\pi}:\mathbb{R} ^2\rightarrow\mathbb{R} ^2$ è biettiva e trasforma rette in rette, per il Teorema \ref{thm:LinGen2} allora questa è un'applicazione affine, nella fattispecie poiché $g(0)=0$ per costruzione e $L$ è lineare allora $(L\circ g)(0)=0$, per cui $L\circ g$ è lineare. Risulta a questo punto sufficiente far uso dell'inversa di $L$ per ottenere:
	\begin{flalign*}
		g(\alpha x+\beta y)&=L^{-1}((L\circ g)(\alpha x+\beta y))\\&=L^{-1}(\alpha (L\circ g)(x)+\beta (L\circ g)(y))=\alpha g(x)+ \beta g(y) \quad \alpha,\beta\in\mathbb{R}.
	\end{flalign*}
	Poiché la scelta di $x$ e $y$ fatta prima di determinare il piano $\pi$ è arbitraria, segue che $g$ è lineare e ,considerando $f=T^{-1}\circ g$, si conclude che $f$ deve essere affine.
\end{proof}

Così facendo si conclude che ogni trasformazione tra sistemi di riferimento inerziali deve essere necessariamente una trasformazione affine di qualche sorta. Ulteriori osservazioni sperimentali consentiranno di identificare famiglie più ristrette di queste trasformazioni.


\section{Fatti di Fisica Classica}
Quanto scoperto dai fisici fino alla fine dell'ottocento fu sviluppato sulle idee di Galilei e Newton nella loro formulazione classica della meccanica. Infatti, utilizzando i risultati di questa branca della fisica, scienziati come Coulomb, Ampère e Faraday furono in grado di descrivere fenomeni noti da svariati secoli dando vita alla teoria dell'elettromagnetismo.\\
Poiché la nascita della teoria della relatività speciale è strettamente connessa a queste due branche si procederà illustrando brevemente i loro fondamenti. 


\subsection{La meccanica e le trasformazioni di Galileo}\label{sec:MC}
Per delineare gli schemi della meccanica classica, oltre a quanto assunto nella sezione \ref{sec:MathSDRI}, ossia che lo spazio sia tridimensionale, isotropo, omogeneo e che rispetti la geometria euclidea mentre il tempo sia ad una sola dimensione, si prende come assioma che le distanze spaziali e temporali siano assolute, ossia che ogni osservatore concordi sulla misura di queste.\\
Inoltre si assume che, in un riferimento inerziale, le posizioni e le velocità dei punti di un sistema ad un tempo iniziale ne determinino in maniera univoca l'evoluzione $\vec x(t)\in \mathbb{R}^3$ secondo la legge:
\begin{equation}
	m\ddot{\vec{x}}(t)=\vec{F}(t,\vec{x},\dot{\vec{x}})
	\label{equazioneDiNewton}
\end{equation}
dove $m$ è detta massa inerziale e $\vec{F}$ è una funzione caratteristica del sistema detta forza.\\

Si vuole quindi identificare quali applicazioni $\varphi:\mathbb{R}^3\times\mathbb{R}\rightarrow
\mathbb{R}^3\times\mathbb{R}$ consentono di cambiare sistema di riferimento inerziale, ossia quali 
trasformazioni non variano le leggi della natura. Queste applicazioni si chiamano trasformazioni di 
Galileo e, come si è dimostrato nella sezione \ref{sec:MathSDRI}, per soddisfare il principio di relatività devono essere applicazioni affini.
Una generica trasformazione di Galileo è quindi data dalla composizione di tre famiglie di applicazioni:
\begin{itemize}
	\item una generica traslazione spazio temporale dell'origine, dedotta dalla proprietà 
	di omogeneità dello spazio e del tempo:
	\begin{equation}
		\varphi_{\vec{r},s}(t,\vec{x})=(t+s,\vec{x}+\vec{r})
		\label{GalileoTraslazoine}
	\end{equation} 
\item una generica rotazione degli assi spaziali, dovuta alla proprietà di isotropia dello spazio:
\begin{equation}
	\varphi_{G}(t,\vec{x})=(t,G\vec{x}) \qquad G\in M_{3\times3}(\mathbb{R}):G^{-1}=G^t
	\label{GalileoRotazione}
\end{equation} 
	\item una traslazione di moto rettilineo uniforme, ammissibile grazie alle proprietà di muoversi di tale moto dei sistemi 
	di riferimento inerziali:
\begin{equation}
	\varphi_{\vec{v}}(t,\vec{x})=(t,\vec{x}+\vec{v}t)
	\label{GalileoVelocità}
\end{equation} 
\end{itemize}
Si osservi che la rotazione (\ref{GalileoRotazione}) può avvenire solamente rispetto alle direzioni spaziali, infatti se fosse una rotazione di tutto lo spazio-tempo le lunghezze e gli intervalli di tempo non sarebbero più assoluti.\\

La trasformazione (\ref{GalileoVelocità}) è quella che viene comunemente studiata per caratterizzare le trasformazioni di sistemi inerziali. 
Si consideri quindi un sistema $K$, con coordinate $(t,\vec{x})$ e un sistema $K'$, con 
coordinate $(t',\vec{x'})$, in moto a velocità $\vec{V}$ rispetto a $K$, si scriverà allora la 
(\ref{GalileoVelocità}) come:
\begin{equation}
	\vec{x'}=\vec{x}-\vec{V}t, \qquad t'=t.
	\label{GalileoEasy}
\end{equation}
Il principio di relatività impone l'invarianza di ogni legge fisica rispetto alla trasformazione (\ref{GalileoEasy}), è quindi opportuno
studiare come si trasformino gli operatori di differenziazione con questa.
Se si vuole derivare una $f(\vec{x},t)$, dalla regola di Leibniz, si ha:
\begin{equation*}
		\frac{\partial}{\partial t'}=\frac{\partial t}{\partial t'}\frac{\partial}{\partial t}+
		\sum_{i=1}^{3}\frac{\partial x_i}{\partial t}\frac{\partial t}{\partial t'}
		\frac{\partial}{\partial x_i}, \qquad \qquad
		\frac{\partial}{\partial x'_i}=\frac{\partial t}{\partial x'_i}\frac{\partial}{\partial t}+
		\sum_{i=1}^{3}\frac{\partial x_j}{\partial x'_i}\frac{\partial}{\partial x_j}
\end{equation*}
dove $\frac{\partial}{\partial t}$ è la derivata parziale rispetto al tempo nel sistema $K$, $\frac{\partial}{\partial t'}$ è la derivata parziale rispetto al tempo nel sistema $K'$, $\frac{\partial}{\partial x_i}$ è la derivata parziale rispetto alla coordinata i-esima del sistema $K$ e $\frac{\partial}{\partial x_i'}$ è la derivata parziale rispetto alla coordinata i-esima del sistema $K'$.\\Osservando dalla (\ref{GalileoEasy}) che $\frac{\partial t}{\partial t'}=1$, che 
$\frac{\partial t}{\partial x'_i}=0$, che $\frac{\partial x_i}{\partial t}=V_i$ e che 
$\frac{\partial x_j}{\partial x'_i}=\delta_{ij}$, dove $\delta_{ij}$ è una delta di Kronecker, 
si ottengono le trasformazioni degli operatori di differenziazione:
\begin{equation}
	\frac{\partial}{\partial t'}=\frac{\partial}{\partial t}+\vec{V}\cdot\vec{\nabla}, \qquad \qquad
	\frac{\partial}{\partial x'_i}=\frac{\partial}{\partial x_i}
	\label{GalileoDifferenziale}
\end{equation}
dove $\vnabla=(\frac{\partial }{\partial x},\frac{\partial }{\partial y},\frac{\partial }{\partial z})$.\\
Dalla (\ref{GalileoDifferenziale}) si deduce che tutti gli operatori che comprendono solamente derivate 
rispetto alle coordinate spaziali sono lasciati inalterati dalle trasformazioni di Galileo. In particolare si ha che $\vnabla'=(\frac{\partial }{\partial x'},\frac{\partial }{\partial y'},\frac{\partial }{\partial z'})=\vnabla$.\\

Note queste trasformazioni è possibile osservare che il vettore accelerazione $\ddot{\vec{x}}$ risulta galileo invariante, infatti considerando una traiettoria $\vec{x}(t)$:
\begin{equation*}
	\frac{d^2}{dt'^2}(\vec{x'}(t))=\frac{d^2}{dt^2}(\vec{x}(t)-\vec{V}t)=\ddot{\vec{x}}(t)+\frac{d}{dt}(\vec{V})=\ddot{\vec{x}}(t).
\end{equation*} 
Questo fatto, assieme al principio di relatività galileiano applicato alla legge di Newton (\ref{equazioneDiNewton}), 
impone che le forze esercitate su di un punto e misurate in due sistemi inerziali differenti debbano essere le medesime. \\

Analogamente a quanto appena fatto si possono ricavare le trasformazioni per la velocità di un punto in moto con traiettoria $\vec{x}(t)$:
\begin{equation}
	\frac{d}{dt'}(\vec{x'}(t))=\frac{d}{dt}(\vec{x}(t)-\vec{V}t)=\dot{\vec{x}}(t)+\vec{V}.
\end{equation}
Per cui si ottiene dalle trasformazioni di Galileo che le velocità si compongono per somma algebrica.\\


\subsection{L'elettromagnetismo e le equazioni di Maxwell}\label{sec:EquazioniMaxwell}
Per interpretare i fenomeni elettromagnetici, anche in questo caso, è necessario introdurre
una serie di osservazioni sperimentali: in primo luogo esiste una proprietà della materia, 
detta carica elettrica, che non dipende dal sistema di riferimento in cui è misurata e che consente 
ai corpi di interagire con due campi vettoriali: 
il campo elettrico $\vec{E}$ e il campo magnetico $\vec{B}$. Un corpo puntiforme di carica 
$q$ subisce quindi un forza $\vec{F}$ data da:
\begin{equation}
	\vec{F}=q(\vec{E}(\vec{x},t)+\vec{v}\wedge\vec{B}(\vec{x},t)),
	\label{ForzaLorentz}
\end{equation}
dove $\vec{v}$ è il vettore velocità di tale corpo.\\
In secondo luogo gli esperimenti mostrarono che questi due campi rispettano una serie di equazioni 
dette equazioni di Maxwell:
\begin{equation}
	\begin{gathered}
		\vec{\nabla}\cdot\vec{E}=\frac{\rho}{\epsilon_0}, \qquad \qquad \vec{\nabla}\cdot\vec{B}=0, \\
		\vec{\nabla}\wedge\vec{E}=-\frac{\partial\vec{B}}{\partial t}, \qquad \qquad \vec{\nabla}\wedge
		\vec{B}=\mu_0\vec{J}+\epsilon_0\mu_0\frac{\partial\vec{E}}{\partial t},
		\label{EquazioniMaxwell}
	\end{gathered}
\end{equation}
dove $\rho$ è la densità di carica volumetrica, $\vec{J}$ è la densità di corrente di carica superficiale e 
$\epsilon_0$, $\mu_0$ sono due costanti del vuoto.\\

Dalle equazioni di Maxwell (\ref{EquazioniMaxwell}) segue che le cariche sono sorgenti del campo 
elettrico mentre le correnti lo sono per 
il campo Magnetico. Per esempio è possibile mostrare che una carica puntiforme è sorgente di campo elettrico e 
tramite la forza di Lorentz (\ref{ForzaLorentz}) è possibile derivare la legge sperimentale di Coulomb.\\
Si consideri una carica puntiforme $Q$ posta nell'origine di un sistema di riferimento e si prenda una sfera $\mathcal{S}$ di raggio $R$ 
contenete tale carica. Integrando la prima equazione di Maxwell e facendo uso del teorema di Gauss si ottiene:
\begin{equation*}
	\int_\mathcal{S}\vec{\nabla}\cdot\vec{E}\ d^3x=\int_\mathcal{S}\frac{\rho}{\epsilon_0}\ d^3x=\frac{Q}{\epsilon_0}=\oint_{\partial\mathcal{S}}\vec{E}\cdot\hat{n}\ d\Sigma
\end{equation*}
dove si è indicato con $\hat{n}$ il versore normale alla superficie $\mathcal{S}$ nel punto in cui si valuta l'integrando.\\
Si osservi che essendo lo spazio 
isotropo e la carica puntiforme allora qualsiasi rotazione degli assi coordinati mantiene immutato il sistema, ne segue che 
il Campo Elettrico generato dalla carica deve essere radiale e costante su superfici sferiche centrate nell'origine. Infatti un'ipotetica seconda carica 
fissa a distanza $R$ dall'origine del sistema di riferimento deve percepire sempre la medesima forza, indipendentemente dalla rotazione effettuata, 
che risulta connessa al campo elettrico per mezzo della (\ref{ForzaLorentz}), inoltre sempre dalle Equazioni di Maxwell, supponendo assenza di correnti 
si ha che il rotore di $\vec{E}$ risulta nullo. Grazie a queste deduzioni l'integrale di superficie si riduce a:
\begin{equation*}
	\oint_{\partial\mathcal{S}}\vec{E}\cdot\hat{n}\ d\Sigma=4\pi R^2|\vec{E}|=\frac{Q}{\epsilon_0}.
\end{equation*}	
Tenendo conto che $\vec{E}$ è radiale, come si è dedotto, si ha la legge di Coulomb
\begin{equation}
	\vec{E}=\frac{1}{4\pi\epsilon_0}\frac{Q}{R^2}\hat{r} \quad \Rightarrow \quad \vec{F}=q\vec{E}=\frac{1}{4\pi\epsilon_0}\frac{qQ}{R^2}\hat{r}
\end{equation} 
dove si è indicato con $\hat{r}$ il versore radiale in coordinate sferiche.\\

Infine è importante per trattare la teoria della relatività osservare che è possibile ottenere, calcolando il rotore di ambo 
i membri delle ultime due equazioni di Maxwell
\begin{equation*}
	\vec{\nabla}\wedge\vec{\nabla}\wedge\vec{E}=-\frac{\partial}{\partial t}\vnabla\wedge\vec{B},
	\qquad \vec{\nabla}\wedge\vec{\nabla}\wedge\vec{B}=\mu_0\epsilon_0\frac{\partial}{\partial t}
	\vnabla\wedge\vec{E},
\end{equation*}
e supponendo assenza di cariche, per cui $\rho=0$ e $\vec{J}=0$, due equazioni che descrivono 
onde di campo elettrico e magnetico nel vuoto:
\begin{equation}
	\vnabla^2\vec{E}=\mu_0\epsilon_0\frac{\partial^2\vec{E}}{\partial t^2} \qquad \vnabla^2\vec{B}=
	\mu_0\epsilon_0\frac{\partial^2\vec{B}}{\partial t^2}
	\label{OndeEMVuoto}
\end{equation}
dove si è indicato l'operatore laplaciano con la notazione $\vnabla^2=(\frac{\partial^2 }{\partial x^2},\frac{\partial^2 }{\partial y^2},\frac{\partial^2 }{\partial z^2})$.\\
Queste onde si propagano con una velocità $\frac{1}{\sqrt{\mu_0\epsilon_0}}=299792458\  \frac{m}{s}$, 
che corrisponde con precisione ai valori sperimentalmente misurati della velocità della luce.\\ Maxwell stesso suppose allora che questa fosse quindi da intendere come un fenomeno elettromagnetico e successivi esperimenti, come quelli di Hertz, confermarono tale ipotesi.\\
Sulla base della teoria ondulatoria classica è però necessario identificare un mezzo nel quale queste onde possano 
propagarsi e rispetto al quale la loro velocità di propagazione debba essere intesa. Per questo motivo alla fine dell'ottocento venne ipotizzata l'esistenza di tale mezzo detto etere luminifero.

\section{L'articolo di Einstein del 1905}
La Meccanica newtoniana e l'elettromagnetismo di Maxwell si rivelarono agli occhi dei fisici dell'ottocento 
incompatibili fra loro, 
poiché le equazioni di Maxwell non risultarono invarianti per le trasformazioni di Galilei. 
Proprio per questo motivo i fisici dell'epoca dovettero rivalutare i principi alla base delle leggi 
della natura fino a quando l'incompatibilità trovò una soluzione nel 1905 con la teoria relatività ristretta di Einstein.
Si ripercorreranno ora i passi che Einstein stesso indicò nel suo articolo \cite{Einstein1905} del 1905.
\subsection{La non invarianza delle equazioni di Maxwell}
\label{Sec:nonInvMax}
 Si considerino due sistemi di riferimento $K$ e $K'$, inerziali e reciprocamente in moto in maniera tale che in $K'$ l'origine del sistema $K$ risulti in moto a velocità costante $\vec{V}$. Allora in ogni sistema si misureranno rispettivamente $\vec{E},\ \vec{B}$ e $\vec{E'},\ \vec{B'}$.\\ Il principio di relatività galileiana impone che questi due campi, nei loro sistemi di riferimento, soddisfino le equazioni di Maxwell (\ref{EquazioniMaxwell}). Inoltre, siccome la forza di Lorentz (\ref{ForzaLorentz}) deve essere la medesima in tutti i sistemi di riferimento inerziali, considerando una carica $q$ in moto con velocità $\vec{v}$ in $K$ e $\vec{V}+\vec{v}$ in $K'$, essendo $q$ invariante deve valere:
 \begin{flalign*}
		\vec{F'}=\vec{F}\quad&\Rightarrow\quad \vec{E'}+\vec{v}\wedge\vec{B'}=\vec{E}+(\vec{V}+\vec{v})\wedge\vec{B}\\
         &\Rightarrow\quad \vec{E'}+\vec{v}\wedge(\vec{B'}-\vec{B})=\vec{E}+\vec{V}\wedge\vec{B'}.
 \end{flalign*}
Così facendo si possono ottenere delle relazioni tra $\vec{E},\ \vec{B}$ e $\vec{E'},\ \vec{B'}$ che costituiscono quindi le trasformazioni dei campi elettrici e magnetici tra sistemi inerziali. Queste però possono dipendere esclusivamente dalle proprietà dei due sistemi di riferimento considerati e nella fattispecie dalla loro velocità reciproca. Per questo motivo, il termine contenente $\vec{v}$, ossia la velocità della carica nel sistema $K$, deve annullarsi. Quindi le trasformazioni risultano:
\begin{equation}
	\begin{cases}
		\vec{E'}(t,\vec{x'})=\vec{E}(t,\vec{x'})+\vec{V}\wedge\vec{B}(t,\vec{x})\\
		\vec{B'}(t,\vec{x'})=\vec{B}(t,\vec{x})
	\end{cases}.
	\label{TrasfGalileiEB}
\end{equation}
Bisogna ora studiare come si trasformino le grandezze generatici dei campi: $\rho$ e $\vec{J}$. Considerando un volume $\Delta V$, in cui è presente una carica $\Delta q$. Allora, la densità di carica è definita come
\begin{equation*}
	\rho=\lim_{\Delta V\rightarrow 0}\frac{\Delta q}{\Delta V}.
\end{equation*} Visto che le lunghezze sono assunte essere assolute allora devono esserlo pure i volumi e, essendo la carica non dipendente dal sistema di riferimento, si conclude che pure la densità di carica non dipende dal sistema di riferimento. Per quanto riguarda la densità di corrente superficiale, definita come $\vec{J}=\rho\vec{v}$,  è sufficiente applicare le trasformazioni delle velocità tra due sistemi in moto reciproco a velocità $\vec{V}$ per ottenere:
\begin{equation}
	\vec{J'}=\vec{J}-\rho\vec{V}.
\end{equation}
I risultati appena ottenuti consentono di studiare l'invarianza o meno delle equazioni di Maxwell per le trasformazioni di Galilei.\\

Si consideri la prima equazione di Maxwell in $K'$ e la si supponga valida in $K$. Trasformando $E'$ in $E$ e per gli operatori di differenziazione secondo la (\ref{GalileiDifferenziale}), si ottiene una quantità che deve annullarsi affinché sia valida questa equazione anche in $K'$, come richiesto dal principio di relatività.
\begin{flalign*}
	&\vnabla'\cdot\vec{E'}-\frac{\rho}{\epsilon_0}=\vnabla\cdot(\vec{E}+\vec{V}\wedge\vec{B})-\frac{\rho}{\epsilon_0}\\
	&=\left(\vnabla\cdot\vec{E}-\frac{\rho}{\epsilon_0}\right)-\vec{V}\cdot(\vnabla\wedge\vec{B})=-\vec{V}\cdot(\vnabla\wedge\vec{B}).
\end{flalign*}
Il primo termine tra parentesi è identicamente nullo (poiché valgono le equazioni di Maxwell in $K$) mentre l'ultimo termine non è sempre nullo, nella fattispecie in presenza di campi elettrici variabili nel tempo. Questo implica che quindi la prima equazione di Maxwell non sia invariante per le trasformazioni di Galilei.\\

Se si studia la seconda con lo stesso procedimento si scopre che questa invece è invariante:
\begin{equation*}
	\vnabla'\cdot\vec{B'}=\vnabla\cdot\vec{B}=0.
\end{equation*}

Analogamente per la terza equazione:
\begin{flalign*}
	&\vnabla'\wedge\vec{E'}+\frac{\partial \vec{B'}}{\partial t'}=
	\vnabla\wedge\vec{E'}+\frac{\partial \vec{B'}}{\partial t}+(\vec{V}\cdot\vnabla)\vec{B'}\\
	&=\vnabla\wedge(\vec{E}+\vec{V}\wedge\vec{B})+\frac{\partial \vec{B}}{\partial t}+(\vec{V}\cdot\vnabla)\vec{B}\\
	&=\left(\vnabla\wedge\vec{E}+\frac{\partial \vec{B}}{\partial t}\right)+\vnabla\wedge\vec{V}\wedge\vec{B}+(\vec{V}\cdot\vnabla)\vec{B}=0.
\end{flalign*} 
Infatti il termine tra parentesi è identicamente nullo poiché in $K$ vale la terza equazione di Maxwell e, ricordando che $\vec{V}$ è costante, si annullano gli addendi restanti se sviluppati tramite le regole di differenziazione:
\begin{equation*}
	\vnabla\wedge\vec{V}\wedge\vec{B}+(\vec{V}\cdot\vnabla)\vec{B}=-(\vec{V}\cdot\vnabla)\vec{B}+(\vec{V}\cdot\vnabla)\vec{B}=0.
\end{equation*}
L'ultima equazione di Maxwell è invece non-invariante. Infatti, sempre con il medesimo procedimento, si ottiene:
\begin{flalign*}
	&\vnabla '\wedge\vec{B'}-\mu_0\epsilon_0\frac{\partial \vec{E'}}{\partial t'}-\mu_0\vec{J'}=
	\vnabla\wedge\vec{B'}-\mu_0\epsilon_0\frac{\partial \vec{E'}}{\partial t}-\mu_0\epsilon_0(\vec{V}\cdot\vnabla)\vec{E'}-\mu_0\vec{J'} &&\\
	&=\vnabla\wedge\vec{B}-\mu_0\epsilon_0\frac{\partial \vec{E}}{\partial t}-\mu_0\epsilon_0\vec{V}\wedge\frac{\partial \vec{B}}{\partial t}-\mu_0\epsilon_0(\vec{V}\cdot\vnabla)(\vec{E}+\vec{V}\wedge\vec{B})-\mu_0\vec{J}+\mu_0\vec{V}\rho &&\\
	&=\left(\vnabla\wedge\vec{B}-\mu_0\epsilon_0\frac{\partial \vec{E}}{\partial t}-\mu_0\vec{J}\right)-\mu_0\epsilon_0\vec{V}\wedge\frac{\partial \vec{B}}{\partial t}-\mu_0\epsilon_0(\vec{V}\cdot\vnabla)(\vec{E}+\vec{V}\wedge\vec{B})+\mu_0\vec{V}\rho &&\\
	&=-\mu_0\epsilon_0\vec{V}\wedge(-\vnabla\wedge\vec{E})-\mu_0\epsilon_0(\vec{V}\cdot\vnabla)(\vec{E}+\vec{V}\wedge\vec{B})+\mu_0\vec{V}\rho &&\\
	&=-\mu_0\epsilon_0(\vec{V}\cdot\vnabla)(\vec{V}\wedge\vec{B})+\mu_0\vec{V}\rho
\end{flalign*}
che, con le assunzioni fin qui fatte, non è nullo in generale.\\

Si è quindi giunti alla conclusione che la teoria di Maxwell e la meccanica di Newton non sono conciliabili.
\subsection{I postulati di Einstein}\label{Sec:postulati}
Per giungere ad una formulazione coerente della dinamica dei corpi carichi Einstein, nel suo articolo del 1905$^{\cite{Einstein1905}}$, propose di modificare gli assunti alla base della meccanica classica per prediligere un modello coerente con l'elettromagnetismo di Maxwell.\\Infatti all'epoca erano noti alcuni risultati sperimentali che giocavano a sfavore della concezione classica della meccanica, primo fra tutti l'esperimento di Michelson e Morley che avrebbe dovuto consentire di misurare la velocità della Terra rispetto all'etere luminifero, detta vento d'etere. L'esperimento ebbe un esito inaspettato, infatti non fu possibile misurare alcun vento d'etere portando i fisici a tre possibili spiegazioni: o l'Etere si muove assieme alla Terra, o l'apparato sperimentale si contrae lungo la direzione del moto terrestre oppure non esiste alcun Etere e la luce si propaga alla medesima velocità in ogni direzione e per ogni osservatore.\\
\newpage
Einstein pose quindi a fondamento della sua teoria due postulati:
\begin{itemize}
    \item Il principio di relatività: basato sull'assunzione che esistano  una serie di sistemi di riferimento detti inerziali, reciprocamente in moto rettilineo uniforme, in cui le leggi della fisica sono identicamente valide.
    \item Il principio di costanza della velocità della luce: il quale asserisce che la luce nello spazio vuoto si propaghi sempre con modulo della velocità determinato ed identico per ogni osservatore inerziale, che si indicherà con $c$. 
\end{itemize}
Il primo si è lo stesso principio della meccanica classica mentre il secondo è una diretta conseguenza dell'esperimento di Michelson e Morley il cui risultato viene spiegato senza la necessità dell'introduzione dell'etere e di un sistema di riferimento privilegiato. Inoltre si continua ad intendere spazio e tempo come due enti omogenei e isotropi.\\

La primissima conseguenza dell'assunzione di questi due postulati è la non validità delle trasformazioni di Galileo. Infatti secondo queste un moto di velocità $\vec{v}$ in un sistema $K$, se osservato in un sistema $K'$, nel quale $K$ si muove a velocità $\vec{V}$, risulterà in un moto a velocità $\vec{v}+\vec{V}$. Il secondo postulato però richiede che se tale moto è di un fascio di luce questo debba risultare sia in $K$ che in $K'$ in un moto con modulo della velocità pari a $c$, in totale disaccordo con le trasformazioni di Galileo.\\

Inoltre nel suo articolo Einstein stesso propose, dopo aver enunciato i postulati, un esperimento mentale che consente di mostrare come questi siano in diretto conflitto con le assunzioni classiche dell'assolutezza del tempo e delle lunghezze. Si considerino due orologi reciprocamente a riposo posizionati in due punti detti $A$ e $B$. Einstein osservò che ogni orologio è in grado di misurare intervalli temporali solamente per eventi che avvengono nello stesso punto in cui ognuno è posizionato, questo poiché diventa necessario tener conto della velocità finita della luce che quindi non propagandosi istantaneamente genera dei ritardi nella percezione degli eventi lontani.\\ Il secondo postulato consente però di sincronizzare gli orologi così che sia possibile confrontare i tempi misurati in $A$ e in $B$. In primo luogo si ipotizzi di far partire all'istante $t_A=0$, misurato dal primo orologio, un fascio di luce che viaggia da $A$ e giunge in $B$ quando l'orologio posizionato in tale punto segna un tempo $t_B$. In $B$ il fascio è riflesso e fa ritorno in $A$ quando il relativo orologio segna un tempo $t_A$. Poiché per il principio di costanza della velocità della luce il fascio luminoso deve propagarsi in ogni direzione con la stessa velocità e la distanza tra i due orologi è fissata e costante allora il tempo impiegato dalla luce per andare da $A$ a $B$ e vice versa deve essere il medesimo. Si conclude quindi che i due orologi sono sincronizzati solamente se vale:
\begin{equation}
    2t_B=t_A.
    \label{SinconizazioneOrologi}
\end{equation}
Chiaramente se l'orologio nel punto $A$ si è rivelato sincrono con quello nel punto $B$, tramite la procedura appena descritta, è chiaro che è altrettanto vero che quello posto in $B$ è sincrono con quello posto in $A$ e che se si considera un orologio posto in un terzo punto $C$ che risulta sincrono con quello posto in $B$ allora questo terzo orologio è sincrono con quello nel punto $A$.\\Così facendo è possibile assegnare un immaginario orologio ad ogni punto dello spazio in maniera tale che siano tutti sincroni tra loro e sia possibile determinare quando due eventi lontani fra loro avvengono nello stesso istante in un determinato sistema di riferimento inerziale.\\
Fatta propria questa osservazione è possibile procedere analizzando l'esperimento mentale.
\begin{figure}[H]
    \centering
    \begin{tikzpicture}
        \draw[->,black] (0,0)node [anchor=north east]{$K$} -- (4,0) node[anchor=north ]{$x$};
        \draw[->,black] (0,0) -- (0,2) node[anchor= east]{$y$};
        \draw[fill=blue] (0.5,0) rectangle node[anchor=north,scale=0.8,white]{Regolo} (3,1);
        \draw[black] (0.5,-0.1) node[anchor=north]{$x_A$} -- (0.5,0.1);
        \draw[black] (3,-0.1) node[anchor=north]{$x_B$} -- (3,0.1);
        \draw[yellow,->,thick] (0.5,1.4) -- (3,1.4);
        \draw[yellow,<-,thick] (0.5,1.2) -- (3,1.2);

        \draw[->,black] (0+7,0)node [anchor=north east]{$K'$} -- (4.5+7,0) node[anchor=north ]{$x'$};
        \draw[->,black] (0+7,0) -- (0+7,2) node[anchor= east]{$y'$};
        \draw[fill=blue] (1.5+7,0) rectangle node[anchor=north,scale=0.8,white]{Regolo} (3.5+7,1);
        \draw[black] (1+7,-0.1) node[anchor=north]{$x_{A'}'$} -- (1+7,0.1);
        \draw[black] (1.5+7,-0.1) node[anchor=north west]{$x_{C'}'$} -- (1.5+7,0.1);
        \draw[black] (3+7,-0.1) node[anchor=north]{$x_{B'}'$} -- (3+7,0.1);
        \draw[yellow,->,thick] (1+7,1.4) -- (3+7,1.4);
        \draw[yellow,<-,thick] (1.5+7,1.2) -- (3+7,1.2);
        \draw[black,->,thick] (2.5+7,.5) -- (4+7,.5) node[anchor=west]{$\vec{v}$};
    \end{tikzpicture}
    \caption{Rappresentazione dell'esperimento mentale di Einstein.}
    \label{fig:esperMentale}
\end{figure}
Si consideri un regolo di lunghezza $l$ in un sistema di riferimento ad esso solidale detto $K$. Si considerino anche due orologi sincroni posti nelle due estremità del regolo dette $A$ e $B$, se si ripete quanto appena fatto per la sincronizzazione degli orologi si deve osservare che il tempo impiegato dalla luce per muoversi da un estremo ad un altro è il medesimo in entrambe le direzioni.\\ In un sistema $K'$ si osserva il regolo in moto a velocità $v$ lungo la direzione in cui la parte lunga del regolo poggia. Sia detto $A'$ il punto del sistema $K'$ in cui si osserva l'emissione del fascio di luce dalla prima estremità del regolo, $B'$ il punto, sempre in $K'$, in cui si osserva la riflessione del fascio nel secondo estremo del regolo ed infine $C'$ il punto in $K'$ in cui si osserva il fascio fare ritorno alla prima estremità. In ognuno dei tre punti è presente un orologio sincronizzato con gli altri del sistema $K'$ in maniera da osservare l'emissione del fascio ad un tempo $t'_{A'}=0$. Se $t'_{B'}$ è l'istante in cui si osserva la riflessione in $B'$, $t'_{C'}$ è l'istante in cui il fascio fa ritorno al primo estremo del regolo e $l'$ è la lunghezza del regolo misurata nel sistema $K'$ allora è possibile determinare in funzione di questi tempi le distanze tra i punti $A',\ B'$ e $C'$, infatti queste dipenderanno in parte dalla distanza percorsa del regolo e in parte dalla sua lunghezza:
\begin{flalign*}
    &\Delta x'_{A'B'}=x'_{B'}-x'_{A'}=v(t'_{B'}-t'_{A'})+l'=vt'_{B'}+l',\\
    &\Delta x'_{B'C'}=x'_{B'}-x'_{C'}=l'-v(t'_{C'}-t'_{B'}).
\end{flalign*}
Queste due distanze sono quelle percorse dal fascio luminoso rispettivamente in tempi $t'_{B'}$ e $t'_{C'}-t'_{B'}$, per cui si ottiene:
\begin{flalign*}
    \Delta x'_{A'B'}=ct'_{B'}=vt'_{B'}+l' \quad &\Rightarrow\qquad t'_{B'}=\frac{l'}{c-v},\\
    \Delta x'_{B'C'}=c(t'_{C'}-t'_{B'})=l'-v(t'_{C'}-t'_{B'}) \qquad &\Rightarrow\quad t'_{C'}-t'_{B'}=\frac{l'}{c+v}.
\end{flalign*}

Questa semplice osservazione consente di concludere che i postulati enunciati da Einstein non ammettono la possibilità di assumere tempi assoluti: infatti l'osservatore in $K'$ osserva che il fascio di luce impiega più tempo per giungere al secondo estremo di quanto ne trascorra tra la riflessione e il suo ritorno al primo estremo, mentre invece l'osservatore in $K$, solidale con il regolo, osserva tempi identici per questi due tragitti. Una volta sviluppata matematicamente questa nuova teoria della relatività si osserverà che in maniera analoga pure le lunghezze non possono più essere considerate assolute.

\section{Le trasformazioni di Lorentz}
Come si è già visto, i postulati di Einstein non risultano compatibili con le trasformazioni di Galilei, 
per questo motivo il primo passo compiuto da Einstein stesso fu quello di identificare matematicamente quali siano 
le nuove trasformazioni dei sistemi di riferimento inerziali che derivano direttamente dai nuovi postulati.\\
In questa sezione si procederà a ricavare le trasformazioni di Lorentz seguendo i ragionamenti di Fock \cite{Fock} e di Landau \cite{Landau}, 
per poi discuterne le implicazioni.  
\subsection{Derivazione}
\label{sec:DervTrasfLorentz}
Si vogliono trovare le trasformazioni che, rispettando i due postulati di Einstein, trasformano i vettori dello spazio-tempo $\mathbb{R}\times\mathbb{R}^3$  misurati in un sistema di riferimento inerziale $K$ in quelli misurati in un secondo sistema di riferimento inerziale $K'$. Il principio di relatività impone che tutte le leggi della fisica siano valide sia in $K$ che in $K'$ e in modo particolare il principio di costanza della velocità della luce. Si può richiedere questa condizione nel seguente modo: sia emesso rispetto a $K$ un segnale luminoso all'istante $t_0$ e nel punto $\vec{r}_0$, allora in un istante $t_1$ si osserverà il segnale in $\vec{r}_1$ tale che rispettando il secondo postulato, la propagazione avvenga a velocità con modulo costante e pari a $c$, così che:
\begin{equation}
    |\vec{r}_1-\vec{r}_0|^2=(x_1-x_0)^2+(y_1-y_0)^2+(z_1-z_0)^2=c^2(t_1-t_0)^2
    \label{luceK}
\end{equation}
e se in $K'$ si osserva lo stesso fascio di luce emesso all'istante $t_0'$ nel punto $\vec{r'}_0$ analogamente in un istante $t_1'$ il segnale giungerà in $\vec{r'}_1$ e per il principio di costanza della velocità della luce anche in questo sistema di riferimento varrà la seguente espressione.
\begin{equation}
    |\vec{r'}_1-\vec{r'}_0|^2=(x'_1-x'_0)^2+(y'_1-y'_0)^2+(z'_1-z'_0)^2=c^2(t'_1-t'_0)^2
    \label{luceK'}
\end{equation}
Il luogo dei punti in $\mathbb{R}\times\mathbb{R}^3$ che soddisfa queste relazioni è detto cono di luce ed è la rappresentazione spaziotemporale del moto luminare. I postulati di Einstein impongono che la trasformazione che si sta cercando trasformi sempre punti del cono di luce in $K$ in altri punti del cono di luce in $K'$.\\ 

Come si è visto nella Sezione \ref{Sec:postulati} gli istanti temporali in cui si verifica un preciso evento non sono più assoluti e diversi osservatori di diversi sistemi di riferimento potrebbero non concordare sulla loro misura. Per questo motivo è importate iniziare a ragionare utilizzando vettori appartenenti a $\mathbb{R}^4$, detti quadrivettori\footnote{Poiché con l'uso dei quadrivettori la coordinata temporale è considerata al pari di quelle spaziali si dirà da ora in poi che questi appartengono a $\mathbb{R}^4$ e non a $\mathbb{R}\times\mathbb{R}^3$, seppur queste due notazioni rappresentino lo stesso spazio vettoriale.}, così da considerare trasformazioni anche della coordinata temporale. Le relazioni (\ref{luceK}) e (\ref{luceK'}) suggeriscono di utilizzare come quadrivettore $r=(ct,x,y,z)$ poiché così facendo è possibile definire la norma quadra di un quadrivettore tramite il prodotto righe per colonne con una matrice $g$ detta matrice metrica
\begin{flalign*}
    g = \begin{pmatrix}
        1 & 0 & 0 & 0\\
        0 & -1 & 0 & 0\\
        0 & 0 & -1 & 0\\
        0 & 0 & 0 & -1
        \end{pmatrix}\quad
        \Rightarrow \quad |r|^2=(ct,x,y,z)\ g
        \begin{pmatrix}
            ct\\
            x\\
            y\\
            z
        \end{pmatrix}
        =c^2t^2-x^2-y^2-z^2
\end{flalign*}
e in tal modo si descriverà un quadrivettore rappresentante una traiettoria nello spazio-tempo di un fascio di luce indicando che la sua norma deve essere nulla.\\ Un secondo modo equivalente consiste nel considerare il quadrivettore $r=(ict,x,y,z)$, dove $i^2=-1$, il che consente di poter utilizzare il regolare prodotto scalare euclideo per definire la norma di un quadrivettore ottenendo la stessa espressione della precedente convenzione. Per passare da una convenzione all'altra è sufficiente far uso di un cambio di base $P$ che mappi i vettori della base canonica in una nuova base $\{e_0'=ie_0,\ e_1'=e_1,\ e_2'=e_2,\ e_3'=e_3\}$.
\begin{equation}
    P=\begin{pmatrix}
        i & 0 & 0 & 0\\
        0 & 1 & 0 & 0\\
        0 & 0 & 1 & 0\\
        0 & 0 & 0 & 1
        \end{pmatrix}
        \qquad \qquad \qquad
        P^{-1}=\begin{pmatrix}
            -i & 0 & 0 & 0\\
            0 & 1 & 0 & 0\\
            0 & 0 & 1 & 0\\
            0 & 0 & 0 & 1
            \end{pmatrix}
    \label{PiP}
\end{equation}\\

La trasformazione che si vuole trovare è quindi una trasformazione $f:\mathbb{R}^4\rightarrow\mathbb{R}^4$, invertibile e che trasforma quadrivettori con norma nulla in altri quadrivettori con norma nulla. L'invertibilità è necessaria affinché sia possibile trasformare sia da un sistema di riferimento $K$ ad uno $K'$ sia viceversa. Questa trasformazione deve essere una trasformazione affine affinché il principio di relatività si soddisfatto, come si è dimostrato nella Sezione \ref{sec:MathSDRI}; un'ulteriore dimostrazione, valida solamente per la relatività speciale, è fornita nell'appendice \ref{chap:LinearitàLorentz}.\\
La proprietà di mantenere la norma nulla, che è equivalente alla richiesta di soddisfare i postulati di Einstein, implica che  $|f(r)|^2=\lambda |r|^2$, con $\lambda$ costante, questo fatto algebrico è dimostrato nel lemma \ref{lemm:A2} dell'appendice \ref{chap:LinearitàLorentz}.\\ Si considerino $3$ sistemi di riferimento $ K, K_1$ e $K_2$, tali per cui in $K$ l'origine di $K_1$ risulti in moto a velocità $\vec{v}_1$ e l'origine di $K_2$ risulti in moto a velocità $\vec{v}_2$. Sia $r$ un quadrivettore in $K$ e $r_1$ e $r_2$ il medesimo quadrivettore rispettivamente in $K_1$ e $K_2$, è possibile considerare tre trasformazioni $f_1$, $f_2$ $f_{12}$, tali che $f_1(r_1)=f_2(r_2)=r$ e $f_{12}(r_1)=r_2$, per cui si avrà:
\begin{equation*}
    |r_1|^2=\lambda_1 |r|^2 \quad |r_2|^2=\lambda_2 |r|^2 \quad |r_1|^2=\lambda_{12} |r_2|^2 \quad  \Rightarrow |r_1|^2=\frac{\lambda_1}{\lambda_2}|r_2|^2=\lambda_{12}|r_2|^2.
\end{equation*}
Ogni costante $\lambda$ può dipendere esclusivamente dalla trasformazione considerata, ossia dalle velocità reciproche dei sistemi di riferimento tra cui questa agisce, infatti non è possibile ammettere una dipendenza da un qualche quadrivettore $r$ siccome questa violerebbe la proprietà di omogeneità dello spazio-tempo. Analogamente non è possibile supporre che $\lambda$ dipenda dalla direzione o dal verso delle velocità reciproche altrimenti sarebbe possibile definire una direzione preferenziale violando l'isotropia dello spazio-tempo. Si conclude che deve quindi valere la seguente relazione:
\begin{equation}
    \frac{\lambda(|\vec{v_1}|)}{\lambda(|\vec{v_2}|)}=\lambda(|\vec{v}_{12}|).
    \label{fraclambda}
\end{equation} 
Infine si osservi che la norma della velocità reciproca $|\vec{v}_{12}|$ dipenderà dalle direzioni in cui sono orientate le velocità $\vec{v}_1$ e $\vec{v}_2$, infatti se le due sono identiche la velocità reciproca dovrà essere nulla mentre se sono in norma uguale ma in direzioni differenti sarà possibile ottenere solamente velocità non nulle. Nella relazione (\ref{fraclambda}), appena ottenuta, $\lambda(|\vec{v}_{12}|)$ è quindi dipendente dalle direzioni di $\vec{v}_1$ e $\vec{v}_2$ mentre il rapporto $\frac{\lambda(|\vec{v_1}|)}{\lambda(|\vec{v_2}|)}$ permane di valore fissato al variare dell'angolo tra $\vec{v}_1$ e $\vec{v}_2$. Si conclude quindi che $\lambda$ deve assumere un valore costante indipendentemente dalla trasformazione considerata, inoltre dalla (\ref{fraclambda}) è immediato concludere che tale costante è proprio pari a $1$.\\

La trasformazione $f$ deve quindi anche conservare le norme dei vettori $r$ dello spazio tempo. Se si fa uso della convenzione per cui $r=(ict,x,y,z)$ si ottiene che il luogo dei punti con norma fissata $R$ è dato da una 4-sfera
\begin{equation}
    r^2=(ict)^2+x^2+y^2+z^2=R^2
    \label{4-sfera}
\end{equation} 
e questo insieme di punti permane una 4-sfera solo per rotazioni e traslazioni, per cui l'applicazione lineare della trasformazione affine che si sta cercando deve essere una rotazione. Questa generica rotazione può essere decomposta in rotazioni nei piani $xy,$ $xz,$ $yz,$ $xt,$ $yt$ e $yt$: le rotazioni nei primi tre piani corrispondono alle classiche rotazioni spaziali mentre sono le ultime tre quelle più peculiari.\\ Si consideri ora una rotazione nel piano $xt$, le altre due rotazioni sono analizzabili in maniera del tutto analoga, questa può essere facilmente espressa nella forma:
\begin{equation*}
   \begin{pmatrix}
    ict'\\x'\\y'\\z'
   \end{pmatrix}
   =\begin{pmatrix}
    \cos\theta & -\sin\theta & 0 & 0\\
    \sin\theta & \cos\theta & 0 & 0\\
    0& 0 & 1 & 0\\
    0& 0 & 0 & 1\\
   \end{pmatrix}
   \begin{pmatrix}
    ict\\x\\y\\z
   \end{pmatrix}\qquad
   \Rightarrow \qquad
   \begin{cases}
    t'=t\cos\theta+i\frac{x}{c}\sin\theta\\
    x'=ict\sin\theta+x\cos\theta\\
    y'=y\\
    z'=z
   \end{cases}
\end{equation*}
dove $t',x',y',z'$ sono le coordinate misurate in $K'$ e $t,x,y,z$ sono quelle misurate in $K$. Siccome si sta studiando la trasformazione tra due sistemi inerziali e quindi in moto rettilineo uniforme l'uno rispetto l'altro solamente nel piano $xt$ è naturale porre l'origine di $K'$ in moto a velocità costante $V$ lungo l'asse $x$, coincidente con $x'$ siccome non si sono effettuate rotazioni spaziali. Se si considera la trasformazione del punto che ha come immagine l'origine di $K'$ si ottiene:
\begin{equation*}
    \begin{cases}
        t'=t\cos\theta+i\frac{x}{c}\sin\theta\\
        0=ict\sin\theta+x\cos\theta\\
        0=y\\
        0=z
       \end{cases}
    \quad \Rightarrow \quad
    V=\frac{x}{t}=-ic\tan\theta \quad \Rightarrow \quad
    \begin{cases}
        \cos\theta=\frac{1}{\sqrt{1-\frac{V^2}{c^2}}}\\
        \sin\theta=\frac{i\frac{V}{c}}{\sqrt{1-\frac{V^2}{c^2}}}
    \end{cases}.
\end{equation*}
Definendo $\gamma=\frac{1}{\sqrt{1-\frac{V^2}{c^2}}}$ la trasformazione diventa:
\begin{equation*}
    \begin{pmatrix}
     ict'\\x'\\y'\\z'
    \end{pmatrix}
    =\begin{pmatrix}
     \gamma & -i\frac{V}{c}\gamma & 0 & 0\\
     i\frac{V}{c}\gamma & \gamma & 0 & 0\\
     0& 0 & 1 & 0\\
     0& 0 & 0 & 1\\
    \end{pmatrix}
    \begin{pmatrix}
     ict\\x\\y\\z
    \end{pmatrix}\qquad
    \Leftrightarrow   \qquad
    \begin{cases}
     t'=(t-\frac{V}{c^2}x)\gamma\\
     x'=(x-Vt)\gamma\\
     y'=y\\
     z'=z
    \end{cases}.
 \end{equation*}
 Siccome la convenzione $r=(ict,x,y,z)$ non è quella comunemente utilizzata ma è stata adoperata solamente per evidenziare il carattere geometrico della trasformazione, è necessario ricondursi alla convenzione consueta dove $r=(ct,x,y,z)$ e come si è già detto è necessario effettuare un cambio di base tramite le matrici della (\ref{PiP}):
 \begin{equation*}
    P^{-1}\begin{pmatrix}
        \gamma & -i\frac{V}{c}\gamma & 0 & 0\\
        i\frac{V}{c}\gamma & \gamma & 0 & 0\\
        0& 0 & 1 & 0\\
        0& 0 & 0 & 1\\
       \end{pmatrix}
       P=\begin{pmatrix}
        \gamma & -\frac{V}{c}\gamma & 0 & 0\\
        -\frac{V}{c}\gamma & \gamma & 0 & 0\\
        0& 0 & 1 & 0\\
        0& 0 & 0 & 1\\
       \end{pmatrix}.
 \end{equation*}
 Questa è convenzionalmente riconosciuta come la trasformazione di Lorentz\footnote{Una trasformazione di Lorentz di questo tipo è più propriamente detta Boost di Lorentz.} $\Lambda$ da $K$ a $K'$ con $K'$ in moto a velocità $v$ lungo l'asse $x$ rispetto a $K$:
 \begin{equation}
    \Lambda=
    \begin{pmatrix}
        \gamma & -\frac{V}{c}\gamma & 0 & 0\\
        -\frac{V}{c}\gamma & \gamma & 0 & 0\\
        0& 0 & 1 & 0\\
        0& 0 & 0 & 1\\
       \end{pmatrix}
       \qquad
       \begin{cases}
        t'=(t-\frac{V}{c^2}x)\gamma\\
        x'=(x-Vt)\gamma\\
        y'=y\\
        z'=z
       \end{cases}
       \qquad \gamma=\frac{1}{\sqrt{1-\frac{V^2}{c^2}}}.
       \label{TrasformazioneLorentz}
 \end{equation}
Come si è già detto la trasformazione inversa di $\Lambda$, che consente di trasformare i vettori di $K'$ in quelli di $K$, è ricavabile invertendo la matrice appena ottenuta.\\

Analoghe considerazioni possono essere fare per sistemi in moto in direzioni differenti a qui corrisponderanno trasformazioni descritte da rotazioni di piani differenti. Inoltre è possibile comporre tutte le trasformazioni di Lorentz tramite il prodotto righe per colonne con altre trasformazioni di Lorentz o rotazioni spaziali così da ottenere arbitrarie trasformazioni dei sistemi di riferimento inerziali.\\

Si osservi che in tutte queste trasformazioni appare un fattore che si è chiamato $\gamma$, caratteristico della trasformazione che si sta considerando. Questo fattore è dipendente esclusivamente dal modulo della velocità reciproca dei due sistemi di riferimento e assume un comportamento caratteristico e fondamentale per la teoria della relatività: in generale $\gamma\geq 1$ ed è pari a $1$ solo per $V=0$ mentre tende ad $\infty$ per $V$ che tendono a $\pm c$.\\
Per $|V|>c$ il fattore $\gamma\in\mathbb{C}$, in questo modo ipotetici sistemi di riferimento in moto a velocità superluminare darebbero origine a trasformazioni che includono coordinate complesse e quindi di senso non fisico, questo suggerisce, come si osserverà più avanti, che $c$ costituisca la velocità limite del moto.\\
Per piccoli valori di $V$ rispetto a $c$, ossia nel limite in cui $c$ è infinitamente grande, detto limite classico poiché coincide con la descrizione classica secondo cui la luce si propaga istantaneamente:
\begin{equation}
    \gamma=\frac{1}{\sqrt{1-\frac{V^2}{c^2}}}=1+\frac12\frac{V^2}{c^2}+\frac38\frac{V^4}{c^4}+o\bigg(\frac{V^4}{c^4}\bigg).
    \label{limiteClassicoGamma}
\end{equation}
Così facendo, considerando $c\rightarrow\infty$ e la (\ref{limiteClassicoGamma}) al prim'ordine, le trasformazioni di Lorentz divengono quelle di Galilei, per questo motivo la meccanica classica risulta solamente una prima approssimazione della corretta descrizione delle realtà.


\subsection{Dilatazione dei tempi e contrazione delle lunghezze}\label{sec:ContrazioneDilatazione}
Come si è già visto il concetto di tempo assoluto non è conciliabile con i postulati di Einstein, questo fatto è ora deducibile dalle Trasformazioni di Lorentz (\ref{TrasformazioneLorentz}).\\
Si consideri in un sistema di riferimento inerziale $K$ il moto rettilineo ed uniforme a velocità $V$ di un corpo che emette un segnale luminoso ogni $\Delta t$. Considerando un secondo sistema di riferimento inerziale $K'$ solidale con tale corpo e tale da osservarlo nell'origine degli assi spaziali, allora le trasformazioni di Lorentz risulteranno:
\begin{equation*}
    \begin{cases}
        t=(t'+\frac{V}{c^2}x')\gamma\\
        x=(x'+Vt')\gamma\\
        y=y'\\
        z=z'
    \end{cases}
    \Longleftrightarrow \quad
    \begin{cases}
        t'=(t-\frac{V}{c^2}x)\gamma\\
        x'=(x-Vt)\gamma\\
        y'=y\\
        z'=z
    \end{cases}
    \qquad \gamma=\frac{1}{\sqrt{1-\frac{V^2}{c^2}}}.
\end{equation*}
Si possono quindi correlare i tempi di emissione misurati in $K$ con quelli in $K'$ considerando la trasformazione di Lorentz rispetto al corpo posto nell'origine del sistema $K'$ (r'=(ct',0,0,0)): 
\begin{equation}
    \begin{cases}
        t=t'\gamma\\
        x=Vt'\gamma\\
        y=0\\
        z=0
    \end{cases}
    \Rightarrow \Delta \tau=t'_1-t'_0=\frac{(t_1-t_0)}{\gamma}=\frac{\Delta t}{\gamma}=\Delta t \sqrt{1-\frac{V^2}{c^2}}.
    \label{dilatazioneTempi}
\end{equation}
L'intervallo di tempo $\Delta \tau$, misurato nel sistema $K'$ solidale al corpo, è detto tempo proprio di tale corpo e, considerando che $\gamma$ assume valori maggiori di $1$ per ogni $V\neq 0$, risulta l'intervallo di tempo tra due eventi più breve (tra tutti gli intervalli $\Delta t$ misurati in ogni sistema di riferimento inerziale).\\

Si consideri adesso un parallelepipedo esteso di dimensioni $l_1\ l_2\ l_3$ rispettivamente lungo gli assi $x,\ y,\ z$ del sistema $K$ e $\lambda_1\ \lambda_2\ \lambda_3$ rispettivamente lungo gli assi $x',\ y',\ z'$ del sistema $K'$. Operativamente le dimensioni dell'oggetto sono misurate ponendo dei regoli diretti lungo gli assi di ogni sistema di riferimento e non in moto in essi, quindi ad un determinato istante si misurano contemporaneamente le posizioni delle estremità del corpo utilizzando i regoli. Si osservi che è necessario effettuare misure contemporanee della posizione di ogni coppia di estremità per assicurarsi di non misurare anche una componente dovuta allo spostamento del corpo avvenuto tra le misure non contemporanee. Si effettui la misura come è appena stata descritta nell'istante $t=0$, si ottene, facendo uso delle trasformazioni di Lorentz: 
\begin{equation}
    \begin{cases}
        0=(t'+\frac{V}{c^2}\lambda_1)\gamma\\
        l_1=(\lambda_1+Vt')\gamma\\
        l_2=\lambda_2\\
        l_3=\lambda_3
    \end{cases}
    \Rightarrow
    \begin{cases}
        t'=-\frac{V}{c^2}\lambda_1\\
        l_1=(1-\frac{V^2}{c^2})\lambda_1\gamma\\
        l_2=\lambda_2\\
        l_3=\lambda_3
    \end{cases}
    \Rightarrow
    \begin{cases}
        \lambda_1=l_1\gamma=\frac{l_1}{\sqrt{1-\frac{V^2}{c^2}}}\\
        \lambda_2=l_2\\
        \lambda_3=l_3
    \end{cases}.
    \label{contrazioneLunghezze}
\end{equation}   
 Si osserva quindi che anche le lunghezze non possono più essere considerate assolute. La lunghezza di un corpo misurata nella direzione del suo moto risulta contratta rispetto alla medesima lunghezza misurata nel sistema $K'$ solidale al corpo, detta lunghezza propria. Inoltre, analogamente a come si è osservato per il tempo proprio, si ha che $\gamma > 1$, per ogni $V\neq0$, da cui si deduce che la lunghezza propria è la lunghezza maggiore tra tutte quelle misurabili in differenti sistemi di riferimento inerziali.\\
 
 Per concludere, è possibile constatare che il fenomeno della contrazione delle lunghezze influenza anche il volume di un corpo, nel caso del parallelepipedo sopra considerato il volume $\mathcal{V}_0$ misurato in $K'$ diminuirà dello stesso fattore della lunghezza $\lambda_1$ se misurato in $K$:
 \begin{equation}
    V=l_1\ l_2\ l_3=\frac{\lambda_1\ \lambda_2\ \lambda_3}{\gamma}=\frac{V_0}{\gamma}.
    \label{contrazioneVolumi}
 \end{equation}
 Quest'ultima osservazione mette in luce un'ulteriore differenza tra la teoria che si sta descrivendo e la teoria classica: infatti, poiché il volume di un corpo può mutare in base al suo moto, nella teoria della relatività non è possibile fare uso del così detto corpo rigido.

\input{Ch1/Trasformazioni di Lorentz/Trasfromazione Velocità}

\section{Quadrivettori e spazio-tempo}
\label{sec:4-vettori}

Nella sezione \ref{sec:DervTrasfLorentz} è stato introdotto il concetto di 4-vettore (quadrivettore) tramite il 4-vettore posizione, successivamente si è osservato che è possibile costruire altri 4-vettori, come la 4-velocità, che si trasformano anch'essi con le trasformazioni di Lorentz. Volendo essere più precisi si chiameranno 4-vettori tutti i vettori di $\mathbb{R}^4$ che si trasformano come il 4-vettore posizione.\\ 
La struttura non più euclidea dello spazio-tempo richiede una maggior attenzione nella trattazione dei 4-vettori: è quindi necessario distinguere 4-vettori controvarianti, indicati con $A^{\mu}=(A^0,A^1,A^2,A^3)$, e covarianti, indicati con $A_{\mu}=(A_0,A_1,A_2,A_3)$. I primi si trasformano con le trasformazioni di Lorentz $\Lambda_\nu^\mu$ (\ref{TrasformazioneLorentz}), mentre i secondi con la trasformazione inversa e trasposta, così che utilizzando la convenzione di Einstein\footnote{La convenzione di Einstein è una notazione abbreviata dell'operazione di sommatoria: ogni coppia di indici ripetuti corrisponde ad una sommatoria sottintesa su tale indice. Un esempio può essere: $\sum_k A_{i,k}B^{j,k}= A_{i,k}B^{j,k}$} tali trasformazioni si scrivono:
\begin{equation}
    A'^\mu =\Lambda_\nu^\mu A^\nu \qquad \qquad A_\mu '=(\Lambda^{-1} )^\nu_\mu A_\nu.
\end{equation}
Utilizzando 4-vettori covarianti e controvarianti è possibile ottenere una quantità scalare invariante per le trasformazioni di Lorentz:
\begin{eqnarray*}
    A'^\mu B'_\mu=\Lambda_\alpha^\mu A^\alpha (\Lambda^{-1} )^\beta_\mu B_\beta=A^\alpha \delta_\alpha^\beta B_\beta=A^\alpha B_\alpha
\end{eqnarray*}
dove $ \delta_\alpha^\beta$ è la delta di Kronecker data dal prodotto righe per colonne della matrice della trasformazione di Lorentz con la sua inversa. La quantità appena ottenuta è il prodotto scalare dei 4-vettori $A$ e $B$, questo è esprimibile, come si è visto nella sezione \ref{sec:DervTrasfLorentz}, attraverso l'uso della matrice metrica $g_{\mu \nu}$, questa osservazione consente di trovare una relazione tra componenti controvarianti e covarianti. Fissato un 4-vettore $A_\mu$ allora $\forall B^\nu$: 
\begin{equation}
    B^\mu A_\mu=B^\mu g_{\mu\nu} A^\nu \quad \Rightarrow\quad A_\mu=g_{\mu\nu}A^\nu \ \Leftrightarrow\ A^0=A_0,\ A^i=-A_i \ \ i\in\{1,2,3\}.
\end{equation}

Si consideri ora il 4-vettore posizione $x_\mu$, come si è già visto il luogo dei punti nei quali $x^\mu x_\mu=0$ rappresenta moti che avvengono alla velocità della luce ed è detto cono di luce. In analogia con il 4-vettore posizione ed il cono di luce (Fig. \ref{fig:conoLuce}) è possibile classificare un qualsiasi 4-vettore in base al suo modulo quadro:
\begin{multicols}{2}
    \begin{figure}[H]
        \centering
        \begin{tikzpicture}
            \draw[->,black,thick] (0,-0.5) -- (0,1.5) node[anchor=east]{$ct$};
            \draw[->,black,thick] (-2,0) -- (2,0) node[anchor=west]{$\mathbb{R}^3$};
            \draw[dashed,gray,thick] (-0.5,-0.5) -- (1.5,1.5)node[anchor=west, black]{$(ct)^2=|\vec{x}|^2$};
            \draw[dashed,gray,thick] (0.5,-0.5) -- (-1.5,1.5);
            \draw[->,black] (0,0) -- (-0.3,1) node[anchor=south east, scale=.5,fill=white]{Tipo tempo};
            \draw[->,black] (0,0) -- (1,1) node[anchor=north west, scale=.5]{Tipo luce};
            \draw[->,black] (0,0) -- (1,0.2) node[anchor=south west, scale=.5,fill=white]{Tipo spazio};
        \end{tikzpicture}
        \caption{Cono di luce nel piano}
        \label{fig:conoLuce}
    \end{figure}

    \begin{flalign*}
        A^\mu A_\mu=(A^0)^2-|\vec{A}|^2
        \begin{cases}
            >0\ \ \text{Tipo tempo}\\
            =0\ \ \text{Tipo luce}\\
            <0\ \ \text{Tipo spazio}\\
        \end{cases}
    \end{flalign*}

\end{multicols}
Poiché come è già stato anticipato $c$ rappresenta un limite per le velocità allora si può enunciare il seguente principio: \emph{dall'origine di un sistema di riferimento l'informazione non può raggiungere 4-vettori posizione di tipo spazio, ossia al di fuori del cono di luce}.\\

Il concetto di 4-vettore può essere esteso ad oggetti a più indici che si trasformano con le trasformazioni di Lorentz detti 4-tensori, come la matrice metrica o la delta di Kronecker. Anche le componenti di questi oggetti possono essere covarianti o controvarianti, in base a come queste si trasformino, si fa quindi uso della medesima notazioni con indici bassi e alti tenendo conto che però un 4-tensore può avere anche indici misti.\\

Infine si osservi che prendendo un campo scalare $\Phi(t,x,y,z)$ e calcolandone il 4-gradiente $\partial_\mu\Phi=(\frac{1}{c}\frac{\partial\Phi}{\partial t},\frac{\partial\Phi}{\partial x^1},\frac{\partial\Phi}{\partial x^2},\frac{\partial\Phi}{\partial x^3})$ questo risulta essere un 4-vettore in coordinate covarianti, infatti secondo la regola di Leibniz:
\begin{equation*}
    \partial'_\mu\Phi=\partial_\nu\Phi\ \frac{\partial x^\nu}{\partial x'^\mu}=\partial_\nu\Phi\ (\Lambda^{-1})_\mu^\nu.
\end{equation*}
Analogamente è possibile calcolare la 4-divergenza di un 4-vettore, questa però risulta uno scalare Lorentz invariante infatti:
\begin{equation*}
    \partial'_\mu A'^\mu=\frac{\partial A^\delta}{\partial x^\nu} \frac{\partial x^\nu}{\partial x'^\mu}\Lambda^\mu_\delta=\frac{\partial A^\delta}{\partial x^\nu}(\Lambda^{-1})_\mu^\nu \Lambda^\mu_\delta=\partial_\nu A^\nu=\frac{1}{c}\frac{\partial A^0}{\partial t}+\vnabla\cdot \vec{A}.
\end{equation*} 
Considerazioni di questo tipo possono essere fatte per altri operatori di differenziazione che possono anche agire su 4-tensori risultanti quindi in altri 4-tensori o in scalari, questo poiché sia gli operatori di differenziazione, tramite la regola di Leibniz, sia i 4-tensori, per definizione, si trasformano con $\Lambda$\footnote{Questa osservazione è valida solamente in relatività speciale dove la trasformazione delle coordinate è lineare, in relatività generale infatti non è più vero che la derivata di un 4-tensore è ancora un 4-tensore o che la 4-divergenza è uno scalare.}. La notazione ad indici alti e bassi diventa quindi utile per evidenziare come si possano costruire quantità tensoriali o scalari invarianti sfruttando quanto appena detto: infatti per esempio la notazione $\partial_\mu A^\mu$ ricorda il prodotto scalare, che come si è visto da luogo ad un invariante di Lorentz come avviene anche per la 4-divergenza.

\section{Trasformazioni del campo elettromagnetico}
\label{sec:trasfEM}
Nella Sezione \ref{Sec:nonInvMax}, si è visto come le equazioni di Maxwell (Sezione \ref{sec:EquazioniMaxwell}) non risultino Galilei invarianti, si procederà ora a derivare le trasformazioni relativistiche del campo elettrico e magnetico in maniera analoga a come fece Einstein nel suo articolo del 1905$^{\cite{Einstein1905}}$.\\

Per prima cosa è opportuno ricavare le trasformazioni degli operatori di differenziazione considerando le trasformazioni di Lorentz (\ref{TrasformazioneLorentz}) tra due sistemi $K$ e $K'$ in moto reciproco a velocità $V$. Usando la regola della derivata di funzione composta si ottiene:
\begin{equation}
    \frac{\partial}{\partial x}=\gamma\frac{\partial}{\partial x'}-\gamma \frac{V}{c^2}\frac{\partial}{\partial t'},\quad \frac{\partial}{\partial y}=\frac{\partial}{\partial y'},\quad \frac{\partial}{\partial z}=\frac{\partial}{\partial z'},\quad \frac{\partial}{\partial t}=\gamma \frac{\partial}{\partial t'}-\gamma V\frac{\partial}{\partial x'}.
    \label{trasfLorentzDiff}
\end{equation}
Siccome la relatività presuppone che la teoria di Maxwell sia corretta, è naturale partire proprio da questa, infatti si considereranno  due delle quattro equazioni di Maxwell:
\begin{equation}
    \vnabla \wedge \vec{E}=-\frac{\partial \vec{B}}{\partial t}, \qquad \vnabla\cdot\vec{B}=0.\label{maxwellMaSolo2}
\end{equation}
Se si studia la prima trasformando gli operatori di derivazione, componente per componente, secondo le (\ref{trasfLorentzDiff}), si può dedurre come debba trasformarsi il campo elettromagnetico.
\begin{flalign}
    &\frac{\partial E_z}{\partial y}-\frac{\partial E_y}{\partial z}=-\frac{\partial B_x}{\partial t}\ &\quad  \frac{\partial E_z}{\partial y'}-\frac{\partial E_y}{\partial z'}&=-\bigg(\frac{\partial B_x}{\partial t'}-V\frac{\partial B_x}{\partial x'}\bigg)\gamma&&\label{Maxwell3comp1trasf}\\
    &\frac{\partial E_x}{\partial z}-\frac{\partial E_z}{\partial z}=-\frac{\partial B_y}{\partial t}\ &\Rightarrow\quad  \frac{\partial E_x}{\partial z'}-\frac{\partial E_z}{\partial x'}+\gamma\frac{V}{c^2}\frac{\partial E_z}{\partial t'}&=-\bigg(\frac{\partial B_y}{\partial t'}-V\frac{\partial B_y}{\partial x'}\bigg)\gamma&&\\
    &\frac{\partial E_y}{\partial x}-\frac{\partial E_x}{\partial y}=-\frac{\partial B_z}{\partial t}\ &\quad  \frac{\partial E_y}{\partial x'}-\gamma\frac{V}{c^2}\frac{\partial E_y}{\partial t'}-\frac{\partial E_x}{\partial y'}&=-\bigg(\frac{\partial B_z}{\partial t'}-V\frac{\partial B_z}{\partial x'}\bigg)\gamma.&&
\end{flalign}
Infatti, siccome le equazioni di Maxwell devono valere sia in $K$ che in $K'$, è possibile identificare quali termini devono corrispondere alle coordinate di $\vec{E'}$ e $\vec{B'}$ nelle equazioni trasformate affinché queste si riducano a nuove equazioni di Maxwell:
\begin{flalign*}
    &\frac{\partial E_x}{\partial z'}-\frac{\partial }{\partial x'}\bigg[\gamma\bigg(E_z+VB_y\bigg)\bigg]=-\frac{\partial }{\partial t'}\bigg[\gamma\bigg(B_y+\frac{V}{c^2}E_z\bigg)\bigg],\\
    &\frac{\partial }{\partial x'}\bigg[\gamma\bigg(E_y-VB_z\bigg)\bigg]-\frac{\partial E_x}{\partial y'}=-\frac{\partial }{\partial t'}\bigg[\gamma\bigg(B_z-\frac{V}{c^2}E_y\bigg)\bigg].
\end{flalign*}
Così facendo si ottengono le trasformazioni di tutte le componenti tranne che per $B_x$:
\begin{flalign}
   & E'_{x'}=E_x,\qquad&E'_{y'}=(E_y-VB_z)\gamma,\qquad &E'_{z'}=(E_z+VB_y)\gamma,&&\nonumber\\
   & &B'_{y'}=(B_y+\frac{V}{c^2}E_z)\gamma,\qquad &B'_{z'}=(B_z-\frac{V}{c^2}E_y)\gamma.&&\nonumber
\end{flalign}
Per conoscere come si trasforma $B_x$ si sostituiscano le espressioni di $E_y$ e $E_z$, ricavabili dalle trasformazioni appena ottenute, nella (\ref{Maxwell3comp1trasf}) così che utilizzando la (\ref{maxwellMaSolo2}) e il procedimento precedentemente adoperato si ottiene
\begin{equation*}
    \frac{\partial E'_z}{\partial y'}-\frac{\partial E'_y}{\partial z'}=V(\vnabla'\cdot\vec{B'})-\frac{\partial B_x}{\partial t'}=-\frac{\partial B_x}{\partial t'}\qquad\Rightarrow\qquad B'_{x'}=B_x.
\end{equation*}
Queste trasformazioni, essendo ricavate dall'uso combinato della relatività e dell'elettromagnetismo di Maxwell, sono quindi coerenti con entrambe. \\

Si osservi che nel limite classico (dove $c$ è infinitamente grande) le trasformazioni ottenute divengono quelle ricavate nella Sezione \ref{Sec:nonInvMax} dalle prescrizioni della meccanica classica.

\chapter{Meccanica relativistica}
\section{Il principio di minima azione}
\label{sec:MinimaAzione}
La meccanica classica, brevemente introdotta nel capitolo \ref{sec:MC}, descrive il moto di sistemi di corpi tramite l'equazione di Newton. Si può dimostrare che questa però è del tutto equivalente ad un principio più generale noto come \emph{principio di minima azione}\footnote{Questo principio è esposto da Landau nella sua trattazione della meccanica classica \cite{Landau1}.}.\\ Questo principio assume che per ogni sistema meccanico sia possibile costruire una funzione $\mathcal{L}(q_1...q_n,\dot{q}_1...\dot{q}_n,t)$ detta \emph{lagrangiana} (dove $q_1...q_n$ sono le coordinate generalizzate e $\dot{q}_1...\dot{q}_n$ sono le velocità generalizzate date dagli $n$ gradi di libertà del sistema, che è uso comune indicare solamente con $q$ e $\dot{q}$) che lo descrive.\\ In questo modo il principio di minima azione afferma che: se si suppone che ad un istante $t_1$ iniziale e ad un istante finale $t_2$ il sistema si trovi a coordinate fissate l'evoluzione del sistema avverrà secondo un moto $(q(t),\dot{q}(t))$ tale che il funzionale 
\begin{equation}
    \label{Azione}
    \mathcal{S}[q(t)] =\int_{t_1}^{t_2} \mathcal{L}(q(t),\dot{q}(t),t)\ dt,
\end{equation}
detto \emph{azione}, assuma un valore stazionario, Nella fattispecie si può dimostrare che per intervalli sufficientemente piccoli assume il valore minimo.\\

Si vuole ora, dato questo principio, ottenere le equazioni del moto: si consideri una funzione $q(t)=(q_1(t),...,q_n(t))$ e una seconda funzione arbitraria $\delta q(t)=(\delta q_1(t),...,\delta q_n(t))$, detta variazione di $q(t)$, tale che $\delta q (t_1)=\delta q(t_2)=0$ e che assuma valori piccoli nel suo dominio $[t_1,t_2]$. In questo modo si può costruire una seconda funzione $q(t)+\delta q(t)$ che soddisfa ancora le ipotesi per cui devono essere fissate le coordinate agli istanti $t_1$ e $ t_2$. 
La variazione $\delta S$, data dalla variazione $\delta q$, deve risultare nulla affinché $q(t)$ sia la funzione che rende stazionaria l'azione. Il principio di minima azione è quindi equivalente a richiedere:
\begin{equation}
    \int_{t_1}^{t_2} \mathcal{L}(q+\delta q,\dot{q}+\delta \dot{q},t)\ dt-\int_{t_1}^{t_2} \mathcal{L}(q,\dot{q},t)\ dt=\int_{t_1}^{t_2}\sum_{i=1}^{n}\bigg[\frac{\partial\mathcal{L} }{\partial q_i}\delta q_i+\frac{\partial\mathcal{L} }{\partial \dot{q}_i}\delta \dot{q}_i \bigg]\ dt=0,
\end{equation}
nella quale si è espressa esplicitamente la variazione dell'azione rispetto alla variazione $\delta q$ e alla sua derivata prima $\delta \dot{q}$ tramite uno sviluppo di Taylor al prim'ordine con le derivate parziali della lagrangiana $\frac{\partial\mathcal{L} }{\partial q_i},\ \frac{\partial\mathcal{L} }{\partial \dot{q}_i}$.\\
Integrando per parti gli addendi contenenti $\delta \dot{q}_i=\frac{d}{dt}\delta q_i$ e ricordando che $\delta q (t)$ si annulla in $t_1$ e in $t_2$, si ottiene
\begin{equation*}
    \int_{t_1}^{t_2}\sum_{i=1}^{n}\bigg[\frac{\partial\mathcal{L} }{\partial q_i}\delta q_i-\frac{d}{dt}\frac{\partial\mathcal{L} }{\partial \dot{q}_i}\delta q_i \bigg]\ dt+\sum_{i=1}^n\frac{\partial\mathcal{L} }{\partial \dot q_i}\delta \dot{q}_i\bigg|_{t_1}^{t_2}=\int_{t_1}^{t_2}\sum_{i=1}^{n}\bigg[\frac{\partial\mathcal{L} }{\partial q_i}-\frac{d}{dt}\frac{\partial\mathcal{L} }{\partial \dot{q}_i} \bigg]\delta q_i\ dt=0.
\end{equation*}
Poiché questo integrale deve annullarsi per ogni $\delta q$ che rispetti le ipotesi fin qui fatte è necessario che si annulli identicamente l'integranda. Nella fattispecie ogni addendo singolarmente (siccome ognuno di questi è moltiplicato per una componente di $\delta q(t)$ che è una funzione arbitraria).\\Si ottengono quindi le \emph{equazioni di Eulero-Lagrange} che consentono di determinare $q(t)$:
\begin{equation}
    \label{eulero-lagrange}
    \frac{d}{dt}\frac{\partial\mathcal{L} }{\partial \dot{q}_i}-\frac{\partial\mathcal{L} }{\partial q_i}=0 \qquad \forall i\in{1,2,...,n}.
\end{equation}

Date le coordinate generalizzate $q(t)=(q_1(t),...,q_n(t))$ si definisco i momenti generalizzati associati a tali coordinate:
\begin{equation}
    p_{q_i}=\frac{\partial\mathcal{L} }{\partial \dot{q}_i}.
    \label{DefMomentoGeneralizzato}
\end{equation}
Se la lagrangiana non dipende esplicitamente da una coordinata $q_i$ allora si dice che questa è una coordinata ciclica. Dalle equazioni di Eulero-Lagrange segue che sulla curva del moto il momento associato a tale coordinata resta costante nel tempo.\\
Oltre ai momenti è utile definire l'energia meccanica del sistema:
\begin{equation}
    E=\sum_i \frac{\partial\mathcal{L} }{\partial \dot{q}_i}\dot{q}_i-\mathcal{L}.
    \label{DefEnergiaMeccanica}
\end{equation}
Se $\mathcal{L}$ non dipende esplicitamente dal tempo $E$ risulta una quantità conservata durante il moto. Infatti derivando rispetto al tempo la lagrangiana sulla curva del moto si ha:
\begin{flalign*}
    \frac{d \mathcal{L} }{dt}\bigg|_{q(t)}&=\sum_i\frac{\partial\mathcal{L} }{\partial q_i}\dot{q}_i+\sum_i\frac{\partial\mathcal{L} }{\partial \dot{q}_i}\ddot{q}_i+\frac{\partial\mathcal{L} }{\partial t}=\sum_i\bigg(\frac{d}{dt}\frac{\partial\mathcal{L} }{\partial \dot{q_i}}\bigg)\dot{q}_i+\sum_ip_i\ddot{q}_i+\frac{\partial\mathcal{L} }{\partial t}&&\\&=\sum_i\dot{p_i}\dot{q}_i+\sum_ip_i\ddot{q}_i+\frac{\partial\mathcal{L} }{\partial t}=\sum_i\frac{d}{dt}(p_i\dot{q_i})+\frac{\partial\mathcal{L} }{\partial t}\qquad \Rightarrow\qquad \frac{dE}{dt}\bigg|_{q(t)}=-\frac{\partial\mathcal{L} }{\partial t}.&&
\end{flalign*}

L'azione può essere intesa anche come funzione delle coordinate e del tempo. In questo caso a differenza del funzionale (\ref{Azione}) si mantengono libere le coordinate e gli istanti finali, che divengono le variabili di dipendenza dell'azione, e si calcola l'integrale lungo la curva del moto. Così facendo l'azione è la funzione
\begin{equation}
    S (\tilde q,t_2)=\int_{t_1}^{\tilde q,t_2} \mathcal{L} (q,\dot q,t)\ dt.
\end{equation}
Questa è nota come \emph{funzione d'azione} ed è indicata con la lettera $S$ (per distinguerla dal funzionale d'azione $\mathcal{S} $).\\
Si consideri ora una variazione delle coordinate finale $\tilde q_i\rightarrow \tilde q_i+\delta \tilde q_i$. Poiché ogni coppia di coordinate iniziali e finali determina (tramite il principio di minima azione) la traiettoria del moto, a $\delta \tilde q$ corrisponde una variazione $\delta q(t)$ di tale curva che mantiene invariate solamente le coordinate nell'istante iniziale. La variazione della funzione d'azione si calcola quindi come per il funzionale azione e si ottiene:
\begin{equation*}
    \delta S=\int_{t_1}^{t_2}\sum_{i=1}^{n}\bigg[\frac{\partial\mathcal{L} }{\partial \dot  q_i}\delta q_i-\frac{d}{dt}\frac{\partial\mathcal{L} }{\partial \dot{q}_i}\delta q_i \bigg]\ dt+\sum_{i=1}^n+\frac{\partial\mathcal{L} }{\partial \dot q_i}\delta q_i\bigg|_{t_1}^{t_2}.
\end{equation*}
Ricordando che questo integrale è calcolato su curve del moto si ha che l'integrando deve annullarsi (poiché devono essere soddisfatte le equazioni di Eulero-Lagrange). Mentre, come si appena osservato, $\delta q_i(t_1)=0$ ma $\delta q_i(t_2)=\delta \tilde q\neq0$, per cui si deduce che:
\begin{equation*}
    \delta S=\sum_{i=1}^n\frac{\partial\mathcal{L} }{\partial \dot q_i}\delta q_i\bigg|_{t_2}\ \Rightarrow\ \frac{\partial S}{\partial \tilde q_i}= \frac{\partial\mathcal{L} }{\partial \dot q_i}=p_{q_i}.
\end{equation*}
Se ora si considera la derivata totale rispetto al tempo di $S(q,t)$, per il teorema fondamentale del calcolo integrale, si ha che questa è proprio la lagrangiana $\mathcal{L} $ e quindi:
\begin{equation*}
    \frac{dS}{dt}=\sum_{i=1}^n\frac{\partial S }{\partial  q_i}\dot{q_i}+\frac{\partial S }{\partial t}=\mathcal{L}\quad \Rightarrow\quad \frac{\partial S }{\partial t}=\mathcal{L} -\sum_{i=1}^n\frac{\partial S }{\partial  q_i}\dot{q_i}=-E.
\end{equation*} 
Si osservi che quanto appena descritto necessita solamente del principio di minima azione e di nessun'altra assunzione sperimentale. Per questo motivo questa trattazione della meccanica è utilizzabile anche nell'ambito della relatività dove sarà la lagrangiana a dover tenere conto di quanto descritto nel Capitolo 1.
\section{La lagrangiana relativistica}\
Si procederà in questa sezione ricavando la lagrangiana relativistica e quanto segue dalla forma di questa. Siccome una generica lagrangiana è data dalla somma di una parte di corpo libero $\mathcal{L}_{lib}$ con una seconda di interazione $\mathcal{L}_{int}$ è opportuno in primis dedurre la forma di $\mathcal{L}_{lib}$ studiando una particella libera.
\subsection{La particella libera}
Si consideri il moto di una particella nello spazio tempo, il principio di minima azione determina tale moto indipendentemente dalla scelta del sistema di riferimento in cui esso è descritto.
Come si è già visto nella sezione \ref{sec:DervTrasfLorentz} le trasformazioni di Lorentz conservano le lunghezze nello spazio-tempo, per questo motivo tale lunghezza per una curva del moto è un candidato ad essere l'azione relativistica. Una curva di questo tipo può essere parametrizzata dal tempo misurato nel sistema in cui si osserva il moto, così facendo la curva sarà data dal 4-vettore posizione $x^\mu(t)=(ct,\vec{x}(t))$. La lunghezza della curva, e la supposta azione, è quindi:
\begin{equation}
    \label{azioneRel}
    \mathcal{S}[x^\mu(t)]=\alpha\int_{t_1}^{t_2} |\dot{x}^\mu(t)|\ dt= \alpha c\int_{t_1}^{t_2} \sqrt{1-\frac{|\vec{v}|^2}{c^2}}\ dt=\alpha c\int_{\tau_1}^{\tau_2}d\tau,
\end{equation}
dove si è indicato $\vec{v}=\dot{\vec{x}}$ che corrisponde al vettore tridimensionale velocità e con $\tau$ il tempo proprio della particella (sezione \ref{sec:ContrazioneDilatazione}), ossia il tempo misurato in un sistema di riferimento inerziale istantanemtne solidale al moto, ossia nel quale la velocità è nulla.\\

A questo punto si deve verificare se questa azione nel limite classico, ossia quando il rapporto $\frac{|\vec{v}|^2}{c^2}$ tende a $0$, è riconducibile all'azione classica. Ricordando che la lagrangiana classica della particella libera è data da $\frac{m|\vec{v}|^2}{2} $:
\begin{equation*}
    \mathcal{L} =\alpha c \sqrt{1-\frac{|\vec{v}|^2}{c^2}}\longrightarrow \alpha c - \frac{\alpha |\vec{v}|^2}{2c^2}\qquad \text{per}\ \frac{|\vec{v}|^2}{c^2}\rightarrow 0.
\end{equation*}
A meno di costanti moltiplicative e additive che non mutano la forma delle soluzioni delle equazioni di Eulero Lagrange, il limite classico ottenuto è quindi riconducibile alla lagrangiana classica se $\alpha=-mc$.\\
Questa evidenza quindi corrobora\footnote{In realtà solamente un confronto tra i risultati teorici che si dedurranno e quelli sperimentali può giustificare l'ipotesi iniziale.} l'ipotesi iniziale così che sia lecito affermare che la lagrangiana relativistica della particella libera sia:
\begin{equation}\label{FreeLagRel}
    \mathcal{L} = -mc^2\sqrt{1-\frac{|\vec{v}|^2}{c^2}}.
\end{equation}
\subsection{Energia e momenti}\label{sec:LagRelEnMo}
Identificata la lagrangiana relativistica per la particella libera è ora possibile sfruttare tutti gli strumenti illustarti nella sezione \ref{sec:MinimaAzione} per studiare grandezze quali l'energia e i momenti di un sistema.\\

La prima quantità che si può studiare è il momento associato alle coordinate cartesiane, questo è il vettore  $(\frac{\partial \mathcal{L} }{\partial v_x},\ \frac{\partial \mathcal{L} }{\partial v_y},\ \frac{\partial \mathcal{L} }{\partial v_z})$ noto come impulso $\vec p$: dalla (\ref{FreeLagRel}) risulta
\begin{equation}
    \vec{p}=m\vec{v}\gamma=\frac{m\vec{v}}{\sqrt{1-\frac{|\vec{v}|^2}{c^2}}}.
    \label{impulsoRel}
\end{equation}
Anche relativisticamente, come in meccanica classica, l'impulso di una particella libera risulta conservato, infatti tutte le coordinate della particella libera sono cicliche. Questo implica che il moto di una particella libera è rettilineo ed uniforme.\\
Si osservi che nel limite classico, in cui $|\vec{v}|<<c$, l'impulso diviene la classica espressione $\vec p=m\vec v$, mentre per velocità tendenti in modulo a $c$ l'impulso tende ad infinito.\\

Analogamente è possibile calcolare l'energia della particella libera $E = \vec{p}\cdot\vec{v}-\mathcal{L} $ che risulta una quantità conservata poiché non vi è dipendenza esplicita dal tempo nella lagrangiana. Calcolandola dalla (\ref{FreeLagRel}) si ottiene:
\begin{equation}
    E = mc^2\gamma=\frac{mc^2}{\sqrt{1-\frac{|\vec{v}|^2}{c^2}}}.\label{energiaRel}
\end{equation}
In questo caso è opportuno osservare che per particelle libere a riposo l'energia non è nulla, come accade in meccanica classica, bensì assume il valore $E_0=mc^2$ noto come energia a riposo della particella. Sottraendo all'energia totale del corpo il termine di energia a riposo si ottiene la quantità di energia esclusivamente dovuta al moto, ossia l'energia cinetica:
\begin{equation}
    T=mc^2\gamma-mc^2=mc^2(\gamma-1)
\end{equation}
Anche in questo caso studiando il limite classico si ritrova l'energia cinetica classica mentre nel limite di velocità prossime a $c$ si ottiene che l'energia del corpo tende ad infinito. Quest'ultima osservazione consente di affermare senza alcuna ombra di dubbio che $c$ è la velocità limite di ogni moto, infatti un qualsiasi corpo dotato di massa necessita di un'energia infinita per poter essere accelerato fino a tale velocità.\\

Talvolta è utile, per ottenere equazioni differenziali di più facile risoluzione di quelle date dalle equazioni di Eulero Lagrange, esprimere il vettore velocità facendo uso dell'impulso e dell'energia, infatti dalle \eqref{impulsoRel} \eqref{energiaRel} si ha che:
\begin{equation}
    \vec v=\frac{\vec p}{m\gamma}=\frac{c^2\vec p}{mc^2\gamma}=\frac{c^2\vec p}{E}.\label{velPE}
\end{equation}    
Questa espressione è indipendente dal fattore $\gamma$ e se di determinano $\vec p$ e $E$ in funzione del tempo è integrabile per ottenere l'equazione del moto.\\

Si consideri ora il 4-gradiente della funzione d'azione, siccome $S(q,t)$ è uno scalare Lorentz invariante, questo è un 4-vettore covariante (sezione \ref{sec:4-vettori}). Come si è mostrato nella sezione \ref{sec:MinimaAzione}, l'azione è legata ad energia e momenti dalle sue derivate parziali: $\vnabla S=\vec{p}$ e $\frac{\partial S }{\partial t}=-E$. Facendo uso di queste relazioni si ottiene la definizione del \emph{4-impulso}:
\begin{equation}
   \frac{\partial S}{\partial x^\mu}= \bigg(\frac{1}{c}\frac{\partial S }{\partial t},\frac{\partial S }{\partial x},\frac{\partial S }{\partial y},\frac{\partial S }{\partial z}\bigg)=\bigg(-\frac{E}{c},\vec{p}\bigg)\quad \Rightarrow \quad
   p^\mu=-\frac{\partial S }{\partial x_\mu}=\bigg(\frac{E}{c},\vec{p}\bigg).
\end{equation}
Studiando la variazione dell'azione \eqref{azioneRel} per variazioni della curva del moto nello spazio-tempo è possibile determinare alcune informazioni utili sul 4-impulso e sul moto delle particelle libere.
Considerazioni di carattere variazionale di questo tipo sono compatibili con il principio di minima azione, infatti la curva del moto $\vec{q}(t)$, se espressa in coordinate cartesiane, è rappresentata nello spazio-tempo dal 4-vettore $x^\mu(t)=(ct,\vec{x}(t))$. Di questa si possono considerare variazioni $\delta x^\mu(t)$ che soddisfino le ipotesi del principio di minima azione $\delta x^\mu(t_1)=\delta x^\mu(t_2)=0$, allora:
\begin{equation*}
    \mathcal{S}[x^\mu(t)]=-mc \int_{t_1}^{t_2} |\dot{x}^\mu(t)|\ dt=-mc \int_{t_1}^{t_2} \sqrt{\frac{d x^\mu}{dt}\frac{d x_\mu}{dt}}\ dt.
\end{equation*}
La variazione prima di questo integrale, sviluppando l'integrando in serie di Taylor al prim'ordine rispetto a $\frac{d x^\mu}{dt}$ e integrando per parti, risulta:
\begin{flalign}
    \delta \mathcal{S}&=-mc \int_{t_1}^{t_2} \frac{1}{2\sqrt{\dot x^\nu(t)\dot x_\nu(t)}}2\frac{dx^\mu}{dt}\frac{d}{dt}\delta x_\mu(t)\ dt\nonumber\\&=-\frac{mc}{\sqrt{\dot x^\nu(t)\dot x_\nu(t)}}\frac{dx^\mu}{dt}\delta x_\mu(t)\bigg|_{t_1}^{t_2}+mc \int_{t_1}^{t_2}\delta x_\mu(t) \frac{d}{dt}\bigg(\frac{1}{\sqrt{\dot x^\nu(t)\dot x_\nu(t)}}\frac{dx^\mu}{dt}\bigg)\ dt.\label{dSFreePart}
\end{flalign}
Si osservi che si fissa $\delta x^\mu(t_1)=\delta x^\mu(t_2)=0$, come si è fatto per determinare le equazioni di Eulero Lagrange nella sezione \ref{sec:MinimaAzione}, il primo addendo si annulla, quindi, poiché per il principio di minima azione $\delta \mathcal{S}$ deve essere nullo, l'integrando del secondo addendo della \eqref{dSFreePart} deve annullarsi $\forall \delta x_\mu$. Osservando che $\frac{1}{\sqrt{\dot x^\nu(t)\dot x_\nu(t)}}=\frac{\gamma}{c}$ e $\frac{d}{d\tau}=\gamma\frac{d}{dt}$, dove $\tau$ indica il tempo proprio, si ottiene:
\begin{equation}
    \frac{d}{dt}\bigg(\frac{\gamma}{c} \frac{dx^\mu}{dt}\bigg)=\frac{d}{dt}\bigg(\frac{1}{c} \frac{dx^\mu}{d\tau}\bigg)=\frac{1}{c}\frac{du^\mu}{dt}=0\label{mru4-d},
\end{equation}
ovvero che la 4-velocità è costante e la traiettoria nello spazio-tempo è quindi una retta per particelle libere, poichè la 4-velocità è tangente a tale curva.\\

Si consideri invece il caso in cui l'integrale d'azione sia calcolato sulla traiettoria del moto ma il secondo estremo d'integrazione non sia fissato, in questo caso il primo addendo della \eqref{dSFreePart} non si annulla in $t_2$ mentre ad annullarsi, per la \eqref{mru4-d}, è l'integrale a secondo addendo. Ricordando $\frac{1}{|\dot{x}^\mu(t)|}=\frac{\gamma}{c}$ e $\frac{d}{d\tau}=\gamma\frac{d}{dt}$ si ha che il 4-impulso è quindi il prodotto tra la 4-velocità è la massa della particella libera:
\begin{equation}
    \delta S=-m\gamma\frac{dx^\mu}{dt}\delta x_\mu=-m\frac{dx^\mu}{d\tau}\delta x_\mu=-m u^\mu\delta x_\mu \quad \Rightarrow \quad p^\mu=m u^\mu,
\end{equation}

Poiché come si è visto energia ed impulso costituiscono un 4-vettore è opportuno, analogamente a quanto accade per la 4-velocità, determinare come queste due quantità si trasformino. Considerando la trasformazione di Lorentz $p'^\mu=\Lambda_\nu^\mu p^\nu$ si ha:
\begin{equation}
    p'_x=\bigg(p_x-\frac{V}{c^2}E\bigg)\gamma,\qquad p'_y=p_y,\qquad p'_z=p_z,\qquad E'=(E-Vp_x)\gamma.
    \label{TrasfLorentzEI}
\end{equation}
Il formalismo quadrivettoriale garantisce che il modulo del 4-impulso sia Lorentz invariante, è quindi utile determinare tale quantità in un sistema di riferimento inerziale solidale al moto della particella, in questo sistema l'impulso e l'energia cinetica sono nulli per cui, detto $p'^\mu$ il 4-impulso in tale sistema, si ha
\begin{equation*}
    p'^\mu p'_\mu=\frac{E_0^2}{c^2}=m^2c^2.
\end{equation*}  
Siccome tale modulo è lo stesso in ogni sistema di riferimento inerziale si ottene la relazione tra l'energia e l'impulso di una particella:
\begin{equation}
    p^\mu p_\mu=\frac{E^2}{c^2}-|\vec p |^2=m^2c^2.
    \label{relazioneEnergiaImpulso}
\end{equation}
Da quest'ultima relazione, se non si determinano i valori delle derivate parziali dell'azione, è possibile ottenere \emph{l'equazione di Hamilton Jacobi} relativistica:
\begin{equation}
    p^\mu p_\mu=\frac{\partial S }{\partial x^\mu}\frac{\partial S }{\partial x_\mu}=\bigg(\frac{1}{c}\frac{\partial S }{\partial t}\bigg)^2-\bigg(\frac{\partial S }{\partial x}\bigg)^2-\bigg(\frac{\partial S }{\partial y}\bigg)^2-\bigg(\frac{\partial S }{\partial z}\bigg)^2=m^2c^2.
\end{equation}

%\section{Meccanica lagrangiana covariante}Preso da \cite{Barone}\\
Le trasformazioni di Lorentz mettono in luce come la dimensione temporale si da considerare al pari di quelle spaziali. Inoltre, come si è già visto nella Sezione precedente, nell'ambito della relatività è più conveniente studiare quantità 4-vettoriali piuttosto che le grandezze caratteristiche della meccanica classica. Per questi motivi è opportuno analizzare anche un formulazione covariante$^{\cite{Barone}}$ della meccanica.\\

Si consideri una curva nello spazio-tempo $x^\mu$ parametrizzata da una variabile priva di significato fisico $\xi$. Il principio di minima azione descritto nella Sezione \ref{sec:MinimaAzione} richiede che fissati l'istante iniziale e finale le coordinate spaziali occupate dal sistema in tali istanti siano anch'esse determinate a priori. Si noti che associare ad un istante una posizione spaziale equivale geometricamente a fissare un punto nello spazio-tempo $\tilde{x}^\mu$. Da un punto di vista quadridimensionale è quindi necessario richiedere che per due valori $\xi_1$ e $\xi_2$ si abbia $x^\mu(\xi_1)=x^\mu_{iniziale}$ e $x^\mu(\xi_1)=x^\mu_{finale}$ con $x^\mu_{iniziale}$ $x^\mu_{finale}$ fissati.\\



\input{Ch2/Più particelle}

\chapter{L'elettromagnetismo nella teoria della relatività}
\section{La lagrangiana di interazione}
\label{sec:LagEMInt}
Si consideri una carica $q$ piccola\footnote{Questa assunzione è necessaria in quanto si vuole considerare inizialmente che la carica non modifichi il campo elettromagnetico che ne influenza il moto.}, a questa è associata una lagrangiana $\mathcal{L} $ costituita da una parte di corpo libero e una parte d'interazione con i campi elettrici e magnetici esterni. Si vuole ricavare la forma di questa seconda parte.\\Come si è detto nella Sezione \ref{sec:EquazioniMaxwell}, tale carica subisce la forza di Lorentz e siccome la relatività assume la formulazione dell'elettromagnetico di Maxwell come la formulazione corretta si può procedere ricavando la forma della lagrangiana d'interazione tra carica e campi da questa forza. Secondo la teoria di Maxwell è possibile descrivere i due campi tramite un \emph{potenziale scalare} $\phi(t,\vec x)$ e un \emph{potenziale vettore} $\vec{A}(t,\vec x)$ tali che:
\begin{equation}
    \vec{E}=-\vnabla\varphi-\frac{\partial \vec{A}}{\partial t} \qquad \vec B=\vnabla\wedge\vec A.\label{PotenzialiEM}
\end{equation}
È opportuno osservare che queste due quantità sono in realtà definite a meno di termini opportuni che non mutano i rispettivi campi: il potenziale $\varphi$ è definito a meno di una costante rispetto alle coordinate spaziali e il potenziale vettore $\vec A$ è definito a meno di un gradiente di un campo scalare.  Questa arbitrarietà prende il nome di \emph{simmetria di gauge}.\\
Utilizzando questi due potenziali la forza di Lorentz assume la forma:
\begin{flalign}
    \vec F&=q\bigg(\vec E(t,\vec x)+\vec v\wedge\vec B(t,\vec x)\bigg)=q\bigg(-\vnabla\varphi(t,\vec x)-\frac{\partial \vec{A}}{\partial t}(t,\vec x)+\vec v\wedge(\vnabla\wedge\vec A(t,\vec x))\bigg)\nonumber\\
    &=q\bigg(-\vnabla\varphi(t,\vec x)-\frac{\partial \vec{A}}{\partial t}(t,\vec x)+\vnabla(\vec v\cdot\vec A(t,\vec x))-(\vec v\cdot \vnabla)\vec A(t,\vec x)\bigg)\nonumber\\
    &=q\vnabla\bigg[-\varphi(t,\vec x)+\vec v\cdot \vec A(t,\vec x)\bigg]-q\frac{d}{dt}\vec A(t,\vec x)\nonumber\\
    &=-\bigg[\frac{d}{dt}\vnabla_{\vec v}-\vnabla\bigg](-q\varphi(t,\vec x)+q\vec v\cdot\vec A(t,\vec x)),\qquad \qquad\vnabla_{\vec{v}}=\bigg(\frac{\partial}{\partial v_x},\frac{\partial}{\partial v_y},\frac{\partial}{\partial v_z}\bigg).\label{FLorentzLagrangiana}
\end{flalign}
Si noti che nella \eqref{FLorentzLagrangiana} si è ottenuta una quantità scalare su cui agisce l'operatore di Eulero-Lagrange in forma vettoriale (Sezione \ref{sec:MinimaAzione}). Infatti, in meccanica lagrangiana, tale operatore, agendo sulla lagrangiana d'interazione, fornisce proprio la forza che media l'interazione, cambiata di segno.\\ La lagrangiana d'interazione elettromagnetica è quindi
\begin{equation}
    \label{lagrangianaInt}
    \mathcal{L}_{Int}=-q\varphi(t,\vec x)+q\vec v\cdot\vec A(t,\vec x),
\end{equation}
nota questa l'azione è data dall'integrale della somma di questa con la lagrangiana di corpo libero:
\begin{equation}
    \mathcal{S} [\vec x(t)]=\int_{t_1}^{t_2}\bigg(-mc^2\sqrt{1-\frac{|\vec v|^2}{c^2}}-q\varphi(t,\vec x)+q\vec v\cdot\vec A(t,\vec x)\bigg)\ dt.\label{AzioneFree+Int}
\end{equation}
Come si è fatto per il caso della particella libera si può procedere calcolando il vettore impulso generalizzato e l'energia della particella carica interagente con campi elettrici e magnetici. Dalle definizioni \eqref{DefMomentoGeneralizzato} e \eqref{DefEnergiaMeccanica}, utilizzate con $\mathcal{L} =\mathcal{L}_{Libera}+\mathcal{L}_{Int}$, si ottiene:
\begin{equation}
    \vec{P}=\frac{m\vec v}{\sqrt{1-\frac{|\vec v|^2}{c^2}}}+q\vec A=\vec{p}+q\vec A,\qquad E=\frac{mc^2}{\sqrt{1-\frac{|\vec v|^2}{c^2}}}+q\phi,\label{energiaImpulsoIntEM}
\end{equation}
dove si è indicato con $\vec P$ l'impulso generalizzato e con $\vec p$ l'impulso della particella libera, che viene in generale chiamato solamente impulso.\\Studiando sistemi di particelle interagenti con campi si indica l'energia che possiederebbe la particella se fosse libera con $E_{Lib}=mc^2\gamma$, questa è l'energia cinetica del sistema a meno di una costante che è l'energia a riposo della particella. Così facendo l'energia totale è $E=E_{Lib}+q\varphi$.\\

I potenziali utilizzati nella lagrangiana \eqref{lagrangianaInt} non sono però quantità misurabili, per cui, per studiare questi sistemi, è opportuno ricavarli dalle loro definizioni \eqref{PotenzialiEM} conoscendo la forma del campo elettrico e magnetico. Nel caso più semplice possibile, ossia di campi costanti nel tempo ed uniformi\footnote{Con uniformi si intende costanti nelle tre dimensioni spaziali.}, i potenziali sono 
\begin{equation}
    \varphi=-\vec E\cdot\vec x, \qquad \qquad \vec A=\frac{\vec B\wedge \vec x}{2},\label{PotenzialiCostanti}
\end{equation}
infatti dalle definizioni si ha:
\begin{equation*}
    -\vnabla{\varphi}=(\vec E\cdot\vnabla)\vec x=\vec{E},\qquad \vnabla\wedge\vec A=\frac{1}{2}[ (\vnabla\cdot\vec x)\vec B-(\vec B\cdot \vnabla)\vec x]=\frac{1}{2}(3\vec B -\vec B)=\vec B.
\end{equation*}

\section{Il moto di particelle cariche in campi costanti}
Lo studio del moto di una particella soggette a potenziali di forma arbitraria risulta molto complicato, in particolar modo nel contesto della relatività (nella quale le equazioni di Eulero-Lagrange danno origine ad equazioni differenziali più complesse delle loro analoghe classiche), per questo motivo si limiterà lo studio al caso di campi costanti. 
\subsection{Campo elettrico costante}
Si consideri un campo solamente elettrico, ossia in cui $\vec B=0$, costante nel tempo ed uniforme. Questo campo è orientato in una direzione spaziale determinata, il sistema quindi perde la sua proprietà di isotropia spaziale ma lo spazio permane isotropo. Per questo motivo è possibile orientare a piacere gli assi del sistema di riferimento così che: $\vec E$ risulti parallelo all'asse $x$ e l'asse $y$ parallelo alla velocità iniziale della carica e quindi il moto avvenga nel piano $xy$. Dalla \eqref{PotenzialiCostanti} si ha che il potenziale scalare può quindi essere espresso come $\varphi=|\vec E|x$, con $|\vec E|$ intensità del campo.
Utilizzando questo potenziale dalle equazioni di Eulero-Lagrange si ha:
\begin{equation*}
    \dot p_x=q|\vec E|,\quad\dot p_y=0,\quad\dot p_z=0 \quad \Rightarrow \quad p_x=q|\vec E|t,\quad p_y=p_0,\quad p_z=0.
\end{equation*}
Facendo uso della relazione energia-impulso \eqref{relazioneEnergiaImpulso} si ottiene l'energia di corpo libero della particella ad un determinato istante:
\begin{equation*}
    E_{Lib}=\sqrt{m^2c^4+c^2p_0^2+(cq|\vec E|t)^2}=mc^2\gamma.
\end{equation*}
Questa espressione dell'energia consente tramite la \eqref{velPE} di ottenere un'espressione istante per istante della velocità del corpo:
\begin{equation*}
    v_x=\frac{p_xc^2}{E_{Lib}}=\frac{c^2q|\vec E|t}{\sqrt{m^2c^4+c^2p_0^2+(cq|\vec E|t)^2}},\quad v_y=\frac{p_yc^2}{E_{Lib}}=\frac{c^2p_0}{\sqrt{m^2c^4+c^2p_0^2+(cq|\vec E|t)^2}}.
\end{equation*}
Integrando rispetto al tempo queste tre espressioni si ottengono le equazioni del moto a meno di costanti additive:
\begin{flalign}
    x(t)&=\int_{0}^{t}\frac{c^2q|\vec E|t}{\sqrt{m^2c^4+c^2p_0^2+(cq|\vec E|t)^2}}\ dt=\frac{\sqrt{m^2c^4+c^2p_0^2+(cq|\vec E|t)^2}}{q|\vec E|}+C_x,\label{MotoEConstX}\\
    y(t)&=\int_{0}^{t}\frac{c^2p_0}{\sqrt{m^2c^4+c^2p_0^2+(cq|\vec E|t)^2}}\ dt=\frac{p_0c}{q|\vec E|}\text{arsinh}\bigg(\frac{cq|\vec E|t}{\sqrt{m^2c^4+c^2p_0^2}}\bigg)+C_y,\label{MotoEConstY}\\
    z(t)&=C_z.
\end{flalign}
Posti $x(0)=y(0)=z(0)=0$, se si esprime nell'equazione del moto \eqref{MotoEConstY} il tempo rispetto ad $y$ e lo si sostituisce nell'equazione del moto \eqref{MotoEConstX}, si ottiene la traiettoria che compie la particella nel piano $xy$:
\begin{equation}
    x(y)=\frac{\sqrt{m^2c^4+c^2p_0^2}}{q|\vec E|}\bigg[\cosh\bigg(\frac{q|\vec E|y}{c\sqrt{m^2c^4+c^2p_0^2}}\bigg)-1\bigg].
\end{equation}
\begin{figure}[H]
    \centering
    \begin{subfigure}{.3\textwidth}
        \begin{tikzpicture}
           \draw[->] (0,-.5)--(0,3)node[anchor=north east]{$ct$};
           \draw[->] (-.5,0)--(3.4,0)node[anchor=north east]{$x$};
           \draw[scale=.6, domain=0:4, smooth, variable=\x, blue] plot ( {sqrt(1+\x*\x)-1},{\x}) node[anchor= south east, color=black ]{$x(t)$};
           \draw[scale=.6, domain=0:4, smooth, variable=\x, yellow,ultra thick,dotted] plot ( {\x},{\x}) node[anchor= north west, color=black ]{$x=ct$};
           \draw[scale=.6, domain=1:4, smooth, variable=\x, yellow,dotted, ultra thick] plot ( {\x-1},{\x});
       \end{tikzpicture}   
       \end{subfigure}
       \begin{subfigure}{.3\textwidth}
        \begin{tikzpicture}
           \draw[->] (0,-.5)--(0,3)node[anchor=north east]{$ct$};
           \draw[->] (-.5,0)--(3.4,0)node[anchor=north east]{$y$};
           \draw[scale=.6, domain=0:2.1, smooth, variable=\x, blue] plot ( {\x},{sinh(\x)}) node[anchor= south east, color=black ]{$y(t)$};
           \draw[scale=.6, domain=0:4, smooth, variable=\x, yellow,ultra thick,dotted] plot ( {\x},{\x}) node[anchor= north west, color=black ]{$y=ct$};
       \end{tikzpicture}   
       \end{subfigure}
    \begin{subfigure}{.3\textwidth}
     \begin{tikzpicture}
        \draw[thick,red,->] (.5,0.5)--(1.5,.5);
        \draw[thick,red,->] (.5,1)--(1.5,1);
        \draw[thick,red,->] (.5,1.5)--(1.5,1.5);
        \draw[thick,red,->] (.5,2)--(1.5,2);
        \draw[->] (0,-.5)--(0,3)node[anchor=north east]{$y$};
        \draw[->] (-.5,0)--(3.4,0)node[anchor=north east]{$x$};
        \draw[thick,red,->] (.5,0)--(1.5,0)node[anchor=south west, black]{$\vec{E}$};
        \draw[scale=.5, domain=0:2.5, smooth, variable=\x, blue] plot ( {cosh(\x)-1},{\x}) node[anchor= south, color=black ]{$x(y)$};
    \end{tikzpicture}   
    \end{subfigure}
    \caption{Rappresentazione grafica delle equazioni del moto ottenute.}
    \label{fig:motoECost}
\end{figure}
 In figura \ref{fig:motoECost} è possibile apprezzare che per $t\rightarrow \infty$ si ha $v_x\rightarrow c$ e $v_y\rightarrow 0$: infatti anche se la forza di Lorentz è diretta solo nella direzione dell'asse $x$ la particella decelera anche nella direzione dell'asse $y$ affinché il moto  avvenga all'interno del cono di luce.

\subsection{Campo magnetico costante}
Se si considera solamente un campo magnetico uniforme, come si è fatto per il campo elettrico costante, è possibile orientare il sistema di riferimento in maniera tale da avere tale campo diretto lungo l'asse $z$. In questo modo il potenziale vettore dalla relazione \eqref{PotenzialiCostanti} risulta $\vec A=\frac{|\vec B|}{2}(-y,x,0)$, dove $|\vec B|$ è il modulo del campo magnetico.\\ Si osservi che, essendo la lagrangiana di questo sistema indipendente dal tempo, l'energia del sistema si conserva e dalla \eqref{energiaImpulsoIntEM} questa (in assenza di campi elettrici) è $E=mc^2\gamma$.\\Usando le equazioni di Eulero-Lagrange e la relazione \eqref{velPE} si ottiene:
\begin{equation*}
    \dot{p}_x=qv_y |\vec B|=\frac{d}{dt}\bigg(\frac{v_xE}{c^2}\bigg),\quad \dot{p}_y=-qv_x|\vec B|=\frac{d}{dt}\bigg(\frac{v_yE}{c^2}\bigg)\Rightarrow \dot v_x=v_y\frac{|\vec B|q}{m\gamma},\quad\dot v_y=-v_x\frac{|\vec B|q}{m\gamma}.
\end{equation*}
Per risolvere questo sistema di equazioni differenziali si osservi che, se definisce la quantità $v_x+iv_y$, dove $i^2=-1$, è possibile ricondursi ad un'equazione differenziale a variabili separabili:
\begin{equation*}
    \frac{d}{dt}(v_x+iv_y)=-i\frac{|\vec B|q}{m\gamma}(v_x+iv_y)\quad\Rightarrow\quad v_x+iv_y=v_0e^{-i\frac{|\vec B|q}{m\gamma}t+i\alpha}
\end{equation*}
dove $\alpha=\arctan\frac{v_y(0)}{v_x(0)}$ e $v_0^2=v_x^2(t)+v_y^2(t)$ sono due costanti da determinare.\\
Separando la parte immaginaria e la parte reale di quanto ottenuto e integrando nuovamente si ottengono le equazioni del moto:
\begin{flalign}
    v_x(t)&=v_0\cos\bigg(\frac{|\vec B|q}{m\gamma}t+\alpha\bigg)\quad&\Rightarrow\quad &x(t)=x_0+\frac{v_0m\gamma}{|\vec B|q}\sin\bigg(\frac{|\vec B|q}{m\gamma}t+\alpha\bigg)&&\\
    v_y(t)&=-v_0\sin\bigg(\frac{|\vec B|q}{m\gamma}t+\alpha\bigg)\quad&\Rightarrow\quad &y(t)=y_0+\frac{v_0m\gamma}{|\vec B|q}\cos\bigg(\frac{|\vec B|q}{m\gamma}t+\alpha\bigg)&&\\
    v_z(t)&=v_z \quad &\Rightarrow\quad &z(t)=z_0+v_zt.&&
\end{flalign}
Proiettando il moto nel piano $xy$ si ottiene una circonferenza centrata in $(x_0,y_0)$ e di raggio $r=\frac{v_0m\gamma}{|\vec B|q}$ che nel complesso dà origine ad un'elica nello spazio (Fig. \ref{fig:motoBCost}).
\begin{figure}[H]
    \centering
    \begin{subfigure}{.3\textwidth}
        \begin{tikzpicture}
           \draw[->] (0,-.5)--(0,3)node[anchor=north east]{$y$};
           \draw[->] (-.5,0)--(3.4,0)node[anchor=north east]{$x$};
           \draw[color=blue,] (1.5,1.7) circle (1);
           \draw[->] (0,0) -- (1.5,1.7) node[anchor=south]{$(x_0,y_0$)};
           \draw[<->] (1.5,1.7) --node[anchor=north]{$r$} (2.5,1.7);
           \draw[<-] (2.8,1.7)node[anchor=south west]{$\omega=\frac{|\vec B|q}{m\gamma}$} arc(0:90:1.3) ;
       \end{tikzpicture}
    \end{subfigure}   
    \begin{subfigure}{.3\textwidth}
     \begin{tikzpicture}
        \draw[thick,green,->] (.5,0.5)--(1.5,.5);
        \draw[thick,green,->] (.5,1)--(1.5,1);
        \draw[thick,green,->] (.5,1.5)--(1.5,1.5);
        \draw[thick,green,->] (.5,2)--(1.5,2);
        \draw[thick,green,->] (.5,-.5)--(1.5,-.5)node[anchor=south west, black]{$\vec{B}$};
        \draw[->] (0,-.5)--(0,2)node[anchor=north east]{$x$};
        \draw[->] (0.2,.2)--(-1,-1)node[anchor=north east]{$y$};
        \draw[->] (-.5,0)--(5,0)node[anchor=north east]{$z$};
        \draw[thick,green,->] (.5,0)--(1.5,0);
        \draw[decoration={aspect=0.3, segment length=3mm, amplitude=20,coil},decorate,color=blue] (-0.5,.8) -- (4,.8); 
    \end{tikzpicture}   
    \end{subfigure}
    \caption{Rappresentazione grafica delle equazioni del moto ottenute.}
    \label{fig:motoBCost}
\end{figure}
 Il risultato ottenuto è simile a quanto è predetto dalla meccanica classica, differisce da questa per il valore esatto della frequenza di rotazione e del raggio che dipendo da $\gamma$ e si riconducono alle loro forme newtoniane nel limite classico per cui $\gamma \approx 1$.
\section{Il tensore elettromagnetico}
\subsection{Il 4-potenziale}
Nella Sezione \ref{sec:trasfEM} si è mostrato in che modo il campo elettrico e il campo magnetico si trasformino nel passaggio tra sistemi di riferimento inerziali. È quindi opportuno mostrare come il potenziale $\varphi$ e il potenziale vettore $\vec A$ si trasformino di conseguenza.\\Se si considera una trasformazione di Lorentz tra due sistemi di riferimento ($K$ e $K'$) in moto reciproco, a velocità $V$ lungo l'asse $x$, si hanno le seguenti trasformazioni degli operatori di differenziazione e del campo elettrico:
\begin{flalign}
    &\frac{\partial}{\partial x}=\gamma\frac{\partial}{\partial x'}-\gamma \frac{V}{c^2}\frac{\partial}{\partial t'},\qquad \frac{\partial}{\partial y}=\frac{\partial}{\partial y'},\qquad \frac{\partial}{\partial z}=\frac{\partial}{\partial z'},\quad \frac{\partial}{\partial t}=\gamma \frac{\partial}{\partial t'}-\gamma V\frac{\partial}{\partial x'},\label{lorentzdiff3}&&\\
    & E_x=E'_{x'},\qquad E_{y}=(E'_{y'}+VB'_{z'})\gamma,\qquad E_{z}=(E_{z'}'-VB_{y'}')\gamma.&&\label{trasfE3}
\end{flalign}
Nel sistema $K'$ sono definiti dalla \eqref{PotenzialiEM} i potenziali $\varphi'$ e $\vec{A'}$, questi danno origine ai campi $\vec{E'}$ e $\vec{B'}$. Utilizzando le trasformazioni \eqref{lorentzdiff3} nella definizione del potenziale \eqref{PotenzialiEM} si ottiene, per la prima componente del campo elettrico nel sistema $K$, che
\begin{flalign}
    E_x&=-\frac{\partial\varphi}{\partial x}-\frac{\partial A_x}{\partial t}=-\gamma\frac{\partial\varphi}{\partial x'}+\gamma\frac{V}{c^2}\frac{\partial\varphi}{\partial t'}-\gamma\frac{\partial A_x}{\partial t'}+\gamma V\frac{\partial A_x}{\partial x'}&&\nonumber\\&=-\frac{\partial}{\partial x'}\bigg[\varphi-VA_x\bigg]\gamma-\frac{\partial}{\partial t'}\bigg[A_x-\frac{V}{c^2}\varphi\bigg]\gamma.&&\label{trasfExPhiA}
\end{flalign}
 Si osservi che per la \eqref{trasfE3} $E_x$ si trasforma tramite l'identità, per cui la relazione \eqref{trasfExPhiA}, appena ottenuta, è pari a $E'_{x'}$. Questa è in forma equivalente alla definizione dei potenziali \eqref{PotenzialiEM}, per cui si ha che:
\begin{equation}
    \varphi'=\gamma\bigg(\varphi-VA_x\bigg),\qquad A_{x'}'=\gamma\bigg(A_x-\frac{V}{c^2}\varphi\bigg).
\end{equation} 
Trasformando la componente $E_y$ secondo la \eqref{trasfE3} e facendo uso delle definizioni dei potenziali \eqref{PotenzialiEM} in $K'$ si ottiene, tramite le trasformazioni degli operatori di differenziazione \eqref{lorentzdiff3}, un'espressione riconducibile alle definizioni dei potenziali in $K$:
\begin{flalign*}
    E_y&=\gamma(E'_{y'}+VB'_{z'})=\gamma\bigg[-\frac{\partial\varphi'}{\partial y'}-\frac{\partial A_{y'}'}{\partial t'}+V\frac{\partial A_{y'}'}{\partial x'}-V\frac{\partial A_{x'}'}{\partial y'}\bigg] &&\\ &=-\frac{\partial}{\partial y'}\bigg[\varphi'+VA'_{x'}\bigg]\gamma-\frac{\partial A_{y'}'}{\partial t'}\gamma+V\frac{\partial A_{y'}'}{\partial x'}\gamma=-\frac{\partial}{\partial y}\bigg[\varphi'+VA'_{x'}\bigg]\gamma-\frac{\partial A_{y'}'}{\partial t}.&&
\end{flalign*}
Da questo segue che: 
\begin{equation}
    \varphi=\gamma\bigg(\varphi'+VA_{x'}'\bigg),\qquad A_y=A_{y'}'.
\end{equation}
Analoghe considerazioni sulla componente $E_z$ consentono di determinare che $A_z=A_{z'}'$. Così facendo si è mostrato che il potenziale scalare e il potenziale vettore si trasformano come un 4-vettore. Si può infatti costruire il \emph{4-potenziale}:
\begin{equation}
    A^\mu=\bigg(\frac{\varphi}{c},\vec A\bigg),\qquad A_\mu=\bigg(\frac{\varphi}{c},-\vec A\bigg).\label{4-potenziale}
\end{equation}

\subsection{Equazioni del moto 4-vettoriali}
\label{sec:4-equazioniMotoEM}
Utilizzando il 4-potenziale \eqref{4-potenziale} è possibile esprimere l'azione \eqref{AzioneFree+Int}, di un particella interagente con campi elettromagnetici, tramite l'uso di 4-vettori. Infatti:
\begin{equation*}
    q\phi-\vec v\cdot \vec A=\frac{1}{\gamma}A^\mu u_\mu=A^\mu\frac{dx_\mu}{dt} \qquad \Rightarrow \qquad  \mathcal{S} [x^\mu(t)]=-\int_{t_1}^{t_2}\bigg(mc|\dot x^\mu(t)|+qA^\mu \dot x_\mu(t)\bigg)\ dt
\end{equation*}
in questo modo è evidente che questa azione sia ancora Lorentz invariante (siccome parametrizzando la curva d'integrazione rispetto al tempo proprio\footnote{La scelta della parametrizzazione è arbitraria.} ci si riconduce ad un prodotto scalare integrato rispetto ad una quantità invariante, che è quindi a sua volta Lorentz invariante).\\

Si vogliono ora studiare le variazioni dell'azione $\delta \mathcal{S}$ dovute a variazioni della traiettoria del corpo $\delta x^\mu(t)$. Queste danno luogo a variazioni $\delta \dot x^\mu(t)=\frac{d}{dt}\delta x^\mu(t)$ e $\delta A^\mu=\frac{\partial A^\mu}{\partial x^\nu}\delta x^\nu$, (poiché il 4-potenziale dipende dal punto nello spazio-tempo in cui è calcolato).\\ Per utilizzare il principio di minima azione si calcola la variazione prima che è data da:
\begin{equation*}
    \delta \mathcal{S} [x^\mu(t)]=-\int_{t_1}^{t_2}\bigg(mc\frac{\dot x^\mu(t)\delta\dot x_\mu(t)}{\sqrt{\dot x^\nu(t)\dot x_\nu(t)}}+qA^\mu\delta \dot x_\mu(t)+q\dot x_\mu(t)\frac{\partial A^\mu}{\partial x_\nu}\delta x_\nu\bigg)\ dt
\end{equation*} 
ottenuta dall'espansione in serie di Taylor dell'integrando rispetto a $x^\mu$ e $\dot x^\mu(t)$.\\
Integrando per parti i primi due addendi dell'integrando si ottiene:
\begin{equation*}
    \int_{t_1}^{t_2}\bigg(mc\frac{d}{dt}\frac{\dot x^\mu(t)}{\sqrt{\dot x^\nu(t)\dot x_\nu(t)}}+\frac{d}{dt}qA^\mu-q\dot x_\nu(t)\frac{\partial A^\nu}{\partial x_\mu}\bigg)\delta x_\mu(t)\ dt-\bigg(mc\frac{\dot x^\mu(t)}{\sqrt{\dot x^\nu(t)\dot x_\nu(t)}}+qA^\mu\bigg)\delta x_\mu(t)\bigg|_{t_1}^{t_2},
\end{equation*} 
dove il termine dopo l'integrale si annulla (siccome per le ipotesi del principio di minima azione $\delta x_\mu(t_1)=\delta x_\mu(t_2)=0$). A questo punto affinché $x^\mu(t)$ renda estremale l'azione per $\delta x_\mu(t)$ arbitrario è necessario che tutto il termine tra parentesi tonde nell'integrando si annulli identicamente. Esplicitando, tramite la regola della derivata composta, la derivata totale rispetto al tempo di $A^\mu$ e ricordando che $\frac{1}{\sqrt{\dot x^\nu(t)\dot x_\nu(t)}}=\frac{\gamma}{c}$ e si ottiene:
\begin{equation*}
    \frac{mc}{\sqrt{\dot x^\nu(t)\dot x_\nu(t)}}\frac{d}{dt}\frac{dx^\mu}{dt}+q\frac{d}{dt}A^\mu-q\frac{dx_\nu}{dt} \frac{\partial A^\nu}{\partial x_\mu}=m\gamma\frac{d}{dt}\frac{dx^\mu}{dt}+q\frac{\partial A^\mu}{\partial x_\nu}\frac{dx_\nu}{dt}-q\frac{dx_\nu}{dt} \frac{\partial A^\nu}{\partial x_\mu}=0.
\end{equation*}
Infine moltiplicando tutta l'ultima uguaglianza per $\gamma$ e facendo uso della relazione $\frac{d}{d\tau}=\gamma\frac{d}{dt}$ si ricava l'equazione di Newton\footnote{L'equazione \eqref{4-EqNewton4EM} è così chiamata poiché ricorda in forma l'equazione di Newton classica.} 4-dimensionale:
\begin{equation}
    m\frac{du^\mu}{d\tau}=q\bigg(\frac{\partial A^\nu}{\partial x_\mu}-\frac{\partial A^\mu}{\partial x_\nu}\bigg)u_\nu=qF^{\mu\nu}\ u_\nu,\label{4-EqNewton4EM}
\end{equation}
dove è stato introdotto il \emph{4-tensore elettromagnetico} $F^{\mu\nu}=\frac{\partial A^\nu}{\partial x_\mu}-\frac{\partial A^\mu}{\partial x_\nu}$. Se si calcolano esplicitamente tutte le derivate parziali di questo 4-tensore, dalla \eqref{PotenzialiEM}, si ha:
\begin{equation}
    \label{4-tensoreEM}
    F^{\mu\nu}=
    \begin{pmatrix}
        0&-E_x/c&-E_y/c&-E_z/c\\
        E_x/c&0&-B_z&B_y\\
        E_y/c&B_z&-0&-B_x\\
        E_z/c&-B_y&B_x&-0\\
 \end{pmatrix}\qquad
 F_{\mu\nu}=
 \begin{pmatrix}
     0&E_x/c&E_y/c&E_z/c\\
     -E_x/c&0&-B_z&B_y\\
     -E_y/c&B_z&-0&-B_x\\
     -E_z/c&-B_y&B_x&-0\\
\end{pmatrix}
\end{equation}

Si osservi che, per $\mu=1,2,3$, la \eqref{4-EqNewton4EM} esprime proprio la forza di Lorentz \eqref{ForzaLorentz}.\\ Per $\mu=0$ si ottiene
\begin{equation}
    \frac{d}{d\tau}(mc\gamma)=\frac{q}{c}\bigg[\frac{\partial \vec A}{\partial t}-\vnabla \varphi\bigg]\cdot \vec v\gamma \qquad \Rightarrow \qquad \frac{d}{dt}(mc^2\gamma)=q\frac{\partial \vec A}{\partial t}\cdot \vec v - q\vnabla \varphi\cdot \vec v
\end{equation}
che esprime la variazione di energia di corpo libero del sistema, ossia la potenza.\\

Se invece si valuta la variazione dell'azione variando il secondo estremo di integrazione e integrando su una curva del moto (come si è già fatto nella Sezione \ref{sec:LagRelEnMo}) si ottiene:
\begin{equation*}
    \delta S=-\bigg(mc\frac{\dot x^\mu(t)}{\sqrt{\dot x^\nu(t)\dot x_\nu(t)}}+qA^\mu\bigg)\delta x_\mu=-(mu^\mu+qA^\mu)\delta x_\mu \quad \Rightarrow \quad -\frac{\partial S}{\partial x_\mu}=mu^\mu+qA^\mu.
\end{equation*}
Il 4-gradiente dell'azione consente quindi di definire il 4-momento generalizzato della particella:
\begin{equation}
    P^\mu=mu^\mu+qA^\mu=\bigg(\frac{mc^2\gamma+q\varphi}{c},\vec p+q\vec A\bigg)
\end{equation}
che come nel caso della particella libera ha come componenti l'energia del sistema e il suo momento generalizzato, già calcolati nella \eqref{energiaImpulsoIntEM}.
\subsection{Invarianti del campo elettromagnetico}
Come si è fatto per i 4-vettori, si vogliono identificare le quantità caratteristiche del 4-tensore elettromagnetico \eqref{4-tensoreEM} che sono invarianti per trasformazioni di Lorentz. Il modo più semplice per ottenere tali quantità è di contrarre il 4-tensore elettromagnetico su altre quantità tensoriali opportune.\\

In primo luogo è possibile contrarre il 4-tensore elettromagnetico controvariante con quello covariante (come si è già visto nella Sezione \ref{sec:4-vettori} questa è una quantità Lorentz invariante):
\begin{equation}
    F^{\mu\nu}F_{\mu\nu}=2\biggl(|\vec B|^2-\frac{|\vec E|^2}{c^2}\biggr)=\text{invariante}.
\end{equation}
Questo implica direttamente che se in un sistema di riferimento vale $|\vec B|^2c^2\geq|\vec E|^2$ (oppure $|\vec B|^2c^2\leq|\vec E|^2$) allora questa relazione vale anche per ogni altra coppia di campi elettrici e magnetici in ogni altro sistema di riferimento inerziale.\\

Un secondo invariante può essere costruito contraendo il 4-tensore elettromagnetico controvariante e quello covariante con il simbolo di Levi-Civita. Quest'ultimo è un oggetto a più indici così definito in ogni sistema di riferimento:
\begin{equation}
    e^{\mu\nu\delta\lambda}=\begin{cases}
        1\ \text{se}\ (\mu,\nu,\delta,\lambda)\ \text{sono un permutazione pari di}\ (0,1,2,3)\\
        -1\ \text{se}\ (\mu,\nu,\delta,\lambda)\ \text{sono un permutazione dispari di}\ (0,1,2,3)\\
        0\ \text{se almeno due indici sono uguali}
    \end{cases}.\label{LeviCivita}
\end{equation}
Si ha quindi che è Lorentz invariante (siccome si sta contraendo quantità covarianti con quantità controvarianti):
\begin{equation}
    e^{\mu\nu\delta\lambda}F_{\mu\nu}F^{\delta\lambda}=-8\frac{\vec E\cdot\vec B}{c}=\text{invariante}.
\end{equation}
Questa seconda quantità invariante consente di affermare che: se in un sistema di riferimento il campo elettrico è normale a quello magnetico allora in ogni altro sistema di riferimento si avranno campi perpendicolari.\\ È utile osservare che, facendo uso delle trasformazioni di Lorentz per i campi elettromagnetici (ricavate nella Sezione \ref{sec:trasfEM}), è possibile scegliere arbitrariamente un secondo sistema di riferimento inerziale in cui i valori assunti dai due campi sono di più facile utilizzo. Gli unici vincoli su questa scelta sono proprio dati dalle due quantità invarianti determinate. In questo modo se $\vec E \cdot \vec B\neq0$ si può identificare un riferimento in cui i due campi sono paralleli mentre se $\vec E \cdot \vec B=0$ è possibile identificare un riferimento nel quale uno dei due campi è identicamente nullo.\\

Per quanto detto fino ad ora non è chiaro se di questi invarianti ve ne possano essere altri. Per rendere evidente che quelli trovati sono gli unici è possibile ragionare nel seguente modo: per prima cosa si osservi che gli invarianti che si sono individuati sono propri dei campi e quindi slegati dal formalismo che si utilizza. Per questo motivo è utile introdurre il seguente vettore a componenti complesse che ancora rappresenta il tensore elettromagnetico ma con un formalismo alternativo:
\begin{equation*}
    \vec{F}=\frac{\vec E}{c} +i\vec B \qquad i^2=-1.
\end{equation*}
Utilizzando le trasformazioni del campo elettrico e magnetico (ricavate nella Sezione \ref{sec:trasfEM}) si possono ottenere le trasformazioni per questo vettore:
\begin{equation*}
    \begin{cases}
        F'_x=\frac{E_x}{c}+iB_x\\
        F'_y=\gamma\frac{E_y}{c}-\frac{V}{c}\gamma B_z+i\gamma B_y+i\frac{V}{c^2}\gamma E_z\\
        F'_z=\gamma\frac{E_z}{c}+\frac{V}{c}\gamma B_y+i\gamma B_z-i\frac{V}{c^2}\gamma E_y
    \end{cases}
    \quad\Rightarrow\quad
    \vec{F'}=\begin{pmatrix}
        1&0&0\\
        0&\gamma&-i\frac{V}{c}\gamma\\
        0&i\frac{V}{c}\gamma&\gamma
    \end{pmatrix}
    \begin{pmatrix}
        F_x\\F_y\\F_z
    \end{pmatrix}.
\end{equation*}
Questa trasformazione è una rotazione delle componenti $y$ e $z$ di $\vec{F}$ (infatti il determinante della matrice che esprime la trasformazione è identicamente pari a $1$). Un vettore conserva una sola quantità sotto rotazioni, ossia il suo modulo, che in questo caso questo è pari a:
\begin{equation*}
    |\vec F|^2=\frac{|\vec E|^2}{c^2}-|\vec B|^2+2i\frac{\vec E\cdot\vec B}{c}.
\end{equation*}
Siccome questa quantità è complessa si ha che per essere invariante devono essere indipendentemente invarianti la sua parte reale e la sua parte immaginaria. Queste sono proprio, a meno di fattori moltiplicativi, gli invarianti precedentemente individuati.
\section{Trasformazioni di gauge}\label{sec:gauge}
Come si è già osservato nella Sezione \ref{sec:LagEMInt}, le definizioni \eqref{PotenzialiEM} dei potenziali $\varphi$ e $\vec A$ non determinano in maniera univoca il campo elettrico e il campo magnetico. Nella fattispecie il potenziale vettore è definito a meno di un gradiente di un campo scalare $\xi(t,\vec x)$. Per cui la trasformazione
\begin{equation}
     \vec{A}\longrightarrow\vec {A'}=\vec A + \vnabla \xi \label{trasfGauge1}
\end{equation}
lascia invariato il campo magnetico, poiché il rotore di un gradiente è una quantità identicamente nulla: 
\begin{equation*}
    \vec B\longrightarrow\vec{B'}=\vnabla\wedge\vec {A'}=\vnabla\wedge[\vec A + \vnabla \xi]=\vnabla\wedge\vec A =\vec B.
\end{equation*}
La trasformazione \eqref{trasfGauge1} può però modificare la forma del campo elettrico, infatti $\xi$ può dipendere dal tempo e tramite la definizione dei potenziali \eqref{PotenzialiEM} si ha che:
\begin{equation*}
    \vec E\longrightarrow\vec{E'}=-\vnabla \varphi-\frac{\partial\vec A'}{\partial t}=-\vnabla \varphi-\frac{\partial\vec A}{\partial t}-\frac{\partial}{\partial t}\vnabla\xi=\vec{E}-\frac{\partial}{\partial t}\vnabla\xi.
\end{equation*}
Per ovviare a questo problema si consideri una seconda trasformazione da utilizzare assieme alla \eqref{trasfGauge1}:
\begin{equation}
    \varphi \longrightarrow \varphi'=\varphi-\frac{\partial\xi}{\partial t}
    \label{trasfGauge2}
\end{equation}
in questo modo (supponendo sufficiente regolarità di $\xi$ per scambiare l'ordine delle derivate parziali) la trasformazione del campo elettrico risulta:
\begin{equation*}
    \vec E\longrightarrow\vec{E'}=-\vnabla \varphi'-\frac{\partial\vec A'}{\partial t}=-\vnabla \varphi+\frac{\partial}{\partial t}\vnabla\xi-\frac{\partial\vec A}{\partial t}-\frac{\partial}{\partial t}\vnabla\xi=\vec{E}.
\end{equation*}
Le due trasformazioni \eqref{trasfGauge1} e \eqref{trasfGauge2} lasciano quindi invariati campo elettrico e magnetico e prendono il nome di \emph{trasformazioni di gauge}. Si osservi che queste trasformazioni sarebbero potute esser ricavate considerando quali trasformazioni del 4-potenziale non mutano la lagrangiana di interazione \eqref{lagrangianaInt}. È noto infatti che la lagrangiana è definita a meno di derivate totali rispetto al tempo e questo consente di trasformare:
\begin{equation*}
    \mathcal{L} \longrightarrow\mathcal{L} '=\mathcal{L} +q\frac{d\xi}{dt},
\end{equation*}
dove $\xi(t,\vec x)$ è una funzione scalare. Ne consegue che esplicitando la lagrangiana di interazione si ottiene:
\begin{equation*}
    \mathcal{L}' =-q\varphi+\vec v\cdot\vec A+q\bigg(\frac{\partial \xi}{\partial t}+\vnabla\xi\cdot\vec v\bigg)=-q\bigg(\varphi-\frac{\partial \xi}{\partial t}\bigg)+\vec v\cdot\bigg(\vec A+\vnabla\xi\bigg),
\end{equation*}
ossia le trasformazioni \eqref{trasfGauge1} e \eqref{trasfGauge2}.\\Infine si osservi che queste trasformazioni in forma 4-vettoriale diventano:
\begin{equation}
    A^\mu\longrightarrow  A'^\mu=A^\mu-\frac{\partial \xi}{\partial x_\mu}=A^\mu-\partial^\mu\xi.\label{4-trasfGauge}
\end{equation}

Le trasformazioni di gauge possono essere molto utili per semplificare la risoluzione delle equazioni di Maxwell. Infatti, sostituendo i potenziali \eqref{PotenzialiEM} nella prima e ultima equazione di Maxwell \eqref{EquazioniMaxwell}, si ottengono due equazioni differenziali che se risolte determinano i potenziali stessi:
\begin{flalign}
    &\vnabla^2\varphi+\vnabla\cdot \frac{\partial \vec A}{\partial t}=-\frac{\rho}{\epsilon_0},\label{MaxwellPote1}\\
    &\vnabla^2\vec A-\epsilon_0\mu_0\frac{\partial^2 \vec A}{\partial t^2}-\vnabla\bigg(\vnabla\cdot\vec A +\epsilon_0\mu_0\frac{\partial \varphi}{\partial t}\bigg)=-\mu_0\vec J.\label{MaxwellPote2}
\end{flalign}
Si possono quindi annullare identicamente alcuni addendi facendo uso di apposite funzioni $\xi$. In funzione di quale di questi si vuole annullare si hanno diversi gauge.

\subsection{Gauge di Lorentz}
Il gauge di Lorentz è la richiesta che valga:
\begin{equation}
    \vnabla \cdot \vec A+\epsilon_0\mu_0\frac{\partial \varphi}{\partial t}=0.\label{gaugediLorentz}
\end{equation}
Questa condizione è la richiesta che si annulli identicamente tutto l'addendo tra parentesi tonde dell'equazione di Maxwell \eqref{MaxwellPote2} che diventa un'equazione delle onde non omogenea. Analogamente, se si suppone una sufficiente regolarità dei potenziali (affinché sia possibile scambiare l'ordine derivate parziali temporali e spaziali\footnote{Questa condizione nella trattazione dei prossimi gauge, per non doversi ripetere, sarà sottointesa.}) e si sostituisce il gauge di Lorentz nella equazione \eqref{MaxwellPote1}, anche questa diviene un'equazione delle onde non omogenea:
\begin{flalign*}
    &\vnabla^2\varphi-\epsilon_0\mu_0\frac{\partial^2 \varphi}{\partial t^2}=-\frac{\rho}{\epsilon_0},\\
    &\vnabla^2\vec A-\epsilon_0\mu_0\frac{\partial^2 \vec A}{\partial t^2}=-\mu_0\vec J.
\end{flalign*}
Inoltre, se si suppone l'assenza di termini di sorgente di campo, si ottengono due equazioni equivalente alle equazioni delle onde elettromagnetiche nel vuoto \eqref{OndeEMVuoto}.\\

Si osservi che in questo caso si ha il vantaggio di avere un'invarianza relativistica di questa condizione: infatti, utilizzando il 4-potenziale e ricordando che $\epsilon_0\mu_0=\frac{1}{c^2}$, la condizione \eqref{gaugediLorentz} diventa $\partial_\mu A^\mu=0$ (che è un'equazione invariante sotto trasformazioni di Lorentz).\\ Infine si osservi che questo gauge mantiene un certo grado di libertà nella scelta dei potenziali. Infatti se si compie un'ulteriore trasfromazione di gauge $(\varphi,\ \vec A)\longrightarrow(\varphi',\ \vec{A'})$ la condizione di gauge di Lorentz diviene:
\begin{equation*}
    \vnabla \cdot \vec A+\vnabla^2\xi+\epsilon_0\mu_0\frac{\partial \varphi}{\partial t}-\epsilon_0\mu_0\frac{\partial^2 \xi}{\partial t^2}=0
\end{equation*}
che è riconducibile nuovamente al gauge di Lorentz solo se $\xi$ soddisfa anch'essa un'equazione delle onde (così che tutti i termini dove questa compare nell'ultima equazione ricavata si annullino). Sotto questa condizione è quindi possibili applicare altre trasformazioni che semplifichino ulteriormente il problema.
\subsection{Gauge di Coulomb}
Il gauge di Coulomb consiste nella condizione:
\begin{equation}
    \vnabla\cdot{\vec A}=0,\label{gaugediCoulomb}
\end{equation}
questa non è Lorentz invariante a differenza del gauge di Lorentz. Sostituendola nelle equazioni del potenziale \eqref{MaxwellPote1} e \eqref{MaxwellPote2} si ottengono le seguenti:
\begin{flalign*}
    &\vnabla^2\varphi=-\frac{\rho}{\epsilon_0},\\
    &\vnabla^2\vec A-\epsilon_0\mu_0\frac{\partial^2 \vec A}{\partial t^2}-\epsilon_0\mu_0\vnabla\frac{\partial \varphi}{\partial t}=-\mu_0\vec J,
\end{flalign*}
dove la prima equazione è anche nota come equazione di Poisson.\\

In questo caso esiste un unico potenziale vettore che soddisfa il gauge di Coulomb. Infatti affinché una trasformazione di gauge $\vec A\longrightarrow\vec{A'}$ soddisfi la \eqref{gaugediCoulomb} deve valere dappertutto
\begin{equation*}
    \vnabla\cdot\vec{A'}=\vnabla\cdot(\vec A + \vnabla\xi)=\vnabla^2\xi=0.
\end{equation*}
Questa condizione è soddisfatta se $\xi$ è ovunque spazialmente costante, il che rende la trasformazione di gauge $\vec{A'}=\vec{A}$. Lo stesso non vale per $\varphi$ e per la dipendenza temporale di $\xi$. 
\subsection{Gauge temporale}
Il gauge temporale è dato dalla condizione:
\begin{equation}
    \varphi=0, \label{gaugeTemporale}
\end{equation}
anch'essa non è Lorentz invariante. Se sostituita nelle \eqref{MaxwellPote1} e \eqref{MaxwellPote2} si ha:
\begin{flalign*}
    &\vnabla\cdot \frac{\partial \vec A}{\partial t}=-\frac{\rho}{\epsilon_0},\\
    &\vnabla^2\vec A-\epsilon_0\mu_0\frac{\partial^2 \vec A}{\partial t^2}-\vnabla(\vnabla\cdot\vec A)=-\mu_0\vec J.
\end{flalign*}
\subsection{Gauge di radiazione}
Infine il gauge di radiazione consiste nella richiesta simultanea del gauge di Coulomb \eqref{gaugediCoulomb} e del gauge temporale \eqref{gaugeTemporale}:
\begin{equation}
    \vnabla\cdot\vec A=0,\qquad\qquad\qquad\varphi=0\label{gaugediRadiazione}
\end{equation}
ma queste condizioni sono compatibili tra di loro solo se vale $\rho=0$. Infatti imponendo prima la condizione del gauge di Coulomb si ha la relazione:
\begin{equation*}
    \vnabla^2\varphi=-\frac{\rho}{\epsilon_0}.
\end{equation*}
Come si è mostrato discutendo il gauge di Coulomb, questa prima condizione determina univocamente $\vec A$ ma non $\varphi$. In virtù di questo fatto si può scegliere $\varphi=0$ ma dalla relazione appena ricavata si ha che deve valere anche $\rho=0$.\\

Le condizioni \eqref{gaugediRadiazione} applicate alle equazioni di Maxwell per i potenziali \eqref{MaxwellPote1} e \eqref{MaxwellPote2} risultano in una sola equazione:
\begin{flalign*}
    &\vnabla^2\vec A-\epsilon_0\mu_0\frac{\partial^2 \vec A}{\partial t^2}=0
\end{flalign*}
la soluzione di questa è un potenziale vettore che descrive un'onda elettromagnetica nel vuoto.


\section{Elettromagnetismo in forma 4-vettoriale}
Nella Sezione \ref{sec:4-equazioniMotoEM} è stato ricavato il 4-tensore elettromagnetico $F^{\mu\nu}=\partial^\mu A^\nu-\partial^\nu A^\mu\ $\footnote{In questa formula si è adottata la notazione alternativa, che verrà utilizzata anche successivamente, per le derivate parziali, già introdotta nella Sezione \ref{sec:4-vettori}, secondo la quale: $\partial^\mu=\frac{\partial\ }{\partial x_\mu}$ e $\partial_\mu=\frac{\partial\ }{\partial x^\mu}$.}:
\begin{equation*}
        F^{\mu\nu}=
    \begin{pmatrix}
        0&-E_x/c&-E_y/c&-E_z/c\\
        E_x/c&0&-B_z&B_y\\
        E_y/c&B_z&-0&-B_x\\
        E_z/c&-B_y&B_x&-0\\
 \end{pmatrix},\qquad
 F_{\mu\nu}=
 \begin{pmatrix}
     0&E_x/c&E_y/c&E_z/c\\
     -E_x/c&0&-B_z&B_y\\
     -E_y/c&B_z&-0&-B_x\\
     -E_z/c&-B_y&B_x&-0\\
\end{pmatrix}.
\end{equation*}
Questo tensore fa le veci dei campi elettromagnetici nel formalismo 4-vettoriale dello spazio-tempo della relatività.
È quindi ora opportuno ricavare anche la forma delle equazioni di Maxwell \eqref{EquazioniMaxwell} 4-vettoriale così da completare il quadro dell'elettromagnetismo nel formalismo della relatività.
\subsection{4-corrente}\label{sec:4-corrente}
Nella Sezione \ref{sec:EquazioniMaxwell} si è già illustrato come sperimentalmente la carica elettrica sia una quantità invariante sotto cambio di sistema di riferimento. Questo, essendo un fatto sperimentale, continua a valere anche nella teoria relatività. Sempre nella Sezione \ref{sec:EquazioniMaxwell}, sono state introdotte due quantità sorgenti del campo elettro magnetico: la densità di carica volumetrica e la densità di corrente superficiale. Rispettivamente:
\begin{equation}
    \rho=\lim_{\Delta V\rightarrow 0}\frac{\Delta q}{\Delta V},\qquad\qquad \vec J=\rho\vec v(t,\vec x),\label{defRhoJ}
\end{equation}
dove $\Delta q$ è la carica contenuta nel volume $\Delta V$ e $\vec v$ è il campo di velocità dato dal moto delle singole cariche che compongono la distribuzione descritta da $\rho$.\\

Si supponga di avere due sistemi di riferimento: $K^0$ nel quale tutte le cariche sono ferme e un secondo $K$ nel quale queste si muovono a velocità $\vec{v}$. In questo modo dalla \eqref{defRhoJ} si possono definire due densità di carica, $\rho_0$ in $K^0$ e $\rho$ in $K$, tali che, presa una regione di spazio $\mathcal{V}_0$ e la sua trasformata secondo un boost di Lorentz $\mathcal{V}$, si abbia:
\begin{equation*}
    Q=\int_{\mathcal{V}_0}\rho_0(t^{(0)},\vec{ x^{(0)}})\ d^3x^{(0)}=\int_{\mathcal{V}}\rho(t,\vec x)\ d^3x.
\end{equation*}
Nella Sezione \ref{sec:ContrazioneDilatazione} si è mostrato che un parallelepipedo, in seguito ad un boost di Lorentz, subisce una contrazione del suo volume di un fattore $\gamma$. Questo stesso ragionamento può essere applicato all'elemento di volume infinitesimo in coordinate cartesiane che si trasforma: $d^3x^{(0)}=d^3x\gamma$. Così facendo si ottiene, dai due integrali sopra, considerando che il dominio di integrazione è arbitrario:
\begin{equation}
    \int_{\mathcal{V}}\gamma\rho_0(\Lambda^{-1}(t,\vec{ x}))\ d^3x-\int_{\mathcal{V}}\rho(t,\vec x)\ d^3x=0\quad\Rightarrow\quad\gamma\rho_0=\rho.
\end{equation}
Questa è la trasformazione della densità di carica da un sistema solidale alle cariche ad un altro sistema di rifeirmento inerziale. Si osservi che, per questa trasformazione, $\rho_0$ è uno scalare Lorentz invariante al pari del tempo proprio (infatti in ogni sistema di riferimento inerziale è possibile, misurando $\rho$ e la velocità del moto delle cariche, ottenere il valore di $\rho_0$ e che risulta il medesimo per tutti gli osservatori).\\
Dalla definizione di densità di corrente superficiale \eqref{defRhoJ} si ha quindi che nel sistema di riferimento $K$ vale $\vec{J}=\rho\vec v=\rho_0\gamma\vec{v}$.\\
Se ora si considera il 4-vettore velocità e lo si moltiplica per $\rho_0$ si ottiene un secondo 4-vettore (poiché quest'ultima è un invariante):
\begin{equation}
    J^\mu=u^\mu\rho_0=(\rho_0c,\vec v\rho_0)\gamma=(\rho c,\vec J),\label{4-corrente}
\end{equation} 
questo 4-vettore è chiamato \emph{4-corrente}.\\
Si osservi che se si calcola la 4-divergenza della 4-corrente si ottiene:
\begin{equation}
    \partial_\mu J^\mu=\frac{\partial \rho}{\partial t}+\vnabla\cdot\vec J,\label{continuitàCarica}
\end{equation}
che è parte dell' \emph{equazione di continuità della carica}. Inoltre questa quantità è uno scalare Lorentz invariante\footnote{In questo caso accade che, trasformando la 4-corrente e gli operatori di differenziazione, si ha: $\partial'_\mu J'^\mu=\partial_\nu (\Lambda^{-1})_\mu^\nu \Lambda_\lambda^\mu J^\lambda=\partial_\nu \delta_\lambda^\nu J^\lambda=\partial_\nu J^\nu$.}, per cui se si valuta la \eqref{continuitàCarica} nel sistema $K^0$ (dove risulta nulla essendo le cariche in tale sistema immobili) si ha che deve essere nulla in ogni riferimento inerziale, così che valga in ognuno di essi l'equazione di continuità: $\partial_\mu J^\mu=0$.

\subsection{Equazioni di Maxwell 4-vettoriali}\label{sec:4-Maxwell}
Si procederà ora a ricavare la forma 4-vettoriale delle equazioni di Maxwell. Per farlo si considerino le equazioni di Maxwell in cui sono state sostituiti il potenziale $\varphi$ e il potenziale vettore $\vec A$ (ottenuti nella Sezione \ref{sec:gauge}, Eq. \eqref{MaxwellPote1} e Eq. \eqref{MaxwellPote2}):
\begin{flalign*}
    &\vnabla^2\varphi+\vnabla\cdot \frac{\partial \vec A}{\partial t}=-\frac{\rho}{\epsilon_0},\\
    &\vnabla^2\vec A-\epsilon_0\mu_0\frac{\partial^2 \vec A}{\partial t^2}-\vnabla\bigg(\vnabla\cdot\vec A +\epsilon_0\mu_0\frac{\partial \varphi}{\partial t}\bigg)=-\mu_0\vec J.
\end{flalign*}
Le restanti equazioni di Maxwell, una volta sostituiti i potenziali, si riducono a semplici identità.\\
Indicando con lettere latine i soli indici ${1,2,3}$, gli operatori di differenziazione che appaiono nelle equazioni precedenti si scrivono:
\begin{equation*}
    \vnabla=\partial_i,\qquad \vnabla^2=\partial_i\partial_i,\qquad\frac{\partial}{\partial t}=c\partial_0.
\end{equation*}
In questo modo, utilizzando il potenziale vettore $A^\mu=(\frac{\varphi}{c},\vec A)$, le due equazioni di Maxwell \eqref{MaxwellPote1} e \eqref{MaxwellPote2} assumono la forma:
\begin{flalign*}
    &\partial_i\partial_i A^0+\partial_i\partial_0A^i=-\frac{\rho}{\epsilon_0}c,\\
    &\partial_i\partial_iA^k-\frac{\epsilon_0\mu_0}{c^2}\partial_0\partial_0A^k-\partial_k\bigg(\partial_iA^i +\frac{\epsilon_0\mu_0}{c^2}\partial_0A^0\bigg)=-\mu_0J^k,
\end{flalign*}
Ricordando che $\epsilon_0\mu_0=\frac{1}{c^2}$ e che nel passare da vettori covarianti a controvarianti la matrice metrica cambia solo il segno delle componenti spaziali di un 4-vettore si ottiene:
\begin{flalign*}
    &-\partial_i\partial^i A^0+\partial_i\partial^0A^i=\partial_i\partial^0A^i-\partial_i\partial^i A^0=-\mu_0\rho c,\\
    &-\partial_i\partial^iA^k-\partial_0\partial^0A^k+\partial^k(\partial_iA^i +\partial_0A^0)=\partial^k\partial_\mu A^\mu-\partial_\mu\partial^\mu A^k=-\mu_0J^k.
\end{flalign*}
A questo punto è necessario supporre che i potenziali abbiano sempre una regolarità tale da soddisfare le ipotesi del teorema di Schwarz, così che si possa scambiare l'ordine delle derivate parziali. Così facendo, sommando (e sottraendo) nella prima equazione opportuni termini si ha:
\begin{flalign*}
    &\partial_i\partial^0A^i-\partial_i\partial^i A^0+\partial_0\partial^0A^0-\partial_0\partial^0A^0=\partial_\mu\partial^0A^\mu-\partial_\mu\partial^\mu A^0=-\mu_0\rho c,\\
    &\partial^k\partial_\mu A^\mu-\partial_\mu\partial^\mu A^k=\partial_\mu\partial ^kA^\mu-\partial_\mu\partial^\mu A^k=-\mu_0J^k.
\end{flalign*}
Infine riconoscendo in queste equazioni che $\partial^\nu A^\mu-\partial^\mu A^\nu=-F^{\mu\nu}\ $\footnote{$F^{\mu\nu}$ dalla sua definizione è antisimmetrico per cui $F^{\mu\nu}=-F^{\nu\mu}$} e $\rho c=J^0$ si trova
\begin{flalign*}
    &\partial_\mu(\partial^0A^\mu-\partial^\mu A^0)=-\partial_\mu F^{\mu 0}=-\mu_0 J^0,\\
    &\partial_\mu(\partial ^kA^\mu-\partial^\mu A^k)=-\partial_\mu F^{\mu k}=-\mu_0J^k.
\end{flalign*}
Queste se riunite in un'unica equazione rappresentano la forma 4-vettoriale delle equazioni di Maxwell:
\begin{equation}
    \partial_\mu F^{\mu\nu}=\mu_0J^\nu.\label{4-Maxwell}
\end{equation} 
Così facendo risulta evidente la covarianza di queste equazioni, che è stata richiesta in primo luogo per ricavare la relatività speciale.
\chapter{Introduzione alla teoria dei campi}

\section{Formalismo lagrangiano per la teoria dei campi}
In relatività l'interazione di una particella carica con un campo elettromagnetico è descritta da un'azione nella quale compaiono accoppiate la 4-velocità della particella e il 4-potenziale del campo. Questa è adatta a descrivere solamente particelle con carica piccola (tale da non modificare il campo elettromagnetico). Per descrivere il moto di cariche capaci di modificare il campo stesso è necessario che nell'azione relativistica compaia un termine che, tramite il principio di minima azione, determini anche la dinamica del campo. La teoria che riesce in questo intento è la \emph{teoria dei campi}.
\label{sec:lagCampi}
Seguendo di passi di Barone \cite{Barone} si vuole ora descrivere questo formalismo.
\subsection{Il principio di minima azione per un campo}
Per prima cosa si supponga che per un campo $\varphi(x^\mu)$ sia possibile costruire una funzione $\mathfrak{L}(\varphi,\partial_\mu\varphi,x^\mu)$, chiamata \emph{densità di lagrangiana}, che ne descrive le proprietà.
Si definisce allora l'azione di un campo il seguente funzionale:
\begin{equation}
    \mathcal{S}[\varphi]=\int\int_{\Omega}\mathfrak{L}(\varphi,\partial_\mu\varphi,x^\mu)\ d^3x\ dt,\label{azioneCampo}
\end{equation}
dove l'integrale è calcolato su un volume 4-dimensionale $\Omega$ \footnote{Questo volume può anche non essere finito.} nello spazio-tempo.\\
Il principio di minima azione afferma in questo caso che: supponendo che $\varphi$ e $\partial_\mu\varphi$ si annullano con sufficiente rapidità sulla frontiera $\partial\Omega$, il campo $\varphi(t,x)$ realmente osservato nel sistema è tale da rendere stazionaria l'azione \eqref{azioneCampo}. Usando questo principio si possono ricavare le equazioni di Eulero-Lagrange per un campo.\\

 Si consideri un secondo campo (detto variazione di $\varphi$) $\delta\varphi(t,x)$ che assume valori piccoli in $\Omega$ e si annulla sulla sua frontiera. Il campo $\varphi+\delta\varphi$ rispetta ancora le ipotesi del principio di minima azione. Si osservi che così facendo le coordinate e il 4-volume $\Omega$ non vengono variate. Affinché $\varphi$ sia il campo che rende stazionaria l'azione deve essere nulla la variazione $\delta\mathcal{S}$ generata da $\delta\varphi$:
\begin{equation}
    \delta\mathcal{S}=\mathcal{S}[\varphi+\delta\varphi]-\mathcal{S}[\varphi]=\int\int_{\Omega}\bigg[\frac{\partial\mathfrak{L}}{\partial \varphi}\delta\varphi+\frac{\partial\mathfrak{L}}{\partial \partial_\mu\varphi}\delta\partial_\mu\varphi\bigg]\ d^3x\ dt=0,\label{minimaAzioneCampiFull}
\end{equation}
dove l'integrando è lo sviluppo di Taylor al prim'ordine della densità di lagrangiana rispetto alle variabili che vengono variate.\\
In primo luogo si osservi che, siccome si sta considerando una sola variazione del campo e non delle coordinate, si ha
\begin{equation*}
    \delta\partial_\mu\varphi=\partial_\mu(\varphi+\delta\varphi)-\partial_\mu\varphi=\partial_\mu\varphi+\partial_\mu\delta\varphi-\partial_\mu\varphi=\partial_\mu\delta\varphi,
\end{equation*}
per cui la \eqref{minimaAzioneCampiFull} diventa:
\begin{equation}
    \delta\mathcal{S}=\int\int_{\Omega}\bigg[\frac{\partial\mathfrak{L}}{\partial \varphi}\delta\varphi+\frac{\partial\mathfrak{L}}{\partial \partial_\mu\varphi}\partial_\mu\delta\varphi\bigg]\ d^3x\ dt=0.\label{minimaAzioneCampiFull'}
\end{equation}
In più dimensioni vale la formula di integrazione per parti\footnote{Questa è data integrando: $\partial_\mu(f(x^\mu)V^\mu(x^\mu))=f(x^\mu)\partial_\mu V^\mu(x^\mu)+V^\mu(x^\mu)\partial_\mu f(x^\mu)$.} per un prodotto $V^\mu(x^\mu)\partial_\mu f(x^\mu) $ integrato su un 4-volume $\mathcal{V}$:
\begin{equation*}
    \int_{\mathcal{V} }V^\mu(x^\mu)\partial_\mu f(x^\mu)\ d^4x=\int_{\mathcal{V} }\partial_\mu(f(x^\mu)V^\mu(x^\mu))\ d^4x-\int_{\mathcal{V} }f(x^\mu)\partial_\mu V^\mu(x^\mu)\ d^4x.
\end{equation*}
Utilizzando il Teorema di Gauss nel primo addendo della parte di destra di quest'ultima equazione si ha:
\begin{equation*}
\int_{\mathcal{V} }V^\mu(x^\mu)\partial_\mu f(x^\mu)\ d^4x=\int_{\partial\mathcal{V} }f(x^\mu)V^\mu(x^\mu)n_\mu\ d\sigma-    \int_{\mathcal{V} }f(x^\mu)\partial_\mu V^\mu(x^\mu)\ d^4x,
\end{equation*}
dove $d\sigma$ è l'elemento di 4-superfice di $\partial\mathcal{V}$ e $n_\mu$ è il 4-vettore normale ad esso.\\
Integrando per parti in questo modo il secondo addendo ad integrando della \eqref{minimaAzioneCampiFull'} si ha:
\begin{equation*}
    \int\int_{\Omega}\bigg[\frac{\partial\mathfrak{L}}{\partial \varphi}\delta\varphi+\frac{\partial\mathfrak{L}}{\partial \partial_\mu\varphi}\partial_\mu\delta\varphi\bigg]\ d^3x\ dt=\int\int_{\Omega}\bigg[\frac{\partial\mathfrak{L}}{\partial \varphi}\delta\varphi-\partial_\mu\frac{\partial\mathfrak{L}}{\partial \partial_\mu\varphi}\delta\varphi\bigg]\ d^3x\ dt+\int_{\partial\Omega }\frac{\partial\mathfrak{L}}{\partial \partial_\mu\varphi}\delta\varphi\ d\sigma=0,
\end{equation*}
dove $d\sigma$ è l'elemento di 4-superfice di $\partial\Omega$ e $n_\mu$ è il 4-vettore normale ad esso.\\
Per le ipotesi ($\delta\varphi$ si annulla su $\partial\Omega$) l'ultimo addendo si annulla e si ottiene:
\begin{equation*}
    \delta\mathcal{S}=\int\int_{\Omega}\bigg[\frac{\partial\mathfrak{L}}{\partial \varphi}-\partial_\mu\frac{\partial\mathfrak{L}}{\partial \partial_\mu\varphi}\bigg]\delta\varphi\ d^3x\ dt.
\end{equation*}
Siccome questo integrale deve annullarsi per ogni variazione $\delta\varphi$ (che è arbitraria) è necessario che si annulli l'integrando. Si ottiene quindi:
\begin{equation}
    \partial_\mu\frac{\partial\mathfrak{L}}{\partial \partial_\mu\varphi}-\frac{\partial\mathfrak{L}}{\partial \varphi}=0\label{euleroLagrnageCampo}
\end{equation}
che è l'\emph{equazione di Eulero-Lagrange} per un campo scalare $\varphi$.\\Volendo ottenere le equazioni differenziali per un campo vettoriale $\varphi_\nu(x^\mu)$ è possibile considerare ogni componente come un campo scalare che è soluzione di una \eqref{euleroLagrnageCampo}. In questo modo si hanno tante equazioni di Eulero-Lagrange quante le componenti:
\begin{equation}
    \partial_\mu\frac{\partial\mathfrak{L}}{\partial \partial_\mu\varphi_\nu}-\frac{\partial\mathfrak{L}}{\partial \varphi_\nu}=0\label{euleroLagrnageCampi}.
\end{equation}
Infine si osservi che $\mathfrak{L}$ è definita a meno di una costante moltiplicativa e di una divergenza di un campo vettoriale. Infatti moltiplicando la densità di lagrangiana per una costante le equazioni differenziali che si ottengono dalla \eqref{minimaAzioneCampiFull'} non cambiano. Sommando invece a $\mathfrak{L}$ una divergenza di un campo scalare $\partial_\mu A^\mu(x^\nu)$ e usando il Teorema di Gauss si ottiene nell'azione un termine pari a:
\begin{equation*}
    \int\int_{\Omega}\partial_\mu A^\mu \ d^3x\ dt=\int_{\partial\Omega} A^\nu n_\nu\ d\sigma,
\end{equation*}
dove $d\sigma$ è l'elemento di 4-superfice di $\partial\Omega$ e $n_\mu$ è il 4-vettore normale ad esso.\\ Per ipotesi la variazione di questo termine deve annullarsi (siccome i campi da cui può dipendere $\mathfrak{L}$ devono annullarsi sulla frontiera di $\Omega$) non contribuendo alle equazioni che si ottengono dalle equazioni di Eulero-Lagrange \eqref{minimaAzioneCampiFull'}.\\

Per un sistema di più campi si definiscono i campi \emph{canonicamente coniugati}:
\begin{equation}
    \pi^\nu(x^\mu)=\frac{\partial \mathfrak{L}}{\partial\partial_{0}\varphi_\nu}(x^\mu)\label{campo coniugato}.
\end{equation}
Si definisce anche la \emph{densità di energia} tramite la trasformata di Legendre della densità di lagrangiana:
\begin{equation}
    \mathfrak{E} =\sum_{\nu}\pi^\nu\partial_0\varphi_\nu-\mathfrak{L} \label{densktà di energia}.
\end{equation}

\subsection{Il modello della corda vibrante}
Uno dei sistemi più semplici per cui è possibile utilizzare la teoria dei campi è la corda omogenea vibrante. Un sistema di questo tipo è modellizzato come un corpo elastico esteso in una dimensione e di grandezza trascurabile nelle direzioni a questa trasversali. La corda, disposta orizzontalmente come in Figura \ref{fig:corda}, può quindi oscillare verticalmente. Per descrivere tale oscillazione si definisce un campo scalare $\varphi(t,x)$ che descrive lo spostamento del punto $x$ della corda dall'equilibrio all'istante $t$.
\begin{figure}[H]
    \centering
    \begin{tikzpicture}
        \draw[pattern=north west lines, pattern color=black!30,draw=white] (0,-1.2) rectangle (-1.7,1.2);
        \draw[pattern=north west lines, pattern color=black!30,draw=white] (12,-1.2) rectangle (12+1.7,1.2);
        \draw[dashed,gray,thick](0,0)node[black,anchor=east]{$x=0$}--(12,0)node[black,anchor=west]{$x=L$};
        \draw[black,ultra thick](0,-1.2)--(0,1.2);
        \draw[black,ultra thick](12,-1.2)--(12,1.2);
        \draw[blue,thick] (0,0) .. controls (4.7,2) and (5.4,-1.5) .. (12,0);
        \draw[<->](2.3,0)node[anchor=north]{$x$}--(2.3,.6)node[anchor=south]{$\varphi(t,x)$};
    \end{tikzpicture}
    \caption{Schema del sistema corda e del campo degli spostamenti $\varphi$.}
    \label{fig:corda}
\end{figure} 
Si osservi che è necessario che la corda (di lunghezza finita) sia vincolata ai suoi estremi, infatti tale richiesta rende soddisfatta l'ipotesi che il campo degli spostamenti si annulli sulla frontiera del dominio ove esso è definito. \\

Classicamente la lagrangiana meccanica del sistema è data da un termine di energia cinetica $T(v)$ ed un termine di energia potenziale $U(x)$:
\begin{equation*}
    \mathcal{S} =\int_{t_1}^{t_2}[T-U]\ dt.
\end{equation*}
Si possono valutare questi due termini considerando il contributo di ogni punto della corda. Il campo degli spostamenti $\varphi(t,x)$ definisce il campo delle velocità di ogni punto della corda $v(t,x)=\frac{\partial \varphi}{\partial t}(t,x)$, così facendo il contributo cinetico è dato da:
\begin{equation*}
    T=\int_{0}^{L}\frac{\rho}{2}v^2(t,x)\ dx=\int_{0}^{L}\frac{\rho}{2}\bigg[\frac{\partial \varphi}{\partial t}(t,x)\bigg]^2\ dx,
\end{equation*}
dove $\rho$ è la densità di massa lineare della corda.\\
Per determinare il termine potenziale è necessario fare alcune assunzioni che semplifichino la trattazione. In primo Luogo si considererà un potenziale elastico che dipende dall'allungamento che subisce la corda. In questo modo, considerando un tratto di corda compreso tra $x$ e $x+\Delta x$, l'allungamento di tale tratto, durante un'oscillazione, è:
\begin{flalign*}
    \Delta l=\Delta x\sqrt{\bigg(\frac{\varphi(t,x+\Delta x)-\varphi(t,x)}{\Delta x}
    \bigg)^2+1}-\Delta x.
\end{flalign*}
Il potenziale del tratto $\Delta x$ di corda è invece:
\begin{equation*}
    U_{\Delta x}=\frac{k}{2}(\Delta l)^2,
\end{equation*}
dove $k$ è la constante elastica di quel preciso tratto.\\
In secondo luogo si suppone che la vibrazione della corda dia origine a piccole oscillazioni, tali che, su un tratto di corda definito come prima, valga $\varphi(t,x+\Delta x)-\varphi(t,x)\ll\Delta x$. Così facendo l'allungamento di questo tratto è approssimatile secondo Taylor con:
\begin{equation*}
    \Delta l\approx\Delta x\bigg[\bigg(\frac{\varphi(t,x+\Delta x)-\varphi(t,x)}{\Delta x}
    \bigg)^2+1\bigg]-\Delta x=\Delta x\bigg(\frac{\varphi(t,x+\Delta x)-\varphi(t,x)}{\Delta x}
    \bigg)^2.
\end{equation*}
Infine è necessario studiare cosa accade nel limite in cui questo tratto di corda tende ad essere infinitesimo, così da poter studiare le proprietà puntuali della corda e non di singoli tratti. In questo limite si ha che :
\begin{equation*}
    \lim_{\Delta x\rightarrow 0}\frac{\varphi(t,x+\Delta x)-\varphi(t,x)}{\Delta x}=\frac{\partial \varphi}{\partial x}.
\end{equation*}
Inoltre va osservato che la costante elastica $k$ dipende dal tratto di corda considerato (nella fattispecie dalla sua lunghezza siccome la corda è omogenea). Questo giustifica l'introduzione di un nuovo parametro definito come:
\begin{equation*}
    a=\lim_{\Delta x\rightarrow 0}k\Delta x.
\end{equation*}
In questo modo il contributo di energia potenziale diviene (considerando $a$ costante per omogeneità della corda)
\begin{equation*}
    U=\int_{0}^{L} \frac{a}{2}\bigg(\frac{\partial \varphi}{\partial x}\bigg)^2\ dx
\end{equation*}
e quindi l'azione del sistema diviene:
\begin{equation}
    \mathcal{S} =\int_{t_1}^{t_2}\int_{0}^{L}\bigg[ \frac{\rho}{2}\bigg(\frac{\partial \varphi}{\partial t}\bigg)^2-\frac{a}{2}\bigg(\frac{\partial \varphi}{\partial x}\bigg)^2\bigg]\ dx\ dt.
\end{equation}
In questo modo si è ottenuta una densità di lagrangiana con la quale è possibile utilizzare le equazioni di Eulero-Lagrnage, dalle quali si ottiene l'equazione differenziale delle onde su corda:
\begin{equation*}
    \frac{\partial^2 \varphi}{\partial t^2}=\frac{a}{\rho}\frac{\partial^2 \varphi}{\partial x^2}.
\end{equation*}
Le soluzioni di questa equazione sono onde che si propagano a velocità $\sqrt{\frac{a}{\rho}}$ sulla corda.
\section{Densità di lagrangiana elettromagnetica}
Una volta delineato il formalismo che sta alla base della teoria dei campi è possibile applicare i risultati ottenuti al campo elettromagnetico. Per prima cosa si procederà, seguendo i ragionamenti di Barone \cite{Barone}, identificando quale possa essere la densità di lagrangiana di tale campo.
\subsection{Campo elettromagnetico libero}
Come si è fatto per la lagrangiana meccanica della particella libera, anche in questo caso non è possibile dimostrare matematicamente quale debba essere la densità di lagrangiana corretta. Infatti tale procedimento è riservato agli esperimenti (tant'è vero che da questa si ritroveranno le equazioni di Maxwell che sono di natura sperimentale). \\
Per procedere è quindi necessario fare un'ipotesi che consentirà di determinare la forma di $\mathfrak{L}$. Infatti si è già osservato che l'azione di una particella interagente con un campo elettromagnetico esterno è uno scalare Lorentz invariante. È ragionevole supporre che quindi anche l'azione del campo elettromagnetico libero debba essere uno scalare Lorentz invariante. Nella Sezione \ref{sec:invariantiEM} si sono identificate due quantità invarianti proprie del campo elettromagnetico:
\begin{equation*}
   F^{\mu\nu} F_{\mu\nu},\qquad e^{\mu\nu\delta\lambda}F_{\mu\nu}F_{\delta \lambda}.
\end{equation*}
Si osservi però che la seconda di queste due non può essere utilizzata all'interno dell'integrale \eqref{azioneCampo}. Infatti, ricordando che il 4-tensore elettromagnetico è definito come $F^{\mu\nu}=\partial^\mu A^\nu-\partial^\nu A^\mu$, si può esprimere questa come una 4-divergenza:
\begin{flalign*}
    e^{\mu\nu\delta\lambda}F_{\mu\nu}F_{\delta \lambda}&=e^{\mu\nu\delta\lambda}(\partial_\mu A_\nu-\partial_\nu A_\mu) (\partial_\delta A_\lambda-\partial_\lambda A_\delta)\\&=e^{\mu\nu\delta\lambda}\partial_\mu A_\nu\partial_\delta A_\lambda-e^{\mu\nu\delta\lambda}\partial_\nu A_\mu\partial_\delta A_\lambda-e^{\mu\nu\delta\lambda}\partial_\mu A_\nu\partial_\lambda A_\delta+e^{\mu\nu\delta\lambda}\partial_\nu A_\mu\partial_\lambda A_\delta\\&=4e^{\mu\nu\delta\lambda}\partial_\mu A_\nu\partial_\delta A_\lambda=4e^{\mu\nu\delta\lambda}\partial_\mu A_\nu\partial_\delta A_\lambda+4e^{\mu\nu\delta\lambda} A_\nu\partial_\mu\partial_\delta A_\lambda-4e^{\mu\nu\delta\lambda} A_\nu\partial_\mu\partial_\delta A_\lambda\\&=4\partial_\mu(e^{\mu\nu\delta\lambda} A_\nu\partial_\delta A_\lambda)-4e^{\mu\nu\delta\lambda} A_\nu\partial_\mu\partial_\delta A_\lambda=4\partial_\mu(e^{\mu\nu\delta\lambda} A_\nu\partial_\delta A_\lambda),
\end{flalign*}
dove si è fatto uso del Teorema si Schwarz per riconoscere $e^{\mu\nu\delta\lambda} A_\nu\partial_\mu\partial_\delta A_\lambda=0$. Infatti $A_\nu\partial_\mu\partial_\delta A_\lambda$ è simmetrico rispetto a $\mu,\ \delta$ e risulta quindi nullo nella contrazione con un 4-tensore antisimmetrico. Come si è visto, all'interno dell'integrale d'azione \eqref{azioneCampo} termini di questo tipo non influenzano l'evoluzione del sistema.\\
Per questo motivo si può ipotizzare che la densità di lagrangiana sia:
\begin{equation}
    \mathfrak{L} =\alpha F^{\mu\nu} F_{\mu\nu},\label{lagEM}
\end{equation}
dove $\alpha$ è un fattore da determinare sperimentalmente.\\
Si possono quindi utilizzare le equazioni di Eulero-Lagrange \eqref{euleroLagrnageCampi} per ottenere le equazioni differenziali la cui soluzione determinerà il campo elettromagnetico. Si osservi in primo luogo che la lagrangiana \eqref{lagEM} è esprimibile in funzione del 4-potenziale (questo è un insieme di soli 4 campi scalari invece delle 3 componenti elettriche e 3 componenti magnetiche di $F^{\mu\nu}$):
\begin{flalign*}
    \alpha F^{\mu\nu} F_{\mu\nu}&=\alpha(\partial^\mu A^\nu-\partial^\nu A^\mu)(\partial_\mu A_\nu-\partial_\nu A_\mu)\\&=\alpha(\partial^\mu A^\nu\partial_\mu A_\nu-\partial^\mu A^\nu\partial_\nu A_\mu-\partial^\nu A^\mu\partial_\mu A_\nu+\partial^\nu A^\mu\partial_\nu A_\mu)\\&=2\alpha(\partial^\mu A^\nu\partial_\mu A_\nu-\partial^\mu A^\nu\partial_\nu A_\mu)=2\alpha(\partial^\mu A^\nu\partial_\mu A_\nu-\partial_\mu A_\nu\partial^\nu A^\mu)
\end{flalign*}
Utilizzando le equazioni \eqref{euleroLagrnageCampi} per il 4-potenziale (osservando che non vi è dipendenza esplicita da $A_\nu$) si ottiene:
\begin{flalign*}
    0=\partial_\mu\frac{\partial\mathfrak{L} }{\partial\partial_\mu A_\nu}-\frac{\partial\mathfrak{L} }{\partial A_\nu}=4\alpha\partial_\mu(\partial^\mu A^\nu-\partial^\nu A^\mu)=4\alpha\partial_\mu F^{\mu\nu}.
\end{flalign*}
Come si è visto nella Sezione \ref{sec:4-Maxwell}, le equazioni così ottenute sono proprio le equazioni di Maxwell \eqref{4-Maxwell} in assenza di cariche:
\begin{equation*}
    \partial_\mu F^{\mu\nu}=0.
\end{equation*}
\subsection{Campo elettromagnetico interagente con cariche}
Come si è fatto per la lagrangiana meccanica di corpo libero è ora necessario considerare le interazioni del campo con particelle cariche. Anche in questo caso è necessario ipotizzare la forma della densità di lagrangiana di interazione.\\
È ragionevole ipotizzare che questo termine sia il medesimo responsabile dell'interazione di una particella con un campo tramite la forza di Lorentz \eqref{lagrangianaInt}. Utilizzando la 4-corrente (introdotta nella Sezione \ref{sec:4-corrente}) è possibile riconoscere in tale termine una densità di lagrangiana, infatti:
\begin{equation*}
    \int_{t_1}^{t_2} q\dot{x}^\mu A_\mu\ dt=\int_{t_1}^{t_2} \dot{x}^\mu A_\mu\ \int_{\mathcal{V} }\rho\ d^3x\ dt=\int_{t_1}^{t_2} \int_{\mathcal{V} }\rho \dot{x}^\mu A_\mu\ d^3x\ dt=\int_{t_1}^{t_2} \int_{\mathcal{V} }J^\mu A_\mu\ d^3x\ dt,
\end{equation*}
dove $\mathcal{V}$ è il volume in cui si vuole determinare il campo e $\rho$ la densità di carica.\\
La densità di lagrangiana diventa quindi:
\begin{equation}
    \mathfrak{L}=\alpha F^{\mu\nu} F_{\mu\nu}-J^\mu A_\mu.
\end{equation}
Non resta quindi che utilizzare le equazioni di Eulero-Lagrange \eqref{euleroLagrnageCampi} per verificare se le ipotesi fatte sono corrette e per determinare $\alpha$. Il primo addendo di ogni equazione di Eulero-Lagrange restituisce lo stesso termine del caso di campo libero. Il secondo addendo è invece ora non nullo e contribuisce nel seguente modo:
\begin{equation*}
    \partial_\mu\frac{\partial\mathfrak{L} }{\partial\partial_\mu A_\nu}-\frac{\partial\mathfrak{L} }{\partial A_\nu}=4\alpha\partial_\mu F^{\mu\nu}+J^\nu=0.
\end{equation*}
Ponendo $\alpha=-\frac{1}{4\mu_0}$ queste sono le equazioni di Maxwell \eqref{4-Maxwell}. La densità di lagrangiana di un campo elettromagnetico in presenza di cariche è quindi:
\begin{equation}
    \mathfrak{L}=-\frac{1}{4\mu_0} F^{\mu\nu} F_{\mu\nu}-J^\mu A_\mu.
\end{equation}
L'azione complessiva del sistema risulta invece:
\begin{equation}
    \mathcal{S} =-\int_{t_1}^{t_2}\bigg[mc\sqrt{\dot x^\mu\dot x_\mu} +\int_{\mathcal{V} }\bigg(\frac{1}{4\mu_0} F^{\mu\nu} F_{\mu\nu}+J^\mu A_\mu\bigg)\ d^3x\bigg]\ dt.
\end{equation}
Questa determina contemporaneamente il moto di un particella carica, dovuto al campo elettromagnetico, e l'evoluzione di quest'ultimo in seguito al moto della particella.
\subsection{Campi coniugati e densità di energia}
Per concludere si vuole applicare il formalismo descritto nella Sezione \ref{sec:lagCampi} per calcolare la densità di energia di un campo elettromagnetico. Per prima cosa si determineranno i campi coniugati al 4-potenziale. Dalla definizione \eqref{campo coniugato} e ricordando la forma del 4-tensore elettromagnetico \eqref{4-tensoreEM} si ha:
\begin{flalign*}
    \pi^\nu&=-\frac{1}{4\mu_0} \frac{\partial F^{\mu\nu} F_{\mu\nu}}{\partial\partial_{0}A_\nu}=-\frac{1}{4\mu_0} \frac{\partial }{\partial\partial_{0}A_\nu}[(\partial^\mu A^\nu-\partial^\nu A^\mu)(\partial_\mu A_\nu-\partial_\nu A_\mu)]\\&=-\frac{1}{4\mu_0} \frac{\partial }{\partial\partial_{0}A_\nu}(\partial^\mu A^\nu\partial_\mu A_\nu-\partial^\mu A^\nu\partial_\nu A_\mu-\partial^\nu A^\mu\partial_\mu A_\nu+\partial^\nu A^\mu\partial_\nu A_\mu)\\&=-\frac{1}{2\mu_0} \frac{\partial }{\partial\partial_{0}A_\nu}(\partial^\mu A^\nu\partial_\mu A_\nu-\partial^\mu A^\nu\partial_\nu A_\mu)=-\frac{1}{2\mu_0} \frac{\partial }{\partial\partial_{0}A_\nu}(\partial^\mu A^\nu\partial_\mu A_\nu-\partial_\mu A_\nu\partial^\nu A^\mu)\\&=-\frac{1}{4\mu_0} (\partial^0A^\nu-\partial^\nu A^0)=-\frac{1}{\mu_0} F^{0\nu}=\frac{1}{\mu_0} \bigg(0,\frac{\vec E}{c}\bigg).
\end{flalign*}
Per proseguire nel calcolo della densità di energia è necessario valutare il prodotto scalare $\pi^\mu\partial_0 A_\mu$. Questo conto può essere semplificato utilizzando un gauge opportuno (Sezione \ref{sec:gauge}). Il gauge temporale ($\varphi=0$) infatti è tale da far valer la relazione $E^k=-\partial_t A^k$ (siccome, come detto nella Sezione \ref{sec:LagEMInt}, in generale vale $\vec E=-\vec{\nabla}\varphi-\partial_t \vec A$). Si ha quindi:
\begin{equation*}
    \pi^\mu\partial_0 A_\mu=\frac{1}{\mu_0}E_k\frac{1}{c}\partial_t A_k=\frac{\epsilon_0\mu_0}{\mu_0}E_kE_k=\epsilon_0 |\vec E|^2.
\end{equation*}
Calcolando a questo punto la trasformata di Legendre della densità di lagrangiana del campo libero si ottiene che la densità di energia è:
\begin{flalign*}
    \mathfrak{E} &=\pi^\mu\partial_0 A_\mu-\mathfrak{L} =\pi^\mu\partial_0 A_\mu+\frac{1}{4\mu_0} F^{\mu\nu} F_{\mu\nu}=\epsilon_0 |\vec E|^2+\frac{2}{4\mu_0}\bigg(|\vec B|^2-\frac{|\vec E|^2}{c^2}\bigg)\\&=\epsilon_0 |\vec E|^2+\frac{1}{2\mu_0}|\vec B|^2-\frac{\epsilon_0\mu_0}{2\mu_0}|\vec E|^2=\frac{1}{2}\epsilon_0|\vec E|^2+\frac{1}{2\mu_0}|\vec B|^2.
\end{flalign*}
Un campo elettromagnetico in una regione di spazio $\mathcal{V} $ possiede quindi un'energia data da:
\begin{equation}
    E=\int_{\mathcal{V} }\bigg(\frac{1}{2}\epsilon_0|\vec E|^2+\frac{1}{2\mu_0}|\vec B|^2\bigg)\ d^3x.
\end{equation}
Questo risultato è coerente quanto già noto dalla teoria di Maxwell.

\appendix
\input{Appendici/Sulla linearità delle trasformazioni di Lorentz}


\chapter*{Ringraziamenti}
\addcontentsline{toc}{chapter}{Ringraziamenti}
\markboth{Ringraziamenti}{Ringraziamenti}
Al termine di questi tre anni non posso non ringraziare tutte le persone che mi hanno accompagnato fino ad oggi. Primi tra tutti desidero ringraziare i miei genitori: fin da quando ero bambino hanno coltivato la mia curiosità da cui è nata la passione per la fisica. Grazie di cuore per l'amore che mi avete dato supportandomi sempre, specialmente quando ho incontrato le prime difficoltà della vita.\\Voglio ringraziare anche il professor Paolo Albano, che mi ha seguito nella preparazione di questa tesi, per tutto il tempo e le attenzioni che mi ha dedicato. Grazie al suo lavoro e alla sua disponibilità di confronto non solo ho avuto modo di apprendere gli argomenti trattati in questa sede ma ho anche potuto approfondire e consolidare molti aspetti fondamentali di quanto ho studiato in questi anni. Da lei ho anche imparato come approcciarsi ad un lavoro come questo e a come scriverlo affinché risulti chiaro e comprensibile.\\ Desidero ringraziare Ester, Luca Bergamini e Samuele che, fin da prima dell'inizio dell'università, sono sempre stati al mio fianco con la loro amicizia. La vostra vicinanza mi ha aiutato non solo durante la pandemia ma anche nel momento in cui ho dovuto lasciare casa per venire a vivere da solo a Bologna. Un particolare ringraziamento va a Sua Eccellenza Gianni Sacchi che in questi tre anni mi ha supportato specialmente nei momenti di crisi personale. Grazie a tutti voi di essere sempre stati per me un porto sicuro in cui fare ritorno.\\Desidero ringraziare tutti gli amici che ho incontrato a Bologna. Per primo Marco Piolanti che, tra gli alti e i bassi delle nostre vite universitarie, è sempre stato un vero amico su cui poter contare. Grazie per tutte le serate passate assieme e tutte le nostre infinite discussioni filosofiche. Non posso dimenticare Cristina con cui ho abitato in questi anni. Grazie per avermi sopportato come coinquilino ed esserti prestata come cavia per i miei esperimenti di cucina. Ringrazio anche Andrea, Damiano, Federico, Gabriele, Lorenzo, Luca, Matteo Bonazzi, Matteo Zandi, Marco Benazzi e Marco Mullazani con cui ho condiviso questo cammino di lezione in lezione, di esame in esame e di giornata in giornata. Tutti i pranzi, tutti i caffè al ginseng e tutti gli integrali complessi che abbiamo condiviso rimarranno per sempre un ricordo indelebile di questi anni. Infine, ma non per importanza, voglio ringraziare tutti i ragazzi del Treno dei Clochard. Ogni venerdì sera mi avete accolto come amico condividendo con me le vostre settimane. 


\bibliographystyle{plain}
\bibliography{ref}
\addcontentsline{toc}{chapter}{Bibliografia}

\end{sloppypar}
\end{document}
