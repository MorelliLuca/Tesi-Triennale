\documentclass[12pt,a4paper]{report}

\usepackage[italian]{babel}

\usepackage{newlfont}

\usepackage{color}
\textwidth=450pt\oddsidemargin=0pt

\usepackage{graphicx}

\usepackage{titlesec}

\usepackage{pgfplots}
\pgfplotsset{compat=1.16}

\usepackage[none]{hyphenat}

\usepackage{amssymb}

\usepackage{amsmath}

\usepackage{amsthm}
\newtheorem{lemma}{Lemma}
\newtheorem{sublemma}{Lemma}[section]
\numberwithin{equation}{section}

\usepackage{multicol}

\usepackage[margin=1.2in]{geometry}

\usepackage{fancyhdr}
\pagestyle{fancy}
\fancyhead[RO,LE]{\textbf{Titolo Tesi}}
\fancyhead[LO,RE]{Luca Morelli}

\newcommand{\vnabla}{\vec{\nabla}}

\begin{document}

\begin{titlepage}

\begin{center}
	{{\Large{\textsc{Alma Mater Studiorum $\cdot$ Universit\`a di Bologna}}}} 
	\rule[0.1cm]{15.8cm}{0.1mm}
	\rule[0.5cm]{15.8cm}{0.6mm}
	\\\vspace{3mm}
	
	{\small{\bf Scuola di Scienze \\ 
			Dipartimento di Fisica e Astronomia\\
			Corso di Laurea in Fisica}}
	
\end{center}

\vspace{23mm}

\begin{center}\textcolor{black}{
		%
		% INSERIRE IL TITOLO DELLA TESI
		%
		{\LARGE{\bf Aspetti fisici e matematici della teoria della relatività ristretta}}\\
}\end{center}

\vspace{50mm} \par \noindent

\begin{minipage}[t]{0.47\textwidth}
	%
	% INSERIRE IL NOME DEL RELATORE CON IL RELATIVO TITOLO DI DOTTORE O PROFESSORE
	%
	\large{\bf Relatore: \vspace{2mm}\\\textcolor{black}{
				Prof. Paolo Albano}}\\\\
\end{minipage}
%
\hfill
%
\begin{minipage}[t]{0.47\textwidth}\raggedleft \textcolor{black}{
		{\large{\bf Presentata da:
				\vspace{2mm}\\
				%
				% INSERIRE IL NOME DEL CANDIDATO
				%
				Luca Morelli}}}
\end{minipage}

\vspace{40mm}

\begin{center}
	%
	% INSERIRE L'ANNO ACCADEMICO
	%
	Anno Accademico \textcolor{black}{ 2022/2023}
\end{center}


\end{titlepage}

\vspace*{10pt}
\begin{center}
	\large\textbf{Abstract}\normalsize
\end{center}
\vspace*{10pt}
\addcontentsline{toc}{section}{Abstract}
\markboth{Abstract}{Abstract}
\begin{changemargin}{1cm}{1cm}
Questa tesi contiene un'esposizione della teoria della relatività ristretta. Partendo dalla fisica di fine ottocento si è descritto lo sviluppo della teoria a partire dal lavoro di Einstein del 1905.  In seguito si sono considerate le principali caratteristiche possedute dalla formulazione di una meccanica coerente con la teoria della relatività. Per concludere si è data un'esposizione relativistica dell'elettromagnetismo terminando con una breve introduzione alla teoria dei campi.
Si è prestata particolare attenzione ad alcuni aspetti matematici della teoria prevalentemente riconducibili ai concetti di invarianza e simmetria.
\end{changemargin}

\tableofcontents

\chapter*{Introduzione}   
\addcontentsline{toc}{section}{Introduzione}

La fisica sviluppatasi fino alla fine dell'ottocento descriveva con precisione i fenomeni meccanici noti sulla base di una serie di principi sperimentali. Queste osservazioni sperimentali definiscono classicamente i concetti di spazio e tempo e i sistemi di riferimento inerziali (particolari sistemi nei quali le leggi della fisica sono sempre le medesime).\\
Contemporaneamente, gli studi ottocenteschi sull'elettromagnetismo avevano consentito di modellizzare tutti i fenomeni elettromagnetici osservati tramite le equazioni di Maxwell (una serie di equazioni differenziali che descrivono le proprietà dei campi elettrici e magnetici). Queste equazioni e nella fattispecie le loro soluzioni d'onda (che sono le onde elettromagnetiche come la luce visibile) misero in evidenza l'incompatibilità di questa teoria di Maxwell con la meccanica classica. In modo particolare ci si rese conto che le equazioni di Maxwell non risultano identiche in ogni sistema di riferimento inerziale (secondo la meccanica classica).\\
Nel 1905 Albert Einstein propose, nel suo articolo \emph{"Zur Elektrodynamik bewegter Körper"} \cite{Einstein1905}, un'interpretazione risolutiva degli esperimenti che cercavano di chiarire quale teoria fosse corretta tra la meccanica classica e l'elettromagnetismo di Maxwell. Einstein riformulò i principi sperimentali alla base della meccanica postulando la costanza della velocità della luce in ogni sistema di riferimento inerziale (come suggerivano le equazioni di Maxwell). Sulla base dei nuovi principi di Einstein (noti come postulati) è possibile costruire l'intera teoria della relatività ristretta. Il fulcro di questa teoria è costituito dalle trasformazioni di Lorentz che consentono di passare da un sistema di riferimento inerziale ad un secondo.\\
Lo sviluppo di questa teoria modificò radicalmente le idee di spazio e tempo radicate nella fisica, enti assoluti secondo la concezione newtoniana. Inoltre si capì che non è più possibile fare distinzione tra spazio e tempo ma è necessario intenderli come un unico spazio-tempo dotato di un suo preciso formalismo.\\
Nel Capitolo 1 si approfondiranno tutti questi aspetti.\\

Una volta sviluppati i fondamenti della teoria della relatività fu necessario delineare la meccanica che descrive il moto dei corpi coerentemente con i postulati di Einstein. Nel Capitolo 2 si procederà a dare una descrizione della meccanica relativistica nel formalismo lagrangiano per una particella libera e per sistemi di più particelle.\\

L'elettromagnetismo, la cui descrizione maxwelliana risultò corretta, si rivela quindi come la naturale descrizione di una interazione relativistica che tenga conto di quanto scoperto da Einstein. Per questo motivo nel Capitolo 3 si descriverà come tradurre nel formalismo dello spazio-tempo relativistico tutti i concetti propri dell'elettromagnetismo: dalle cariche e le correnti fino a i campi elettrici e magnetici e i loro potenziali.\\

Per concludere, nel Capitolo 4, si farà una breve introduzione della teoria dei campi e di come questa possa essere utilizzata per descrivere i campi elettromagnetici. Nella fattispecie si ricaverà la densità di lagrangiana del campo elettromagnetico e si mostrerà come questa consenta di ottenere le equazioni di Maxwell.

\chapter{Fatti di Fisica Classica}
Nella seconda metà del diciannovesimo secolo la fisica era costituita fondamentalmente dalla meccanica, 
dalla termodinamica e dall'elettromagnetismo.\\ La meccanica, fondata da Newton e Galileo, descriveva il 
moto dei corpi e di fatto fungeva da modello per tutta la fisica.\\L'elettromagnetismo invece, in seguito 
a numerosi esperimenti, aveva trovato una completa spiegazione nelle Equazioni di Maxwell.\\
Poichè la nascita della Teoria della Relatività Speciale è strettamente connessa a queste due branche della fisica 
è necessario illustrare brevemente i loro fondamenti. 


\subsection{La meccanica e le trasformazioni di Galileo}
Come fondamento delle meccanica classica vi sono un serie di osservazioni di carattere sperimentale, queste possono essere utilizzate come assiomi da cui dedurre le leggi del moto dei corpi.\\

Il primo fatto sperimentale che viene assunto è che lo spazio sia tridimensionale, isotropo, omogeneo e che rispetti la geometria euclidea mentre il tempo sia ad una sola dimensione. Queste assunzioni, come si è visto nella sezione \ref{sec:MathSDRI}, consentono di definire cosa sia un sistema di riferimento. Inoltre si prende come assioma che le distanze spaziali e temporali siano assolute, ossia che ogni osservatore concordi sulla misura di queste.\\

Il secondo fatto sperimentale prende il nome di principio di relatività\footnote{In meccanica classica è anche detto principio di relatività galileiano} e, basandosi sulla nozione di sistema di riferimento inerziale (sezione \ref{sec:MathSDRI}), afferma che: \emph{in ogni sistema di riferimento inerziale tutte le leggi della fisica sono identiche}.\\

Infine si assume che, in un riferimento inerziale, le posizioni e le velocità dei punti di un sistema ad un tempo iniziale ne determinino in maniera univoca l'evoluzione $\vec x(t)\in \mathbb{R}^3$ secondo la legge:
\begin{equation}
	m\ddot{\vec{x}}(t)=\vec{F}(t,\vec{x},\dot{\vec{x}})
	\label{equazioneDiNewton}
\end{equation}
dove $m$ è detta massa inerziale e $\vec{F}$ è una funzione caratteristica del sistema detta forza.\\

Si vuole quindi identificare quali applicazioni $\varphi:\mathbb{R}^3\times\mathbb{R}\rightarrow
\mathbb{R}^3\times\mathbb{R}$ consentono di cambiare sistema di riferimento inerziale, ossia quali 
trasformazioni non variano le leggi della natura. Queste applicazioni si chiamano trasformazioni di 
Galileo e, come si è dimostrato nella sezione \ref{sec:MathSDRI}, per soddisfare il principio di relatività devono essere applicazioni affini.
Una generica trasformazione di Galileo è quindi data dalla composizione di tre famiglie di applicazioni:
\begin{itemize}
	\item una generica traslazione spazio temporale dell'origine, dedotta dalla proprietà 
	di omogeneità dello spazio e del tempo:
	\begin{equation}
		\varphi_{\vec{r},s}(t,\vec{x})=(t+s,\vec{x}+\vec{r})
		\label{GalileoTraslazoine}
	\end{equation} 
\item una generica rotazione degli assi spaziali, dovuta alla proprietà di isotropia dello spazio:
\begin{equation}
	\varphi_{G}(t,\vec{x})=(t,G\vec{x}) \qquad G\in M_{3\times3}(\mathbb{R}):G^{-1}=G^t
	\label{GalileoRotazione}
\end{equation} 
	\item una traslazione di moto rettilineo uniforme, ammissibile grazie alle proprietà di muoversi di tale moto dei sistemi 
	di riferimento inerziali:
\begin{equation}
	\varphi_{\vec{v}}(t,\vec{x})=(t,\vec{x}+\vec{v}t)
	\label{GalileoVelocità}
\end{equation} 
\end{itemize}
Quest'ultima tipologia di trasformazione è quella che viene comunemente studiata per caratterizzare le trasformazioni di sistemi inerziali. 
Si consideri quindi un sistema $K$, con coordinate $(t,\vec{x})$ e un sistema $K'$, con 
coordinate $(t',\vec{x'})$, in moto a velocità $\vec{V}$ rispetto a $K$, si scriverà allora la 
(\ref{GalileoVelocità}) come:
\begin{equation}
	\vec{x'}=\vec{x}-\vec{V}t \qquad t'=t
	\label{GalileoEasy}
\end{equation}
Si noti che questa trasformazione non muta la forma del vettore accelerazione $\ddot{\vec{x}}$, infatti considerando una traiettoria $\vec{x}(t)$:
\begin{equation*}
	\frac{d^2}{dt'^2}(\vec{x'}(t))=\frac{d^2}{dt^2}(\vec{x}(t)-\vec{V}t)=\ddot{\vec{x}}(t)+\frac{d}{dt}(\vec{V})=\ddot{\vec{x}}(t)
\end{equation*} 
questo fatto, assieme al principio di relatività galileiano applicato alla legge di Newton (\ref{equazioneDiNewton}), 
impone che le forze esercitate su di un punto e misurate in due sistemi inerziali differenti debbano essere le medesime. \\

Analogamente a quanto appena fatto si possono ricavare le trasformazioni per la velocità di un punto in moto con traiettoria $\vec{x}(t)$:
\begin{equation}
	\frac{d}{dt'}(\vec{x'}(t))=\frac{d}{dt}(\vec{x}(t)-\vec{V}t)=\dot{\vec{x}}(t)+\vec{V}
\end{equation}
Per cui si ottiene dalle trasformazioni di Galileo che le velocità si compongono per somma algebrica.\\

Per determinare l'invarianza di un legge fisica rispetto alle trasformazioni (\ref{GalileoEasy}) può essere necessario 
studiare come si trasformino gli operatori di differenziazione. 
Se si vuole derivare una $f(\vec{x},t)$, dalla regola di Leibniz, si ha:
\begin{equation*}
	\begin{gathered}
		\frac{\partial}{\partial t'}=\frac{\partial t}{\partial t'}\frac{\partial}{\partial t}+
		\sum_{i=1}^{3}\frac{\partial x_i}{\partial t}\frac{\partial t}{\partial t'}
		\frac{\partial}{\partial x_i} \\
		\frac{\partial}{\partial x'_i}=\frac{\partial t}{\partial x'_i}\frac{\partial}{\partial t}+
		\sum_{i=1}^{3}\frac{\partial x_j}{\partial x'_i}\frac{\partial}{\partial x_j}
	\end{gathered}
\end{equation*}
dove $\frac{\partial}{\partial t}$ è la derivata parziale rispetto al tempo nel sistema $K$, $\frac{\partial}{\partial t'}$ è la derivata parziale rispetto al tempo nel sistema $K'$, $\frac{\partial}{\partial x_i}$ è la derivata parziale rispetto alla coordinata i-esima del sistema $K$ e $\frac{\partial}{\partial x_i'}$ è la derivata parziale rispetto alla coordinata i-esima del sistema $K'$.\\
Osservando dalla (\ref{GalileoEasy}) che $\frac{\partial t}{\partial t'}=1$, che 
$\frac{\partial t}{\partial x'_i}=0$, che $\frac{\partial x_i}{\partial t}=V_i$ e che 
$\frac{\partial x_j}{\partial x'_i}=\delta_{ij}$, dove $\delta_{ij}$ è una delta di Kronecker, 
si ottengono le trasformazioni degli operatori di differenziazione desiderate:
\begin{equation}
	\frac{\partial}{\partial t'}=\frac{\partial}{\partial t}+\vec{V}\cdot\vec{\nabla} \qquad \qquad
	\frac{\partial}{\partial x'_i}=\frac{\partial}{\partial x_i}
	\label{GalileoDifferenziale}
\end{equation}
dove $\vnabla=(\frac{\partial }{\partial x},\frac{\partial }{\partial y},\frac{\partial }{\partial z})$.\\
Dalla (\ref{GalileoDifferenziale}) si deduce che tutti gli operatori che comprendono solamente derivate 
rispetto alle coordinate spaziali sono lasciati inalterati dalle trasformazioni di Galileo. In particolare si ha che $\vnabla'=(\frac{\partial }{\partial x'},\frac{\partial }{\partial y'},\frac{\partial }{\partial z'})=\vnabla$.
\section{L'elettromagnetismo e le equazioni di Maxwell}
Per interpretare i fenomeni elettromagnetici, anche in questo caso, è necessario introdurre
una serie di osservazioni sperimentali: in primo luogo esiste una proprietà della materia 
detta carica elettrica, che non dipende dal sistema di riferimento in cui è misurata, che consente 
ai corpi di interagire con due campi vettoriali: 
il campo Elettrico $\vec{E}$ e il campo Magnetico $\vec{B}$.\\ Un corpo puntiforme di carica 
$q$ interagendo con questi subisce un forza data da:
\begin{equation}
	\vec{F}=q(\vec{E}(\vec{x},t)+\vec{\dot{x}}\wedge\vec{B}(\vec{x},t))
	\label{ForzaLorentz}
\end{equation}
In secondo luogo gli esperimenti mostrarono che questi due campi rispettano una serie di equazioni 
dette Equazioni di Maxwell:
\begin{equation}
	\begin{gathered}
		\vec{\nabla}\cdot\vec{E}=\frac{\rho}{\epsilon_0} \qquad \qquad \vec{\nabla}\cdot\vec{B}=0 \\
		\vec{\nabla}\wedge\vec{E}=-\frac{\partial\vec{B}}{\partial t} \qquad \qquad \vec{\nabla}\wedge
		\vec{B}=\mu_0\vec{J}+\epsilon_0\mu_0\frac{\partial\vec{E}}{\partial t}
		\label{EquazioniMaxwell}
	\end{gathered}
\end{equation}
dove $\rho$ è la densità di carica volumetrica, $\vec{J}$ è la densità di corrente superficiale e 
$\epsilon_0$, $\mu_0$ sono due costanti del vuoto.\\

Dalle Equazioni di Maxwell (\ref{EquazioniMaxwell}) segue che le cariche sono sorgenti del campo 
Elettrico mentre le correnti lo sono per 
il campo Magnetico.\\ 

Inoltre si può ottenere, calcolando il rotore di ambo 
i membri delle ultime due
\begin{equation*}
	\vec{\nabla}\wedge\vec{\nabla}\wedge\vec{E}=-\frac{\partial}{\partial t}\vnabla\wedge\vec{B} 
	\qquad \vec{\nabla}\wedge\vec{\nabla}\wedge\vec{B}=\mu_0\epsilon_0\frac{\partial}{\partial t}
	\vnabla\wedge\vec{E}
\end{equation*}
e supponendo assenza di cariche per cui $\rho=0$ e $\vec{J}=0$, due equazioni che descrivono 
onde di campo Elettrico e Magnetico nel vuoto:
\begin{equation}
	\vnabla^2\vec{E}=\mu_0\epsilon_0\frac{\partial^2\vec{E}}{\partial t^2} \qquad \vnabla^2\vec{B}=
	\mu_0\epsilon_0\frac{\partial^2\vec{B}}{\partial t^2}
\end{equation}

Queste onde si propagano con una velocità $\frac{1}{\sqrt{\mu_0\epsilon_0}}=2.99795\ \ 10^8\  \frac{m}{s}$, 
che corrisponde con precisione ai valori sperimentalmente misurati della velocità della luce. 
Sulla base della teoria ondulatoria classica è però necessario identificare un mezzo nel quale queste onde possano 
propagarsi e rispetto al quale la loro velocità di propagazione deve essere intesa. Per questo motivo alla fine dell'ottocento venne ipotizzata l'esistenza di tale mezzo detto Etere Luminifero.

 

\chapter{L'articolo del 1905}
La Meccanica Newtoniana e l'Elettromagnetismo di Maxwell si rivelarono a gli occhi dei fisici dell'ottocento 
incompatibili fra loro, 
poichè le Equazioni di Maxwell non risultarono invarianti per le trasformazioni di Galileo. 
Proprio per questo motivo i fisici dell'epoca dovettero rivalutare i principi alla base delle leggi 
della natura fino a quando l'incompatibilità trovò una soluzione nel 1905 con la Relatività Ristretta di Einstein.
 Si ripercorreranno ora i passi che Einstein stesso indicò nel suo articolo del 1905 \cite{Einstein1905}.
\subsection{La non invarianza delle equazioni di Maxwell}
\label{Sec:nonInvMax}
 Si considerino due sistemi di riferimento $K$ e $K'$, inerziali e reciprocamente in moto in maniera tale che in $K$ l'origine del sistema $K'$ risulti in moto a velocità costante $\vec{V}$. Allora in ogni sistema si misureranno rispettivamente $\vec{E},\ \vec{B}$ e $\vec{E'},\ \vec{B'}$.\\ Il principio di relatività galileiana impone che questi due campi, nei loro sistemi di riferimento, soddisfino le equazioni di Maxwell (\ref{EquazioniMaxwell}). Inoltre, siccome la forza di Lorentz (\ref{ForzaLorentz}) deve essere la medesima in tutti i sistemi di riferimento inerziali, considerando una carica $q$ in moto con velocità $\vec{v}$ in $K$ e $\vec{v}-\vec{V}$ in $K'$, essendo $q$ invariante deve valere:
 \begin{flalign*}
		\vec{F'}=\vec{F}\quad&\Rightarrow\quad \vec{E'}+(\vec{v}-\vec{V})\wedge\vec{B'}=\vec{E}+\vec{v}\wedge\vec{B}\\
         &\Rightarrow\quad \vec{E'}+\vec{v}\wedge(\vec{B'}-\vec{B})=\vec{E}+\vec{V}\wedge\vec{B'}.
 \end{flalign*}
Così facendo si possono ottenere delle relazioni tra $\vec{E},\ \vec{B}$ e $\vec{E'},\ \vec{B'}$ che costituiscono quindi le trasformazioni dei campi elettrici e magnetici tra sistemi inerziali secondo le prescrizioni della meccanica classica. Queste però possono dipendere esclusivamente dalle proprietà dei due sistemi di riferimento considerati e nella fattispecie dalla loro velocità reciproca. Per questo motivo, il termine contenente $\vec{v}$, ossia la velocità della carica nel sistema $K$, deve annullarsi. Quindi le trasformazioni risultano:
\begin{equation}
	\begin{cases}
		\vec{E'}(t,\vec{x'})=\vec{E}(t,\vec{x})+\vec{V}\wedge\vec{B}(t,\vec{x})\\
		\vec{B'}(t,\vec{x'})=\vec{B}(t,\vec{x})
	\end{cases}.
	\label{TrasfGalileiEB}
\end{equation}
Si considerino ora le leggi di trasformazione delle grandezze generatici dei campi: $\rho$ e $\vec{J}$. Sia $\Delta V$ un volume in cui è presente una carica $\Delta q$. Allora, la densità di carica è definita come:
\begin{equation*}
	\rho=\lim_{\Delta V\rightarrow 0}\frac{\Delta q}{\Delta V}.
\end{equation*} Siccome le lunghezze sono assunte "assolute" allora devono esserlo pure i volumi e, essendo la carica non dipendente dal sistema di riferimento, si conclude che pure la densità di carica non dipende dal sistema di riferimento. Per quanto riguarda la densità di corrente superficiale, definita come $\vec{J}=\rho\vec{v}$,  è sufficiente applicare le trasformazioni delle velocità tra due sistemi in moto reciproco a velocità $\vec{V}$ per avere:
\begin{equation}
	\vec{J'}=\vec{J}-\rho\vec{V}.
\end{equation}
I risultati appena ottenuti consentono di studiare l'invarianza o meno delle equazioni di Maxwell per le trasformazioni di Galilei.\\

Si si supponga valida in $K$ la prima equazione di Maxwell e si studi la medesima in $K'$. Trasformando $E'$ in $E$ e gli operatori di differenziazione secondo la (\ref{GalileiDifferenziale}), si ottiene una quantità che deve annullarsi (ossia ricondursi ad un'identità) affinché sia valida questa equazione anche in $K'$, come richiesto dal principio di relatività.
\begin{flalign*}
	\vnabla'\cdot\vec{E'}-\frac{\rho}{\epsilon_0}&=\vnabla\cdot(\vec{E}+\vec{V}\wedge\vec{B})-\frac{\rho}{\epsilon_0}\\
	&=\left(\vnabla\cdot\vec{E}-\frac{\rho}{\epsilon_0}\right)-\vec{V}\cdot(\vnabla\wedge\vec{B})=-\vec{V}\cdot(\vnabla\wedge\vec{B}).
\end{flalign*}
Il primo termine tra parentesi è identicamente nullo (poiché valgono le equazioni di Maxwell in $K$) mentre l'ultimo termine non è sempre nullo, nella fattispecie in presenza di campi elettrici variabili nel tempo. Questo implica che quindi la prima equazione di Maxwell non sia invariante per le trasformazioni di Galilei.\\

Se si studia la seconda con lo stesso procedimento si scopre che questa invece è invariante:
\begin{equation*}
	\vnabla'\cdot\vec{B'}=\vnabla\cdot\vec{B}=0.
\end{equation*}

Analogamente per la terza equazione:
\begin{flalign*}
	\vnabla'\wedge\vec{E'}+\frac{\partial \vec{B'}}{\partial t'}&=
	\vnabla\wedge\vec{E'}+\frac{\partial \vec{B'}}{\partial t}+(\vec{V}\cdot\vnabla)\vec{B'}\\
	&=\vnabla\wedge(\vec{E}+\vec{V}\wedge\vec{B})+\frac{\partial \vec{B}}{\partial t}+(\vec{V}\cdot\vnabla)\vec{B}\\
	&=\left(\vnabla\wedge\vec{E}+\frac{\partial \vec{B}}{\partial t}\right)+\vnabla\wedge\vec{V}\wedge\vec{B}+(\vec{V}\cdot\vnabla)\vec{B}=0.
\end{flalign*} 
Infatti il termine tra parentesi è identicamente nullo (poiché in $K$ vale la terza equazione di Maxwell). Inoltre, ricordando che $\vec{V}$ è costante e usando le regole di differenziazione:
\begin{equation*}
	\vnabla\wedge\vec{V}\wedge\vec{B}+(\vec{V}\cdot\vnabla)\vec{B}=-(\vec{V}\cdot\vnabla)\vec{B}+(\vec{V}\cdot\vnabla)\vec{B}=0.
\end{equation*}
L'ultima equazione di Maxwell è invece non-invariante. Infatti si ottiene:
\begin{flalign*}
	&\vnabla '\wedge\vec{B'}-\mu_0\epsilon_0\frac{\partial \vec{E'}}{\partial t'}-\mu_0\vec{J'}=
	\vnabla\wedge\vec{B'}-\mu_0\epsilon_0\frac{\partial \vec{E'}}{\partial t}-\mu_0\epsilon_0(\vec{V}\cdot\vnabla)\vec{E'}-\mu_0\vec{J'} &&\\
	&=\vnabla\wedge\vec{B}-\mu_0\epsilon_0\frac{\partial \vec{E}}{\partial t}-\mu_0\epsilon_0\vec{V}\wedge\frac{\partial \vec{B}}{\partial t}-\mu_0\epsilon_0(\vec{V}\cdot\vnabla)(\vec{E}+\vec{V}\wedge\vec{B})-\mu_0\vec{J}+\mu_0\vec{V}\rho &&\\
	&=\left(\vnabla\wedge\vec{B}-\mu_0\epsilon_0\frac{\partial \vec{E}}{\partial t}-\mu_0\vec{J}\right)-\mu_0\epsilon_0\vec{V}\wedge\frac{\partial \vec{B}}{\partial t}-\mu_0\epsilon_0(\vec{V}\cdot\vnabla)(\vec{E}+\vec{V}\wedge\vec{B})+\mu_0\vec{V}\rho &&\\
	&=-\mu_0\epsilon_0\vec{V}\wedge(-\vnabla\wedge\vec{E})-\mu_0\epsilon_0(\vec{V}\cdot\vnabla)(\vec{E}+\vec{V}\wedge\vec{B})+\mu_0\vec{V}\rho &&\\
	&=-\mu_0\epsilon_0(\vec{V}\cdot\vnabla)(\vec{V}\wedge\vec{B})+\mu_0\vec{V}\rho
\end{flalign*}
che, con le assunzioni fin qui fatte, non è nullo in generale.\\

Si è quindi giunti alla conclusione che la teoria di Maxwell e la meccanica di Newton non sono conciliabili. Infatti le trasformazioni di Galieli non consentono di avere contemporaneamente invarianza della forza e delle equazioni di Maxwell.
\subsection{I postulati di Einstein}\label{Sec:postulati}
Per giungere ad una formulazione coerente della dinamica dei corpi carichi Einstein, nel suo articolo del 1905$^{\cite{Einstein1905}}$, propose di modificare gli assunti alla base della meccanica classica per prediligere un modello coerente con l'elettromagnetismo di Maxwell.\\Infatti all'epoca erano noti alcuni risultati sperimentali che giocavano a sfavore della concezione classica della meccanica, primo fra tutti l'esperimento di Michelson e Morley che avrebbe dovuto consentire di misurare la velocità della Terra rispetto all'etere luminifero, detta vento d'etere. L'esperimento ebbe un esito inaspettato, infatti non fu possibile misurare alcun vento d'etere portando i fisici a tre possibili spiegazioni: o l'Etere si muove assieme alla Terra, o l'apparato sperimentale si contrae lungo la direzione del moto terrestre oppure non esiste alcun Etere e la luce si propaga alla medesima velocità in ogni direzione e per ogni osservatore.\\

Einstein pose quindi a fondamento della sua teoria due postulati:
\begin{itemize}
    \item Il principio di relatività: basato sull'assunzione che esistano  una serie di sistemi di riferimento detti inerziali, reciprocamente in moto rettilineo uniforme, in cui le leggi della fisica sono identicamente valide.
    \item Il principio di costanza della velocità della luce: il quale asserisce che la luce nello spazio vuoto si propaghi sempre con modulo della velocità determinato ed identico per ogni osservatore inerziale, che si indicherà con $c$. 
\end{itemize}
Il primo si rifà al principio di relatività galileiano mentre il secondo è una diretta conseguenza dell'esperimento di Michelson e Morley il cui risultato viene spiegato senza la necessità dell'introduzione dell'etere e di un sistema di riferimento privilegiato. Inoltre si continua ad intendere spazio e tempo come due enti omogenei e isotropi.\\

La primissima conseguenza dell'assunzione di questi due postulati è la non validità delle trasformazioni di Galileo. Infatti secondo queste un moto di velocità $\vec{v}$ in un sistema $K$, se osservato in un sistema $K'$, nel quale $K$ si muove a velocità $\vec{V}$, risulterà in un moto a velocità $\vec{v}+\vec{V}$. Il secondo postulato però richiede che se tale moto è di un fascio di luce questo debba risultare sia in $K$ che in $K'$ in un moto con modulo della velocità pari a $c$, in totale disaccordo con le trasformazioni di Galileo.\\

Inoltre nel suo articolo Einstein stesso propose, dopo aver enunciato i postulati, un'esperimento mentale che consente di mostrare come questi siano in diretto conflitto con le assunzioni classiche dell'assolutezza del tempo e delle lunghezze. Si considerino due orologi reciprocamente a riposo posizionati in due punti detti $A$ e $B$. Einstein osservò che ogni orologio è in grado di misurare intervalli temporali solamente per eventi che avvengono nello stesso punto in cui ognuno è posizionato, questo poiché diventa necessario tener conto della velocità della luce che quindi propagandosi genera dei ritardi nella percezione degli eventi lontani.\\ Il secondo postulato consente però di sincronizzare gli orologi così che sia possibile confrontare i tempi misurati in $A$ e in $B$. In primo luogo si ipotizzi di far partire all'istante $t_A=0$, misurato dal primo orologio, un fascio di luce che viaggia da $A$ e giunge in $B$ quando l'orologio posizionato in tale punto segna un tempo $t_B$. In $B$ il fascio è riflesso e fa ritorno in $A$ quando il relativo orologio segna un tempo $t_A$. Poiché per il principio di costanza della velocità della luce il fascio luminoso deve propagarsi in ogni direzione con la stessa velocità e la distanza tra i due orologi è fissata e costante allora il tempo impiegato dalla luce per andare da $A$ a $B$ e vice versa deve essere il medesimo. Si conclude quindi che i due orologi sono sincronizzati solamente se vale:
\begin{equation}
    2t_B=t_A
    \label{SinconizazioneOrologi}
\end{equation}
Chiaramente se l'orologio nel punto $A$ si è rivelato sincrono con quello nel punto $B$, tramite la procedura appena descritta, è chiaro che è altrettanto vero che quello posto in $B$ è sincrono con quello posto in $A$ e che se si considera un orologio posto in un terzo punto $C$ che risulta sincrono con quello posto in $B$ allora questo terzo orologio è sincrono con quello nel punto $A$.\\Così facendo è possibile assegnare un immaginario orologio ad ogni punto dello spazio in maniera tale che siano tutti sincroni tra loro e sia possibile determinare quando due eventi lontani fra loro avvengono nello stesso istante in un determinato sistema di riferimento inerziale.\\
Fatta propria questa osservazione è possibile procedere analizzando l'esperimento mentale: si consideri un regolo di lunghezza $l$ in un sistema di riferimento ad esso solidale detto $K$. Si considerino anche due orologi sincroni posti nelle due estremità del regolo dette $A$ e $B$. In un sistema $K'$ si osserva il regolo in moto a velocità $v$ lungo la direzione in cui la parte lunga del regolo poggia. Sia detto $A'$ il punto del sistema $K'$ in cui si osserva l'emissione del fascio di luce dalla prima estremità del regolo, $B'$ il punto, sempre in $K'$, in cui si osserva la riflessione del fascio nel secondo estremo del regolo ed infine $C'$ il punto in $K'$ in cui si osserva il fascio fare ritorno alla prima estremità. In ognuno dei tre punti è presente un orologio sincronizzato con gli altri del sistema $K'$ in maniera da osservare l'emissione del fascio ad un tempo $t'_{A'}=0$. Se $t'_{B'}$ è l'istante in cui si osserva la riflessione in $B'$, $t'_{C'}$ è l'istante in cui il fascio fa ritorno al primo estremo del regolo e $l'$ è la lunghezza del regolo misurata nel sistema $K'$ allora è possibile determinare in funzione di questi tempi le distanze tra i punti $A',\ B'$ e $C'$, infatti queste dipenderanno in parte dalla distanza percorsa del regolo e in parte dalla sua lunghezza:
\begin{flalign*}
    &\Delta x'_{A'B'}=x'_{B'}-x'_{A'}=v(t'_{B'}-t'_{A'})+l'=vt'_{B'}+l'\\
    &\Delta x'_{B'C'}=x'_{B'}-x'_{C'}=l'-v(t'_{C'}-t'_{B'})
\end{flalign*}
Queste due distanze sono quelle percorse dal fascio luminoso rispettivamente in tempi $t'_{B'}$ e $t'_{C'}-t'_{B'}$, per cui si ottiene:
\begin{flalign*}
    \Delta x'_{A'B'}=ct'_{B'}=vt'_{B'}+l' \quad &\Rightarrow\qquad t'_{B'}=\frac{l'}{c-v}\\
    \Delta x'_{B'C'}=c(t'_{C'}-t'_{B'})=l'-v(t'_{C'}-t'_{B'}) \qquad &\Rightarrow\quad t'_{C'}-t'_{B'}=\frac{l'}{c+v}
\end{flalign*}
Questa semplice osservazione consente di concludere che i postulati enunciati da Einstein non ammettono la possibilità di assumere tempi assoluti: infatti l'osservatore in $K'$ osserva che il fascio di luce impiega più tempo per giungere al secondo estremo di quanto ne trascorra tra la riflessione e il suo ritorno al primo estremo, mentre invece l'osservatore in $K$, solidale con il regolo, osserva tempi identici per questi due tragitti. Una volta sviluppata matematicamente questa nuova teoria della relatività si osserverà che in maniera analoga pure le lunghezze non possono più essere considerate assolute.


\appendix
\chapter{Lemmi e teoremi per le trasformazioni di Lorentz}
\label{chap:LinearitàLorentz}
 Nelle seguenti pagine sono riportati alcuni risultati utili per la derivazione delle trasformazioni di Lorentz tra cui una dimostrazione dell'affinità delle trasformazioni di Lorentz riportata da Fock in \cite{Fock}.\\

\section{Risultati generali}


\begin{lemma}
    Siano $l_1: \mathbb{R}^n\rightarrow\mathbb{R}$ una forma lineare e $q:\mathbb{R}^n\rightarrow\mathbb{R}$ una forma quadratica. Se vale $q|_{\ker f}=0$ allora $\exists l_2: \mathbb{R}^n\rightarrow\mathbb{R}$ forma lineare tale che:
    \begin{equation*}
        q=l_1l_2.
    \end{equation*}\label{lemm:A1}
\end{lemma}
\begin{proof}
    Per prima cosa si osservi che $\exists A\in M_{n\times n}(\mathbb{R} )$ tale che:
    \begin{equation}
        q(v)=\langle v,Av\rangle,\qquad \forall v\in\mathbb{R}^n,\label{qvAv}
    \end{equation}
    dove si è indicato con $\langle \cdot ,\cdot \rangle$ il prodotto scalare euclideo.\\Inoltre $\exists u\in\mathbb{R}^n$ tale che:
    \begin{equation}
        l_1(v)=\langle u,v\rangle,\qquad \forall v\in\mathbb{R}^n.\label{luv}\
    \end{equation}
    Si osservi che è possibile decomporre $v$ in una componente ortogonale ad $u$ ed una parallela:
    \begin{equation}
        v=v^\parallel +v^\bot =\frac{u \langle u,v\rangle}{\|u\|^2}+v-\frac{u\langle u,v\rangle}{\|u\|^2}.
    \label{Al1Decomposizione}
    \end{equation}  
    Esprimendo $q(v)$ secondo questa decomposizione si ottengono 3 addendi:
    \begin{flalign*}
            q(v)&=\langle  v^\parallel +v^\bot,A(v^\parallel +v^\bot)\rangle\\
            &=\ \langle v^\parallel ,Av^\parallel \rangle+2\langle v^\bot,Av^\parallel\rangle+\langle v^\bot,Av^\bot\rangle.
    \end{flalign*}
    Per la \eqref{qvAv} il termine $\langle v^\bot,Av^\bot\rangle$ è pari a $q(v^\bot)$ e per costruzione $l_1(v^\bot)=\langle u,v^\bot\rangle=0$. Ne segue che deve annullarsi $q(v^\bot)$ (per ipotesi $q|_{\ker f}=0$) per cui:
   \begin{equation*}
    q(v)=\langle v^\parallel ,Av^\parallel \rangle+2\langle v^\bot,Av^\parallel\rangle.
   \end{equation*}
   Sostituendo in quest'ultima relazione le espressioni di $v^\bot$ e $v^\parallel$, date dalla (\ref{Al1Decomposizione}), si ha:
   \begin{flalign*}
        q(v)&=\langle u ,Au \rangle\left(\frac{\langle u,v\rangle}{\|u\|^2} \right)^2 +2 \langle v^\bot,Au\rangle \frac{\langle u,v\rangle}{\|u\|^2}\\
        &=\ \langle u ,Au \rangle\left(\frac{\langle u,v\rangle}{\|u\|^2} \right)^2 +2 \langle v,Au\rangle \frac{\langle u,v\rangle}{\|u\|^2} -2\langle u ,Au \rangle\left(\frac{\langle u,v\rangle}{\|u\|^2} \right)^2\\
   \end{flalign*}
   Infine, se si raccoglie un termine $\langle u,v\rangle$ e si sommano i termini uguali, si ottiene un'espressione per $q(v)$ composta da un termine lineare rispetto a $v$ moltiplicato per il prodotto scalare $\langle u,v\rangle$:
   \begin{equation}
    q(v)=\langle u,v\rangle\left[2 \frac{\langle v,Au\rangle}{\|u\|^2}-\frac{\langle u ,Au \rangle\langle u,v\rangle}{\|u\|^4} \right]=l_1(v)l_2(v) \qquad \forall v\in\mathbb{R}^n
   \end{equation}
\end{proof}
Sia $v\in\mathbb{R}^4$, per comodità da ora in poi si scriverà $v=(v_0,v_1,v_2,v_3)=(v_0,\vec{v})$.
\begin{lemma}
     Sia $q:\mathbb{R}^4\rightarrow\mathbb{R}$ una forma quadratica tale che $\bar{v}_0^2-|\vec{\bar{v}}|^2=0$ implica $ q(\bar{v})=0$. Allora $\exists \lambda $ tale che $ q(v)=\lambda(v_0^2-|\vec{v}|^2)\ \forall v\in\mathbb{R}^4$.
    \label{lemm:A2}
\end{lemma}
\begin{proof}
    Si osservi che $q(v)$ può essere scomposto, seguendo le regole del prodotto righe per colonne $q(v)=v^tAv$ (dove $A\in M_{3\times3}(\mathbb{R})$ simmetrica), nella seguente somma:
    \begin{equation*}
        q(v)=a_{00}v_0^2+2v_0(a_{01}v_1+a_{02}v_2+a_{03}v_3)+\hat{q}(\vec{v})
    \end{equation*}
    dove $\hat{q}$ è ancora una forma quadratica in $\mathbb{R}^3$ che agisce su $\vec{v}$.\\Si consideri $\bar{v}=(-|\vec{v}|,\vec{v})$. Per costruzione $(\bar{v}_0)^2-|\vec{\bar{v}}|^2=|\vec{v}|^2-|\vec{v}|^2=0$, per cui per le ipotesi si ha $q(\bar{v})=0$. Essendo $q(\bar{v})=0$, posto $v=(v_0,\vec v)$, si può scrivere:
    \begin{flalign}
        \label{Al2Decomposizione}
            q(v)&=q(v)-q(\bar{v})\\\nonumber
            &=a_{00}(v_0^2-|\vec{v}|^2)+2(v_0-|\vec{v}|)(a_{01}v_1+a_{02}v_2+a_{03}v_3)+\hat{q}(\vec{v})-\hat{q}(\vec{\bar{v}})\\\nonumber
            &=a_{00}(v_0^2-|\vec{v}|^2)+2(v_0-|\vec{v}|)(a_{01}v_1+a_{02}v_2+a_{03}v_3)
    \end{flalign}
   inoltre calcolando la (\ref{Al2Decomposizione}) per $\bar{v}$ si ha:
   \begin{equation*}
    q(\bar{v})=-2|\vec{v}|(a_{01}v_1+a_{02}v_2+a_{03}v_3)=0.
   \end{equation*}
  Essendo $v$ arbitrario si ha che $a_{01}=a_{02}=a_{03}=0$ e quindi la tesi:
   \begin{equation}
    q(v)=a_{00}(v_0^2-|\vec{v}|^2)\quad \forall v\in\mathbb{R}^4
   \end{equation}
\end{proof}
\newpage
\section{Una dimostrazione della linearità delle trasformazioni di Lorentz}
Siano $x,x'\in\mathbb{R}^4$ due vettori dello spazio-tempo misurati in due sistemi di riferimento inerziali $K$ e $K'$. Si vogliono determinare le proprietà dell'applicazione $f:\mathbb{R}^4\rightarrow\mathbb{R}^4$ che trasforma i vettori misurati in $K$ negli stessi vettori misurati in $K'$ e viceversa considerando validi il principio di relatività e il principio di costanza della velocità della luce.\\

Per prima cosa si supporà che $f$ sia sufficentemente regolare, almeno $C^2$, ed è necessario che sia invertibile poichè deve poter trasformare sia da $K$ a $K'$, sia viceversa. Questa condizione è equivalente a richiedere che $\det J_f(x)\neq 0\ \forall x\in\mathbb{R}^4$ ($J_f(x)$ jacobiana di $f$).\\

Si considerino ora $\xi, \gamma \in \mathbb{R}^4$, rispettivamente detti punto iniziale e velocità della parametrizzazione. Questi consentono di parametrizzare un moto rettilineo uniforme nel seguente modo:
\begin{equation*}
    x=\xi+s\gamma \qquad s\in \mathbb{R}.
\end{equation*}
Infatti è possibile esplicitare la dipendenza di $x_0$ da $s$ per ottenere la forma di un moto rettilineo uniforme. Per questo motivo è necessario che $\gamma_0\neq 0$. 
\begin{equation*}
    s=\frac{x_0-\xi_0}{\gamma_0} \qquad \Rightarrow \qquad x_i=\xi_i+\frac{x_0-\xi_0}{\gamma_0}\gamma_i, \qquad i=1,2,3,
\end{equation*}
Secondo il principio di relatività se si calcola $f(x)=x'$ è necessario che $x'=\xi'+s'\gamma'$, così che un moto rettilineo uniforme resti tale in ogni sistema di riferimento inerziale. Derivando $f(\xi+s\gamma)$ rispetto al parametro $s'$ si ottiene:
\begin{equation*}
    \frac{dx'}{ds'}=\gamma'=J_f(\xi+s\gamma)\frac{ds}{ds'}\gamma.
\end{equation*}
Si consideri ora la i-esima componente di $\gamma'$, dove $i=0,1,2,3$\footnote{In questa appendice non si fa uso della convenzione per cui le lettere latine rappresentano solamente gli indici $1,2,3$.}, e la si divida per $\gamma_0$:
\begin{equation*}
    \frac{\gamma'_i}{\gamma'_0}=\frac{\sum_{k=0}^3\partial_kf_i(\xi+s\gamma)\gamma_k}{\sum_{k=0}^3\partial_kf_0(\xi+s\gamma)\gamma_k}
    \label{ARappGamma}
\end{equation*} 
dove $\partial_k=\frac{\partial}{\partial x_k}$. Da ora in poi si utilizzerà la convenzione degli indici ripetuti di Einstein per cui $\sum_k x_k y_k=x_k y_k$.\\Il rapporto $\frac{\gamma'_i}{\gamma'_0}$ è costante da cui segue immediatamente che $\frac{d}{ds}\frac{\gamma'_i}{\gamma'_0}=0$  e quindi:
\begin{flalign*}
        \frac{d}{ds}&\frac{\partial_kf_i(\xi+s\gamma)\gamma_k}{\partial_jf_0(\xi+s\gamma)\gamma_j}=\\&
    =\frac{\left(\partial^2_{hn}f_i(\xi+s\gamma)\gamma_h\gamma_n\right)\left(\partial_kf_0(\xi+s\gamma)\gamma_k\right)
    - \left(\partial_kf_i(\xi+s\gamma)\gamma_k\right)\left(\partial^2_{hn}f_0(\xi+s\gamma)\gamma_h\gamma_n\right)}{\left(\partial_jf_0(\xi+s\gamma)\gamma_j\right)^2}=0    
\end{flalign*}
che può annullarsi solo se si annulla il numeratore della frazione. Da questa osservazione e dal fatto che preso un $x\in \mathbb{R}^4 $ esistono $ \gamma \in\mathbb{R}^4, s \in \mathbb{R}$ tali che $x=\xi+s\gamma$ si ottiene:
\begin{equation}
    \left(\partial^2_{hn}f_i(x)\gamma_h\gamma_n\right)\left(\partial_kf_0(x)\gamma_k\right)
    = \left(\partial_kf_i(x)\gamma_k\right)\left(\partial^2_{hn}f_0(x)\gamma_h\gamma_n\right) \quad i=0,1,2,3  
    \label{ABigPSum}
\end{equation}
Siccome si è supposto inizialmente $\det J_f(x)\neq 0$, $ \forall\gamma\in \mathbb{R}^4$ non nullo, deve necessariamente esistere almeno un $j$ tale che $\partial_kf_j(x)\gamma_k\neq 0$. Inoltre, se si suppone di prendere $\bar{\gamma}\neq0$ tale che $\partial_kf_0(x)\bar{\gamma}_k=0$, per la (\ref{ABigPSum}) necessariamente anche $\partial^2_{hn}f_0(x)\bar{\gamma}_h\bar{\gamma}_n=0$.\\

Si applichi quindi il Lemma \ref{lemm:A1} che, in virtù di quanto appena osservato ($\partial_kf_0(x)\bar{\gamma}_k=0\Rightarrow \partial^2_{hn}f_0(x)\bar{\gamma}_h\bar{\gamma}_n=0$), consente di scrivere:
\begin{equation*}
    \partial^2_{hn}f_0(x)\gamma_h\gamma_n=\ \langle \psi(x),\gamma\rangle\ (\partial_kf_0(x)\gamma_k) \qquad   \forall x,\gamma\in \mathbb{R}^4 \quad i=0,1,2,3 \quad \psi(x)\in\mathbb{R}^4
\end{equation*}
che inserito nella (\ref{ABigPSum}) risulta in:
\begin{equation*}
    \left(\partial^2_{hn}f_i(x)\gamma_h\gamma_n\right)\left(\partial_kf_0(x)\gamma_k\right)
    = \left(\partial_kf_i(x)\gamma_k\right)\ \langle \psi(x),\gamma\rangle\ (\partial_kf_0(x)\bar{\gamma}_k).
\end{equation*}
Infine, siccome $\gamma'_0\neq 0$ (affinché sia $x'$ sia un moto rettilineo uniforme) a maggior ragione la sua espressione data dalla jacobiana di $f$ soddisfa $(\partial_kf_0(x)\gamma_k\frac{ds}{ds'})\neq 0$. È quindi possibile elidere da ambo i lati tale membro:
\begin{equation}
    \partial^2_{hn}f_i(x)\gamma_h\gamma_n=\left(\partial_kf_i(x)\gamma_k\right)\ \langle \psi(x),\gamma\rangle \qquad   \forall x,\gamma\in \mathbb{R}^4 \quad i=0,1,2,3.
    \label{APArtialScal}
\end{equation}
Si osservi che  $\partial^2_{hn}f_i(x)$ è una matrice simmetrica (per il teorema di Schwarz) per cui dall'espressione appena ottenuta si può scrivere:
\begin{equation}
    \partial^2_{hk}f_i(x)=\frac{1}{2}\left(\partial_kf_i(x)\psi_h(x)+\partial_hf_i(x)\psi_k(x)\right)
    \label{APArtialScalSimm}
\end{equation}

Si consideri ora $x$ e $x'$ parametrizzazioni del moto di un raggio di luce. Per il principio di costanza della velocità della luce se in $K$ $x$ è parametrizzato sul cono di luce, lo stesso deve avvenire in $K'$, il che è esprimibile matematicamente con le due condizioni:
\begin{equation*}
    \begin{gathered}
        \gamma_0^2-(\gamma_1^2+\gamma_2^2+\gamma_3^2)=0,\\
        (\gamma_0')^2-[(\gamma_1')^2+(\gamma_2')^2+(\gamma_3')^2]=0.
    \end{gathered}
\end{equation*}
Poiché si può esprimere ogni componente di $\gamma'$ tramite la jacobiana di $f$ e $\gamma$, come già si è fatto calcolando $\frac{\gamma'_i}{\gamma'_0}$:
\begin{equation*}
    \left(\partial_kf_0(x)\gamma_k\right)^2-\left[\sum_{i=1}^3\left(\partial_kf_i(x)\gamma_k\right)^2\right]=0
\end{equation*}
Il principio di costanza della velocità della luce consente di utilizzare il Lemma \ref{lemm:A2} (poiché l'annullarsi di una forma quadratica implica l'annullarsi dell'altra) da cui si ha che:
\begin{equation*}
    \left(\partial_kf_0(x)\gamma_k\right)^2-\left[\sum_{i=1}^3\left(\partial_kf_i(x)\gamma_k\right)^2\right]=
    \lambda(x)\left[\gamma_0^2-(\gamma_1^2+\gamma_2^2+\gamma_3^2)\right]
\end{equation*}
Sia ora $g_{ij}$ una matrice tale per cui $g_{00}=1$, $g_{ii}=-1$ per $i=1,2,3$ e $g_{ij}=0$ se $i\neq j$, così facendo:
\begin{equation*}
    \begin{gathered}
        \lambda(x)\left[\gamma_0^2-(\gamma_1^2+\gamma_2^2+\gamma_3^2)\right]= 
        \lambda(x)g_{kh}\gamma_k\gamma_h=g_{ij}(\partial_kf_i(x)\gamma_k)(\partial_hf_j(x)\gamma_h)\\
    \end{gathered}
\end{equation*}
\begin{equation}
   \Rightarrow\qquad \lambda(x)g_{hk}=g_{ij}(\partial_kf_i(x))(\partial_hf_j(x)).
    \label{APartialDelta}
\end{equation}
Si derivi ora rispetto a $x_s $ ad ambo i membri così da ottenere:
\begin{equation*}
    g_{hk} \partial_s\lambda =2g_{ij}\partial_{ks}^2f_i(x)\partial_hf_j(x).
\end{equation*}
Sostituendo la derivata seconda parziale tramite la (\ref{APArtialScalSimm}) questa espressione è semplificabile e si ottiene:
\begin{equation}
    g_{hk} \partial_s\lambda =g_{ij} \left[\partial_kf_i(x)\psi_s+\partial_sf_i(x)\psi_k\right]\partial_hf_j(x).
\end{equation}    
Infine applicando nuovamente la (\ref{APartialDelta}) si possono sostituire tutti i prodotti di derivate parziali:
\begin{equation*}
    g_{hk} \partial_s\lambda =g_{kh}\lambda\psi_s+g_{sh}\lambda \psi_k.
\end{equation*}
Si considerino queste due differenti casistiche:
\begin{itemize}
    \item siano $k=h\neq s$ allora
    \begin{equation*}
        g_{hk} \partial_s\lambda =g_{kh}\lambda\psi_s \Rightarrow \partial_s\lambda=\psi_s\lambda,
    \end{equation*}
    \item siano $k=h= s$ allora
    \begin{equation*}
        g_{hk} \partial_s\lambda =2g_{kh}\lambda\psi_s \Rightarrow \partial_s\lambda=2\psi_s\lambda.
    \end{equation*}
\end{itemize}
Queste due relazioni implicano però che $\lambda\psi_i=0$ (per $i=0,1,2,3$). Però $\lambda$  non può annullarsi altrimenti la trasformazione $f$ mapperebbe ogni moto in un moto a velocità luminare, il che violerebbe il principio di relatività.\\Segue quindi che $\psi_i=0$ (per $i=0,1,2,3$), per cui dalla (\ref{APArtialScalSimm}) risulta:
\begin{equation}
    \partial_{kh}^2f_i(x)=0 \qquad \qquad \forall i,k,h=0,1,2,3 \forall x \quad\in \mathbb{R}^4
\end{equation}
per cui $f(x)$ può essere esclusivamente una trasformazione affine.

\bibliographystyle{plain}
\bibliography{ref}

\end{document}
