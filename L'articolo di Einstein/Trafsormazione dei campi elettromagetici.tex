Nella sezione \ref{Sec:nonInvMax} si è visto come le equazioni di Maxwell (sezione \ref{sec:EquazioniMaxwell}) non risultino Galileo invarianti, si procederà ora a derivare le trasformazioni relativistiche del campo elettrico e magnetico in maniera analoga a come fece Einstein nel suo articolo del 1905$^{\cite{Einstein1905}}$.\\

Per prima cosa è opportuno ricavare le trasformazioni degli operatori di differenziazione considerando le trasformazioni di Lorentz (\ref{TrasformazioneLorentz}) tra due sistemi $K$ e $K'$ in moto reciproco a velocità $V$. Usando la regola di Leibniz si ottiene:
\begin{equation}
    \frac{\partial}{\partial x'}=\gamma\frac{\partial}{\partial x}-\gamma \frac{V}{c^2}\frac{\partial}{\partial t},\quad \frac{\partial}{\partial y'}=\frac{\partial}{\partial y},\quad \frac{\partial}{\partial z'}=\frac{\partial}{\partial z},\quad \frac{\partial}{\partial t'}=\gamma \frac{\partial}{\partial x}-\gamma\frac{V}{c^2}\frac{\partial}{\partial x}.
    \label{trasfLorentzDiff}
\end{equation}
Siccome la relatività presuppone che la teoria di Maxwell sia corretta, è naturale partire proprio da questa, infatti si considereranno  due delle quattro equazioni di Maxwell:
\begin{equation}
    \vnabla \wedge \vec{E}=-\frac{\partial \vec{B}}{\partial t}, \qquad \vnabla\cdot\vec{B}=0.\label{maxwellMaSolo2}
\end{equation}
Se si studia la prima trasformando gli operatori di derivazione, componente per componente, secondo le (\ref{trasfLorentzDiff}), si ottengono delle trasformazioni per il campo elettromagnetico.
\begin{flalign}
    &\frac{\partial E_z}{\partial y}-\frac{\partial E_y}{\partial z}=-\frac{\partial B_x}{\partial t}\ &\quad  \frac{\partial E_z}{\partial y'}-\frac{\partial E_y}{\partial z'}&=-\bigg(\frac{\partial B_x}{\partial t'}-v\frac{\partial B_x}{\partial x'}\bigg)\gamma&&\label{Maxwell3comp1trasf}\\
    &\frac{\partial E_x}{\partial z}-\frac{\partial E_z}{\partial z}=-\frac{\partial B_y}{\partial t}\ &\Rightarrow\quad  \frac{\partial E_x}{\partial z'}-\frac{\partial E_z}{\partial x'}+\gamma\frac{V}{c^2}\frac{\partial E_z}{\partial t'}&=-\bigg(\frac{\partial B_y}{\partial t'}-v\frac{\partial B_y}{\partial x'}\bigg)\gamma&&\\
    &\frac{\partial E_y}{\partial x}-\frac{\partial E_x}{\partial y}=-\frac{\partial B_z}{\partial t}\ &\quad  \frac{\partial E_y}{\partial x'}-\gamma\frac{V}{c^2}\frac{\partial E_y}{\partial t'}-\frac{\partial E_x}{\partial y'}&=-\bigg(\frac{\partial B_z}{\partial t'}-v\frac{\partial B_z}{\partial x'}\bigg)\gamma.&&
\end{flalign}
Infatti siccome le equazioni di Maxwell devono valere sia in $K$ che in $K'$ è possibile identificare quali termini devono corrispondere alle coordinate di $\vec{E'}$ e $\vec{B'}$ nelle equazioni trasformate affinché queste si riducano a nuove equazioni di Maxwell:
\begin{flalign*}
    &\frac{\partial E_x}{\partial z'}-\frac{\partial }{\partial x'}\bigg[\gamma\bigg(E_z+vB_y\bigg)\bigg]=-\frac{\partial }{\partial t'}\bigg[\gamma\bigg(B_y+\frac{v}{c^2}E_z\bigg)\bigg],\\
    &\frac{\partial }{\partial x'}\bigg[\gamma\bigg(E_y-vB_z\bigg)\bigg]-\frac{\partial E_x}{\partial y'}=-\frac{\partial }{\partial t'}\bigg[\gamma\bigg(B_z-\frac{v}{c^2}E_y\bigg)\bigg].
\end{flalign*}
Così facendo si ottengono le trasformazioni di tutte le componenti tranne che per $B_x$:
\begin{flalign}
   & E'_{x'}=E_x,\qquad&E'_{y'}=(E_y-vB_z)\gamma,\qquad &E'_{z'}=(E_z+vB_y)\gamma,&&\nonumber\\
   & &B'_{y'}=(B_y+\frac{V}{c^2}E_z)\gamma,\qquad &B'_{z'}=(B_z-\frac{V}{c^2}E_y)\gamma.&&\nonumber
\end{flalign}
Per conoscere come si trasforma $B_x$ si sostituiscano le espressioni di $E_y$ e $E_z$, ricavabili dalle trasformazioni appena ottenute, nella (\ref{Maxwell3comp1trasf}) così che utilizzando la (\ref{maxwellMaSolo2}) e il procedimento precedentemente adoperato si ottenga:
\begin{equation*}
    \frac{\partial E'_z}{\partial y'}-\frac{\partial E'_y}{\partial z'}=v(\vnabla'\cdot\vec{B'})-\frac{\partial B_x}{\partial t'}=-\frac{\partial B_x}{\partial t'}\qquad\Rightarrow\qquad B'_{x'}=B_x.
\end{equation*}
Queste trasformazioni, essendo ricavate dall'uso combinato della relatività e dell'elettromagnetismo di Maxwell, sono quindi coerenti con entrambe. 