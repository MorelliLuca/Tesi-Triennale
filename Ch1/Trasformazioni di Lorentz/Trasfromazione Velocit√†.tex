\subsection{Trasformazione delle velocità}
Si è già osservato che le trasformazioni di Galilei non risultano compatibili con i postulati di Einstein in quanto ammettono che un osservatore potrebbe misurare la velocità della luce con un valore differente da $c$. Dalle trasformazioni di Lorentz è possibile ricavare quali sono le effettive trasformazioni delle velocità nella teoria della relatività.\\

Per prima cosa è necessario studiare come si trasformi un generico moto $(ct,\vec r(t))$ quando si cambia sistema di riferimento da $K$ a $K'$, tale per cui nel primo il secondo si muove a velocità $V$ lungo l'asse $x$, si avrà allora la trasformazione:
\begin{equation*}
    \begin{cases}
        t=(t'+\frac{V}{c^2}r'_x(t'))\gamma\\
        r_x=(r_x'(t')+Vt')\gamma\\
        r_y=r_y'(t')\\
        r_z=r_z'(t')
    \end{cases}
    \Longleftrightarrow \quad
    \begin{cases}
        t'=(t-\frac{V}{c^2}r_x(t))\gamma\\
        r_x'=(r_x(t)-Vt)\gamma\\
        r_y'=r_y(t)\\
        r_z'=r_z(t)
    \end{cases}
    \qquad \gamma=\frac{1}{\sqrt{1-\frac{V^2}{c^2}}}.
\end{equation*}
Se si calcolano le componenti della velocità in $K'$ facendo uso della regola della derivata di funzione composta e delle trasformazioni di $(ct,\vec r(t))$ si ottiene:
\begin{flalign*}
    &v_{x'}'=\frac{dr'_{x'}}{dt'}=\gamma\frac{dt}{d t'}\frac{d}{d t}(r_x(t)-Vt)=\gamma^2\bigg(1+\frac{Vv_{x'}'}{c^c}\bigg)(v_x-V)\\
    &v_{y'}'=\frac{dr'_{y'}}{dt'}=\gamma\frac{dt}{d t'}\frac{d}{d t}(r_y(t))=\gamma^2\bigg(1+\frac{Vv_{x'}'}{c^c}\bigg)v_y\\
    &v_{z'}'=\frac{dr'_{z'}}{dt'}=\gamma\frac{dt}{d t'}\frac{d}{d t}(r_z(t))=\gamma^2\bigg(1+\frac{Vv_{x'}'}{c^c}\bigg)v_z\\
    &\frac{d t}{dt'}=\gamma\bigg(1+\frac{Vv_{x'}'}{c^2}\bigg).
\end{flalign*}
Risolvendo algebricamente per le componenti $v_{x'}',\ v_{y'}'$ e $v_{z'}'$ si ottengono le trasformazioni:
\begin{equation}
    \begin{cases}
        v_{x'}'=\frac{v_x-V}{1-\frac{Vv_x}{c^2}}\\
    v_{y'}'=\frac{v_y}{\gamma(1-\frac{Vv_x}{c^2})}\\
    v_{z'}'=\frac{v_z}{\gamma(1-\frac{Vv_x}{c^2})}
    \end{cases}
    \qquad \qquad \gamma=\frac{1}{\sqrt{1-\frac{V^2}{c^2}}}.
    \label{TrasformazioniVelocitàLorentz}
\end{equation}

Si noti che queste trasformazioni per le velocità rispettano il principio di costanza della velocità della luce: se si considera il moto di un raggio di luce lungo l'asse $x$ di $K$ si ha\footnote{Questo caso in realtà è generale poiché è sufficiente comporre la trasformazione di Lorentz con una rotazione degli assi spaziali perché ci si riconduca a questo caso.} 
\begin{equation*}
    v'=\frac{c-V}{1-\frac{Vc}{c^2}}=\frac{c-V}{c-V}c=c.
\end{equation*}

Infine si consideri un corpo in moto $\vec r(t)$ in $K$ e si derivi il quadrivettore posizione rispetto al tempo proprio del corpo $\tau$:
\begin{equation}
    u=\frac{d }{d\tau}(ct,\vec r(t))=\bigg(c\frac{dt}{d\tau},\vec v\frac{dt}{d\tau}\bigg)=(c,\vec v)\gamma_0 \qquad \gamma_0=\frac{1}{\sqrt{1-\frac{|\vec v|^2}{c^2}}}
    \label{defQuadrivelocità}
\end{equation}
dove $\frac{dt}{d\tau}$ è stato calcolato dalla (\ref{dilatazioneTempi}).\\
Siccome le trasfromazioni di Lorentz sono applicazioni lineari e non dipendono da $\tau$ è possibile trasformare il quadrivettore posizione prima di derivarlo così che sia possibile trasformare $u$ da $K$ a $K'$ tramite la matrice $\Lambda$ (\ref{TrasformazioneLorentz}), in tal caso si ottiene:
\begin{equation*}
       (c,  v_x',  v_y', v_z')\gamma_0'=\Lambda    
    \begin{pmatrix}
        c \\ v_x \\ v_y \\ v_z
     \end{pmatrix}\gamma_0=\bigg(\bigg(c-\frac{Vv_x}{c}\bigg)\gamma, (v_x-V)\gamma, v_y,  v_z\bigg)\gamma_0
\end{equation*}
dove $\gamma$ dipende dalla velocità reciproca dei sistemi di rifermento mentre $\gamma_0$ e $\gamma'_0$ dipendono dalla velocità del corpo in $K$ e $K'$. Dalla prima componente trasformata si ottiene che:
\begin{equation}
    \frac{\gamma_0'}{\gamma_0}=\bigg(1-\frac{Vv_x}{c^2}\bigg)\gamma
\end{equation}
che se sostituita nelle altre componenti spaziali consente di ottenere le trasformazioni delle velocità (\ref{TrasformazioniVelocitàLorentz}), queste infatti non sono altro che le trasformazioni di Lorentz del quadrivettore $u$, chiamato quadrivelocità. Si osservi che il modulo quadro di $u$ è costante:
\begin{equation*}
    |u|^2=\frac{c^2-|\vec v|^2}{1-\frac{|\vec v|^2}{c^2}}=c^2.
\end{equation*} 