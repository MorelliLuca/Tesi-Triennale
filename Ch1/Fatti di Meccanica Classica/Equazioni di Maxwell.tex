\subsection{L'elettromagnetismo e le equazioni di Maxwell}\label{sec:EquazioniMaxwell}
Per interpretare i fenomeni elettromagnetici, anche in questo caso, è necessario introdurre
una serie di osservazioni sperimentali: in primo luogo esiste una proprietà della materia, 
detta carica elettrica, che non dipende dal sistema di riferimento in cui è misurata e che consente 
ai corpi di interagire con due campi vettoriali: 
il campo elettrico $\vec{E}$ e il campo magnetico $\vec{B}$. Un corpo puntiforme di carica 
$q$ subisce quindi un forza $\vec{F}$ data da:
\begin{equation}
	\vec{F}=q(\vec{E}(\vec{x},t)+\vec{v}\wedge\vec{B}(\vec{x},t)),
	\label{ForzaLorentz}
\end{equation}
dove $\vec{v}$ è il vettore velocità di tale corpo.\\
In secondo luogo gli esperimenti mostrarono che questi due campi rispettano una serie di equazioni 
dette equazioni di Maxwell:
\begin{equation}
	\begin{gathered}
		\vec{\nabla}\cdot\vec{E}=\frac{\rho}{\epsilon_0}, \qquad \qquad \vec{\nabla}\cdot\vec{B}=0, \\
		\vec{\nabla}\wedge\vec{E}=-\frac{\partial\vec{B}}{\partial t}, \qquad \qquad \vec{\nabla}\wedge
		\vec{B}=\mu_0\vec{J}+\epsilon_0\mu_0\frac{\partial\vec{E}}{\partial t},
		\label{EquazioniMaxwell}
	\end{gathered}
\end{equation}
dove $\rho$ è la densità di carica volumetrica, $\vec{J}$ è la densità di corrente di carica superficiale e 
$\epsilon_0$, $\mu_0$ sono due costanti del vuoto.\\

Dalle equazioni di Maxwell (\ref{EquazioniMaxwell}) segue che le cariche sono sorgenti del campo 
elettrico mentre le correnti lo sono per 
il campo Magnetico. Per esempio è possibile mostrare che una carica puntiforme è sorgente di campo elettrico e 
tramite la forza di Lorentz (\ref{ForzaLorentz}) è possibile derivare la legge sperimentale di Coulomb.\\
Si consideri una carica puntiforme $Q$ posta nell'origine di un sistema di riferimento e si prenda una sfera $\mathcal{S}$ di raggio $R$ 
contenete tale carica. Integrando la prima equazione di Maxwell e facendo uso del teorema di Gauss si ottiene:
\begin{equation*}
	\int_\mathcal{S}\vec{\nabla}\cdot\vec{E}\ d^3x=\int_\mathcal{S}\frac{\rho}{\epsilon_0}\ d^3x=\frac{Q}{\epsilon_0}=\oint_{\partial\mathcal{S}}\vec{E}\cdot\hat{n}\ d\Sigma
\end{equation*}
dove si è indicato con $\hat{n}$ il versore normale alla superficie $\mathcal{S}$ nel punto in cui si valuta l'integrando.\\
Si osservi che essendo lo spazio 
isotropo e la carica puntiforme allora qualsiasi rotazione degli assi coordinati mantiene immutato il sistema, ne segue che 
il Campo Elettrico generato dalla carica deve essere radiale e costante su superfici sferiche centrate nell'origine. Infatti un'ipotetica seconda carica 
fissa a distanza $R$ dall'origine del sistema di riferimento deve percepire sempre la medesima forza, indipendentemente dalla rotazione effettuata, 
che risulta connessa al campo elettrico per mezzo della (\ref{ForzaLorentz}), inoltre sempre dalle Equazioni di Maxwell, supponendo assenza di correnti 
si ha che il rotore di $\vec{E}$ risulta nullo. Grazie a queste deduzioni l'integrale di superficie si riduce a:
\begin{equation*}
	\oint_{\partial\mathcal{S}}\vec{E}\cdot\hat{n}\ d\Sigma=4\pi R^2|\vec{E}|=\frac{Q}{\epsilon_0}.
\end{equation*}	
Tenendo conto che $\vec{E}$ è radiale, come si è dedotto, si ha la legge di Coulomb
\begin{equation}
	\vec{E}=\frac{1}{4\pi\epsilon_0}\frac{Q}{R^2}\hat{r} \quad \Rightarrow \quad \vec{F}=q\vec{E}=\frac{1}{4\pi\epsilon_0}\frac{qQ}{R^2}\hat{r}
\end{equation} 
dove si è indicato con $\hat{r}$ il versore radiale in coordinate sferiche.\\

Infine è importante per trattare la teoria della relatività osservare che è possibile ottenere, calcolando il rotore di ambo 
i membri delle ultime due equazioni di Maxwell
\begin{equation*}
	\vec{\nabla}\wedge\vec{\nabla}\wedge\vec{E}=-\frac{\partial}{\partial t}\vnabla\wedge\vec{B},
	\qquad \vec{\nabla}\wedge\vec{\nabla}\wedge\vec{B}=\mu_0\epsilon_0\frac{\partial}{\partial t}
	\vnabla\wedge\vec{E},
\end{equation*}
e supponendo assenza di cariche, per cui $\rho=0$ e $\vec{J}=0$, due equazioni che descrivono 
onde di campo elettrico e magnetico nel vuoto:
\begin{equation}
	\vnabla^2\vec{E}=\mu_0\epsilon_0\frac{\partial^2\vec{E}}{\partial t^2} \qquad \vnabla^2\vec{B}=
	\mu_0\epsilon_0\frac{\partial^2\vec{B}}{\partial t^2}
\end{equation}
dove si è indicato l'operatore laplaciano con la notazione $\vnabla^2=(\frac{\partial^2 }{\partial x^2},\frac{\partial^2 }{\partial y^2},\frac{\partial^2 }{\partial z^2})$.\\
Queste onde si propagano con una velocità $\frac{1}{\sqrt{\mu_0\epsilon_0}}=299792458\  \frac{m}{s}$, 
che corrisponde con precisione ai valori sperimentalmente misurati della velocità della luce.\\ Maxwell stesso suppose allora che questa fosse quindi da intendere come un fenomeno elettromagnetico e successivi esperimenti, come quelli di Hertz, confermarono tale ipotesi.\\
Sulla base della teoria ondulatoria classica è però necessario identificare un mezzo nel quale queste onde possano 
propagarsi e rispetto al quale la loro velocità di propagazione debba essere intesa. Per questo motivo alla fine dell'ottocento venne ipotizzata l'esistenza di tale mezzo detto etere luminifero.