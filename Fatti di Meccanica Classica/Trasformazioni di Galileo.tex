\section{La meccanica e le trasformazioni di Galileo}
La meccanica classica, in tutte le sue possibili formulazioni, ha come fondamento una serie 
di osservazioni sperimentali che vengono utilizzate come principi da cui dedurre le leggi del moto.\\

Il primo fatto sperimentale che viene assunto è che lo spazio sia tridimensionale, isotropo, omogeneo e che rispetti la 
geometria euclidea mentre il tempo sia ad una sola dimensione. Inoltre si assume che le distanze spaziali e il tempo siano assoluti,
 ossia che ogni osservatore concordi sulla misura di queste.\\
Sulla base di queste assunzioni si può quindi scegliere un punto dello spazio-tempo come 
origine di un sistema di riferimento, ossia uno spazio vettoriale 
$\mathbb{R}^3\times\mathbb{R}$ che ha come vettore nullo il punto scelto. Da questo vengono scelte tre direzioni 
spaziali arbitrarie lungo cui sono orientati tre assi cartesiani, identificabili con la base di $\mathbb{R}^3$, e 
dal corrispondente punto temporale si inizia a misurare il tempo. Si osservi che tali scelte sono arbitrarie, 
come lo è la direzione degli assi corrispondenti ai vettori della base, questo poiché spazio e tempo sono isotropi 
ed omogenei.\\

Il secondo fatto sperimentale prende il nome di Principio di Relatività Galileiano e consiste 
nell'assunzione che esistano una serie di sistemi di riferimento detti inerziali, caratterizzati 
dalla proprietà di essere reciprocamente in moto rettilineo uniforme, in cui le leggi della natura 
in ogni istante assumono la stessa forma.\\

Infine si assume che, in un riferimento inerziale, le posizioni e le velocità dei punti di un sistema ad un tempo iniziale ne determinino 
in maniera univoca l'evoluzione secondo la legge:
\begin{equation}
	m\ddot{\vec{x}}=\vec{F}(\vec{x},\dot{\vec{x}},t)
	\label{equazioneDiNewton}
\end{equation}
dove $m$ è detta massa inerziale e $\vec{F}$ è una funzione caratteristica del sistema detta forza.\\

Si vuole quindi identificare quali applicazioni $\varphi:\mathbb{R}^3\times\mathbb{R}\rightarrow
\mathbb{R}^3\times\mathbb{R}$ consentono di cambiare sistema di riferimento inerziale, ossia quali 
trasformazioni non variano le leggi della natura. Queste applicazioni si chiamano Trasformazioni di 
Galileo e dalle evidenze sperimentali si conclude che queste sono costituite dalle 
composizioni di tre famiglie di applicazioni:
\begin{itemize}
	\item una generica traslazione spazio temporale dell'origine, dedotta dalla proprietà 
	di omogeneità dello spazio e del tempo:
	\begin{equation}
		\varphi_{\vec{r},s}(\vec{x},t)=(\vec{x}+\vec{r},t+s)
		\label{GalileoTraslazoine}
	\end{equation} 
\item una generica rotazione degli assi spaziali, dovuta alla proprietà di isotropia dello spazio:
\begin{equation}
	\varphi_{G}(\vec{x},t)=(G\vec{x},t) \qquad G\in M_{3\times3}(\mathbb{R}):G^{-1}=G^t
	\label{GalileoRotazione}
\end{equation} 
	\item una traslazione di moto rettilineo uniforme, ammissibile grazie alle proprietà dei sistemi 
	di riferimento inerziali:
\begin{equation}
	\varphi_{\vec{v}}(\vec{x},t)=(\vec{x}+\vec{v}t,t)
	\label{GalileoVelocità}
\end{equation} 
\end{itemize}
Quest'ultima tipologia di trasformazione è quella che viene comunemente studiata per caratterizzare le trasformazioni di sistemi inerziali. 
Si consideri quindi un sistema $K$, con coordinate $(\vec{x},t)$ e un sistema $K'$, con 
coordinate $(\vec{x'},t')$, in moto a velocità $\vec{V}$ rispetto a $K$, si scriverà allora la 
(\ref{GalileoVelocità}) come:
\begin{equation}
	\vec{x'}=\vec{x}-\vec{V}t \qquad t'=t
	\label{GalileoEasy}
\end{equation}
Si noti che questa trasformazione, essendo lineare, non muta la forma del vettore $\ddot{\vec{x}}$, infatti:
\begin{equation*}
	\frac{d^2}{dt'^2}(\vec{x'})=\frac{d^2}{dt^2}(\vec{x}-\vec{V}t)=\ddot{\vec{x}}+\frac{d}{dt}(\vec{V})=\ddot{\vec{x}}
\end{equation*} 
questo fatto, assieme al Principio di Relatività Galileiano applicato alla Legge di Newton (\ref{equazioneDiNewton}), 
impongono che le forze esercitate su di un punto e misurate in due sistemi inerziali differenti debbano essere le medesime. \\

Analogamente a quanto appena fatto si possono ricavare le trasformazioni per la velocità di un punto:
\begin{equation}
	\frac{d}{dt'}(\vec{x'})=\frac{d}{dt}(\vec{x}-\vec{V}t)=\dot{\vec{x}}+\vec{V}
\end{equation}
Si osserva che generale le Trasformazioni di Galileo compongono le velocità per somma algebrica.\\

Per determinare l'invarianza di un legge fisica rispetto alle trasformazioni (\ref{GalileoEasy}) può essere necessario 
studiare come si trasformino gli operatori di differenziazione. 
Se si vuole derivare una $f(\vec{x},t)$, dalla regola di Leibniz, si ha:
\begin{equation*}
	\begin{gathered}
		\frac{\partial}{\partial t'}=\frac{\partial t}{\partial t'}\frac{\partial}{\partial t}+
		\sum_{i=1}^{3}\frac{\partial x_i}{\partial t}\frac{\partial t}{\partial t'}
		\frac{\partial}{\partial x_i} \\
		\frac{\partial}{\partial x'_i}=\frac{\partial t}{\partial x'_i}\frac{\partial}{\partial t}+
		\sum_{i=1}^{3}\frac{\partial x_j}{\partial x'_i}\frac{\partial}{\partial x_j}
	\end{gathered}
\end{equation*}
osservando dalla (\ref{GalileoEasy}) che $\frac{\partial t}{\partial t'}=1$, che 
$\frac{\partial t}{\partial x'_i}=0$, che $\frac{\partial x_i}{\partial t}=V_i$ e che 
$\frac{\partial x_j}{\partial \xi_i}=\delta_{ij}$, dove $\delta_{ij}$ è una delta di Kronecker, 
si ottengono le trasformazioni degli operatori di differenziazione desiderate:
\begin{equation}
	\frac{\partial}{\partial t'}=\frac{\partial}{\partial t}+\vec{V}\cdot\vec{\nabla} \qquad \qquad
	\frac{\partial}{\partial x'_i}=\frac{\partial}{\partial x_i}
	\label{GalileoDifferenziale}
\end{equation}
Dalla (\ref{GalileoDifferenziale}) si deduce che tutti gli operatori che comprendono solamente derivate 
rispetto alle derivate spaziali sono lasciati inalterati dalle Trasformazioni di Galileo, così che per esempio $\vnabla'=\vnabla$.