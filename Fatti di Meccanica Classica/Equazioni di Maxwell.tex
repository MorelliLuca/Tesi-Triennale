\section{L'elettromagnetismo e le equazioni di Maxwell}
Per interpretare i fenomeni elettromagnetici, anche in questo caso, è necessario introdurre
una serie di osservazioni sperimentali: in primo luogo esiste una proprietà della materia 
detta carica elettrica, che non dipende dal sistema di riferimento in cui è misurata, che consente 
ai corpi di interagire con due campi vettoriali: 
il campo Elettrico $\vec{E}$ e il campo Magnetico $\vec{B}$.\\ Un corpo puntiforme di carica 
$q$ interagendo con questi subisce un forza data da:
\begin{equation}
	\vec{F}=q(\vec{E}(\vec{x},t)+\vec{\dot{x}}\wedge\vec{B}(\vec{x},t))
	\label{ForzaLorentz}
\end{equation}
In secondo luogo gli esperimenti mostrarono che questi due campi rispettano una serie di equazioni 
dette Equazioni di Maxwell:
\begin{equation}
	\begin{gathered}
		\vec{\nabla}\cdot\vec{E}=\frac{\rho}{\epsilon_0} \qquad \qquad \vec{\nabla}\cdot\vec{B}=0 \\
		\vec{\nabla}\wedge\vec{E}=-\frac{\partial\vec{B}}{\partial t} \qquad \qquad \vec{\nabla}\wedge
		\vec{B}=\mu_0\vec{J}+\epsilon_0\mu_0\frac{\partial\vec{E}}{\partial t}
		\label{EquazioniMaxwell}
	\end{gathered}
\end{equation}
dove $\rho$ è la densità di carica volumetrica, $\vec{J}$ è la densità di corrente superficiale e 
$\epsilon_0$, $\mu_0$ sono due costanti del vuoto.\\

Dalle Equazioni di Maxwell (\ref{EquazioniMaxwell}) segue che le cariche sono sorgenti del campo 
Elettrico mentre le correnti lo sono per 
il campo Magnetico.\\ 

Inoltre si può ottenere, calcolando il rotore di ambo 
i membri delle ultime due
\begin{equation*}
	\vec{\nabla}\wedge\vec{\nabla}\wedge\vec{E}=-\frac{\partial}{\partial t}\vnabla\wedge\vec{B} 
	\qquad \vec{\nabla}\wedge\vec{\nabla}\wedge\vec{B}=\mu_0\epsilon_0\frac{\partial}{\partial t}
	\vnabla\wedge\vec{E}
\end{equation*}
e supponendo assenza di cariche per cui $\rho=0$ e $\vec{J}=0$, due equazioni che descrivono 
onde di campo Elettrico e Magnetico nel vuoto:
\begin{equation}
	\vnabla^2\vec{E}=\mu_0\epsilon_0\frac{\partial^2\vec{E}}{\partial t^2} \qquad \vnabla^2\vec{B}=
	\mu_0\epsilon_0\frac{\partial^2\vec{B}}{\partial t^2}
\end{equation}

Queste onde si propagano con una velocità $\frac{1}{\sqrt{\mu_0\epsilon_0}}=2.99795\ \ 10^8\  \frac{m}{s}$, 
che corrisponde con precisione ai valori sperimentalmente misurati della velocità della luce. 
Sulla base della teoria ondulatoria classica è però necessario identificare un mezzo nel quale queste onde possano 
propagarsi e rispetto al quale la loro velocità di propagazione deve essere intesa. Per questo motivo alla fine dell'ottocento venne ipotizzata l'esistenza di tale mezzo detto Etere Luminifero.

 