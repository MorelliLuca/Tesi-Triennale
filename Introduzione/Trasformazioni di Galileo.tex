\section{La meccanica e le trasformazioni di Galileo}
La meccanica classica, in tutte le sue possibili formulazioni, ha come fondamento una serie 
di osservazioni sperimentali che vengono utilizzate come principi da cui dedurre le leggi del moto.\\

Il primo fatto sperimentale che viene assunto è che lo spazio sia tridimensionale, isotropo, omogeneo e che rispetti la 
geometria euclidea mentre il tempo sia ad una sola dimensione e che sia assoluto come sono assolute le distanze spaziali.\\
Sulla base di queste assunzioni si può quindi scegliere un punto dello spazio-tempo come 
origine di un sistema di coordinate o di riferimento, ossia uno spazio vettoriale 
$\mathbb{R}^3\times\mathbb{R}$ che ha come vettore nullo il punto scelto. Osserviamo che il punto 
in questione è arbitrario, come lo è la direzione degli assi corrispondenti ai vettori della base, poichè spazio e tempo sono isotropi 
ed omogenei.\\

Il secondo fatto sperimentale prende il nome di Principio di Relatività Galileiano e consiste 
nell'assunzione che esistano una serie di sistemi di riferimento detti inerziali, caratterizzati 
dalla proprietà di essere reciprocamente in moto rettilineo uniforme, in cui le leggi della natura 
in ogni istante assumono la stessa forma.\\

Infine si assume che le posizioni e le velocità dei punti di un sistema ad un tempo iniziale determinino 
in maniera univoca l'evoluzione del sistema secondo la legge:
\begin{equation}
	m\ddot{\vec{x}}=\vec{F}(\vec{x},\dot{\vec{x}},t)
	\label{equazioneDiNewton}
\end{equation}
dove $m$ è detta massa inerziale e $\vec{F}$ è una funzione caratteristica del sistema detta forza.\\

Vogliamo quindi identificare quali applicazioni $\varphi:\mathbb{R}^3\times\mathbb{R}\rightarrow
\mathbb{R}^3\times\mathbb{R}$ ci consentono di cambiare sistema di riferimento inerziale, ossia quali 
trasformazioni non variano le leggi della natura. Queste applicazioni si chiamano Trasformazioni di 
Galileo ed è immediato dalle ipotesi sperimentali concludere che queste sono costituite dalle 
composizioni di tre famiglie di applicazioni:
\begin{itemize}
	\item una generica traslazione spazio temporale dell'origine, dedotta dalla proprietà 
	di omogeneità dello spazio e del tempo:
	\begin{equation}
		\varphi_{\vec{r},s}(\vec{x},t)=(\vec{x}+\vec{r},t+s)
		\label{GalileoTraslazoine}
	\end{equation} 
\item una generica rotazione degli assi spaziali, dovuta alla proprietà di isotropia dello spazio:
\begin{equation}
	\varphi_{G}(\vec{x},t)=(G\vec{x},t) \qquad G\in M_{3\times3}(\mathbb{R}):G^{-1}=G^t
	\label{GalileoRotazione}
\end{equation} 
	\item una traslazione di moto rettilineo uniforme, ammissibile grazie alle proprietà dei sistemi 
	di riferimento inerziali:
\begin{equation}
	\varphi_{\vec{v}}(\vec{x},t)=(\vec{x}+\vec{v}t,t)
	\label{GalileoVelocità}
\end{equation} 
\end{itemize}
Quest'ultima tipologia di trasformazione è la più importante dal punto di vista fisico ed è quella 
che viene comunemente studiata per caratterizzare le trasformazioni di sistemi inerziali. 
Consideriamo quindi un sistema $K$, con coordinate $(\vec{x},t)$ e un sistema $k$, con 
coordinate $(\vec{\xi},\tau)$, in moto a velocità $\vec{v}$ rispetto a $K$, scriveremo allora la 
(\ref{GalileoVelocità}) come:
\begin{equation}
	\vec\xi=\vec{x}-\vec{v}t \qquad \tau=t
	\label{GalileoEasy}
\end{equation}

Per determinare l'invarianza di un legge fisica rispetto a queste trasformazioni è necessario 
studiare come si trasformino gli operatori di differenziazione con la  (\ref{GalileoEasy}). 
Se vogliamo derivare una $f(\vec{x},t)$, dalla regola di Leibniz, abbiamo:
\begin{equation*}
	\begin{gathered}
		\frac{\partial}{\partial \tau}=\frac{\partial t}{\partial \tau}\frac{\partial}{\partial t}+
		\sum_{i=1}^{3}\frac{\partial x_i}{\partial t}\frac{\partial t}{\partial \tau}
		\frac{\partial}{\partial x_i} \\
		\frac{\partial}{\partial \xi_i}=\frac{\partial t}{\partial \xi_i}\frac{\partial}{\partial t}+
		\sum_{i=1}^{3}\frac{\partial x_j}{\partial \xi_i}\frac{\partial}{\partial x_j}
	\end{gathered}
\end{equation*}
osservando dalla (\ref{GalileoEasy}) che $\frac{\partial t}{\partial \tau}=1$, che 
$\frac{\partial t}{\partial \xi_i}=0$, che $\frac{\partial x_i}{\partial t}=v_i$ e che 
$\frac{\partial x_j}{\partial \xi_i}=\delta_{ij}$, dove $\delta_{ij}$ è una delta di Kronecker, 
otteniamo le trasformazioni degli operatori di differenziazione che stavamo cercando:
\begin{equation}
	\frac{\partial}{\partial \tau}=\frac{\partial}{\partial t}+\vec{v}\cdot\vec{\nabla} \qquad \qquad
	\frac{\partial}{\partial \xi_i}=\frac{\partial}{\partial x_i}
\end{equation}
\newpage