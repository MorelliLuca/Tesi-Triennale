\section*{Notazioni utilizzate}
\vspace*{\fill}
$\frac{d}{dt}$ è la derivata totale rispetto al tempo, indicata anche con un punto.\\
$\frac{\partial}{\partial x}$ è la derivata parziale rispetto a $x$.\\
\subsection*{Notazioni di grandezze tridimensionali}
Il simbolo freccia denota un vettore $\vec{a}\in\mathbb{R}^3$ le cui cartesiane sono indicate con $v_x,\ v_y\ , v_z$.\\
$\vec{x}$ e $\vec{v}$ sono vettori posizione e velocità.\\
$q_i$ e $\dot q_i$ sono le coordinate e le velocità generalizzate. \\
$\vec{p}$ e $E$ sono la quantità di moto o impulso e l'energia.\\
Il campo elettrico e magnetico sono indicati con $\vec{E}\ \text{e}\ \vec{B}$.\\
Si è deciso di fare uso delle unità del Sistema Internazionale così che campo elettrico e magnetico dipendono dalle costanti dielettriche e magnetiche del vuoto $\epsilon_0$ e $\mu_0$.\\
$\vec a\cdot\vec b$ indica il prodotto scalare euclideo.\\
$\vnabla$ è l'operatore $(\frac{\partial}{\partial x},\frac{\partial}{\partial y},\frac{\partial}{\partial z})$.\\
\subsection*{Notazioni di grandezze quadridimensionali}
Con $A^\mu$ si indica in 4-vettore (quadrivettore) controvariante di $\mathbb{R}^4$ mentre con $A_\mu$ se ne intende uno covariante.\\
Le componenti di $A^\mu$ sono date dal variare di $\mu$ tra $0$ e $1$, anche indicate con $A^\mu=(A^0,\vec{A})$.\\
Due indici ripetuti sottintendono una sommatoria $A^\mu B^\mu=\sum_\mu A^\mu B^\mu$.\\
Matrice metrica $g_{\mu \nu}$ con segnatura $(+,-,-,-)$.\\
$|A^\mu|$ è la norma Minkowsky data da: $A^\mu A_\mu=(A^0)^2-(A^1)^2-(A^2)^2-(A^3)^2$.\\
