\section*{Notazioni utilizzate}
\addcontentsline{toc}{section}{Notazioni utilizzate}
\vspace*{\fill}
$\frac{d}{dt}$ è la derivata totale rispetto al tempo, indicata anche con un punto.\\

$\frac{\partial}{\partial x}$ è la derivata parziale rispetto a $x$.\\
\subsection*{Notazioni di grandezze tridimensionali}
Il simbolo $\vec{a}$ denota un vettore di $\mathbb{R}^3$ le cui cartesiane sono indicate con $a_x,\ a_y\ , a_z$.\\
$\vec{x}$ e $\vec{v}$ i sono vettori posizione e velocità.\\
$q_i$ e $\dot q_i$ sono le coordinate e le velocità generalizzate. \\
$\vec{p}$ e $E$ sono la quantità di moto (o impulso) e l'energia.\\
Il campo elettrico e magnetico sono indicati con $\vec{E}\ \text{e}\ \vec{B}$.\\
Si è deciso di fare uso delle unità del Sistema Internazionale così i campi elettrico e magnetico dipendono dalle costanti dielettriche e magnetiche del vuoto $\epsilon_0$ e $\mu_0$.\\
$\vec a\cdot\vec b$ indica il prodotto scalare euclideo.\\
$\vec a\wedge\vec b$ indica il prodotto vettoriale.\\
$\vnabla$ è l'operatore $(\frac{\partial}{\partial x},\frac{\partial}{\partial y},\frac{\partial}{\partial z})$.\\
\subsection*{Notazioni di grandezze quadridimensionali}
Con $A^\mu$ si indica un 4-vettore (quadrivettore) controvariante di $\mathbb{R}^4$ mentre con $A_\mu$ si intende un 4-vettore covariante.\\
Le componenti di $A^\mu$ sono date dal variare di $\mu$ (in generale di lettere greche) tra $0$ e $3$, si usa anche la notazione $A^\mu=(A^0,\vec{A})$.\\
Con lettere latine sono indicate le componenti spaziali di un 4-vettore, così che $A^i$ indichi solamente $A^1,\ A^2,\ A^3$.\\
Due indici ripetuti sottintendono una sommatoria $A^\mu B^\mu=\sum_{\mu=0}^{3} A^\mu B^\mu$.\\
$g_{\mu \nu}$ è la matrice metrica con segnatura $(+,-,-,-)$.\\
$|A|$ è la norma Minkowski data da: $A^\mu A_\mu=(A^0)^2-(A^1)^2-(A^2)^2-(A^3)^2=g_{\mu \nu}A^\mu A^\nu$.\\
