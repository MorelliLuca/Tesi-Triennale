\section{L'elettromagnetismo e le equazioni di Maxwell}
Per interpretare i fenomeni elettromagnetici, anche in questo caso, è necessario introdurre
una serie di osservazioni sperimentali: in primo luogo esiste una proprietà della materia 
detta carica elettrica, che non dipende dal sistema di riferimento in cui è misurata, che consente 
ai corpi di interagire con due campi vettoriali: 
il campo Elettrico $\vec{E}$ e il campo Magnetico $\vec{B}$.\\ Un corpo puntiforme di carica 
$q$ interagendo con questi subisce un forza data da:
\begin{equation}
	\vec{F}=q(\vec{E}(\vec{x},t)+\vec{\dot{x}}\wedge\vec{B}(\vec{x},t))
	\label{ForzaLorentz}
\end{equation}
In secondo luogo gli esperimenti mostrarono che questi due campi rispettano una serie di equazioni 
dette Equazioni di Maxwell:
\begin{equation}
	\begin{gathered}
		\vec{\nabla}\cdot\vec{E}=\frac{\rho}{\epsilon_0} \qquad \qquad \vec{\nabla}\cdot\vec{B}=0 \\
		\vec{\nabla}\wedge\vec{E}=-\frac{\partial\vec{B}}{\partial t} \qquad \qquad \vec{\nabla}\wedge
		\vec{B}=\mu_0\vec{J}+\epsilon_0\mu_0\frac{\partial\vec{E}}{\partial t}
		\label{EquazioniMaxwell}
	\end{gathered}
\end{equation}
dove $\rho$ è la densità di carica volumetrica, $\vec{J}$ è la densità di corrente superficiale e 
$\epsilon_0$, $\mu_0$ sono due costanti del vuoto.\\

Si può ottenere dalle Equazioni di Maxwell (\ref{EquazioniMaxwell}), calcolando il rotore di ambo 
i membri delle ultime due
\begin{equation*}
	\vec{\nabla}\wedge\vec{\nabla}\wedge\vec{E}=-\frac{\partial}{\partial t}\vnabla\wedge\vec{B} 
	\qquad \vec{\nabla}\wedge\vec{\nabla}\wedge\vec{B}=\mu_0\epsilon_0\frac{\partial}{\partial t}
	\vnabla\wedge\vec{E}
\end{equation*}
e supponendo assenza di cariche per cui $\rho=0$ e $\vec{J}=0$, due equazioni che descrivono 
onde di campo Elettrico e Magnetico nel vuoto:
\begin{equation}
	\vnabla^2\vec{E}=\mu_0\epsilon_0\frac{\partial^2\vec{E}}{\partial t^2} \qquad \vnabla^2\vec{B}=
	\mu_0\epsilon_0\frac{\partial^2\vec{B}}{\partial t^2}
\end{equation}

Queste onde si propagano con una velocità $\frac{1}{\sqrt{\mu_0\epsilon_0}}=2.99795\ \ 10^8\  \frac{m}{s}$, 
indipendentemente dal sistema di riferimento, che corrisponde con precisione ai valori
 sperimentalmente misurati della velocità della luce.  

 \section{La non invarianza delle Equazioni di Maxwell}

 Si considerino due sistemi di riferimento $K$ e $K'$, inerziali e reciprocamente in moto a velocità $\vec{V}$, in ogni sistema 
 si misureranno rispettivamente $\vec{E},\ \vec{B}$ e $\vec{E'},\ \vec{B'}$. Il principio di relatività galileiana impone che 
 questi due campi, nei loro sistemi di riferimento, rispettino le Equazioni di Maxwell (\ref{EquazioniMaxwell}), inoltre, siccome la forza (\ref{ForzaLorentz})
 deve essere la medesima in tutti i sistemi di riferimento inerziali, considerando una carica $q$ in moto con velocità $\vec{v}$ 
 in $K$ e $\vec{V}+\vec{v}$ in $K'$ deve valere:
 \begin{equation*}
	\begin{gathered}
		\vec{F'}=\vec{F}\quad\Rightarrow\quad \vec{E'}+\vec{v}\wedge\vec{B'}=\vec{E}+(\vec{V}+\vec{v})\wedge\vec{B}\\
         \Rightarrow\quad \vec{E'}+\vec{v}\wedge(\vec{B'}-\vec{B})=\vec{E}+\vec{V}\wedge\vec{B'}
	\end{gathered}
 \end{equation*}
Così facendo si possono ottenere le trasformazioni dei capi Elettrici e Magnetici tra sistemi inerziali che però devono dipendere esclusivamente dai due sistemi di riferimento 
considerati e nella fattispecie dalla loro velocità reciproca, poichè non ha alcun senso che dipendano dalla velocità di un corpo 
specifico in uno dei due sistemi, per questo motivo il termine contnente $\vec{v}$ deve annullarsi e così le trasformazioni risultano:
\begin{equation}
	\begin{cases}
		\vec{E'}(\vec{x'},t)=\vec{E}(\vec{x},t)+\vec{V}\wedge\vec{B}(\vec{x},t)\\
		\vec{B'}(\vec{x'},t)=\vec{B}(\vec{x},t)
	\end{cases}
	\label{TrasfGalileoEB}
\end{equation}
Bisonga ora studiare come si trasformano le grandezze generatici dei campi:
$\rho$ e $\vec{J}$. Se si considera una volume $\Delta V$, in cui è presente una carica 
$\Delta q$, allora la densità di carica è definita come:
\begin{equation*}
	\rho=\lim_{\Delta V\rightarrow 0}\frac{\Delta q}{\Delta V}
\end{equation*}
Siccome le lunghezze sono assunte essere assolute devono esserlo pure i volumi ed, 
essendo la carica non dipendente dal sistema di riferimento, si conclude che pure la 
densità di carica non lo è. Per quanto riguarda la densità di corrente superficiale, 
definita come $\vec{J}=\rho\vec{v}$,  
è sufficiente applicare le trasformazioni delle velocità tra due sistemi in moto reciproco 
a velocità $\vec{V}$ per ottenere:
\begin{equation}
	\vec{J'}=\vec{J}-\rho\vec{V}
\end{equation}
I risultati appena ottenuti consentono di determinare l'invarianza delle Equazioni di 
Maxwell per le Trasformazioni di Galileo. Considerando la prima equazione in $K'$, 
trasformando $E'$ in $E$ e analogmante per gli operatori di differenziazione secondo la (\ref{GalileoDifferenziale}):
\begin{align*}
	\vnabla'\cdot\vec{E'}-\frac{\rho}{\epsilon_0}=\vnabla\cdot(\vec{E}+\vec{V}\wedge\vec{B})-\frac{\rho}{\epsilon_0}\\
	=\left(\vnabla\cdot\vec{E}-\frac{\rho}{\epsilon_0}\right)-\vec{V}\cdot(\vnabla\wedge\vec{B})=-\vec{V}\cdot(\vnabla\wedge\vec{B})
\end{align*}
Perchè sia valida questa Equazione di Maxwell in $K'$ deve annullarsi questa quantità: 
il termine tra parentesi è identicamente nullo poichè valgono le Equazioni di Maxwell in $K$ 
mentre l'ultimo termine non è sempre nullo, nella fattispecie in presenza di campi eletrici variabili nel tempo, 
questo implica che quindi la prima Equazione di Maxwell non è invariante per le Trasfromazioni di Galileo.
Se si studia la seconda con lo stesso procedimento si scopre che questa invece è invariante:
\begin{equation*}
	\vnabla'\cdot\vec{B'}=\vnabla\cdot\vec{B}=0
\end{equation*}
Analogamente per la terza equazione:
\begin{align*}
	\vnabla'\wedge\vec{E'}+\frac{\partial \vec{B'}}{\partial t'}=
	\vnabla\wedge\vec{E'}+\frac{\partial \vec{B'}}{\partial t}+(\vec{V}\cdot\vnabla)\vec{B'}\\
	=\vnabla\wedge(\vec{E}+\vec{V}\wedge\vec{B})+\frac{\partial \vec{B}}{\partial t}+(\vec{V}\cdot\vnabla)\vec{B}\\
	=\left(\vnabla\wedge\vec{E}+\frac{\partial \vec{B}}{\partial t}\right)+\vnabla\wedge\vec{V}\wedge\vec{B}+(\vec{V}\cdot\vnabla)\vec{B}=0
\end{align*} 
infatti il termine tra parentesi è identicamente nullo poichè in $K$ vale la terza Equazione di Maxwell 
e gli addendi restanti si annullano se sviluppati tramite le regole di differenziazione ricordando che $\vec{V}$ è costante:
\begin{equation*}
	\vnabla\wedge\vec{V}\wedge\vec{B}+(\vec{V}\cdot\vnabla)\vec{B}=-(\vec{V}\cdot\vnabla)\vec{B}+(\vec{V}\cdot\vnabla)\vec{B}=0
\end{equation*}
L'ultima Equazione di Maxwell è invece non invariante infatti sempre con il medesimo procedimento si ottiene:
\begin{align*}
	\vnabla'\wedge\vec{B'}-\mu_0\epsilon_0\frac{\partial \vec{E'}}{\partial t'}-\mu_0\vec{J'}=
	\vnabla\wedge\vec{B'}-\mu_0\epsilon_0\frac{\partial \vec{E'}}{\partial t}-\mu_0\epsilon_0(\vec{V}\cdot\vnabla)\vec{E'}-\mu_0\vec{J'}\\
	=\vnabla\wedge\vec{B}-\mu_0\epsilon_0\frac{\partial \vec{E}}{\partial t}-\mu_0\epsilon_0\vec{V}\wedge\frac{\partial \vec{B}}{\partial t}-\mu_0\epsilon_0(\vec{V}\cdot\vnabla)(\vec{E}+\vec{V}\wedge\vec{B})-\mu_0\vec{J}+\mu_0\vec{V}\rho\\
	=\left(\vnabla\wedge\vec{B}-\mu_0\epsilon_0\frac{\partial \vec{E}}{\partial t}-\mu_0\vec{J}\right)-\mu_0\epsilon_0\vec{V}\wedge\frac{\partial \vec{B}}{\partial t}-\mu_0\epsilon_0(\vec{V}\cdot\vnabla)(\vec{E}+\vec{V}\wedge\vec{B})+\mu_0\vec{V}\rho\\
	=-\mu_0\epsilon_0\vec{V}\wedge(-\vnabla\wedge\vec{E})-\mu_0\epsilon_0(\vec{V}\cdot\vnabla)(\vec{E}+\vec{V}\wedge\vec{B})+\mu_0\vec{V}\rho\\
	=-\mu_0\epsilon_0(\vec{V}\cdot\vnabla)(\vec{V}\wedge\vec{B})+\mu_0\vec{V}\rho
\end{align*}
che non è nullo in generale con le assunzioni fin qui fatte.\\

Si è quindi giunti alla conclusione che la teoria di Maxwell non è conciliabile con 
la meccanica di Newton e viceversa.