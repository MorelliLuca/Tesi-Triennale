\chapter*{Introduzione}   
\addcontentsline{toc}{section}{Introduzione}

La fisica sviluppatasi fino alla fine dell'ottocento descriveva con precisione i fenomeni meccanici noti sulla base di una serie di principi sperimentali. Queste osservazioni sperimentali definiscono classicamente i concetti di spazio e tempo e i sistemi di riferimento inerziali (particolari sistemi nei quali le leggi della fisica sono sempre le medesime).\\
Contemporaneamente, gli studi ottocenteschi sull'elettromagnetismo avevano consentito di modellizzare tutti i fenomeni elettromagnetici osservati tramite le equazioni di Maxwell (una serie di equazioni differenziali che descrivono le proprietà dei campi elettrici e magnetici). Queste equazioni e nella fattispecie le loro soluzioni d'onda (che sono le onde elettromagnetiche come la luce visibile) misero in evidenza l'incompatibilità di questa teoria di Maxwell con la meccanica classica. In modo particolare ci si rese conto che le equazioni di Maxwell non risultano identiche in ogni sistema di riferimento inerziale (secondo la meccanica classica).\\
Nel 1905 Albert Einstein propose, nel suo articolo \emph{"Zur Elektrodynamik bewegter Körper"} \cite{Einstein1905}, un'interpretazione risolutiva degli esperimenti che cercavano di chiarire quale teoria fosse corretta tra la meccanica classica e l'elettromagnetismo di Maxwell. Einstein riformulò i principi sperimentali alla base della meccanica postulando la costanza della velocità della luce in ogni sistema di riferimento inerziale (come suggerivano le equazioni di Maxwell). Sulla base dei nuovi principi di Einstein (noti come postulati) è possibile costruire l'intera teoria della relatività ristretta. Il fulcro di questa teoria è costituito dalle trasformazioni di Lorentz che consentono di passare da un sistema di riferimento inerziale ad un secondo.\\
Lo sviluppo di questa teoria modificò radicalmente le idee di spazio e tempo radicate nella fisica, enti assoluti secondo la concezione newtoniana. Inoltre si capì che non è più possibile fare distinzione tra spazio e tempo ma è necessario intenderli come un unico spazio-tempo dotato di un suo preciso formalismo.\\
Nel Capitolo 1 si approfondiranno tutti questi aspetti.\\

Una volta sviluppati i fondamenti della teoria della relatività fu necessario delineare la meccanica che descrive il moto dei corpi coerentemente con i postulati di Einstein. Nel Capitolo 2 si procederà a dare una descrizione della meccanica relativistica nel formalismo lagrangiano per una particella libera e per sistemi di più particelle.\\

L'elettromagnetismo, la cui descrizione maxwelliana risultò corretta, si rivela quindi come la naturale descrizione di una interazione relativistica che tenga conto di quanto scoperto da Einstein. Per questo motivo nel Capitolo 3 si descriverà come tradurre nel formalismo dello spazio-tempo relativistico tutti i concetti propri dell'elettromagnetismo: dalle cariche e le correnti fino a i campi elettrici e magnetici e i loro potenziali.\\

Per concludere, nel Capitolo 4, si farà una breve introduzione della teoria dei campi e di come questa possa essere utilizzata per descrivere i campi elettromagnetici. Nella fattispecie si ricaverà la densità di lagrangiana del campo elettromagnetico e si mostrerà come questa consenta di ottenere le equazioni di Maxwell.